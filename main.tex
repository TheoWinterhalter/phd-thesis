% !TEX program = xelatex
% !TEX options = -synctex=1 -interaction=nonstopmode -file-line-error --shell-escape "%DOC%"

%----------------------------------------------------------------------------------------
%	PACKAGES AND OTHER DOCUMENT CONFIGURATIONS
%----------------------------------------------------------------------------------------

\documentclass[
  fontsize=10pt,
  twoside=false
  % overfullrule
]{kaobook}

% Choose the language
\usepackage[english]{babel} % Load characters and hyphenation
\usepackage[english=british]{csquotes}	% English quotes

% Load the bibliography package
\usepackage{styles/kaobiblio}
\addbibresource{main.bib} % Bibliography file

% Load mathematical packages for theorems and related environments. NOTE: choose only one between 'mdftheorems' and 'plaintheorems'.
\usepackage{styles/mdftheorems}
%\usepackage{styles/plaintheorems}

% \graphicspath{{examples/documentation/images/}{images/}} % Paths in which to look for images

\makeindex[columns=3, title=Alphabetical Index, intoc] % Make LaTeX produce the files required to compile the index

\makeglossaries % Make LaTeX produce the files required to compile the glossary

\makenomenclature % Make LaTeX produce the files required to compile the nomenclature

% Reset sidenote counter at chapters
%\counterwithin*{sidenote}{chapter}

%%% Then my own packages

\usepackage{fontspec}
\setmainfont[Ligatures=TeX]{Fira Sans}
% \setsansfont[Ligatures=TeX]{Fira Sans}

\usepackage{mathpartir}

% For coq higlighting
\usepackage{minted}

% Meta comment
\newcommand\meta[1]{\noindent\textcolor{blue}{\emph{#1}}}
\newcommand\todo{\meta{TODO}}
\def\ms#1{\todo{} (MS) \meta{#1}}
\def\nt#1{\todo{} (NT) \meta{#1}}
\def\tw#1{\todo{} (TW) \meta{#1}}

% Names
\def\name#1{\textsf{#1}\xspace}
\def\Coq{\name{Coq}}
\def\TemplateCoq{\name{TemplateCoq}}
\def\Equations{\name{Equations}}
\def\Andromeda{\name{Andromeda}}
\def\NuPRL{\name{NuPRL}}
\def\Agda{\name{Agda}}
\def\Epigram{\name{EPIGRAM}}

%----------------------------------------------------------------------------------------

\begin{document}

%----------------------------------------------------------------------------------------
%	BOOK INFORMATION
%----------------------------------------------------------------------------------------

\titlehead{Some text}
\subject{PhD Thesis}

\title{Formalisation and Meta-Theory of Type Theory}
\subtitle{Customise this page according to your needs}

\author{Théo Winterhalter}

\date{\today}

\publishers{An Awesome Publisher}

%----------------------------------------------------------------------------------------

\frontmatter % Denotes the start of the pre-document content, uses roman numerals

%----------------------------------------------------------------------------------------
%	OPENING PAGE
%----------------------------------------------------------------------------------------

%\makeatletter
%\extratitle{
%	% In the title page, the title is vspaced by 9.5\baselineskip
%	\vspace*{9\baselineskip}
%	\vspace*{\parskip}
%	\begin{center}
%		% In the title page, \huge is set after the komafont for title
%		\usekomafont{title}\huge\@title
%	\end{center}
%}
%\makeatother

%----------------------------------------------------------------------------------------
%	COPYRIGHT PAGE
%----------------------------------------------------------------------------------------

\makeatletter
\uppertitleback{\@titlehead} % Header

\lowertitleback{
	\textbf{Disclaimer}\\
	You can edit this page to suit your needs. For instance, here we have a no copyright statement, a colophon and some other information. This page is based on the corresponding page of Ken Arroyo Ohori's thesis, with minimal changes.

	\medskip

	\textbf{No copyright}\\
	\cczero\ This book is released into the public domain using the CC0 code. To the extent possible under law, I waive all copyright and related or neighbouring rights to this work.

	To view a copy of the CC0 code, visit: \\\url{http://creativecommons.org/publicdomain/zero/1.0/}

	\medskip

	\textbf{Colophon} \\
	This document was typeset with the help of \href{https://sourceforge.net/projects/koma-script/}{\KOMAScript} and \href{https://www.latex-project.org/}{\LaTeX} using the \href{https://github.com/fmarotta/kaobook/}{kaobook} class.

	The source code of this book is available at:\\\url{https://github.com/fmarotta/kaobook}

	(You are welcome to contribute!)

	\medskip

	\textbf{Publisher} \\
	First printed in May 2019 by \@publishers
}
\makeatother

%----------------------------------------------------------------------------------------
%	DEDICATION
%----------------------------------------------------------------------------------------

\dedication{
	The harmony of the world is made manifest in Form and Number, and the heart and soul and all the poetry of Natural Philosophy are embodied in the concept of mathematical beauty.\\
	\flushright -- D'Arcy Wentworth Thompson
}

%----------------------------------------------------------------------------------------
%	OUTPUT TITLE PAGE AND PREVIOUS
%----------------------------------------------------------------------------------------

% Note that \maketitle outputs the pages before here

% If twoside=false, \uppertitleback and \lowertitleback are not printed
% To overcome this issue, we set twoside=semi just before printing the title pages, and set it back to false just after the title pages
\KOMAoptions{twoside=semi}
\maketitle
\KOMAoptions{twoside=false}

%----------------------------------------------------------------------------------------
%	PREFACE
%----------------------------------------------------------------------------------------

%\input{chapters/preface.tex}

%----------------------------------------------------------------------------------------
%	TABLE OF CONTENTS & LIST OF FIGURES/TABLES
%----------------------------------------------------------------------------------------

\begingroup % Local scope for the following commands

% Define the style for the TOC, LOF, and LOT
%\setstretch{1} % Uncomment to modify line spacing in the ToC
%\hypersetup{linkcolor=blue} % Uncomment to set the colour of links in the ToC
\setlength{\textheight}{23cm} % Manually adjust the height of the ToC pages

% Turn on compatibility mode for the etoc package
\etocstandarddisplaystyle % "toc display" as if etoc was not loaded
\etocstandardlines % "toc lines as if etoc was not loaded

\tableofcontents % Output the table of contents

\listoffigures % Output the list of figures

% Comment both of the following lines to have the LOF and the LOT on different pages
\let\cleardoublepage\bigskip
\let\clearpage\bigskip

\listoftables % Output the list of tables

\endgroup

%----------------------------------------------------------------------------------------
%	MAIN BODY
%----------------------------------------------------------------------------------------

\mainmatter % Denotes the start of the main document content, resets page numbering and uses arabic numbers
\setchapterstyle{kao} % Choose the default chapter heading style

% \setchapterpreamble[u]{\margintoc}
\chapter{Introduction}
\labch{intro}

\todo{Nicolas: Speak about ``derivation checking''. Where? How?}

Type theory places itself at the interface between programming and formal logic,
feeding off and nourishing both worlds. Proof assistants can be built on it,
making them full-fledged programming languages as well.

My main interest lies in the study of type theory while relying on the tools it
provides: I study type theory \emph{in} type theory.
\todo{Just keep writing what I want to put in here and then write it}

\paradot{Contributions}
My contributions will mainly be found in \arefpart{elim-reflection} and
\arefpart{coq-in-coq} corresponding to the following published
articles~\sidecite{winterhalter:hal-01849166,sozeau2019coq,sozeau:hal-02167423}.
While the chapters before these two parts are mainly introductory and
corresponding to a rough state-of-the-art, they actually contain other
contributions, namely work I did with Andrej Bauer on the cardinal model
in \nrefch{models} that we didn't publish, and work I did with Andrej Bauer
and Philipp Haselwarter on formalising type theory called
\ftt~\sidecite{formaltypetheory} and that I present briefly in
\nrefch{formalisation}.

\paradot{Outline}
\todo{TODO}

\pagelayout{wide} % No margins
\addpart{New Part of the Thesis}
\pagelayout{margin} % Restore margins

%%% General

\newcommand{\inv}{^{-1}}

% \newcommand{\eg}{e.g.\ }
% \newcommand{\ie}{\emph{i.e.,}\xspace}

\newcommand{\paradot}[1]{\paragraph{#1.}}

\newcommand{\exmark}{\ensuremath{\mathsf{x}}}

%% Proof by cases
\newenvironment{caselist}{%
  \begin{list}{{\it Case}}{}%
}{\end{list}%
}
\newenvironment{subcaselist}{%
  \begin{list}{{\it Subcase}}{}%
}{\end{list}%
}
\newenvironment{subsubcaselist}{%
  \begin{list}{{\it Subsubcase}}{}%
}{\end{list}%
}

\newcommand{\nextcase}{\item~}

%%% Type theory

% Meta
\newcommand{\Ax}{\ensuremath{\mathsf{Ax}}}
\newcommand{\Rl}{\ensuremath{\mathsf{R}}}

% Generic entities
\newcommand{\Ga}{\Gamma}   % a context
\newcommand{\D}{\Delta}   % another context
\newcommand{\E}{\Xi}      % another context
\newcommand{\sbs}{\sigma} % a substitution
\newcommand{\sbt}{\theta} % another substitution
\newcommand{\sbr}{\rho}   % a third substitution

%% Syntax
\newcommand{\bnf}{\ \mathrel{{:}{:}{=}}\ }
\newcommand{\bnfor}{\ \mid\ \ }

%% Contexts
\newcommand{\ctxempty}{\bullet} % empty context
% \newcommand{\ctxextend}[2]{#1, #2} % extended context

%\newcommand{\ctxdom}[1]{\mathsf{dom}(#1)} % the domain of a context

%% Substitution

\newcommand{\sto}{\mathop{\leftarrow}}

% Reduction
\newcommand{\red}{\rightarrowtriangle}
\newcommand{\redl}{\leftarrowtriangle}

%% Marker
\newcommand{\stt}[1]{#1^{\mathsf{s}}}
\newcommand{\fm}[1]{#1^{\mathsf{f}}}


% Type formers
\newcommand{\Prod}[1]{\mathop{\Pi(#1).\ }}                 % dependent product
\newcommand{\Sum}[1]{\mathop{\Sigma(#1).}}               % dependent sum
\newcommand{\Eq}[3]{#2 =_{#1} #3}                        % fibrant equality
\newcommand{\Eqs}[3]{#2 \overset{\mathsf{s}}{=}_{#1} #3} % strict equality
\newcommand{\Ty}[1]{\square_{#1}}
\newcommand{\Un}[1]{\mathsf{U}_{#1}}                      % strict universe
\newcommand{\F}[1]{\mathsf{F}_{#1}}                      % fibrant universe
\newcommand{\Type}{\mathsf{Type}}
\newcommand{\Prop}{\mathsf{Prop}}
\newcommand{\nat}{\mathsf{nat}}
% \newcommand{\zero}{\mathsf{zero}}
\newcommand{\zero}{0}
% \newcommand{\natsucc}{\mathsf{succ}}
\newcommand{\natsucc}{\mathbf{S}}
\newcommand{\natrec}{\mathsf{natrec}}
\newcommand{\bool}{\mathsf{bool}}
\newcommand{\unit}{\mathsf{unit}}

% Inductive types
\newcommand{\Ical}{\mathcal{I}}
\newcommand{\at}[1]{\{#1\}}

% Terms
\newcommand{\lam}[2]{\lambda (#1). #2 .}                        % abstraction
\newcommand{\app}[4]{#1\mathbin{@_{#2.#3}} #4}                  % application
\newcommand{\pair}[4]{\langle #3 ; #4 \rangle_{#1.#2}}          % pair
\newcommand{\pio}[3]{\pi_1^{#1.#2}\ #3}                         % first proj
\newcommand{\pit}[3]{\pi_2^{#1.#2}\ #3}                         % second proj
\newcommand{\idpath}[1]{{\mathsf{idpath}_{#1}}\ }               % identity path
\newcommand{\refl}[1]{{\mathsf{refl}}_{#1}\ }                   % reflexivity
\newcommand{\funexts}[5]
  {\stt{\mathsf{funext}}(#1,#2,#3,#4,#5)}                        % strict funext
\newcommand{\funext}[5]
  {\mathsf{funext}(#1,#2,#3,#4,#5)}                             % fibrant funext
\newcommand{\uip}[5]{\mathsf{uip}(#1,#2,#3,#4,#5)}              % uip
\newcommand{\J}[6]{\mathsf{J}(#1,#2,#3,#4,#5,#6)}               % the J elim
\newcommand{\Js}[6]{\mathsf{J}^{\mathsf{s}}(#1,#2,#3,#4,#5,#6)} % the J elim
\newcommand{\ttrue}{\mathsf{true}}
\newcommand{\ffalse}{\mathsf{false}}
\newcommand{\tif}[4]{\mathsf{if}\ #1\ \mathsf{return}\ #2\ \mathsf{then}\ #3\ \mathsf{else}\ #4}
\newcommand{\notfunext}{\mathsf{notfunext}}
\newcommand{\Notfunext}{\mathsf{NotFunExt}}
\newcommand{\tunit}{\mathsf{())}}

% Pattern-matching
\newcommand{\matchs}{\mathsf{match}}
\newcommand{\patt}[2]{\mid #1 \Longrightarrow #2}
\newcommand{\match}[3]{%
  \begin{array}{l}%
    \matchs\ #1 \\ \mathsf{return}\ #2\ \mathsf{with} \\ #3 \\ \mathsf{end}%
  \end{array}%
}

% Vectors
\newcommand{\av}{\vec{a}}
\newcommand{\bv}{\vec{b}}
\newcommand{\iv}{\vec{i}}
\newcommand{\pv}{\vec{p}}
\newcommand{\sv}{\vec{s}}
\newcommand{\Av}{\vec{A}}
\newcommand{\Iv}{\vec{I}}
\newcommand{\Nv}{\vec{N}}
\newcommand{\Pv}{\vec{P}}
\newcommand{\Vv}{\vec{V}}
\newcommand{\Xv}{\vec{X}}
\newcommand{\Yv}{\vec{Y}}

% Judgments
\newcommand{\isctx}[1]{\vdash #1}          % well formed context
\newcommand{\istype}[2]{#1 \vdash #2}      % well formed type
\newcommand{\isterm}[3]{#1 \vdash #2 : #3} % well formed term

\newcommand{\eqtype}[3]{#1 \vdash #2 \equiv #3}      % equal types
\newcommand{\eqterm}[4]{#1 \vdash #2 \equiv #3 : #4} % equal terms

\newcommand{\isjg}[2]{#1 \vdash \mathcal{#2}} % arbitrary judgment

% Ext. judgments
\newcommand{\xisctx}[1]{\vdash_{\exmark} #1}          % well formed context
\newcommand{\xistype}[2]{#1 \vdash_{\exmark} #2}      % well formed type
\newcommand{\xisterm}[3]{#1 \vdash_{\exmark} #2 : #3} % well formed term

\newcommand{\xeqtype}[3]{#1 \vdash_{\exmark} #2 \equiv #3}      % equal types
\newcommand{\xeqterm}[4]{#1 \vdash_{\exmark} #2 \equiv #3 : #4} % equal terms

\newcommand{\xisjg}[2]{#1 \vdash_{\exmark} \mathcal{#2}} % arbitrary judgment

% Translation
\newcommand{\transl}[1]{\llbracket #1 \rrbracket}
\newcommand{\So}{\mathsf{S}}
\newcommand{\HTS}{\mathsf{HTS}}
\newcommand{\Gb}{\overline{\Ga}}
\newcommand{\Db}{\overline{\D}}
\newcommand{\Ab}{\overline{A}}
\newcommand{\Bb}{\overline{B}}
\newcommand{\Tb}{\overline{T}}
\newcommand{\Pb}{\overline{P}}
\newcommand{\eb}{\overline{e}}
\newcommand{\jg}{\mathcal{J}}
\newcommand{\jgb}{\overline{\jg}}
\newcommand{\ub}{\overline{u}}
\newcommand{\vb}{\overline{v}}
\newcommand{\wb}{\overline{w}}
\newcommand{\pb}{\overline{p}}
\newcommand{\tb}{\overline{t}}
\newcommand{\fb}{\overline{f}}
\newcommand{\gb}{\overline{g}}

\newcommand{\transpo}[1]{#1_*}
\newcommand{\otransport}[4]{\mathsf{transport}'_{#1,#2}(#3,#4)}

\newcommand{\translitivity}[2]{\mathsf{transitivity}(#1,#2)}
\newcommand{\otransitivity}[2]{\mathsf{transitivity}'(#1,#2)}

% Operations
\newcommand{\nmax}[2]{\mathsf{max}(#1,#2)}

% Notations
\newcommand{\Heqs}{\cong}
\newcommand{\Heq}[4]{#2 \mathrel{{}_{#1}{\Heqs}_{#3}} #4}
\newcommand{\ir}{\sqsubset}
\newcommand{\Pack}[2]{\mathsf{Pack}\ #1\ #2}
% \newcommand{\ProjO}[3]{\mathsf{Proj}_1\ #1\ #2\ #3}
% \newcommand{\ProjT}[3]{\mathsf{Proj}_2\ #1\ #2\ #3}
% \newcommand{\ProjE}[3]{\mathsf{Proj}_e\ #1\ #2\ #3}
\newcommand{\ProjO}[1]{\mathsf{Proj}_1\ #1}
\newcommand{\ProjT}[1]{\mathsf{Proj}_2\ #1}
\newcommand{\ProjE}[1]{\mathsf{Proj}_\mathsf{e}\ #1}
\newcommand{\llift}[3]{#3 {[#1_1]}^{#2}}
\newcommand{\rlift}[3]{#3 {[#1_2]}^{#2}}
% \newcommand{\lo}[1]{\llift{}{}{#1}}
% \newcommand{\ro}[1]{\rlift{}{}{#1}}
\newcommand{\lo}[1]{#1 \upharpoonleft}
\newcommand{\ro}[1]{#1 \upharpoonright}
\newcommand{\Gp}{\Ga_\mathsf{p}}


% Links to the repository

% For anonymity we should remove the links
\newcommand{\repoURL}{https://github.com/TheoWinterhalter/ett-to-itt/blob/master/theories/}
\newcommand{\rpath}[1]{\href{{\repoURL #1}}{#1}}
% \newcommand{\rpath}[1]{\path{#1}}


\chapter{Eliminating Reflection from Type Theory}

\tw{Taken from the CPP article (for now as is).}

\tw{Cite paper and authors}

\begin{abstract}

  Type theories with equality reflection, such as extensional type
  theory (ETT), are convenient theories in which to formalise
  mathematics, as they make it possible to consider provably equal terms
  as convertible. Although type-checking is undecidable in this context,
  variants of ETT have been implemented, for example in \NuPRL and more
  recently in \Andromeda.
  %
  The actual objects that can be checked are not proof-terms, but
  derivations of proof-terms. This suggests that any derivation of ETT
  can be translated into a typecheckable proof term of intensional type
  theory (ITT).
%
  However, this result, investigated categorically by
  Hofmann in 1995, and 10 years later more
  syntactically by Oury, has never given rise to
  an effective translation.
  %
  In this paper, we provide the first effective syntactical translation
  from ETT to ITT with uniqueness of identity proofs and functional
  extensionality. This translation has been defined and proven correct
  in \Coq and yields an executable plugin that translates a derivation
  in ETT into an actual \Coq typing judgment.  Additionally, we show how
  this result is extended in the context of homotopy type theory to a
  two-level type theory.
\end{abstract}

\section{Introduction}

Type theories with equality reflection, such as extensional type
theory (ETT), are convenient theories in which to formalise
mathematics, as they make it possible to consider provably equal terms as
convertible, as expressed in the following typing rule:
%
\begin{equation}
  \label{eq:reflection}
  \infer[]
    {\xisterm{\Ga}{e}{\Eq{A}{u}{v}}}
    {\xeqterm{\Ga}{u}{v}{A}}
  %
\end{equation}
%
Here, the type $\Eq{A}{u}{v}$ is Martin-Löf's identity type with only
one constructor $\refl{} u : \Eq{A}{u}{u}$ which represents
proofs of equality inside type theory, whereas $u \equiv v : A$ means
that $u$ and $v$ are convertible in the theory---and can thus be
silently replaced by one another in any term.
%
Several variants of ETT have been considered and implemented, for
example in \NuPRL\footnote{Although the reflection rule is provable in \NuPRL,
its calculus is based on realisability rather than on intensional type theory
plus reflection.}~\cite{DBLP:conf/cade/AllenCEKL00} and more recently in
\Andromeda~\cite{andromeda}.
%
The prototypical example of the use of equality reflection is the
definition of a coercion function between two types $A$ and $B$ that
are equal (but not convertible) by taking a term of type $A$ and
simply returning it as a term of type $B$:
\[
\lambda\ A\ B \ (e:A = B)\ (x:A).\ x
: \Pi\ A\ B.\ A = B \to A \to B.
\]
%
%
In intensional type theory (ITT), this term does not type-check
because $x$ of type $A$ can not be given the type $B$ by conversion.
%
In ETT, however, equality reflection can be used to turn the witness
of equality into a proof of conversion and thus the type system
validates the fact that $x$ can be given the type $B$.
%
This means that one needs to guess equality proofs
during type-checking, because the witness of equality has been lost at
the application of the reflection rule. Guessing it was not so hard in this
example but is in general undecidable, as one can for instance encode the
halting problem of any Turing machine as an equality in ETT.
%
That is, the actual objects that can be checked in ETT are not terms,
but instead derivations of terms.
%
It thus seems natural to wonder whether any derivation of ETT can be
translated into a typecheckable term of ITT.
%
And indeed, it is well know that one can find a corresponding term of
the same type in ITT by \emph{explicitly} transporting the term $x$
of type $A$ using the elimination of internal equality on the witness
of equality $e$, noted $\transpo{e}$:
%
\[
  \lambda\ A\ B \ (e:A = B)\ (x:A).\ \transpo{e}\ x
  : \Pi\ A\ B.\ A = B \to A \to B.
\]
%
This can be seen as a way to make explicit the \emph{silent} use of
reflection.
%
Furthermore, by making the use of transport as economic as possible,
the corresponding ITT term can be seen as a compact witness of the
derivation tree of the original ETT term.

This result has first been investigated categorically in the
pioneering work of \cite{hofmann1995conservativity,HofmannPhD},
%
by showing that the term model of ITT can be turned into a model of ETT by
quotienting this model with propositional equality.
%
However, it is not clear how to extend this categorical construction
to an explicit and constructive translation from a derivation in ETT to
a term of ITT.
%
In 2005, this result has been investigated more syntactically by
\cite{oury2005extensionality}. However, his presentation does not give
rise to an effective translation.
%
By an \emph{effective} translation we mean that it is entirely
constructive and can be used to deterministically \emph{compute} the
translation of a given ETT typing derivation.
%
Two issues prevent deriving an \emph{effective} translation from Oury's
presentation, and it is the process of actual formalisation of the
result in a proof assistant that led us to these discoveries. First, his
handling of related contexts is not explicit enough, which we fix by
framing the translation using ideas coming from the parametricity
translation (Section \ref{sec:heteq}). Additionally, Oury's proof
requires an additional axiom in ITT on top of functional
extensionality (FunExt)
and uniqueness of identity proofs, that has no clear motivation and can
be avoided by considering an annotated syntax (Section
\ref{sec:ettsyntax}).

\paragraph*{Contributions.}
In this paper, we present the first effective syntactical translation
from ETT to ITT (assuming uniqueness of identity proofs (UIP) and
FunExt in ITT).
%
By syntactical translation, we mean an explicit translation from a
derivation $\xisterm{\Ga}{t}{T}$ of ETT (the $\exmark$ index testifies that it
is a derivation in ETT) to a context $\Ga'$, term
$t'$ and type $T'$ of ITT such that $\isterm{\Ga'}{t'}{T'}$ in ITT.
%
This translation enjoys the additional property that if $T$ can be
typed in ITT, \ie $\istype{\Ga}{T}$, then $T' \equiv T$.
%
This means in particular that a theorem proven in ETT but whose
statement is also valid in ITT can be automatically transferred to a theorem
of ITT. For instance, one could use a \emph{local} extension of the
\Coq proof assistant with a reflection rule, without being forced to rely on
the reflection in the entire development.

This translation can be seen as a way to build a syntactical model of
ETT from a model of ITT as described more generally
in~\cite{boulier17:next-syntac-model-type-theor} and has been entirely
programmed and
formalised in
\Coq~\cite{coq}. For this, we rely on
\TemplateCoq\footnote{\url{https://metacoq.github.io/metacoq/}}~\cite{DBLP:conf/itp/AnandBCST18},
which provides a reifier for \Coq terms as represented in \Coq's kernel
as well as a formalisation of the type system of \Coq.
%
Thus, our formalisation of ETT is just given by adding the reflection
rule to a subset of the original type system of \Coq.
%
This allows us to extract concrete \Coq terms and types from a closed
derivation of ETT, using a little trick to incorporate Inductive types
and induction. We do not treat cumulativity of universes which is an
orthogonal feature of \Coq's type theory. It would also complicate the
proof which relies on uniqueness of typing.

\paragraph*{Outline of the Paper.}

Before going into the technical development of the translation, we
explain its main ingredients and differences with previous works.
Then, in Section~\ref{sec:syntax-features}, we define the extensional
and intensional type theories we consider. In
Section~\ref{sec:relation}, we define the main ingredient of the
translation, which is a relation between terms of ETT and terms in ITT.
%
Then, the translation is given in Section~\ref{sec:translation}.
Section~\ref{sec:form-with-templ} describes the \Coq formalisation and
Sections~\ref{sec:axioms} and \ref{sec:related-works} discuss
limitations and related work.
%
The main proofs are given in detail in Appendices~\ref{sec:proof-fund-lemma}
and~\ref{sec:corr-transl}.


The \Coq formalisation can be found in
\url{https://github.com/TheoWinterhalter/ett-to-itt}.


\subsection{On the Need for UIP and FunExt}

Our translation targets ITT plus UIP and FunExt,
which correspond to the two following axioms (where $\Ty{i}$ denotes
the universe of types at level $i$):
$$
\begin{array}{r@{~}c@{~}l}
\mathtt{UIP} & : & \Pi(A: \Ty{i})\ (x \ y:A)\ (e \ e' : x = y).\ e = e' \\
\mathtt{FunExt} & : & \Pi(A : \Ty{i})\ (B : A \to \Ty{i})\
(f \ g : \Prod{x:A} B\ x). \\
&& (\Prod{x : A} f\ x = g\ x) \to f = g
\end{array}
$$
%
The first axiom says that any two proofs of the same equality are
equal, and the other one says that two (dependent) functions are equal
whenever they are pointwise equal\footnote{In Homotopy Type
  Theory (HoTT)~\cite{hottbook}, FunExt is
  stated in a more complete way, using the notion of adjoint
  equivalences, but this more complete way collapses to our simpler
  statement in presence of UIP.}.
%
These two axioms are perfectly valid statements of ITT and they can be
proven in ETT.
%
Indeed, UIP can be shown to be equivalent to the Streicher's axiom K
$$
\begin{array}{rcl}
\mathtt{K} & : & \Prod{A: \Ty{i}} \Prod{x :A} \Prod{e : x = x} e = \refl{x} \\
\end{array}
$$
using the elimination on the identity type. But K is provable in ETT
by considering the type
$$
\Prod{A: \Ty{i}} \Prod{x \ y:A} \Prod{e : x = y} e = \refl{x} \\
$$
which is well typed (using the reflection rule to show that $e$ has
type $x= x$) and which can be inhabited by elimination of the identity
type.
%
In the same way, FunExt is provable in ETT because
$$
\begin{array}{cll}
&\Prod{x : A} f\ x = g\ x \\
\to & x : A \vdash f\ x \equiv g\ x &
\mbox{by reflection} \\
\to &  (\lambda (x:A). {f \ x}) \equiv (\lambda{(x:A)}.{g \ x}) &
\mbox{by congruence of $\equiv$} \\
\to &  f \equiv g & \mbox{by $\eta$-law} \\
\to &  f = g
\end{array}
$$

Therefore, applying our translation to the proofs of those theorems in ETT
gives corresponding proofs of the same theorems in ITT.
%
However, UIP is independent from ITT, as first shown by Hofmann and
Streicher using the groupoid model~\cite{groupoid-interp}, which has
recently been extended in the setting of univalent type theory using
the simplicial or cubical models~\cite{kapulkin2012simplicial,coquand:cubical}.
%
Similarly, FunExt is independent from ITT,
it is folklore but has recently been formalised by Boulier \emph{et al.} using a
simple syntactical translation~\cite{boulier17:next-syntac-model-type-theor}.

Therefore, our translation provides proofs of axioms independent from
ITT, which means that the target of the translation already needs to
have both UIP and FunExt.
These last two elements are only necessary in ITT and could be removed from ETT
but this allows us to consider only one syntax for both.
%
Part of our work is to show formally that they are the only axioms
required.

\subsection{Heterogeneous Equality and the Parametricity Translation}
\label{sec:heteq}

The basic idea behind the translation from ETT to ITT is to interpret
conversion using the internal notion of equality, \ie the identity type.
%
But this means that two terms of two convertible types that were comparable
in ETT become comparable in ITT only up-to the equality between the
two types. One possible solution to this problem is to consider a
native heterogeneous equality, such as \emph{John Major
equality} introduced by \cite{mcbride2000dependently}.
%
However, to avoid adding additional axioms to ITT as done by
\cite{oury2005extensionality}, we prefer to encode this
heterogeneous equality using the following dependent sums:
$$\Heq{T}{t}{U}{u} := \Sum{p:\Eq{}{T}{U}} \Eq{}{\transpo{p}\ t}{u}.$$
%

During the translation, the same term occurring twice can be
translated in two different manners, if the corresponding typing
derivations are different. Even the types of the two different
translations may be different.
%
However, we have the strong property that any two translations of the
same term only differ in places where transports of proof of equality have been
injected.
%
To keep track of this property, we introduce the relation $t \sim t'$
between two terms of ITT, of possibly different types.
%
The crux of the proof of the translation is to guarantee that for
every two terms $t_1$ and $t_2$ such that $\isterm{\Ga}{t_1}{T_1}$,
  $\isterm{\Ga}{t_2}{T_2}$ and $t_1 \sim t_2$, there exists $p$ such
  that
  $\isterm{\Ga} {p} {\Heq{{T_1}} {{t_1}} {{T_2}} {{t_2}}}$.
%
However, during the proof, variables of different but (propositionally) equal
types are introduced and the context cannot be maintained to be the same
for both $t_1$ and $t_2$. Therefore, the translation needs to keep
track of this duplication of variables, plus a proof that they are
heterogeneously equal.
%
This mechanism is similar to what happens in the (relational) internal
parametricity translation in ITT introduced by
\cite{bernardy2012proofs} and recently rephrased in the setting of
\TemplateCoq~\cite{DBLP:conf/itp/AnandBCST18}. Namely, a context is not
translated as a telescope of variables, but as a telescope of triples
consisting of two variables plus a witness that they are in the
parametric relation.
%
In our setting, this amounts to consider telescope of triples
consisting of two variables plus a witness that they are
heterogeneously equal. We can express this by considering the
following dependent sums:
\[
\Pack{A_1}{A_2} := \Sum{x:A_1} \Sum{y:A_2} \Heq{A_1}{x}{A_2}{y}.
\]
%
This presentation inspired by the parametricity translation is crucial
in order to get an effective translation, because it is necessary to
keep track of the evolution of contexts when doing the translation on
open terms.
%
This ingredient is missing in Oury's work~\cite{oury2005extensionality},
which prevents him from deducing an effective (\ie constructive and
computable) translation from his theorem.

% \section{Definitions of Extensional and Intensional Type Theories}
% \label{sec:syntax-features}

% This section presents the common syntax, typing and main properties of
% ETT and ITT. Our type theories feature a universe hierarchy, dependent
% products and sums as well as Martin Löf's identity types.

% \subsection{Syntax of ETT and ITT}
% \label{sec:ettsyntax}

% The common syntax of ETT and ITT is given in Figure~\ref{fig:syntax}.
% %
% It features: dependent products $\Prod{x:A} B$, with (annotated)
% $\lambda$-abstractions and (annotated) applications, negative dependent sums
% $\Sum{x:A} B$ with (annotated) projections, sorts $\Ty{i}$, identity types
% $\Eq{A}{u}{v}$ with reflection and elimination as well as terms
% realising UIP and FunExt. Annotating terms with
% otherwise computationally irrelevant typing information is a common
% practice when studying the syntax of type theory precisely (see
% \cite{streicher1993investigations} for a similar example).
% %
% We will write $A \to B$ for $\Prod{\_:A} B$ the non-dependent product /
% function type.

% We consider a fixed universe hierarchy without cumulativity, which
% ensures in particular uniqueness of typing~\eqref{lem:uniq} which is
% important for the translation.

% \begin{figure*}
%   \hrulefill
%   \[
%   \begin{array}{l@{~\,}r@{~\,}l@{\quad}l}
%     s &\bnf& \Ty{i} ~(i \in \mathbb{N}) &\mbox{sorts (universes)} \\
%     T,A,B,t,u,v &\bnf& x \bnfor \lam{x:A}{B} t \bnfor \app{t}{x:A}{B}{u}
%     \bnfor \Prod{x:A} B \bnfor s
%     &\mbox{dependent $\lambda$-calculus} \\
%     &\bnfor& \pair{x:A}{B}{u}{v} \bnfor \pio{x:A}{B}{p} \bnfor \pit{x:A}{B}{p}
%     \bnfor \Sum{x:A} B
%     &\mbox{dependent pairs} \\
%     &\bnfor& \refl{A} u \bnfor \J{A}{u}{x.e.P}{w}{v}{p} \bnfor \Eq{A}{u}{v}
%     &\mbox{propositional equality} \\
%     &\bnfor& \funext{x:A}{B}{f}{g}{e} \bnfor \uip{A}{u}{v}{p}{q}
%     &\mbox{equality axioms} \\
%     \Ga, \D &\bnf& \ctxempty \bnfor \Ga, x:A &\mbox{contexts}
%   \end{array}
%   \]
%   \hrulefill
%   \vspace{-2ex}
%   \caption{Common syntax of ETT and ITT}
%   \label{fig:syntax}
% \end{figure*}

% \paragraph{About Annotations.}
% Although it may look like a technical detail, the use of annotation is more
% fundamental in ETT than it is in ITT (where it is irrelevant and doesn't affect
% the theory). And this is actually one of the main differences between our work
% (and that of Martin \cite{hofmann1995conservativity} who has a similar
% presentation) and the work of \cite{oury2005extensionality}.

% Indeed, by using the standard model where types are interpreted as
% cardinals rather than sets, it is possible to see that the equality
% $\nat \to \nat = \nat \to \bool$ is independent from the theory, it is
% thus possible to assume it (as an axiom, or for those that would still
% not be convinced, simply under a $\lambda$ that would introduce this
% equality).  In that context, the identity map $\lambda(x : \nat).\ x$
% can be given the type $\nat \to \bool$ and we thus type
% $(\lambda(x : \nat).\ x)\ \zero : \bool$.  Moreover, the
% $\beta$-reduction of the non-annotated system used by Oury concludes
% that this expression reduces to $\zero$, but cannot be given the type
% $\bool$ (as we said, the equality $\nat \to \nat = \nat \to \bool$ is
% independent from the theory, so the context is consistent). This means
% we lack subject reduction in this case (or uniqueness of types,
% depending on how we see the issue).  Our presentation has a blocked
% $\beta$-reduction limited to matching annotations:
% $\app{(\lam{x:A}{B}\ t)}{x:A}{B}{u} = t[x \sto u]$, from which subject
% reduction and uniqueness of types follow.

% Although subtle, this difference is responsible for Oury's need for an
% extra axiom. Indeed, to treat the case of equality of applications in
% his proof, he needs to assume the congruence rule for heterogeneous
% equality of applications, which is not provable when formulated with
% John Major equality (Fig.~\ref{fig:jmapp}). Thanks to annotations and
% our notion of heterogeneous equality, we can prove this congruence
% rule for applications.

% \begin{figure}[h]
%   \begin{mathpar}
%     \infer[JMAPP]
%     {\Heq{\forall (x : U_1). V_1}{f_1}{\forall (x : U_2). V_2}{f_2}\\
%       \Heq{U_1}{u_1}{U_2}{u_2}}
%     {\Heq{V_1[x \leftarrow u_1]}{f_1\ u_1}{V_2[x \leftarrow u_2]}{f_2\ u_2}}
%     %
%   \end{mathpar}
%   \caption{Congruence of heterogeneous equality}
%   \label{fig:jmapp}
% \end{figure}

% \subsection{The Typing Systems}

% As usual in dependent type theory, we consider contexts which are
% telescopes whose declarations may depend on any variable already
% introduced. We note $\isterm{\Ga}{t}{A}$ to say that $t$ has type $A$
% in context $\Ga$. $\istype{\Ga}{A}$ shall stand for
% $\isterm{\Ga}{A}{s}$ for some sort $s$ and similarly
% $\eqtype{\Ga}{A}{B}$ stands for $\eqtype{\Ga}{A}{B} : s$.

% We use two relations $(s,s') \in \Ax$ (written $(s,s')$ for short)
% and $(s,s',s'') \in \Rl$ (written $(s,s',s'')$) to constrain
% the sorts in the typing rules for universes, dependent products and
% dependent sums, as is done in any Pure Type System (PTS).
% %
% In our case, because we do not have cumulativity, the rules are as
% follows:
% %
% \begin{mathpar}
%   (\Ty{i}, \Ty{i+1}) \in \Ax

%   (\Ty{i}, \Ty{j}, \Ty{\nmax{i}{j}}) \in \Rl
% \end{mathpar}

% \begin{figure}[tbp]
  \flushleft
  \hrulefill
  \paradot{Well-formedness of contexts}

  \begin{mathpar}
    \infer[]
      { }
      {\isctx{\ctxempty}}
    %

    \infer[]
      {\isctx{\Ga} \\
       \istype{\Ga}{A}
      }
      {\isctx{\Ga, x:A}}
    (x \notin \Ga)
  \end{mathpar}

  \paradot{Types}

  \begin{mathpar}
    \infer[]
      {\isctx{\Ga}}
      {\isterm{\Ga}{s}{s'}}
    (s,s')

    \infer[]
      {\isterm{\Ga}{A}{s} \\
       \isterm{\Ga,x:A}{B}{s'}
      }
      {\isterm{\Ga}{\Prod{x:A} B}{s''}}
    (s,s',s'')

    \infer[]
      {\isterm{\Ga}{A}{s} \\
       \isterm{\Ga,x:A}{B}{s'}
      }
      {\isterm{\Ga}{\Sum{x:A} B}{s''}}
    (s,s',s'')

    \infer[]
      {\isterm{\Ga}{A}{s} \\
       \isterm{\Ga}{u}{A} \\
       \isterm{\Ga}{v}{A}
      }
      {\isterm{\Ga}{\Eq{A}{u}{v}}{s}}
    %
  \end{mathpar}

  \paradot{Structural rules}

  \begin{mathpar}
    \infer[]
      {\isctx{\Ga} \\
       (x : A) \in \Ga
      }
      {\isterm{\Ga}{x}{A}}
    %

    \infer[]
      {\isterm{\Ga}{u}{A} \\
       \eqtype{\Ga}{A}{B} : s
      }
      {\isterm{\Ga}{u}{B}}
    %
  \end{mathpar}

  \paradot{$\lambda$-calculus terms}

  \begin{mathpar}
    \infer[]
      {\isterm{\Ga}{A}{s} \\
       \isterm{\Ga,x:A}{B}{s'} \\
       \isterm{\Ga,x:A}{t}{B}
      }
      {\isterm{\Ga}{\lam{x:A}{B} t}{\Prod{x:A} B}}
    %

    \infer[]
      {\isterm{\Ga}{A}{s} \\
       \isterm{\Ga,x:A}{B}{s'} \\
       \isterm{\Ga}{t}{\Prod{x:A} B} \\
       \isterm{\Ga}{u}{A}
      }
      {\isterm{\Ga}{\app{t}{x:A}{B}{u}}{B[x \sto u]}}
    %

    \infer[]
      {\isterm{\Ga}{u}{A} \\
       \isterm{\Ga}{A}{s} \\
       \isterm{\Ga,x:A}{B}{s'} \\
       \isterm{\Ga}{v}{B[x \sto u]}
      }
      {\isterm{\Ga}{\pair{x:A}{B}{u}{v}}{\Sum{x:A}{B}}}
    %
      \\

    \infer[]
      {\isterm{\Ga}{p}{\Sum{x:A}{B}}}
      {\isterm{\Ga}{\pio{x:A}{B}{p}}{A}}
    %

    \infer[]
      {\isterm{\Ga}{p}{\Sum{x:A}{B}}}
      {\isterm{\Ga}{\pit{x:A}{B}{p}}{B[x \sto \pio{x:A}{B}{p}]}}
    %
  \end{mathpar}

  \paradot{Equality terms}

  \begin{mathpar}
    \infer[]
      {\isterm{\Ga}{A}{s} \\
       \isterm{\Ga}{u}{A}
      }
      {\isterm{\Ga}{\refl{A} u}{\Eq{A}{u}{u}}}
    %

    \infer[]
      {\isterm{\Ga}{e_1,e_2}{\Eq{A}{u}{v}}}
      {\isterm{\Ga}{\uip{A}{u}{v}{e_1}{e_2}}{\Eq{}{e_1}{e_2}}}
    %

    \infer[]
      {\isterm{\Ga}{A}{s} \\
       \isterm{\Ga}{u,v}{A} \\
       \isterm{\Ga, x:A, e:\Eq{A}{u}{x}}{P}{s'} \\
       \isterm{\Ga}{p}{\Eq{A}{u}{v}} \\
       \isterm{\Ga}{w}{P[x \sto u, e \sto \refl{A} u]}
      }
      {\isterm
        {\Ga}
        {\J{A}{u}{x.e.P}{w}{v}{p}}
        {P[x \sto v, e \sto p]}
      }
    %

    \infer[]
      {\isterm{\Ga}{f,g}{\Prod{x:A} B} \\
       \isterm
         {\Ga}
         {e}
         {\Prod{x:A} \Eq{B}{\app{f}{x:A}{B}{x}}{\app{g}{x:A}{B}{x}}}
      }
      {\isterm{\Ga}{\funext{x:A}{B}{f}{g}{e}}{\Eq{}{f}{g}}}
    %
  \end{mathpar}

  \hrulefill
  \vspace{-2ex}
  \caption{Typing rules}
  \label{fig:ty-rules}
\end{figure}


% We give the typing rules of ITT in
% Figure~\ref{fig:ty-rules}.
% %
% The rules are standard and we do not explain them.
% %
% Let us just point out the conversion rule, which says that $u:A$ can
% be given the type $u:B$ when $A \equiv B$, \ie when $A$ and $B$ are
% convertible.
% %
% As the notion of conversion is central in our work---the conversion of
% ETT being translated to an equality in ITT---we provide an exhaustive
% definition of it, with computational conversion rules (including
% $\beta$-conversion or reduction of the elimination principle of
% equality over reflexivity, see Figure~\ref{fig:conv-rules}), however
% congruence conversion rules can be found in Appendix~\ref{sec:more-rules}
% (Figure~\ref{fig:cong-rules}).
% Note that we use Christine Paulin-Möhring's variant of the J rule rather than
% Martin-Löf's original formulation.
% %
% Although pretty straightforward, being precise here is very important,
% as for instance the congruence rule for $\lambda$-terms is the reason
% why FunExt is derivable in ETT. Congruence of
% equality terms is a standard extension of congruence to the new
% principles we add (UIP and FunExt).


% ETT is thus simply an extension of ITT (we write $\vdash_\exmark$ for
% the associated typing judgment) with the reflection rule on equality,
% which axiomatises that propositionally equal terms are convertible
% (see Equation~\ref{eq:reflection}).
% Note that, as already mentioned, in the presence of reflection and
% $\mathsf{J}$, UIP is derivable so we could remove it from ETT, but
% keeping it allows us to share a common syntax which makes the
% statements of theorems simpler and does not affect the development.


% \begin{figure*}[htbp]
  \flushleft
  \hrulefill
  \paradot{Computation}

  \begin{mathpar}
    \infer[]
      {\isterm{\Ga}{A}{s} \\
       \isterm{\Ga,x:A}{B}{s'} \\
       \isterm{\Ga,x:A}{t}{B} \\
       \isterm{\Ga}{u}{A}
      }
      {\eqterm
        {\Ga}
        {\app
          {(\lam{x:A}{B} t)}
          {x:A}
          {B}
          {u}
        }
        {t[x \sto u]}
        {B[x \sto u]}
      }
    %

    \infer[]
      {\isterm{\Ga}{A}{s} \\
       \isterm{\Ga}{u}{A} \\
       \isterm{\Ga, x:A, e:\Eq{A}{u}{x}}{P}{s'} \\
       \isterm{\Ga}{w}{P[x \sto u, e \sto \refl{A} u]}
      }
      {\eqterm
        {\Ga}
        {\J{A}{u}{x.e.P}{w}{u}{\refl{A} u}}
        {w}
        {P[x \sto u, e \sto \refl{A} u]}
      }
    %

    \infer[]
      {\isterm{\Ga}{A}{s} \\
       \isterm{\Ga}{u}{A} \\
       \isterm{\Ga,x:A}{B}{s'} \\
       \isterm{\Ga}{v}{B[x \sto u]}
      }
      {\eqterm
        {\Ga}
        {\pio{x:A}{B}{\pair{x:A}{B}{u}{v}}}
        {u}
        {A}
      }
    %

    \infer[]
      {\isterm{\Ga}{A}{s} \\
       \isterm{\Ga}{u}{A} \\
       \isterm{\Ga,x:A}{B}{s'} \\
       \isterm{\Ga}{v}{B[x \sto u]}
      }
      {\eqterm
        {\Ga}
        {\pit{x:A}{B}{\pair{x:A}{B}{u}{v}}}
        {v}
        {B[x \sto u]}
      }
    %
  \end{mathpar}

  % \paradot{$\eta$-rule}
  %
  % \begin{mathpar}
  %   \infer[]
  %     {\isterm{\Ga}{A}{s} \\
  %      \isterm{\Ga, x:A}{B}{s'} \\
  %      \isterm{\Ga}{f}{\Prod{x:A} B}
  %     }
  %     {\eqterm
  %       {\Ga}
  %       {\lam{x:A}{B} \app{f}{x:A}{B}{x}}
  %       {f}
  %       {\Prod{x:A} B}
  %     }
  %   %
  % \end{mathpar}

  \paradot{Conversion}

  \begin{mathpar}
    \infer[]
      {\eqterm{\Ga}{t_1}{t_2}{T_1} \\
       \eqtype{\Ga}{T_1}{T_2}
      }
      {\eqterm{\Ga}{t_1}{t_2}{T_2}}
    %
  \end{mathpar}
  \hrulefill
  \vspace{-2ex}
  \caption{Main conversion rules (omitting congruence rules)}
  \label{fig:conv-rules}
\end{figure*}


% \subsection{General Properties of ITT and ETT}

% We now state the main properties of both ITT and ETT.
% %
% We do not detail their proof as they are standard and can be found
% in the \Coq formalisation.

% First, although not explicit in the typing system, weakening is
% admissible in ETT and ITT.

% \begin{lemma}[Weakening]
%   \label{lem:weak}
%   If $\isjg{\Ga}{J}$ and $\D$ extends $\Ga$ (possibly interleaving variables)
%   then $\isjg{\D}{J}$.
% \end{lemma}

% Then, as mentioned above, the use of a non-cumulative hierarchy allows
% us to prove that a term $t$ can be given at most one type in a
% context $\Ga$, up-to conversion.

% \begin{lemma}[Uniqueness of typing]
%   \label{lem:uniq}
%   If $\isterm{\Ga}{u}{T_1}$ and $\isterm{\Ga}{u}{T_2}$
%   then $\eqtype{\Ga}{T_1}{T_2}$.
% \end{lemma}

% Finally, an important property of the typing system (seen as a mutual
% inductive definition) is the possibility to deduce hypotheses from
% their conclusion, thanks to inversion of typing. Note that it is
% important here that our syntax is annotated for applications and
% projections as it provides a richer inversion principle.

% \begin{lemma}[Inversion of typing]
%   \label{lem:inversion}
%   \leavevmode
%   \begin{enumerate}
%     \item If $\isterm{\Ga}{x}{T}$ then $(x : A) \in \Ga$ and
%     $\eqtype{\Ga}{A}{T}$.
%     \item If $\isterm{\Ga}{\Ty{i}}{T}$ then $\eqtype{\Ga}{\Ty{i+1}}{T}$.
%     \item If $\isterm{\Ga}{\Prod{x:A}{B}}{T}$ then $\isterm{\Ga}{A}{s}$ and
%     $\isterm{\Ga, x:A}{B}{s'}$ and $\eqtype{\Ga}{s''}{T}$ for some $(s,s',s'')$.
%     \item If $\isterm{\Ga}{\lam{x:A}{B} t}{T}$ then $\isterm{\Ga}{A}{s}$ and
%     $\isterm{\Ga, x:A}{B}{s'}$ and $\isterm{\Ga, x:A}{t}{B}$ and
%     $\eqtype{\Ga}{\Prod{x:A}{B}}{T}$.
%     \item If $\isterm{\Ga}{\app{u}{x:A}{B}{v}}{T}$ then $\isterm{\Ga}{A}{s}$ and
%     $\isterm{\Ga, x:A}{B}{s'}$ and $\isterm{\Ga}{u}{\Prod{x:A}{B}}$ and
%     $\isterm{\Ga}{v}{A}$ and $\eqtype{\Ga}{B[ x \sto u]}{T}$.
%     \item \ldots\ Analogous for the remaining term and type constructors.
%   \end{enumerate}
% \end{lemma}

% \begin{proof}
%   Each case is proven by induction on the derivation
%   (which corresponds to any number of applications of the conversion rule
%   following one introduction rule).
% \end{proof}

% \section{Relating Translated Expressions}
% \label{sec:relation}

% We want to define a relation on terms that equates two terms that are
% the same up to transport.
% %
% This begs the question of what notion of transport is going to be
% used.
% %
% Transport can be defined from elimination of equality as follows:
% %
% \begin{definition}[Transport]
%   Given $\isterm{\Ga}{p}{\Eq{s}{T_1}{T_2}}$ and
%   $\isterm{\Ga}{t}{T_1}$ we define the transport of $t$ along $p$, written
%   $\transpo{p}\ t$, as
%   $\app
%     {\J
%       {s}
%       {T_1}
%       {X.e.\ T_1 \to X}
%       {\lam{x:T_1}{T_1} x}
%       {T_2}
%       {p}
%     }
%     {T_1}
%     {T_2}
%     {t}
%   $ such that $\isterm{\Ga}{\transpo{p}\ t}{T_2}$.
% \end{definition}
% %
% However, in order not to confuse the transports added by the
% translation with the transports that were already present in the
% source, we consider $\transpo{p}$ as part of the syntax in the
% reasoning. It will be unfolded to its definition only after the
% complete translation is performed.
% %
% This idea is not novel as Hofmann already had a $\mathsf{Subst}$ operator that
% was part of his ITT (noted TT\textsubscript{I} in his
% paper~\cite{hofmann1995conservativity}).

% %
% We first define the (purely syntactic) relation $\ir$ between ETT terms
% and ITT terms in Figure~\ref{fig:ir-def} stating that the ITT term is
% simply a decoration of the first term by transports. Its purpose is to
% state how close to the original term its translation is. Then, we extend
% this relation to a similarity relation $\sim$ on ETT terms by taking its
% symmetric and transitive closure:
% \begin{center}$\sim \coloneqq (\ir \cup \ir^{-1})^+$
% \end{center}

% \begin{lemma}[$\sim$ is an equivalence relation]
%   \label{lem:sim-er}
%   $\sim$ is reflexive, symmetric and transitive.
% \end{lemma}

% \begin{proof}
%   For reflexivity we proceed by induction on the term.
% \end{proof}

% \begin{figure}[htbp]
  \flushleft
  \hrulefill

  \begin{mathpar}

    \infer[]
      {t_1 \ir t_2}
      {t_1 \ir \transpo{p}\ t_2}
    %

    \\
    \infer[]
      { }
      {x \ir x}
    %

    \infer[]
      {A_1 \ir A_2 \\
       B_1 \ir B_2
      }
      {\Prod{x:A_1} B_1 \ir \Prod{x:A_2} B_2}
    %

    \infer[]
      {A_1 \ir A_2 \\
       B_1 \ir B_2
      }
      {\Sum{x:A_1} B_1 \ir \Sum{x:A_2} B_2}
    %

    \infer[]
      {A_1 \ir A_2 \\
       u_1 \ir u_2 \\
       v_1 \ir v_2
      }
      {\Eq{A_1}{u_1}{v_1} \ir \Eq{A_2}{u_2}{v_2}}
    %

    \infer[]
      { }
      {s \ir s}
    %

    \infer[]
      {A_1 \ir A_2 \\
       B_1 \ir B_2 \\
       t_1 \ir t_2
      }
      {\lam{x:A_1}{B_1} t_1 \ir \lam{x:A_2}{B_2} t_2}
    %

    \infer[]
      {t_1 \ir t_2 \\
       A_1 \ir A_2 \\
       B_1 \ir B_2 \\
       u_1 \ir u_2
      }
      {\app{t_1}{x:A_1}{B_1}{u_1} \ir \app{t_2}{x:A_2}{B_2}{u_2}}
    %

    \infer[]
      {A_1 \ir A_2 \\
       B_1 \ir B_2 \\
       t_1 \ir t_2 \\
       u_1 \ir u_2
      }
      {\pair{x:A_1}{B_1}{t_1}{u_1} \ir \pair{x:A_2}{B_2}{t_2}{u_2}}
    %

    \infer[]
      {A_1 \ir A_2 \\
       B_1 \ir B_2 \\
       p_1 \ir p_2
      }
      {\pio{x:A_1}{B_1}{p_1} \ir \pio{x:A_2}{B_1}{p_2}}
    %

    \infer[]
      {A_1 \ir A_2 \\
       B_1 \ir B_2 \\
       p_1 \ir p_2
      }
      {\pit{x:A_1}{B_1}{p_1} \ir \pit{x:A_2}{B_2}{p_2}}
    %

    \infer[]
      {A_1 \ir A_2 \\
       u_1 \ir u_2
      }
      {\refl{A_1} u_1 \ir \refl{A_2} u_2}
    %

    \infer[]
      {A_1 \ir A_2 \\
       B_1 \ir B_2 \\
       f_1 \ir f_2 \\
       g_1 \ir g_2 \\
       e_1 \ir e_2
      }
      {\funext{x:A_1}{B_1}{f_1}{g_1}{e_1}
       \ir \funext{x:A_2}{B_2}{f_2}{g_2}{e_2}
      }
    %

    \infer[]
      {A_1 \ir A_2 \\
       u_1 \ir u_2 \\
       v_1 \ir v_2 \\
       p_1 \ir p_2 \\
       q_1 \ir q_2
      }
      {\uip{A_1}{u_1}{v_1}{p_1}{q_1} \ir \uip{A_2}{u_2}{v_2}{p_2}{q_2}}
    %

    \infer[]
      {A_1 \ir A_2 \\
       u_1 \ir u_2 \\
       P_1 \ir P_2 \\
       w_1 \ir w_2 \\
       v_1 \ir v_2 \\
       p_1 \ir p_2
      }
      {\J{A_1}{u_1}{x.e.P_1}{w_1}{v_1}{p_1} \ir
       \J{A_2}{u_2}{x.e.P_2}{w_2}{v_2}{p_2}
      }
    %
  \end{mathpar}

  \hrulefill
  \vspace{-2ex}
  \caption{Relation $\ir$}
  \label{fig:ir-def}
\end{figure}

% %\begin{figure}[htbp]
  \flushleft
  \hrulefill

  \begin{mathpar}
    \infer[]
      { }
      {x \sim x}
    %

    \infer[]
      {t_1 \sim t_2}
      {\transpo{p}\ t_1 \sim t_2}
    %

    \infer[]
      {t_1 \sim t_2}
      {t_1 \sim \transpo{p}\ t_2}
    %

    \infer[]
      {A_1 \sim A_2 \\
       B_1 \sim B_2
      }
      {\Prod{x:A_1} B_1 \sim \Prod{x:A_2} B_2}
    %

    \infer[]
      {A_1 \sim A_2 \\
       B_1 \sim B_2
      }
      {\Sum{x:A_1} B_1 \sim \Sum{x:A_2} B_2}
    %

    \infer[]
      {A_1 \sim A_2 \\
       u_1 \sim u_2 \\
       v_1 \sim v_2
      }
      {\Eq{A_1}{u_1}{v_1} \sim \Eq{A_2}{u_2}{v_2}}
    %

    \infer[]
      { }
      {s \sim s}
    %

    \infer[]
      {A_1 \sim A_2 \\
       B_1 \sim B_2 \\
       t_1 \sim t_2
      }
      {\lam{x:A_1}{B_1} t_1 \sim \lam{x:A_2}{B_2} t_2}
    %

    \infer[]
      {t_1 \sim t_2 \\
       A_1 \sim A_2 \\
       B_1 \sim B_2 \\
       u_1 \sim u_2
      }
      {\app{t_1}{x:A_1}{B_1}{u_1} \sim \app{t_2}{x:A_2}{B_2}{u_2}}
    %

    \infer[]
      {A_1 \sim A_2 \\
       B_1 \sim B_2 \\
       t_1 \sim t_2 \\
       u_1 \sim u_2
      }
      {\pair{x:A_1}{B_1}{t_1}{u_1} \sim \pair{x:A_2}{B_2}{t_2}{u_2}}
    %

    \infer[]
      {A_1 \sim A_2 \\
       B_1 \sim B_2 \\
       p_1 \sim p_2
      }
      {\pio{x:A_1}{B_1}{p_1} \sim \pio{x:A_2}{B_1}{p_2}}
    %

    \infer[]
      {A_1 \sim A_2 \\
       B_1 \sim B_2 \\
       p_1 \sim p_2
      }
      {\pit{x:A_1}{B_1}{p_1} \sim \pit{x:A_2}{B_2}{p_2}}
    %

    \infer[]
      {A_1 \sim A_2 \\
       u_1 \sim u_2
      }
      {\refl{A_1} u_1 \sim \refl{A_2} u_2}
    %

    \infer[]
      {A_1 \sim A_2 \\
       B_1 \sim B_2 \\
       f_1 \sim f_2 \\
       g_1 \sim g_2 \\
       e_1 \sim e_2
      }
      {\funext{x:A_1}{B_1}{f_1}{g_1}{e_1}
       \sim \funext{x:A_2}{B_2}{f_2}{g_2}{e_2}
      }
    %

    \infer[]
      {A_1 \sim A_2 \\
       u_1 \sim u_2 \\
       v_1 \sim v_2 \\
       p_1 \sim p_2 \\
       q_1 \sim q_2
      }
      {\uip{A_1}{u_1}{v_1}{p_1}{q_1} \sim \uip{A_2}{u_2}{v_2}{p_2}{q_2}}
    %

    \infer[]
      {A_1 \sim A_2 \\
       u_1 \sim u_2 \\
       P_1 \sim P_2 \\
       w_1 \sim w_2 \\
       v_1 \sim v_2 \\
       p_1 \sim p_2
      }
      {\J{A_1}{u_1}{x.e.P_1}{w_1}{v_1}{p_1} \sim
       \J{A_2}{u_2}{x.e.P_2}{w_2}{v_2}{p_2}
      }
    %
  \end{mathpar}

  \hrulefill
  \vspace{-2ex}
  \caption{Relation $\sim$}
  \label{fig:sim-def}
\end{figure}


% The goal is to prove that two terms in this relation, that are well-typed in the
% target type theory, are heterogeneously equal. As for this notion, we recall
% the definition we previously gave:
% $\Heq{T}{t}{U}{u} := \Sum{p:\Eq{}{T}{U}} \Eq{}{\transpo{p}\ t}{u}$.
% %
% This definition of heterogeneous equality can be shown to be
% reflexive, symmetric and transitive. Because of UIP, heterogeneous
% equality collapses to equality when taken on the same type.

% \begin{lemma}
%   \label{lem:uip-cong}
%   If $\isterm{\Ga}{e}{\Heq{A}{u}{A}{v}}$
%   then there exists $p$ such that $\isterm{\Ga}{p}{\Eq{A}{u}{v}}$.
% \end{lemma}

% \begin{proof}
%   This holds thanks to UIP on equality, which implies K, and so the
%   proof of $A = A$ can be taken to be reflexivity.
% \end{proof}

% \begin{note}
%   As a corollary, $\Heqs$ on types corresponds to equality.
%   Indeed when we have $\isterm{\Ga}{e}{\Heq{s}{A}{s'}{B}}$ we have
%   that $\Eq{}{s}{s'}$, which implies that $s$ and $s'$ have the same sort
%   and thus are syntactically the same (by an inversion argument).
% \end{note}

% Before we can prove the fundamental lemma stating that two terms in relation
% are heterogeneously equal, we need to consider another construction.
% %
% As explained in the introduction, when proving the property by
% induction on terms, we introduce variables in the context that are
% equal only up-to heterogeneous equality.
% %
% This phenomenon is similar to what happens in the parametricity
% translation~\cite{bernardy2012proofs}.
% %
% Our fundamental lemma on the decoration relation $\sim$ assumes two
% related terms of potentially different types $T1$ and $T2$ to produce an
% heterogeneous equality between them. For induction to go through under
% binders (e.g. for dependent products and abstractions), we hence need to
% consider the two terms under different, but heterogeneously equal
% contexts.
% %
% Therefore, the context we produce will not only be a telescope of
% variables, but rather a telescope of triples consisting of two variables
% of possibly different types, and a witness that they are heterogeneously
% equal.
% %
% To make this precise, we define the following macro:
% %
% \[
% \Pack{A_1}{A_2} := \Sum{x:A_1} \Sum{y:A_2} \Heq{}{x}{}{y}
% \]
% together with its projections
% \begin{mathpar}
%   \ProjO{p} := \pio{}{}{p}

%   \ProjT{p} := \pio{}{}{\pit{}{}{p}}

%   \ProjE{p} := \pit{}{}{\pit{}{}{p}}.
% \end{mathpar}
% %
% We can then extend this notion canonically to contexts of the same
% length that are well formed using the same sorts:
% %
% \[
% \begin{array}{l}
%     \Pack{(\Ga_1, x:A_1)}{(\Ga_2, x:A_2)} := \\
%     (\Pack{\Ga_1}{\Ga_2}),
%     x : \Pack{(\llift{\gamma}{}{A_1})}{(\rlift{\gamma}{}{A_2})} \\
%     \\
%     \Pack{\ctxempty}{\ctxempty} := \ctxempty.
% \end{array}
% \]
% %
% When we pack contexts, we also need to apply the correct projections for
% the types in that context to still make sense. Assuming two contexts
% $\Ga_1$ and $\Ga_2$ of the same length, we can define left and right
% substitutions:
% \[
% \begin{array}{ll}
%   \gamma_1 &:= [ x \leftarrow \ProjO{x}\ |\ (x : \_) \in \Ga_1 ] \\
%   \gamma_2 &:= [ x \leftarrow \ProjT{x}\ |\ (x : \_) \in \Ga_2 ].
% \end{array}
% \]
% These substitutions implement lifting of terms to packed contexts:
% $\isterm{\Ga, \Pack{\Ga_1}{\Ga_2}}{\llift{\gamma}{}{t}}{\llift{\gamma}{}{A}}$
% whenever $\isterm{\Ga, \Ga_1}{t}{A}$ (resp.
% $\isterm{\Ga, \Pack{\Ga_1}{\Ga_2}}{\rlift{\gamma}{}{t}}{\rlift{\gamma}{}{A}}$
% whenever $\isterm{\Ga, \Ga_2}{t}{A}$).

% For readability, when $\Ga_1$ and $\Ga_2$ are understood we will write $\Gp$ for
% $\Pack{\Ga_1}{\Ga_2}$.

% Implicitly, whenever we use the notation $\Pack{\Ga_1}{\Ga_2}$ it means that
% the two contexts are of the same length and well-formed with the same
% sorts.
% %
% We can now state the fundamental lemma.

% \begin{lemma}[Fundamental lemma]
%   \label{lem:sim-cong}
%   Let $t_1$ and $t_2$ be two terms. If $\isterm{\Ga, \Ga_1}{t_1}{T_1}$ and
%   $\isterm{\Ga, \Ga_2}{t_2}{T_2}$ and $t_1 \sim t_2$ then there exists $p$ such
%   that
%   $\isterm{\Ga, \Pack{\Ga_1}{\Ga_2}}
%           {p}
%           {\Heq{\llift{\gamma}{}{T_1}}
%                {\llift{\gamma}{}{t_1}}
%                {\rlift{\gamma}{}{T_2}}
%                {\rlift{\gamma}{}{t_2}}}$.
% \end{lemma}

% \begin{proof}
%   The proof is by induction on the derivation of $t_1 \sim t_2$. We show
%   the three most interesting cases:

%   \begin{itemize}
%   \item \textsc{Var}
%     \[
%       \infer[]
%         { }
%         {x \sim x}
%       %
%     \]
%     If $x$ belongs to $\Ga$, we apply reflexivity---together with uniqueness of
%     typing~\eqref{lem:uniq}---to conclude.
%     Otherwise, $\ProjE{x}$ has the expected type (since
%     $\llift{\gamma}{}{x} \equiv \ProjO{x}$ and $\rlift{\gamma}{}{x} \equiv \ProjT{x}$).

%   \item \textsc{Application}
%     \[
%       \infer[]
%         {t_1 \sim t_2 \\
%          A_1 \sim A_2 \\
%          B_1 \sim B_2 \\
%          u_1 \sim u_2
%         }
%         {\app{t_1}{x:A_1}{B_1}{u_1} \sim \app{t_2}{x:A_2}{B_2}{u_2}}
%       %
%     \]
%     We have $\isterm{\Ga, \Ga_1}{\app{t_1}{x:A_1}{B_1}{u_1}}{T_1}$ and
%     $\isterm{\Ga, \Ga_2}{\app{t_2}{x:A_2}{B_2}{u_2}}{T_2}$ which means by
%     inversion~\eqref{lem:inversion} that the subterms are well-typed.
%     We apply the induction hypothesis and then conclude.
%   \item \textsc{TransportLeft}
%     \[
%       \infer[]
%         {t_1 \sim t_2}
%         {\transpo{p}\ t_1 \sim t_2}
%       %
%     \]
%     We have $\isterm{\Ga, \Ga_1}{\transpo{p}\ t_1}{T_1}$ and
%     $\isterm{\Ga, \Ga_2}{t_2}{T_2}$.
%     By inversion~\eqref{lem:inversion} we have
%     $\isterm{\Ga, \Ga_1}{p}{\Eq{}{T_1'}{T_1}}$ and
%     $\isterm{\Ga, \Ga_1}{t_1}{T_1'}$.
%     By induction hypothesis we have $e$ such that
%     $\isterm{\Ga, \Gp}{e}{\Heq{}{\llift{\gamma}{}{t_1}}{}{\rlift{\gamma}{}{t_2}}}$.
%     From transitivity and symmetry we only need to provide a proof of
%     $\Heq{}{\llift{\gamma}{}{t_1}}{}{\transpo{\llift{\gamma}{}{p}}\ \llift{\gamma}{}{t_1}}$ which is inhabited by
%     $\pair{\_}{\_}{\llift{\gamma}{}{p}}{\refl{} (\transpo{\llift{\gamma}{}{p}}\ \llift{\gamma}{}{t_1})}$.
%   \end{itemize}

%   The complete proof can be found in Appendix~\ref{sec:proof-fund-lemma}.
% \end{proof}

% We can also prove that $\sim$ preserves substitution.

% \begin{lemma}
%   If $t_1 \sim t_2$ and $u_1 \sim u_2$ then
%   $t_1[x \sto u_1] \sim t_2[x \sto u_2]$.
% \end{lemma}

% \begin{proof}
%   We proceed by induction on the derivation of $t_1 \sim t_2$.
% \end{proof}

% \section{Translating ETT to ITT}
% \label{sec:translation}

% \subsection{The Translation}
% \label{sec:the-translation}

% We now define the translations (let us stress the plural here) of an
% extensional judgment. We extend $\ir$ canonically to contexts
% ($\Ga \ir \Gb$ when they bind the same variables and the types are in
% relation for $\ir$).

% Before defining the translation, we define a set
% $\trans{\xisterm{\Ga}{t}{A}}$ of typing judgments
% in ITT associated to a typing judgment $\xisterm{\Ga}{t}{A}$ in ETT.
% %
% The idea is that this set describes all the possible translations that
% lead to the expected property. When
% $\isterm{\Gb}{\tb}{\Ab} \in \trans{\xisterm{\Ga}{t}{A}}$, we say that
% $\isterm{\Gb}{\tb}{\Ab}$ realises $\xisterm{\Ga}{t}{A}$. The
% translation will be given by showing that this set is inhabited by
% induction on the derivation.

% \begin{definition}[Characterisation of possible translations]
%   \leavevmode
%   \begin{itemize}
%     \item For any $\xisctx{\Ga}$ we define $\trans{\xisctx{\Ga}}$ as a set of
%     valid judgments (in ITT) such that
%     $\isctx{\Gb} \in \trans{\xisctx{\Ga}}$ if and only if $\Ga \ir \Gb$.

%     \item Similarly, $\isterm{\Gb}{\tb}{\Ab} \in \trans{\xisterm{\Ga}{t}{A}}$ iff
%     $\isctx{\Gb} \in \trans{\xisctx{\Ga}}$ and $A \ir \Ab$ and $t \ir \tb$.
%   \end{itemize}
% \end{definition}

% In order to better master the shape of the produced realiser, we state the
% following lemma which shows that it has the same head
% type constructor as the type it realises.
% %
% This is important for instance for the case of an application, where we
% do not know a priori if the translated function has a dependent product
% type, which is required to be able to use the typing rule for application.

% \begin{lemma}
%   \label{lem:choose}
%   We can always \emph{choose} types $\Tb$ that have the same head constructor
%   as $T$.
% \end{lemma}

% \begin{proof}
%   Assume we have $\isterm{\Gb}{\tb}{\Tb} \in \trans{\xisterm{\Ga}{t}{T}}$.
%   By definition of $\ir$,
%   $T \ir \Tb$ means that $\Tb$ is shaped
%   $\transpo{p}\ \transpo{q}\ ...\ \transpo{r}\ \Tb'$ with $\Tb'$ having
%   the same head constructor as $T$. By inversion~\eqref{lem:inversion}, the
%   subterms are typable, including $\Tb'$. Actually, from inversion, we
%   even get that the type of $\Tb'$ is a universe. Then,
%   using lemma~\ref{lem:sim-cong} and lemma~
%   \ref{lem:uip-cong}, we get $\isterm{\Gb}{e}{\Eq{}{\Tb}{\Tb'}}$.
%   We conclude with
%   $\isterm{\Gb}{\transpo{e}\ \tb}{\Tb'} \in \trans{\xisterm{\Ga}{t}{T}}$.
% \end{proof}

% Finally, in order for the induction to go through, we need to know
% that when we have a realiser of a derivation $\xisterm{\Ga}{t}{T}$, we can
% pick an arbitrary other type realising $\xistype{\Ga}{T}$ and still
% get a new derivation realising $\xisterm{\Ga}{t}{T}$ with that type.
% %
% This is important for instance for the case of an application, where
% the type of the domain of the translated function may differ from the
% type of the translated argument. So we need to be able to change it \textit{a
% posteriori}.


% \begin{lemma}
%   \label{lem:change-type}
%   When we have $\isterm{\Gb}{\tb}{\Tb} \in \trans{\xisterm{\Ga}{t}{T}}$
%   and $\istype{\Gb}{\Tb'} \in \trans{\xistype{\Ga}{T}}$ then we also have
%   $\isterm{\Gb}{\tb'}{\Tb'} \in \trans{\xisterm{\Ga}{t}{T}}$ for some $\tb'$.
% \end{lemma}

% \begin{proof}
%   By definition we have $T \ir \Tb$ and $T \ir \Tb'$ and thus $T \sim \Tb$ and
%   $T \sim \Tb'$, implying $\Tb \sim \Tb'$ by transitivity~\eqref{lem:sim-er}.
%   By lemma~\ref{lem:sim-cong}
%   (in the case $\Ga_1 \equiv \Ga_2 \equiv \ctxempty$) we get
%   $\isterm{\Gb}{p}{\Heq{}{\Tb}{}{\Tb'}}$ for some $p$.
%   By lemma~\ref{lem:uip-cong} (and lemma~\ref{lem:choose} to give
%   universes as types to $\Tb$ and $\Tb'$) we can assume
%   $\isterm{\Gb}{p}{\Eq{}{\Tb}{\Tb'}}$. Then
%   $\isterm{\Gb}{\transpo{p}\ \tb}{\Tb'}$ is still a translation since $\ir$
%   ignores transports.
% \end{proof}

% We can now define the translation. This is done by mutual induction on
% context well-formedness, typing and conversion derivations. Indeed,
% in order to be able to produce a realiser by induction, we need to show
% that every conversion in ETT is translated as an heterogeneous equality
% in ITT.

% \begin{theorem}[Translation]
%   \label{thm:translation}
%   \leavevmode
%   \begin{itemize}
%     \item If\,\,\,$\xisctx{\Ga}$ then there exists
%     $\isctx{\Gb} \in \trans{\xisctx{\Ga}}$,

%     \item If\,\,\,$\xisterm{\Ga}{t}{T}$ then for any
%     $\isctx{\Gb} \in \trans{\xisctx{\Ga}}$ there exist $\tb$ and $\Tb$ such
%     that $\isterm{\Gb}{\tb}{\Tb} \in \trans{\xisterm{\Ga}{t}{T}}$,

%     \item If\,\,\,$\xeqterm{\Ga}{u}{v}{A}$ then for any
%     $\isctx{\Gb} \in \trans{\xisctx{\Ga}}$ there exist
%     $A \ir \Ab, A \ir \Ab', u \ir \ub, v \ir \vb$ and $\eb$ such that
%     $\isterm{\Gb}{\eb}{\Heq{\Ab}{\ub}{\Ab'}{\vb}}$.
%   \end{itemize}
% \end{theorem}

% \begin{proof}
%   We prove the theorem by induction on the derivation in the
%   extensional type theory. We only show the two most interesting cases
%   of application and conversion.
%   The complete proof is given in Appendix~\ref{sec:corr-transl}.

%   \begin{itemize}
%     \item \textsc{Application}
%     \[
%       \infer[]
%         {\xisterm{\Ga}{A}{s} \\
%          \xisterm{\Ga,x:A}{B}{s'} \\
%          \xisterm{\Ga}{t}{\Prod{x:A} B} \\
%          \xisterm{\Ga}{u}{A}
%         }
%         {\xisterm{\Ga}{\app{t}{x:A}{B}{u}}{B[x \sto u]}}
%       %
%     \]
%     Using IH together with lemmata~\ref{lem:choose} and~\ref{lem:change-type}
%     we get $\isterm{\Gb}{\Ab}{s}$ and $\isterm{\Gb,x:\Ab}{\Bb}{s'}$ and
%     $\isterm{\Gb}{\tb}{\Prod{x:\Ab} \Bb}$ and $\isterm{\Gb}{\ub}{\Ab}$
%     meaning we can conclude
%     $\isterm{\Gb}{\app{\tb}{x:\Ab}{\Bb}{\ub}}{\Bb[x \sto \ub]}
%     \in \trans{\xisterm{\Ga}{\app{t}{x:A}{B}{u}}{B[x \sto u]}}$.

%     \item \textsc{Conversion}
%     \[
%       \infer[]
%         {\xisterm{\Ga}{u}{A} \\
%          \xeqtype{\Ga}{A}{B}
%         }
%         {\xisterm{\Ga}{u}{B}}
%       %
%     \]
%     By IH and lemma~\ref{lem:uip-cong} we have
%     $\isterm{\Gb}{\eb}{\Eq{}{\Ab}{\Bb}}$ which implies
%     $\istype{\Gb}{\Ab} \in \trans{\xistype{\Ga}{A}}$ by
%     inversion~\eqref{lem:inversion}, thus, from lemma~\ref{lem:change-type}
%     and IH we get $\isterm{\Gb}{\ub}{\Ab}$, yielding
%     $\isterm{\Gb}{\transpo{\eb}\ \ub}{\Bb} \in \trans{\xisterm{\Ga}{u}{B}}$.
%   \end{itemize}

% \end{proof}


% \subsection{Meta-theoretical Consequences}
% \label{sec:meta-consequences}

% We can check that all ETT theorems whose type are typable in ITT have
% proofs in ITT as well:

% \begin{corollary}[Preservation of ITT]
%   \label{cor:preservation}
%   If $\xisterm{}{t}{T}$ and $\istype{}{T}$ then there exist $\tb$ such that
%     $\isterm{}{\tb}{T} \in \trans{\xisterm{}{t}{T}}$.
% \end{corollary}
% \begin{proof}
%   Since $\isctx{\ctxempty} \in \trans{\xisctx{\ctxempty}}$, by
%   Theorem~\eqref{thm:translation}, there exists $\tb$ and $\Tb$ such
%   that
%   $\isterm{}{\tb}{\Tb} \in \trans{\xisterm{}{t}{T}}$
%   But as $\istype{}{T}$, we have
%   $\istype{}{T} \in \trans{\xistype{}{T}}$, and,
%   using Lemma~\ref{lem:change-type}, we obtain
%   $\isterm{}{\tb}{T} \in \trans{\xisterm{}{t}{T}}$.
% \end{proof}

% \begin{corollary}[Relative consistency]
%   \label{cor:consistency}
%   Assuming ITT is consistent, there is no term $t$ such that
%   $\xisterm{}{t}{\Prod{A:\Ty{0}}{A}}$.
% \end{corollary}

% \begin{proof}
%   Assume such a $t$ exists. By the Corollary~\ref{cor:preservation},
%   because $\istype{}{\Prod{A:\Ty{0}}{A}}$,
%   there exists $\tb$ such that $\isterm{}{\tb}{\Prod{A:\Ty{0}}{A}}$ which
%   contradicts the assumed consistency of ITT.
% \end{proof}

% \subsection{Optimisations}
% \label{sec:optim}

% Up until now, we remained silent about one thing: the size of the
% translated terms. Indeed, the translated term is a decoration of the
% initial one by transports which appear in many locations. For example,
% at each application we use a transport by lemma \ref{lem:choose} to
% ensure that the term in function position is given a function type. In
% most cases---in particular when translating ITT terms---this produces
% unnecessary transports (often by reflexivity) that we wish to avoid.

% In order to limit the size explosion, in the above we use a different version of
% transport, namely $\mathsf{transport}'$ such that
% %
% \begin{align*}
%   \otransport{A_1}{A_2}{p}{t} &= t &\text{ when } A_1 =_\alpha A_2 \\
%   &= \transpo{p}{t} &\text{ otherwise.}
% \end{align*}
% %
% The idea is that we avoid \emph{trivially} unnecessary transports (we do not
% deal with $\beta$-conversion for instance).
% We extend this technique to the different constructors of equality (symmetry,
% transitivity, \dots) so that they reduce to reflexivity whenever possible.
% Take transitivity for instance:
% %
% \begin{align*}
%   \otransitivity{\refl{} u}{q} &= q \\
%   \otransitivity{p}{\refl{} u} &= p \\
%   \otransitivity{p}{q} &= \transitivity{p}{q}.
% \end{align*}
% %
% We show these \emph{defined terms} enjoy the same typing rules as their
% counterparts and use them instead.
% In practice it is enough to recover the exact same term when it is typed in ITT.

% \section{Formalisation with Template-Coq}
% \label{sec:form-with-templ}

% We have formalised the translation
% in the setting of \TemplateCoq~\cite{DBLP:conf/itp/AnandBCST18} in order to have
% a more precise proof, but also to evidence the fact that the translation is
% indeed constructive and can be used to perform computations.

% \TemplateCoq is a \Coq library that has a representation of \Coq terms
% as they are in \Coq's kernel (in particular using de Bruijn indices for
% variables) and a (partial) implementation of the type checking algorithm
% (not checking guardedness of fixpoints or positivity of inductive types).
% %
% It comes with a \Coq plugin that permits to quote \Coq terms into their
% representations, and to produce \Coq terms from their representation
% (if they indeed denote well-typed terms).
% %
% We have integrated our formalisation within that framework in order to
% ensure our presentations of ETT and ITT are close to \Coq, but also to
% take advantage of the quoting mechanism to produce terms using
% the interactive mode (in particular we get to use tactics).
% %
% Note that we also rely on Mangin and Sozeau's
% \Equations{}~\cite{DBLP:conf/itp/Sozeau10} plugin to derive nice
% dependent induction principles.

% Our formalisation takes full advantage of its easy interfacing with \TemplateCoq:
% we define two theories, namely ETT and ITT, but ITT enjoys a lot of syntactic
% sugar by having things such as transport, heterogeneous equality and packing as
% part of the syntax. The operations regarding these constructors---in particular
% the tedious ones---are written in \Coq and then quoted to finally be
% \emph{realised} in the translation from ITT to \TemplateCoq.

% \paragraph{Interoperability with \TemplateCoq.}
% The translation we define from ITT to \TemplateCoq is not proven
% correct, but it is not really important as it can just be seen as a
% feature to observe the produced terms in a nicer setting. In any case,
% \TemplateCoq does not yet provide a complete formalisation of CIC rules,
% as guard checking of recursive definitions and strict positivity of
% inductive type declarations are not formalised yet.

% Our formalised theorems however do not depend on \TemplateCoq itself and as such
% there is no need to \emph{trust} the plugin.

% We also provide a translation from \TemplateCoq to ETT that we will describe
% more extensively with the examples (Section~\ref{sec:examples}).

% \subsection{Quick Overview of the Formalisation}

% The file \rpath{SAst.v} contains the definition of the (common) abstract syntax
% of ETT and ITT in the form of an inductive definition with de Bruijn
% indices for variables (like in \TemplateCoq).
% Sorts are defined separately in \rpath{Sorts.v} and we will address them later
% in Section~\ref{sec:sorts}.

% \begin{coq}
% Inductive sterm : Type :=
% | sRel (n : nat)
% | sSort (s : sort)
% | sProd (nx : name) (A B : sterm)
% | sLambda (nx : name) (A B t : sterm)
% | sApp (u : sterm) (nx : name) (A B v : sterm)
% | sEq (A u v : sterm)
% | sRefl (A u : sterm)
% | (* ... *) .
% \end{coq}

% The files \rpath{ITyping.v} and \rpath{XTyping.v} define respectively the
% typing judgments for ITT and ETT, using mutual inductive types.
% Then, most of the files are focused on the meta-theory of ITT and can be ignored
% by readers who don't need to see yet another proof of subject reduction.

% The most interesting files are obviously those where the fundamental lemma
% and the translation are formalised: \rpath{FundamentalLemma.v} and
% \rpath{Translation.v}.
% For instance, here is the main theorem, as stated in our formalisation:
% %
% \begin{coq}
% Theorem complete_translation Σ :
%   type_glob Σ ->
%   (forall Γ (h : XTyping.wf Σ Γ), ∑ Γ', Σ |--i Γ' # ⟦ Γ ⟧ ) *
%   (forall Γ t A (h : Σ ;;; Γ |-x t : A)
%    Γ' (hΓ : Σ |--i Γ' # ⟦ Γ ⟧),
%     ∑ A' t', Σ ;;;; Γ' |--- [t'] : A' # ⟦ Γ |--- [t] : A ⟧) *
%   (forall Γ u v A (h : Σ ;;; Γ |-x u = v : A)
%    Γ' (hΓ : Σ |--i Γ' # ⟦ Γ ⟧),
%     ∑ A' A'' u' v' p', eqtrans Σ Γ A u v Γ' A' A'' u' v' p').
% \end{coq}
% %
% Herein \coqe{type_glob Σ} refers to the fact that some global
% context is well-typed, its purpose is detailed in
% Section~\ref{sec:inductives}.
% The fact that the theorem holds in \Coq ensures we can actually
% compute a translated term and type out of a derivation in ETT.

% \subsection{Inductive Types and Recursion}
% \label{sec:inductives}

% In the proof of Section~\ref{sec:translation}, we didn't mention
% anything about inductive types, pattern-matching or recursion as it is
% a bit technical on paper.  In the formalisation, we offer a way to
% still be able to use them, and we will even show how it works in
% practice with the examples (Section\ref{sec:examples}).

% The main guiding principle is that inductive types and induction are orthogonal
% to the translation, they should more or less be translated to
% themselves.
% %
% To realise that easily, we just treat an inductive definition as a way
% to introduce new constants in the theory, one for the type, one for
% each constructor, one for its elimination principle, and one equality
% per computation rule.
% %
% For instance, the natural numbers can be represented by having the following
% constants in the context:
% %
% \[
% \begin{array}{l@{~}c@{~}l}
%   \nat &:& \Ty{0} \\
%   \zero &:& \nat \\
%   \natsucc &:& \nat \to \nat \\
%   \natrec &:& \forall P,\
%   P\ \zero \to (\forall m,\ P\ m \to P\ (\natsucc\ m)) \to
%   \forall n,\ P\ n \\
%   \natrec_\zero &:& \forall P\ P_z\ P_s,\ \natrec\ P\ P_z\ P_s\ \zero = P_z \\
%   \natrec_\natsucc &:& \forall P\ P_z\ P_s\ n,\\
%   &&\natrec\ P\ P_z\ P_s\ (\natsucc\ n) = P_s\ n\ (\natrec\ P\ P_z\ P_s\ n)
% \end{array}
% \]
% %
% Here we rely on the reflection rule to obtain the computational behaviour of the
% eliminator $\natrec$.

% This means for instance that we do not consider inductive types that would only
% make sense in ETT, but we deem this not to be a restriction and to the best of
% our knowledge isn't something that is usually considered in the literature.
% %
% With that in mind, our translation features a global context of typed constants
% with the restriction that the types of those constants should be well-formed
% in ITT. Those constants are thus used as black boxes inside ETT.

% With this we are able to recover what we were missing from
% \Coq, without having to deal with the trouble of proving that the translation
% doesn't break the guard condition of fixed points, and we are instead relying on
% a more type-based approach.

% \subsection{About Universes and Homotopy}
% \label{sec:sorts}

% The experienced reader might have noticed that our treatment of universes
% (except perhaps for the absence of cumulativity) was really superficial and the
% notion of sorts used is rather orthogonal to our main development.
% This is even more apparent in the formalisation. Indeed, we didn't fix a
% specific universe hierarchy, but instead specify what properties it should
% have, in what is reminiscent to a (functional\footnote{Meaning the sort of a
% sort, and the sort of a product are functions, necessary to the uniqueness of
% types~\eqref{lem:uniq}.}) PTS formulation.
% %
% \begin{coq}
% Class Sorts.notion := {
%   sort : Type ;
%   succ : sort -> sort ;
%   prod_sort : sort -> sort -> sort ;
%   sum_sort : sort -> sort -> sort ;
%   eq_sort : sort -> sort ;
%   eq_dec : forall s z : sort, {s = z} + {s <> z} ;
%   succ_inj : forall s z, succ s = succ z -> s = z
% }.
% \end{coq}
% %
% From the notion of sorts, we require functions to get the sort of a sort,
% the sort of a product from the sorts of its arguments, and (crucially) the sort
% of an identity type.
% We also require some measure of decidable equality and injectivity on those.

% This allows us to instantiate this by a lot of different notions including the
% one presented earlier in the paper or even its extension with a universe $\Prop$
% of propositions (like CIC~\cite{bertot2004interactive}). We present here two
% instances that have their own interest.

% \paragraph{$\Type$ in $\Type$.}
% One of the instances we provide is one with only one universe $\Type$, with the
% inconsistent typing rule $\Type : \Type$.
% Although inconsistent, this allows us to interface with \TemplateCoq, without
% the---for the time being---very time-consuming universe constraint checking.

% \paragraph{Homotopy Type System and Two-Level Type Theory.}
% Another interesting application (or rather instance) of our formalisation
% is a translation from Homotopy Type System (HTS)~\cite{hts-sota} to
% Two-Level Type Theory
% (2TT)~\cite{DBLP:journals/corr/AltenkirchCK16,DBLP:journals/corr/AnnenkovCK17}.

% HTS and 2TT arise from the incompatibility between UIP---recall it is provable
% in ETT---and univalence. The idea is to have two distinct notions of equality
% in the theory, a \emph{strict} one satisfying UIP, and a \emph{fibrant} one
% corresponding to the homotopy type theory equality, possibly satisfying
% univalence. This actually induces a separation in the types of the theory:
% some of them are called \emph{fibrant} and the fibrant or homotopic equality
% can only be eliminated on those.
% HTS can be seen as an extension of 2TT with reflection on the strict equality
% just like ETT is an extension of ITT.

% We can recover HTS and 2TT in our setting by taking $\F{i}$ and $\Un{i}$ as
% respectively the fibrant and strict universes of those theories
% (for $i \in \mathbb{N}$), along with the following PTS rules:
% %
% \[
% \begin{array}{l@{~}c@{~}l@{\qquad}l@{~}c@{~}l}
%   (\F{i}, \F{i+1}) &\in& \Ax &
%   (\Un{i}, \Un{i+1}) &\in& \Ax \\
%   (\F{i}, \F{j}, \F{\nmax{i}{j}}) &\in& \Rl &
%   (\F{i}, \Un{j}, \Un{\nmax{i}{j}}) &\in& \Rl \\
%   (\Un{i}, \F{j}, \Un{\nmax{i}{j}}) &\in& \Rl &
%   (\Un{i}, \Un{j}, \Un{\nmax{i}{j}}) &\in& \Rl \\
% \end{array}
% \]
% %
% and the fact that the sort of the (strict) identity type on $A : s$ is
% the \emph{strictified} version of $s$, \ie $\Un{i}$ for $s = \Un{i}$ or
% $s = \F{i}$.
% In order to have the fibrant equality, one simply needs to do as in
% Section~\ref{sec:inductives}.

% In short, the translation from HTS to 2TT is basically the same as the one
% from ETT to ITT we presented in this paper, and this fact is factorised through
% our formalisation.

% \subsection{ETT-flavoured \Coq: Examples}
% \label{sec:examples}

% In this section we demonstrate how our translation can bring extensionality to
% the world of \Coq in action. The examples can be found in
% \rpath{plugin\_demo.v}.

% \paragraph{First, a pedestrian approach.}
% %
% We would like to begin by showing how one can write an example step by step
% before we show how it can be instrumented and automated as a plugin.
% For this we use a self-contained example without any inductive
% types or recursion, illustrating a very simple case of reflection.
% The term we want to translate is our introductory example of transport:
% \[
%   \lambda\ A\ B\ e\ x.\ x : \Pi\ A\ B.\ A = B \to
%   A \to B
% \]
% which relies on the equality $e : A = B$ and reflection to convert $x
% : A$ to $x : B$.
% %
% Of course, this definition isn't accepted in \Coq because this
% conversion is not valid in ITT.
% %
% \begin{coq}
% Fail Definition pseudoid (A B : Type) (e : A = B) (x : A) : B := x.
% \end{coq}
% %
% However, we still want to be able to write it \emph{in some way}, in order to
% avoid manipulating de Bruijn indices directly.
% For this, we use a little trick by first defining a \Coq axiom to represent
% an ill-typed term:
% %
% \begin{coq}
% Axiom candidate : forall A B (t : A), B.
% \end{coq}
% %
% \coqe|candidate A B t| is a candidate \coqe|t| of type
% \coqe|A| to inhabit type \coqe|B|.
% We complete this by adding a notation that is reminiscent to
% \Agda's~\cite{norell2007towards} hole mechanism.
% %
% \begin{coq}
% Notation "'{!' t '!}'" := (candidate _ _ t).
% \end{coq}

% We can now write the ETT function within \Coq.
% %
% \begin{coq}
% Definition pseudoid (A B : Type) (e : A = B) (x : A) : B := {! x !}.
% \end{coq}
% %
% We can then quote the term and its type to \TemplateCoq thanks to the
% \coqe|Quote Definition| command provided by the plugin.
% %
% \begin{coq}
% Quote Definition pseudoid_term :=
%   ltac:(let t := eval compute in pseudoid in exact t).
% Quote Definition pseudoid_type :=
%   ltac:(let T := type of pseudoid in exact T).
% \end{coq}
% %
% The terms that we get are now \TemplateCoq terms, representing \Coq syntax.
% We need to put them in ETT, meaning adding the annotations, and also removing
% the \coqe|candidate| axiom.
% This is the purpose of the \coqe|fullquote| function that we provide
% in our formalisation.
% %
% \begin{coq}
% Definition pretm_pseudoid :=
%   Eval lazy in fullquote (2^18) Σ [] pseudoid_term empty empty nomap.
% Definition tm_pseudoid :=
%   Eval lazy in match pretm_pseudoid with
%               | Success t => t
%               | Error _ => sRel 0
%               end.

% Definition prety_pseudoid :=
%   Eval lazy in fullquote (2^18) Σ [] pseudoid_type empty empty nomap.
% Definition ty_pseudoid :=
%   Eval lazy in match prety_pseudoid with
%                | Success t => t
%                | Error _ => sRel 0
%                end.
% \end{coq}
% %
% \coqe|tm_pseudoid| and \coqe|ty_pseudoid| correspond
% respectively to the ETT representation of \coqe|pseudoid| and its
% type.
% We then produce, using our home-brewed Ltac type-checking tactic, the
% corresponding ETT typing derivation (notice the use of reflection to typecheck).
% %
% \begin{coq}
% Lemma type_pseudoid : Σi ;;; [] |-x tm_pseudoid : ty_pseudoid.
% Proof.
%   unfold tm_pseudoid, ty_pseudoid.
%   ettcheck. cbn.
%   eapply reflection with (e := sRel 1).
%   ettcheck.
% Defined.
% \end{coq}
% %
% We can then translate this derivation, obtain the translated term and then
% convert it to \TemplateCoq.
% %
% \begin{coq}
% Definition itt_pseudoid : sterm :=
%   Eval lazy in
%   let '(_ ; t ; _) :=
%     type_translation type_pseudoid istrans_nil
%   in t.

% Definition tc_pseudoid : tsl_result term :=
%   Eval lazy in
%   tsl_rec (2 ^ 18) Σ [] itt_pseudoid empty.
% \end{coq}
% %
% Once we have it, we \emph{unquote} the term to obtain a \Coq term
% (notice that the only use of reflection has been replaced by a transport).
% %
% \begin{coq}
% fun (A B : Type) (e : A = B) (x : A) => transport e x
%      : forall A B : Type, A = B -> A -> B
% \end{coq}

% \paragraph{Making a Plugin with \TemplateCoq.}
% %
% All of this work is pretty systematic. Fortunately for us,
% \TemplateCoq also features a monad to reify \Coq commands which we can
% use to \emph{program} the translation steps.
% As such we have written a complete procedure, relying on type checkers we
% wrote for ITT and ETT, which can generate equality obligations.

% Thanks to this, the user doesn't have to know about the details of
% implementation of the translation, and stay within the \Coq ecosystem.

% For instance, our previous example now becomes:
% %
% \begin{coq}
% Definition pseudoid (A B : Type) (e : A = B) (x : A) : B := {! x !}.

% Run TemplateProgram (Translate ε "pseudoid").
% \end{coq}
% %
% This produces a \Coq term \coqe{pseudoid'} corresponding to the
% translation (\coqe|ε| is the empty translation
% context, see the next example to understand the need for a
% translation context).
% Notice how the user doesn't even have to provide any proof of equality or
% derivations of any sort. The derivation part is handled by our own typechecker
% while the obligation part is solved automatically by the \Coq obligation mechanism.

% \paragraph{About inductive types.}
% %
% As we promised, our translation is able to handle inductive types.
% For this consider the inductive type of vectors (or length-indexed lists) below,
% together with a simple definition (we will remain in ITT for simplicity).
% %
% \begin{coq}
% Inductive vec A : nat -> Type :=
% | vnil : vec A 0
% | vcons : A -> forall n, vec A n -> vec A (S n).

% Arguments vnil {_}.
% Arguments vcons {_} _ _ _.

% Definition vv := vcons 1 _ vnil.
% \end{coq}
% %
% This time, in order to apply the translation we need to extend the translation
% context with \coqe|nat| and \coqe|vec|.
% %
% \begin{coq}
% Run TemplateProgram (
%     Θ <- TranslateConstant ε "nat" ;;
%     Θ <- TranslateConstant Θ "vec" ;;
%     Translate Θ "vv"
% ).
% \end{coq}
% %
% The command \coqe|TranslateConstant| enriches the current
% translation context with the types of the inductive type and of its
% constructors. The translation context then also contains associative
% tables between our own representation of constants and those of \Coq.
% Unsurprisingly, the translated \Coq term is the same as the original
% term.

% \paragraph{Reversal of vectors.}
% %
% Next, we tackle a motivating example: reversal on vectors.
% Indeed, implementing this operation the same way it can
% be done on lists ends up in the following conversion problem:
% %
% \begin{coq}
% Fail Definition vrev {A n m} (v : vec A n) (acc : vec A m)
% : vec A (n + m) :=
%   vec_rect A (fun n _ => forall m, vec A m -> vec A (n + m))
%            (fun m acc => acc)
%            (fun a n _ rv m acc => rv _ (vcons a m acc))
%            n v m acc.
% \end{coq}
% %
% The recursive call returns a vector of length \coqe|n + S m|
% where the context expects one of length \coqe|S n + m|. In ITT, these
% types are not convertible. This example is thus a perfect fit for ETT where we
% can use the fact that these two expressions always compute to the same thing
% when instantiated with concrete numbers.
% %
% \begin{coq}
% Definition vrev {A n m} (v : vec A n) (acc : vec A m)
% : vec A (n + m) :=
%   vec_rect A (fun n _ => forall m, vec A m -> vec A (n + m))
%            (fun m acc => acc)
%            (fun a n _ rv m acc => {! rv _ (vcons a m acc) !})
%            n v m acc.

% Run TemplateProgram (
%     Θ <- TranslateConstant ε "nat" ;;
%     Θ <- TranslateConstant Θ "vec" ;;
%     Θ <- TranslateConstant Θ "Nat.add" ;;
%     Θ <- TranslateConstant Θ "vec_rect" ;;
%     Translate Θ "vrev"
% ).
% \end{coq}
% %
% This generates four obligations that are all solved automatically. One of
% them contains a proof of \coqe|S n + m = n + S m| while the remaining
% three correspond to the computation rules of addition (as mentioned before,
% \coqe|add| is simply a constant and does not compute in our
% representation, hence the need for equalities).
% %
% The returned term is the following, with only one transport remaining
% (remember our interpretation map removes unnecessary transports).
% \begin{coq}
% fun (A : Type) (n m : nat) (v : vec A n) (acc : vec A m) =>
% vec_rect A
%   (fun n _ => forall m, vec A m -> vec A (n + m))
%   (fun m acc => acc)
%   (fun a n₀ v₀ rv m₀ acc₀ =>
%     transport (vrev_obligation_3 A n m v acc a n₀ v₀ rv m₀ acc₀)
%       (rv (S m₀) (vcons a m₀ acc₀))) n v m acc
% : forall A n m, vec A n -> vec A m -> vec A (n + m)
% \end{coq}

% \subsection{Towards an Interfacing between \Andromeda and \Coq}

% \Andromeda~\cite{andromeda} is a proof assistant implementing ETT in a sense
% that is really close to our formalisation. Aside from a concise nucleus with
% a basic type theory, most things happen with the declaration of constants
% with given types, including equalities to define the computational behaviour
% of eliminators for instance.
% This is essentially what we do in our formalisation.
% Furthermore, their theory relies on $\Type : \Type$, meaning, our modular
% handling of universes can accommodate for this as well.

% All in all, it should be possible in the near future to use our translation
% to produce \Coq terms out of \Andromeda developments.
% %
% Note that this would not suffer from the difficulties in generating typing
% derivations since \Andromeda generates them.

% \subsection{Composition with other Translations}

% This translation also enables the formalisation of translations that
% target ETT rather than ITT and still get mechanised proofs of (relative)
% consistency by composition with this ETT to ITT translation.  This could
% also be used to implement plugins based on the composition of
% translations. In particular, supposing we have a theory which forms a
% subset of ETT and whose conversion is decidable. Using this translation,
% we could formalise it as an embedded domain-specific type theory and
% provide an automatic translation of well-typed terms into witnesses in
% \Coq. This would make it possible to extend conversion with the theory
% of lists for example.

% This would provide a simple way to justify the consistency of
% CoqMT~\cite{DBLP:conf/lpar/JouannaudS17} for example, seeing it as an
% extensional type theory where reflection is restricted to equalities
% on a specific domain whose theory is decidable.


% \begin{comment}
% \section{Extension: Translating HTS to 2TT}
% \label{sec:extens-transl-hts}

% \tw{Keep it as an easy extension? If so, we need to say more as this is no
% longer a direct application.}

% As we mentioned earlier, one of the interesting special cases covered by this
% translation is the translation from 2-level type theory with reflection
% (arguably a variant of Homotopy Type System~\cite{hts-sota} (HTS) to 2-level
% type theory~\cite{altenkirch2016extending} (2TT).
% %
% We thus use two hierarchies $(\F{i})_{i \in \mathbb{N}}$, universes of fibrant
% types, and $(\Un{i})_{i \in \mathbb{N}}$, universes of strict types.
% We give the following axioms for the sorts:
% %
% \begin{align*}
%   (\F{i}, \F{i+1}) &\in \Ax \\
%   (\Un{i}, \Un{i+1}) &\in \Ax \\
%   (\F{i}, \F{j}, \F{\nmax{i}{j}}) &\in \Rl \\
%   (\F{i}, \Un{j}, \Un{\nmax{i}{j}}) &\in \Rl \\
%   (\Un{i}, \F{j}, \Un{\nmax{i}{j}}) &\in \Rl \\
%   (\Un{i}, \Un{j}, \Un{\nmax{i}{j}}) &\in \Rl
% \end{align*}
% %
% We then equip our theory with a fibrant equality type defined as an inductive
% type and the proof takes care of it. We can also add any inductive type that we
% like, as for instance the natural numbers, the unit and empty types.
% \end{comment}

% \section{Limitations and Axioms}
% \label{sec:axioms}

% Currently, the representation of terms and derivations and the
% computational content of the proof only allow us to deal with the
% translation of relatively small terms but we hope to improve that in
% the future. As we have seen, the actual translation involves the
% computational content of lemmata of inversion, substitution, weakening
% and equational reasoning and thus cannot be presented as a simple
% recursive definition on derivations.


% As we already mentioned, the axioms K and FunExt are both
% necessary in ITT if we want the translation to be conservative as they are
% provable in ETT~\cite{hofmann1995conservativity}.
% However, one might still be concerned about having axioms
% as they can for instance hinder canonicity of the system.
% In that respect, K isn't really a restriction since it preserves canonicity.
% The best proof of that is probably \Agda itself which natively features K---in
% fact, one needs to explicitly deactivate it with a flag if they wish to work
% without.

% The case of FunExt is trickier. It should be possible to realise
% the axiom by composing our translation with a setoid
% interpretation~\cite{altenkirch99} which validates it, or by going into a
% system featuring it, for instance by implementing Observational Type
% Theory~\cite{altenkirch2007observational} like
% \Epigram~\cite{mcbride2004epigram}.

% However, these two axioms are not used to define the translation itself,
% but only to witness UIP and function extensionality in the translation to
% \Coq.
% The translation only relies on one axiom, called
% \coqe|conv_trans_AXIOM| in the formalisation, stating that conversion
% of ITT is transitive.
% %
% The proof of this property basically sums up to the confluence of the
% reduction rules of ITT which is out of scope for this paper and has
% recently been formalised in Agda~\cite{Abel:2017:DCT:3177123.3158111} (in a
% simpler setting with only one universe).
% Regardless, this axiom inhabits a proposition (the type of conversion is in
% \coqe|Prop|) and is thus irrelevant for computation. Actually no
% information about the derivation leaks to the production of the ITT term.

% On a different note, the \coqe|candidate| axiom allows us to derive
% \coqe|False| but is merely used to write ill-typed terms in \Coq.
% The translated term will never make us of it and one can always check if a
% term is relying on unsafe assumptions thanks to the
% \coqe|Print Assumptions| command.

% \section{Related Works and Conclusion}
% \label{sec:related-works}

% The seminal works on the precise connection between ETT and ITT go
% back to \cite{streicher1993investigations} and
% \cite{hofmann1995conservativity,HofmannPhD}.
% %
% In particular, the work of Hofmann provides a categorical answer to
% the question of consistency and conservativity of ETT over ITT with
% UIP and FunExt.
% %
% Ten years later, \cite{oury2005extensionality,Oury2006} provided
% a translation from ETT to ITT with
% UIP and FunExt and other axioms (mainly due to
% technical difficulties).
% %
% Although a first step towards a move from categorical semantics to a
% syntactic translation, his work does not stress any constructive
% aspect of the proof and shows that there merely exist translations in
% ITT to a typed term in ETT.

% \cite{van2013explicit} have later proposed and
% formalised a similar translation between a PTS with and without explicit
% conversion. This does not entail anything about ETT to ITT but we can
% find similarities in that there is a witness of conversion between any
% term and itself under an explicit conversion, which internalises
% irrelevance of explicit conversions. This morally corresponds to a
% Uniqueness of Conversions principle.

% The Program \cite{sozeau:icfp07} extension of \Coq performs a
% related coercion insertion algorithm, between objects in subsets on the
% same carrier or in different instances of the same inductive family,
% assuming a proof-irrelevance axiom. Inserting coercions locally is not
% as general as the present translation from ETT to ITT which can insert
% transports in any context.

% In this paper we provide the first effective translation from ETT to ITT
% with UIP and FunExt. The translation has been
% formalised in \Coq using \TemplateCoq, a meta-programming plugin of
% \Coq. This translation is also effective in the sense that we can
% produce in the end a \Coq term using the \TemplateCoq denotation
% machinery.
% %
% With ongoing work to extend the translation to the inductive fragment
% of \Coq, we are paving the way to an extensional version of the \Coq
% proof assistant which could be translated back to its intensional
% version, allowing the user to navigate between the two modes, and in
% the end produce a proof term checkable in the intensional fragment.

% \begin{comment}
% \item Hofmann and Oury of course
% \item Floris and Geuvers~\cite{van2013explicit}
% \item Maybe more about Streicher~\cite
% \item Translations~\cite{boulier17:_next_syntac_model_type_theor}?
%   % \item Maybe we should talk about the intuition of Andrej that annotations
%   %   are needed. His example was the Cardinal model in which it was safe to
%   %   assume equality nat -> nat = nat -> bool from which you can derive 3 : bool
%   %   (if you don't annotate application and then allow reduction) but no
%   %   contradiction that we could find.
% \item We should somewhere talk about Oury's extra axiom. But maybe the place
%   for it is not the introduction.
% \item Uniqueness of typing and lack of cumulativity that doesn't represent
%   a restriction?
% \end{comment}

% \begin{acks}
%   We would like to thank Andrej Bauer and Philipp Haselwarter with whom we had
%   fruitful discussions on the subject, prior to this work.
%   We also would like to thank the attendees of the Aarhus EUTypes 2018 meeting
%   for their insightful feedback on the plugin stemming from the translation.
% \end{acks}

% % \newpage
% % \clearpage

% \newpage
% \onecolumn
% \appendix

% \section{Complementary rules}
% \label{sec:more-rules}

% \begin{figure*}[htbp]
  \flushleft
  \hrulefill
  \paradot{Equivalence relation}

  \begin{mathpar}
    \infer[]
      {\isterm{\Ga}{u}{A}}
      {\eqterm{\Ga}{u}{u}{A}}
    %

    \infer[]
      {\eqterm{\Ga}{u}{v}{A}}
      {\eqterm{\Ga}{v}{u}{A}}
    %

    \infer[]
      {\eqterm{\Ga}{u}{v}{A} \\
       \eqterm{\Ga}{v}{w}{A}
      }
      {\eqterm{\Ga}{u}{w}{A}}
    %
  \end{mathpar}

  \paradot{Congruence of type constructors}

  \begin{mathpar}
    \infer[]
      {\eqterm{\Ga}{A_1}{A_2}{s} \\
       \eqterm{\Ga,x:A_1}{B_1}{B_2}{s'}
      }
      {\eqterm{\Ga}{\Prod{x:A_1} B_1}{\Prod{x:A_2} B_2}{s''}}
    (s,s',s'')

    \infer[]
      {\eqterm{\Ga}{A_1}{A_2}{s} \\
       \eqterm{\Ga,x:A_1}{B_1}{B_2}{s'}
      }
      {\eqterm{\Ga}{\Sum{x:A_1} B_1}{\Sum{x:A_2} B_2}{s''}}
    (s,s',s'')

    \infer[]
      {\eqterm{\Ga}{A_1}{A_2}{s} \\
       \eqterm{\Ga}{u_1}{u_2}{A_1} \\
       \eqterm{\Ga}{v_1}{v_2}{A_1}
      }
      {\eqterm{\Ga}{\Eq{A_1}{u_1}{v_1}}{\Eq{A_2}{u_2}{v_2}}{s}}
    %
  \end{mathpar}

  \paradot{Congruence of $\lambda$-calculus terms}

  \begin{mathpar}
    \infer[]
      {\eqterm{\Ga}{A_1}{A_2}{s} \\
       \eqterm{\Ga,x:A_1}{B_1}{B_2}{s'} \\
       \eqterm{\Ga,x:A_1}{t_1}{t_2}{B_1}
      }
      {\eqterm{\Ga}{\lam{x:A_1}{B_1} t_1}{\lam{x:A_2}{B_2} t_2}{\Prod{x:A_1} B_1}}
    %

    \infer[]
      {\eqterm{\Ga}{A_1}{A_2}{s} \\
       \eqterm{\Ga,x:A_1}{B_1}{B_2}{s'} \\
       \eqterm{\Ga}{t_1}{t_2}{\Prod{x:A_1} B_1} \\
       \eqterm{\Ga}{u_1}{u_2}{A_1}
      }
      {\eqterm
        {\Ga}
        {\app{t_1}{x:A_1}{B_1}{u_1}}
        {\app{t_1}{x:A_1}{B_1}{u_1}}
        {B_1[x \sto u_1]}
      }
    %

    \infer[]
      {\eqterm{\Ga}{A_1}{A_2}{s} \\
       \eqterm{\Ga}{u_1}{u_2}{A_1} \\
       \eqterm{\Ga,x:A_1}{B_1}{B_2}{s'} \\
       \eqterm{\Ga}{v_1}{v_2}{B_1[x \sto u_1]}
      }
      {\eqterm
        {\Ga}
        {\pair{x:A_1}{B_1}{u_1}{v_1}}
        {\pair{x:A_2}{B_2}{u_2}{v_2}}
        {\Sum{x:A_1}{B_1}}
      }
    %

    \infer[]
      {\eqterm{\Ga}{A_1}{A_2}{s} \\
       \eqterm{\Ga,x:A_1}{B_1}{B_2}{s'} \\
       \eqterm{\Ga}{p_1}{p_2}{\Sum{x:A_1}{B_1}}
      }
      {\eqterm{\Ga}{\pio{x:A_1}{B_1}{p_1}}{\pio{x:A_2}{B_2}{p_2}}{A_1}}
    %

    \infer[]
      {\eqterm{\Ga}{A_1}{A_2}{s} \\
       \eqterm{\Ga,x:A_1}{B_1}{B_2}{s'} \\
       \eqterm{\Ga}{p_1}{p_2}{\Sum{x:A_1}{B_1}}
      }
      {\eqterm
        {\Ga}
        {\pit{x:A_1}{B_1}{p_1}}
        {\pit{x:A_2}{B_2}{p_2}}
        {B_1[x \sto \pio{x:A_1}{B_1}{p_1}]}
      }
    %
  \end{mathpar}

  \paradot{Congruence of equality terms}

  \begin{mathpar}
    \infer[]
      {\eqterm{\Ga}{A_1}{A_2}{s} \\
       \eqterm{\Ga}{u_1}{u_2}{A}
      }
      {\eqterm{\Ga}{\refl{A_1} u_1}{\refl{A_2} u_2}{\Eq{A_1}{u_1}{u_1}}}
    %

    \infer[]
      {\eqterm{\Ga}{A_1}{A_2}{s} \\
       \eqterm{\Ga}{u_1}{u_2}{A_1} \\
       \eqterm{\Ga}{v_1}{v_2}{A_1} \\
       \eqterm{\Ga, x:A_1, e:\Eq{A_1}{u_1}{x}}{P_1}{P_2}{s'} \\
       \eqterm{\Ga}{p_1}{p_2}{\Eq{A_1}{u_1}{v_1}} \\
       \eqterm{\Ga}{w_1}{w_2}{P_1[x \sto u_1, e \sto \refl{A_1} u_1]}
      }
      {\eqterm
        {\Ga}
        {\J{A_1}{u_1}{x.e.P_1}{w_1}{v_1}{p_1}}
        {\J{A_2}{u_2}{x.e.P_2}{w_2}{v_2}{p_2}}
        {P[x \sto v_1, e \sto p_1]}
      }
    %

    \infer[]
      {\eqterm{\Ga}{A_1}{A_2}{s} \\
       \eqterm{\Ga,x:A_1}{B_1}{B_2}{s'} \\
       \eqterm{\Ga}{f_1}{f_2}{\Prod{x:A_1} B_1} \\
       \eqterm{\Ga}{g_1}{g_2}{\Prod{x:A_1} B_1} \\
       \eqterm
         {\Ga}
         {e_1}
         {e_2}
         {\Prod{x:A_1}
          \Eq{B_1}{\app{f_1}{x:A_1}{B_1}{x}}{\app{g_1}{x:A_1}{B_1}{x}}}
      }
      {\eqterm
        {\Ga}
        {\funext{x:A_1}{B_1}{f_1}{g_1}{e_1}}
        {\funext{x:A_2}{B_2}{f_2}{g_2}{e_2}}
        {\Eq{}{f_1}{g_1}}
      }
    %

    \infer[]
      {\eqtype{\Ga}{A_1}{A_2} \\
       \eqterm{\Ga}{u_1}{u_2}{A_1} \\
       \eqterm{\Ga}{v_1}{v_2}{A_2} \\
       \eqterm{\Ga}{p_1}{p_2}{\Eq{A_1}{u_1}{v_1}} \\
       \eqterm{\Ga}{q_1}{q_2}{\Eq{A_1}{u_1}{v_1}} \\
      }
      {\eqterm
        {\Ga}
        {\uip{A_1}{u_1}{v_1}{p_1}{q_1}}
        {\uip{A_2}{u_2}{v_2}{p_2}{q_2}}
        {\Eq{}{p_1}{q_1}}
      }
    %
  \end{mathpar}
  \hrulefill
  \vspace{-2ex}
  \caption{Congruence rules}
  \label{fig:cong-rules}
\end{figure*}



% \section{Proof of the fundamental lemma}
% \label{sec:proof-fund-lemma}

% \begin{lemma}[Fundamental lemma]
%   Let $t_1$ and $t_2$ be two terms. If $\isterm{\Ga, \Ga_1}{t_1}{T_1}$ and
%   $\isterm{\Ga, \Ga_2}{t_2}{T_2}$ and $t_1 \sim t_2$ then there exists $p$ such
%   that
%   $\isterm{\Ga, \Pack{\Ga_1}{\Ga_2}}
%           {p}
%           {\Heq{\llift{\gamma}{}{T_1}}
%                {\llift{\gamma}{}{t_1}}
%                {\rlift{\gamma}{}{T_2}}
%                {\rlift{\gamma}{}{t_2}}}$.
% \end{lemma}

% For readability we will abbreviate the left and right substitutions
% $\llift{\gamma}{}{\_}$ and $\rlift{\gamma}{}{\_}$ by $\upharpoonleft$
% and $\upharpoonright$ respectively.

% \begin{proof}
%   We prove it by induction on the derivation of $t_1 \sim t_2$.

%   \begin{itemize}
%     \item \textsc{Var}
%     \[
%       \infer[]
%         { }
%         {x \sim x}
%       %
%     \]
%     If $x$ belongs to $\Ga$, we apply reflexivity---together with uniqueness of
%     typing~\eqref{lem:uniq}---to conclude.
%     Otherwise, $\ProjE{x}$ has the expected type (since
%     $\llift{\gamma}{}{x} \equiv \ProjO{x}$ and $\rlift{\gamma}{}{x} \equiv \ProjT{x}$).

%     \item \textsc{TransportLeft}
%     \[
%       \infer[]
%         {t_1 \sim t_2}
%         {\transpo{p}\ t_1 \sim t_2}
%       %
%     \]
%     We have $\isterm{\Ga, \Ga_1}{\transpo{p}\ t_1}{T_1}$ and
%     $\isterm{\Ga, \Ga_2}{t_2}{T_2}$.
%     By inversion~\eqref{lem:inversion} we have
%     $\isterm{\Ga, \Ga_1}{p}{\Eq{}{T_1'}{T_1}}$ and
%     $\isterm{\Ga, \Ga_1}{t_1}{T_1'}$.
%     Then by induction hypothesis we have $e$ such that
%     $\isterm{\Ga, \Gp}{e}{\Heq{}{\lo{t_1}}{}{\ro{t_2}}}$.
%     From transitivity and symmetry we only need to provide a proof of
%     $\Heq{}{\lo{t_1}}{}{\transpo{\lo{p}}\ \lo{t_1}}$ which is inhabited by
%     $\pair{\_}{\_}{\lo{p}}{\refl{} (\transpo{\lo{p}}\ \lo{t_1})}$.

%     \item \textsc{TransportRight}
%     \[
%       \infer[]
%         {t_1 \sim t_2}
%         {t_1 \sim \transpo{p}\ t_2}
%       %
%     \]
%     Similarly.

%     \item \textsc{Product}
%     \[
%       \infer[]
%         {A_1 \sim A_2 \\
%          B_1 \sim B_2
%         }
%         {\Prod{x:A_1} B_1 \sim \Prod{x:A_2} B_2}
%       %
%     \]
%     We have $\isterm{\Ga, \Ga_1}{\Prod{x:A_1} B_1}{T_1}$ and
%     $\isterm{\Ga, \Ga_2}{\Prod{x:A_2} B_2}{T_2}$ so by
%     inversion~\eqref{lem:inversion} we have $\isterm{\Ga, \Ga_1}{A_1}{s_1}$ and
%     $\isterm{\Ga, \Ga_1, x:A_1}{B_1}{s'_1}$ and
%     $\eqtype{\Ga, \Ga_1}{s''_1}{T_1}$ for $(s_1,s'_1,s''_1) \in \Rl$
%     (and similarly with $2$s).
%     By induction hypothesis we have
%     $\isterm{\Ga, \Gp}{p_A}{\Heq{}{\lo{A_1}}{}{\ro{A_2}}}$ and
%     $\isterm
%       {\Ga, \Gp, x : \Pack{A_1}{A_2}}
%       {p_B}
%       {\Heq{}{\lo{B_1}}{}{\ro{B_2}}}
%     $
%     hence the result (using UIP and FunExt, refer to the
%     formalisation and especially to the file \rpath{Quotes.v} for more details
%     on how to realise this equality).

%     \item \textsc{Equality}
%     \[
%       \infer[]
%         {A_1 \sim A_2 \\
%          u_1 \sim u_2 \\
%          v_1 \sim v_2
%         }
%         {\Eq{A_1}{u_1}{v_1} \sim \Eq{A_2}{u_2}{v_2}}
%       %
%     \]
%     We have $\isterm{\Ga, \Ga_1}{\Eq{A_1}{u_1}{v_1}}{T_1}$ and
%     $\isterm{\Ga, \Ga_2}{\Eq{A_2}{u_2}{v_2}}{T_2}$ so, by
%     inversion~\eqref{lem:inversion}, we have
%     $\isterm{\Ga, \Ga_1}{A_1}{s_1}$ and $\isterm{\Ga, \Ga_1}{u_1}{A_1}$ and
%     $\isterm{\Ga, \Ga_1}{v_1}{A_1}$ as well as $\eqtype{\Ga, \Ga_1}{s_1}{T_1}$
%     (and the same with $2$s). By induction hypothesis we thus have
%     $\isterm{\Ga, \Gp}{p_{A}}{\Heq{}{A_1}{}{A_2}}$ and
%     $\isterm{\Ga, \Gp}{p_u}{\Heq{}{u_1}{}{u_2}}$ and
%     $\isterm{\Ga, \Gp}{p_v}{\Heq{}{v_1}{}{v_2}}$.
%     We can thus conclude.

%     \item \textsc{Reflexivity}
%     \[
%       \infer[]
%         { }
%         {s \sim s}
%       %
%     \]
%     This one holds by reflexivity and uniqueness of typing~\eqref{lem:uniq}
%     (indeed, $\lo{s} \equiv s$ and $\ro{s} \equiv s$).

%     \item \textsc{Lambda}
%     \[
%       \infer[]
%         {A_1 \sim A_2 \\
%          B_1 \sim B_2 \\
%          t_1 \sim t_2
%         }
%         {\lam{x:A_1}{B_1} t_1 \sim \lam{x:A_2}{B_2} t_2}
%       %
%     \]
%     We have $\isterm{\Ga, \Ga_1}{\lam{x:A_1}{B_1} t_1}{T_1}$ and
%     $\isterm{\Ga, \Ga_2}{\lam{x:A_2}{B_2} t_2}{T_2}$, thus, by
%     inversion~\ref{lem:inversion} the subterms are well-typed and we can
%     apply induction hypothesis. The conclusion follows similarly to the $\Pi$
%     case.

%     \item \textsc{Application}
%     \[
%       \infer[]
%         {t_1 \sim t_2 \\
%          A_1 \sim A_2 \\
%          B_1 \sim B_2 \\
%          u_1 \sim u_2
%         }
%         {\app{t_1}{x:A_1}{B_1}{u_1} \sim \app{t_2}{x:A_2}{B_2}{u_2}}
%       %
%     \]
%     We have $\isterm{\Ga, \Ga_1}{\app{t_1}{x:A_1}{B_1}{u_1}}{T_1}$ and
%     $\isterm{\Ga, \Ga_2}{\app{t_2}{x:A_2}{B_2}{u_2}}{T_2}$ which means by
%     inversion~\eqref{lem:inversion} that the subterms are well-typed.
%     We apply the induction hypothesis and then conclude.

%     \item \textsc{Reflexivity}
%     \[
%       \infer[]
%         {A_1 \sim A_2 \\
%          u_1 \sim u_2
%         }
%         {\refl{A_1} u_1 \sim \refl{A_2} u_2}
%       %
%     \]
%     We have $\isterm{\Ga, \Ga_1}{\refl{A_1} u_1}{T_1}$ and
%     $\isterm{\Ga, \Ga_2}{\refl{A_2} u_2}{T_2}$ so by
%     inversion~\eqref{lem:inversion} we have $\isterm{\Ga, \Ga_1}{A_1}{s_1}$ and
%     $\isterm{\Ga, \Ga_1}{u_1}{A_1}$
%     (same with $2$s). By IH we have $\Heq{}{\lo{A_1}}{}{\ro{A_2}}$ and
%     $\Heq{\lo{A_1}}{\lo{u_1}}{\ro{A_2}}{\ro{u_2}}$.
%     The proof follows easily.

%     \item \textsc{Funext}
%     \[
%       \infer[]
%         {A_1 \sim A_2 \\
%          B_1 \sim B_2 \\
%          f_1 \sim f_2 \\
%          g_1 \sim g_2 \\
%          e_1 \sim e_2
%         }
%         {\funext{x:A_1}{B_1}{f_1}{g_1}{e_1}
%          \sim \funext{x:A_2}{B_2}{f_2}{g_2}{e_2}
%         }
%       %
%     \]
%     Similar.

%     \item \textsc{UIP}
%     \[
%       \infer[]
%         {A_1 \sim A_2 \\
%          u_1 \sim u_2 \\
%          v_1 \sim v_2 \\
%          p_1 \sim p_2 \\
%          q_1 \sim q_2
%         }
%         {\uip{A_1}{u_1}{v_1}{p_1}{q_1} \sim \uip{A_2}{u_2}{v_2}{p_2}{q_2}}
%       %
%     \]
%     Similar.

%     \item \textsc{J}
%     \[
%       \infer[]
%         {A_1 \sim A_2 \\
%          u_1 \sim u_2 \\
%          P_1 \sim P_2 \\
%          w_1 \sim w_2 \\
%          v_1 \sim v_2 \\
%          p_1 \sim p_2
%         }
%         {\J{A_1}{u_1}{x.e.P_1}{w_1}{v_1}{p_1} \sim
%          \J{A_2}{u_2}{x.e.P_2}{w_2}{v_2}{p_2}
%         }
%       %
%     \]
%     Similar.
%   \end{itemize}
% \end{proof}

% \section{Correctness of the translation}
% \label{sec:corr-transl}
% \begin{theorem}[Translation]
%   \leavevmode
%   \begin{itemize}
%     \item If $\xisctx{\Ga}$ then there exists
%     $\isctx{\Gb} \in \trans{\xisctx{\Ga}}$,

%     \item If $\xisterm{\Ga}{t}{T}$ then for any
%     $\isctx{\Gb} \in \trans{\xisctx{\Ga}}$ there exist $\tb$ and $\Tb$ such that
%     $\isterm{\Gb}{\tb}{\Tb} \in \trans{\xisterm{\Ga}{t}{T}}$,

%     \item If $\xeqterm{\Ga}{u}{v}{A}$ then for any
%     $\isctx{\Gb} \in \trans{\xisctx{\Ga}}$ there exist
%     $A \ir \Ab, A \ir \Ab', u \ir \ub, v \ir \vb$ and $\eb$ such that
%     $\isterm{\Gb}{\eb}{\Heq{\Ab}{\ub}{\Ab'}{\vb}}$.
%   \end{itemize}
% \end{theorem}

% \begin{proof}
%   We prove the theorem by induction on the derivation in the extensional
%   type theory. In most cases we need to assume some $\Gb$, translation of the
%   context, we will implicitly refer to $\Gb$ in such cases as the one given as
%   hypothesis.
%   \leavevmode
%   \begin{itemize}
%     \item \textsc{Empty}
%     \[
%       \infer[]
%         { }
%         {\xisctx{\ctxempty}}
%       %
%     \]
%     We have $\isctx{\ctxempty} \in \trans{\xisctx{\ctxempty}}$.

%     \item \textsc{Extend}
%     \[
%       \infer[]
%         {\xisctx{\Ga} \\
%          \xistype{\Ga}{A}
%         }
%         {\xisctx{\Ga, x:A}}
%       (x \notin \Ga)
%     \]
%     By IH we have $\isctx{\Gb} \in \trans{\xisctx{\Ga}}$ and, using $\Gb$
%     as well as lemma~\ref{lem:choose},
%     $\isterm{\Gb}{\Ab}{s} \in \trans{\xisterm{\Ga}{A}{s}}$.
%     Thus $\isctx{\Gb, x:\Ab} \in \trans{\xisctx{\Ga, x:A}}$.

%     \item \textsc{Sort}
%     \[
%       \infer[]
%         {\xisctx{\Ga}}
%         {\xisterm{\Ga}{s}{s'}}
%       (s,s')
%     \]
%     We have $\isterm{\Gb}{s}{s'} \in \trans{\xisterm{\Ga}{s}{s'}}$.

%     \item \textsc{Product}
%     \[
%       \infer[]
%         {\xisterm{\Ga}{A}{s} \\
%          \xisterm{\Ga,x:A}{B}{s'}
%         }
%         {\xisterm{\Ga}{\Prod{x:A} B}{s''}}
%       (s,s',s'')
%     \]
%     By IH and lemma~\ref{lem:choose} we have $\isterm{\Gb}{\Ab}{s}$,
%     meaning $\isctx{\Gb,x:\Ab} \in \trans{\xisctx{\Ga, x:A}}$,
%     and then $\isterm{\Gb,x:\Ab}{\Bb}{s'}$.
%     We thus conclude
%     $\isterm{\Gb}{\Prod{x:\Ab} \Bb}{s''} \in
%     \trans{\xisterm{\Ga}{\Prod{x:A} B}{s''}}$.

%     \item \textsc{Sigma}
%     \[
%       \infer[]
%         {\xisterm{\Ga}{A}{s} \\
%          \xisterm{\Ga,x:A}{B}{s'}
%         }
%         {\xisterm{\Ga}{\Sum{x:A} B}{s''}}
%       (s,s',s'')
%     \]
%     Similar.

%     \item \textsc{Equality}
%     \[
%       \infer[]
%         {\xisterm{\Ga}{A}{s} \\
%          \xisterm{\Ga}{u}{A} \\
%          \xisterm{\Ga}{v}{A}
%         }
%         {\xisterm{\Ga}{\Eq{A}{u}{v}}{s}}
%       %
%     \]
%     By IH and lemma~\ref{lem:choose} we have $\isterm{\Gb}{\Ab}{s}$,
%     and---using lemma~\ref{lem:change-type}---we also have
%     $\isterm{\Gb}{\ub}{\Ab}$ and $\isterm{\Gb}{\vb}{\Ab}$.
%     Then
%     $\isterm{\Gb}{\Eq{\Ab}{\ub}{\vb}}{s} \in
%     \trans{\xisterm{\Ga}{\Eq{A}{u}{v}}{s}}$.

%     \item \textsc{Variable}
%     \[
%       \infer[]
%         {\xisctx{\Ga} \\
%          (x : A) \in \Ga
%         }
%         {\xisterm{\Ga}{x}{A}}
%       %
%     \]
%     We have $\isctx{\Gb} \in \trans{\xisctx{\Ga}}$ (as we assumed, this is not
%     an instance of the induction hypothesis) and $(x : A) \in \Ga$.
%     By definition of $\Ga \ir \Gb$ we also have some $(x : \Ab) \in \Gb$
%     with $A \ir \Ab$, thus
%     $\isterm{\Gb}{x}{\Ab} \in \trans{\xisterm{\Ga}{x}{A}}$.

%     \item \textsc{Conversion}
%     \[
%       \infer[]
%         {\xisterm{\Ga}{u}{A} \\
%          \xeqtype{\Ga}{A}{B}
%         }
%         {\xisterm{\Ga}{u}{B}}
%       %
%     \]
%     By IH and lemma~\ref{lem:uip-cong} we have
%     $\isterm{\Gb}{\eb}{\Eq{}{\Ab}{\Bb}}$ which implies
%     $\istype{\Gb}{\Ab} \in \trans{\xistype{\Ga}{A}}$ by
%     inversion~\eqref{lem:inversion}, thus, from lemma~\ref{lem:change-type}
%     and IH we get $\isterm{\Gb}{\ub}{\Ab}$, yielding
%     $\isterm{\Gb}{\transpo{\eb}\ \ub}{\Bb} \in \trans{\xisterm{\Ga}{u}{B}}$.

%     \item \textsc{Lambda}
%     \[
%       \infer[]
%         {\xisterm{\Ga}{A}{s} \\
%          \xisterm{\Ga,x:A}{B}{s'} \\
%          \xisterm{\Ga,x:A}{t}{B}
%         }
%         {\xisterm{\Ga}{\lam{x:A}{B} t}{\Prod{x:A} B}}
%       %
%     \]
%     By IH and lemma~\ref{lem:choose} we have $\isterm{\Gb}{\Ab}{s}$ and thus
%     $\isctx{\Gb, x:\Ab} \in \trans{\xisctx{\Ga,x:A}}$, meaning we can apply IH
%     and lemma~\ref{lem:choose} to the second hypothesis to get
%     $\isterm{\Gb,x:\Ab}{\Bb}{s'} \in \trans{\xisterm{\Ga,x:A}{B}{s'}}$ and then
%     IH and lemma~\ref{lem:change-type} to get
%     $\isterm{\Gb,x:\Ab}{\tb}{\Bb} \in \trans{\xisterm{\Ga,x:A}{t}{B}}$.
%     All of this yields
%     $\isterm{\Gb}{\lam{x:\Ab}{\Bb} \tb}{\Prod{x:\Ab} \Bb}
%     \in \trans{\xisterm{\Ga}{\lam{x:A}{B} t}{\Prod{x:A} B}}$.

%     \item \textsc{Application}
%     \[
%       \infer[]
%         {\xisterm{\Ga}{A}{s} \\
%          \xisterm{\Ga,x:A}{B}{s'} \\
%          \xisterm{\Ga}{t}{\Prod{x:A} B} \\
%          \xisterm{\Ga}{u}{A}
%         }
%         {\xisterm{\Ga}{\app{t}{x:A}{B}{u}}{B[x \sto u]}}
%       %
%     \]
%     Using IH together with lemmata~\ref{lem:choose} and~\ref{lem:change-type}
%     we get $\isterm{\Gb}{\Ab}{s}$ and $\isterm{\Gb,x:\Ab}{\Bb}{s'}$ and
%     $\isterm{\Gb}{\tb}{\Prod{x:\Ab} \Bb}$ and $\isterm{\Gb}{\ub}{\Ab}$
%     meaning we can conclude
%     $\isterm{\Gb}{\app{\tb}{x:\Ab}{\Bb}{\ub}}{\Bb[x \sto \ub]}
%     \in \trans{\xisterm{\Ga}{\app{t}{x:A}{B}{u}}{B[x \sto u]}}$.

%     \item \textsc{Pair}
%     \[
%       \infer[]
%         {\xisterm{\Ga}{u}{A} \\
%          \xisterm{\Ga}{A}{s} \\
%          \xisterm{\Ga,x:A}{B}{s'} \\
%          \xisterm{\Ga}{v}{B[x \sto u]}
%         }
%         {\xisterm{\Ga}{\pair{x:A}{B}{u}{v}}{\Sum{x:A}{B}}}
%       %
%     \]
%     Using IH with lemmata~\ref{lem:choose} and~\ref{lem:change-type} we
%     translate all the hypotheses to conclude
%     $\isterm{\Gb}{\pair{x:\Ab}{\Bb}{\ub}{\vb}}{\Sum{x:\Ab}{\Bb}}
%     \in \trans{\xisterm{\Ga}{\pair{x:A}{B}{u}{v}}{\Sum{x:A}{B}}}$.

%     \item \textsc{Proj$_1$}
%     \[
%       \infer[]
%         {\xisterm{\Ga}{p}{\Sum{x:A}{B}}}
%         {\xisterm{\Ga}{\pio{x:A}{B}{p}}{A}}
%       %
%     \]
%     Similar.

%     \item \textsc{Proj$_2$}
%     \[
%       \infer[]
%         {\xisterm{\Ga}{p}{\Sum{x:A}{B}}}
%         {\xisterm{\Ga}{\pit{x:A}{B}{p}}{B[x \sto \pio{x:A}{B}{p}]}}
%       %
%     \]
%     Similar.

%     \item \textsc{Reflexivity}
%     \[
%       \infer[]
%         {\xisterm{\Ga}{A}{s} \\
%          \xisterm{\Ga}{u}{A}
%         }
%         {\xisterm{\Ga}{\refl{A} u}{\Eq{A}{u}{u}}}
%       %
%     \]
%     By IH we have $\isterm{\Gb}{\ub}{\Ab}$ and thus
%     $\isterm{\Gb}{\refl{\Ab} \ub}{\Eq{\Ab}{\ub}{\ub}} \in
%     \trans{\xisterm{\Ga}{\refl{A} u}{\Eq{A}{u}{u}}}$.

%     \item \textsc{J}
%     \[
%       \infer[]
%         {\xisterm{\Ga}{A}{s} \\
%          \xisterm{\Ga}{u,v}{A} \\
%          \xisterm{\Ga, x:A, e:\Eq{A}{u}{x}}{P}{s'} \\
%          \xisterm{\Ga}{p}{\Eq{A}{u}{v}} \\
%          \xisterm{\Ga}{w}{P[x \sto u, e \sto \refl{A} u]}
%         }
%         {\xisterm
%           {\Ga}
%           {\J{A}{u}{x.e.P}{w}{v}{p}}
%           {P[x \sto v, e \sto p]}
%         }
%       %
%     \]
%     By IH and lemma~\ref{lem:choose} we have $\isterm{\Gb}{\Ab}{s}$.
%     From this and IH and lemma~\ref{lem:change-type} we have
%     $\isterm{\Gb}{\ub,\vb}{\Ab}$. We can thus deduce
%     $\isctx{\Gb,x:\Ab,e:\Eq{\Ab}{\ub}{x}} \in \trans{\Ga, x:A, e:\Eq{A}{u}{x}}$
%     which in turn gives us
%     $\isterm{\Gb, x:\Ab,e:\Eq{\Ab}{\ub}{x}}{\Pb}{s'}$.
%     Similarly we also get $\isterm{\Gb}{\pb}{\Eq{\Ab}{\ub}{\vb}}$ and
%     $\isterm{\Gb}{\wb}{\Pb[x \sto \ub, e \sto \refl{\Ab} \ub]}$.
%     All of this allows us to conclude
%     $\isterm
%       {\Gb}
%       {\J{\Ab}{\ub}{x.e.\Pb}{\wb}{\vb}{\pb}}
%       {\Pb[x \sto \vb, e \sto \pb]}
%     \in
%     \trans{
%       \xisterm
%         {\Ga}
%         {\J{A}{u}{x.e.P}{w}{v}{p}}
%         {P[x \sto v, e \sto p]}
%     }$.

%     \item \textsc{Funext}
%     \[
%       \infer[]
%         {\isterm{\Ga}{f,g}{\Prod{x:A} B} \\
%          \isterm
%            {\Ga}
%            {e}
%            {\Prod{x:A} \Eq{B}{\app{f}{x:A}{B}{x}}{\app{g}{x:A}{B}{x}}}
%         }
%         {\isterm{\Ga}{\funext{x:A}{B}{f}{g}{e}}{\Eq{}{f}{g}}}
%       %
%     \]
%     Similar.

%     \item \textsc{UIP}
%     \[
%       \infer[]
%         {\xisterm{\Ga}{e_1,e_2}{\Eq{A}{u}{v}}}
%         {\xisterm{\Ga}{\uip{A}{u}{v}{e_1}{e_2}}{\Eq{}{e_1}{e_2}}}
%       %
%     \]
%     Similar.

%     \item \textsc{Beta}
%     \[
%       \infer[]
%         {\xisterm{\Ga}{A}{s} \\
%          \xisterm{\Ga,x:A}{B}{s'} \\
%          \xisterm{\Ga,x:A}{t}{B} \\
%          \xisterm{\Ga}{u}{A}
%         }
%         {\xeqterm
%           {\Ga}
%           {\app
%             {(\lam{x:A}{B} t)}
%             {x:A}
%             {B}
%             {u}
%           }
%           {t[x \sto u]}
%           {B[x \sto u]}
%         }
%       %
%     \]
%     From IH and the lemmata, we even get the conversion, we conclude using
%     reflexivity.

%     \item \textsc{Proj$_1$-Red}
%     \[
%       \infer[]
%         {\xisterm{\Ga}{A}{s} \\
%          \xisterm{\Ga}{u}{A} \\
%          \xisterm{\Ga,x:A}{B}{s'} \\
%          \xisterm{\Ga}{v}{B[x \sto u]}
%         }
%         {\xeqterm
%           {\Ga}
%           {\pio{x:A}{B}{\pair{x:A}{B}{u}{v}}}
%           {u}
%           {A}
%         }
%       %
%     \]
%     Likewise.

%     \item \textsc{Proj$_2$-Red}
%     \[
%       \infer[]
%         {\xisterm{\Ga}{A}{s} \\
%          \xisterm{\Ga}{u}{A} \\
%          \xisterm{\Ga,x:A}{B}{s'} \\
%          \xisterm{\Ga}{v}{B[x \sto u]}
%         }
%         {\xeqterm
%           {\Ga}
%           {\pit{x:A}{B}{\pair{x:A}{B}{u}{v}}}
%           {v}
%           {B[x \sto u]}
%         }
%       %
%     \]
%     Likewise.

%     \item \textsc{J-Red}
%     \[
%       \infer[]
%         {\xisterm{\Ga}{A}{\Un{i}} \\
%          \xisterm{\Ga}{u}{A} \\
%          \xisterm{\Ga, x:A, e:\Eq{A}{u}{x}}{P}{\Un{j}} \\
%          \xisterm{\Ga}{w}{P[x \sto u, e \sto \refl{A} u]}
%         }
%         {\xeqterm
%           {\Ga}
%           {\J{A}{u}{x.e.P}{w}{u}{\refl{A} u}}
%           {w}
%           {P[x \sto u, e \sto \refl{A} u]}
%         }
%       %
%     \]
%     Likewise.

%     \item \textsc{Conv-Refl}
%     \[
%       \infer[]
%         {\xisterm{\Ga}{u}{A}}
%         {\xeqterm{\Ga}{u}{u}{A}}
%       %
%     \]
%     We conclude from IH and reflexivity of $\Heqs$.

%     \item \textsc{Conv-Sym}
%     \[
%       \infer[]
%         {\xeqterm{\Ga}{u}{v}{A}}
%         {\xeqterm{\Ga}{v}{u}{A}}
%       %
%     \]
%     We conclude from IH and symmetry of $\Heqs$.

%     \item \textsc{Conv-Trans}
%     \[
%       \infer[]
%         {\xeqterm{\Ga}{u}{v}{A} \\
%          \xeqterm{\Ga}{v}{w}{A}
%         }
%         {\xeqterm{\Ga}{u}{w}{A}}
%       %
%     \]
%     We conclude from IH and transitivity of $\Heqs$.

%     % \item \textsc{}
%     % \[
%     %   \infer[]
%     %     {\xisterm{\Ga}{A}{s} \\
%     %      \xisterm{\Ga, x:A}{B}{s'} \\
%     %      \xisterm{\Ga}{f}{\Prod{x:A} B}
%     %     }
%     %     {\xeqterm
%     %       {\Ga}
%     %       {\lam{x:A}{B} \app{f}{x:A}{B}{x}}
%     %       {f}
%     %       {\Prod{x:A} B}
%     %     }
%     %   %
%     % \]
%     % This works like the computation cases.

%     \item \textsc{Conv-Conv}
%     \[
%       \infer[]
%         {\xeqterm{\Ga}{t_1}{t_2}{T_1} \\
%          \xeqtype{\Ga}{T_1}{T_2}
%         }
%         {\xeqterm{\Ga}{t_1}{t_2}{T_2}}
%       %
%     \]
%     By IH (and lemma~\ref{lem:uip-cong}) we have
%     $\isterm{\Gb}{\eb}{\Heq{\Tb_1}{\tb_1}{\Tb'_1}{\tb_2}}$ and
%     $\isterm{\Gb}{p}{\Eq{}{\Tb''_1}{\Tb_2}}$. Also from
%     lemmata~\ref{lem:sim-cong} and~\ref{lem:uip-cong} we have
%     $\Eq{}{\Tb'_1}{\Tb''_1}$ and $\Eq{}{\Tb_1}{\Tb''_1}$, meaning we get
%     $\Eq{}{\Tb'_1}{\Tb_2}$ and $\Eq{}{\Tb_1}{\Tb_2}$.
%     This allows us to conclude by transporting along the aforementioned
%     equalities.

%     \item \textsc{Conv-Prod}
%     \[
%       \infer[]
%         {\xeqterm{\Ga}{A_1}{A_2}{s} \\
%          \xeqterm{\Ga,x:A_1}{B_1}{B_2}{s'}
%         }
%         {\xeqterm{\Ga}{\Prod{x:A_1} B_1}{\Prod{x:A_2} B_2}{s''}}
%       (s,s',s'')
%     \]
%     We conclude exactly like we did in the proof of lemma~\ref{lem:sim-cong}.

%     \item All congruences hold like in proof of
%     lemma~\ref{lem:sim-cong}.

%     \item \textsc{Conv-Eq}
%     \[
%       \infer[]
%         {\xisterm{\Ga}{e}{\Eq{A}{u}{v}}}
%         {\xeqterm{\Ga}{u}{v}{A}}
%       %
%     \]
%     By IH and lemma~\ref{lem:choose} we have
%     $\isterm{\Gb}{\eb}{\Eq{\Ab}{\ub}{\vb}}
%     \in \trans{\xisterm{\Ga}{e}{\Eq{A}{u}{v}}}$ which yields the conclusion we
%     wanted.
%   \end{itemize}
% \end{proof}

\appendix % From here onwards, chapters are numbered with letters, as is the appendix convention

\pagelayout{wide} % No margins
\addpart{Appendix}
\pagelayout{margin} % Restore margins

\setchapterpreamble[u]{\margintoc}
\chapter{CPP Appendix (to go away)}
\labch{cpp-old-appendix}

\section{Complementary rules}
\label{sec:more-rules}

\begin{figure*}[htbp]
  \flushleft
  \hrulefill
  \paradot{Equivalence relation}

  \begin{mathpar}
    \infer[]
      {\isterm{\Ga}{u}{A}}
      {\eqterm{\Ga}{u}{u}{A}}
    %

    \infer[]
      {\eqterm{\Ga}{u}{v}{A}}
      {\eqterm{\Ga}{v}{u}{A}}
    %

    \infer[]
      {\eqterm{\Ga}{u}{v}{A} \\
       \eqterm{\Ga}{v}{w}{A}
      }
      {\eqterm{\Ga}{u}{w}{A}}
    %
  \end{mathpar}

  \paradot{Congruence of type constructors}

  \begin{mathpar}
    \infer[]
      {\eqterm{\Ga}{A_1}{A_2}{s} \\
       \eqterm{\Ga,x:A_1}{B_1}{B_2}{s'}
      }
      {\eqterm{\Ga}{\Prod{x:A_1} B_1}{\Prod{x:A_2} B_2}{s''}}
    (s,s',s'')

    \infer[]
      {\eqterm{\Ga}{A_1}{A_2}{s} \\
       \eqterm{\Ga,x:A_1}{B_1}{B_2}{s'}
      }
      {\eqterm{\Ga}{\Sum{x:A_1} B_1}{\Sum{x:A_2} B_2}{s''}}
    (s,s',s'')

    \infer[]
      {\eqterm{\Ga}{A_1}{A_2}{s} \\
       \eqterm{\Ga}{u_1}{u_2}{A_1} \\
       \eqterm{\Ga}{v_1}{v_2}{A_1}
      }
      {\eqterm{\Ga}{\Eq{A_1}{u_1}{v_1}}{\Eq{A_2}{u_2}{v_2}}{s}}
    %
  \end{mathpar}

  \paradot{Congruence of $\lambda$-calculus terms}

  \begin{mathpar}
    \infer[]
      {\eqterm{\Ga}{A_1}{A_2}{s} \\
       \eqterm{\Ga,x:A_1}{B_1}{B_2}{s'} \\
       \eqterm{\Ga,x:A_1}{t_1}{t_2}{B_1}
      }
      {\eqterm{\Ga}{\lam{x:A_1}{B_1} t_1}{\lam{x:A_2}{B_2} t_2}{\Prod{x:A_1} B_1}}
    %

    \infer[]
      {\eqterm{\Ga}{A_1}{A_2}{s} \\
       \eqterm{\Ga,x:A_1}{B_1}{B_2}{s'} \\
       \eqterm{\Ga}{t_1}{t_2}{\Prod{x:A_1} B_1} \\
       \eqterm{\Ga}{u_1}{u_2}{A_1}
      }
      {\eqterm
        {\Ga}
        {\app{t_1}{x:A_1}{B_1}{u_1}}
        {\app{t_1}{x:A_1}{B_1}{u_1}}
        {B_1[x \sto u_1]}
      }
    %

    \infer[]
      {\eqterm{\Ga}{A_1}{A_2}{s} \\
       \eqterm{\Ga}{u_1}{u_2}{A_1} \\
       \eqterm{\Ga,x:A_1}{B_1}{B_2}{s'} \\
       \eqterm{\Ga}{v_1}{v_2}{B_1[x \sto u_1]}
      }
      {\eqterm
        {\Ga}
        {\pair{x:A_1}{B_1}{u_1}{v_1}}
        {\pair{x:A_2}{B_2}{u_2}{v_2}}
        {\Sum{x:A_1}{B_1}}
      }
    %

    \infer[]
      {\eqterm{\Ga}{A_1}{A_2}{s} \\
       \eqterm{\Ga,x:A_1}{B_1}{B_2}{s'} \\
       \eqterm{\Ga}{p_1}{p_2}{\Sum{x:A_1}{B_1}}
      }
      {\eqterm{\Ga}{\pio{x:A_1}{B_1}{p_1}}{\pio{x:A_2}{B_2}{p_2}}{A_1}}
    %

    \infer[]
      {\eqterm{\Ga}{A_1}{A_2}{s} \\
       \eqterm{\Ga,x:A_1}{B_1}{B_2}{s'} \\
       \eqterm{\Ga}{p_1}{p_2}{\Sum{x:A_1}{B_1}}
      }
      {\eqterm
        {\Ga}
        {\pit{x:A_1}{B_1}{p_1}}
        {\pit{x:A_2}{B_2}{p_2}}
        {B_1[x \sto \pio{x:A_1}{B_1}{p_1}]}
      }
    %
  \end{mathpar}

  \paradot{Congruence of equality terms}

  \begin{mathpar}
    \infer[]
      {\eqterm{\Ga}{A_1}{A_2}{s} \\
       \eqterm{\Ga}{u_1}{u_2}{A}
      }
      {\eqterm{\Ga}{\refl{A_1} u_1}{\refl{A_2} u_2}{\Eq{A_1}{u_1}{u_1}}}
    %

    \infer[]
      {\eqterm{\Ga}{A_1}{A_2}{s} \\
       \eqterm{\Ga}{u_1}{u_2}{A_1} \\
       \eqterm{\Ga}{v_1}{v_2}{A_1} \\
       \eqterm{\Ga, x:A_1, e:\Eq{A_1}{u_1}{x}}{P_1}{P_2}{s'} \\
       \eqterm{\Ga}{p_1}{p_2}{\Eq{A_1}{u_1}{v_1}} \\
       \eqterm{\Ga}{w_1}{w_2}{P_1[x \sto u_1, e \sto \refl{A_1} u_1]}
      }
      {\eqterm
        {\Ga}
        {\J{A_1}{u_1}{x.e.P_1}{w_1}{v_1}{p_1}}
        {\J{A_2}{u_2}{x.e.P_2}{w_2}{v_2}{p_2}}
        {P[x \sto v_1, e \sto p_1]}
      }
    %

    \infer[]
      {\eqterm{\Ga}{A_1}{A_2}{s} \\
       \eqterm{\Ga,x:A_1}{B_1}{B_2}{s'} \\
       \eqterm{\Ga}{f_1}{f_2}{\Prod{x:A_1} B_1} \\
       \eqterm{\Ga}{g_1}{g_2}{\Prod{x:A_1} B_1} \\
       \eqterm
         {\Ga}
         {e_1}
         {e_2}
         {\Prod{x:A_1}
          \Eq{B_1}{\app{f_1}{x:A_1}{B_1}{x}}{\app{g_1}{x:A_1}{B_1}{x}}}
      }
      {\eqterm
        {\Ga}
        {\funext{x:A_1}{B_1}{f_1}{g_1}{e_1}}
        {\funext{x:A_2}{B_2}{f_2}{g_2}{e_2}}
        {\Eq{}{f_1}{g_1}}
      }
    %

    \infer[]
      {\eqtype{\Ga}{A_1}{A_2} \\
       \eqterm{\Ga}{u_1}{u_2}{A_1} \\
       \eqterm{\Ga}{v_1}{v_2}{A_2} \\
       \eqterm{\Ga}{p_1}{p_2}{\Eq{A_1}{u_1}{v_1}} \\
       \eqterm{\Ga}{q_1}{q_2}{\Eq{A_1}{u_1}{v_1}} \\
      }
      {\eqterm
        {\Ga}
        {\uip{A_1}{u_1}{v_1}{p_1}{q_1}}
        {\uip{A_2}{u_2}{v_2}{p_2}{q_2}}
        {\Eq{}{p_1}{q_1}}
      }
    %
  \end{mathpar}
  \hrulefill
  \vspace{-2ex}
  \caption{Congruence rules}
  \label{fig:cong-rules}
\end{figure*}



\section{Proof of the fundamental lemma}
\label{sec:proof-fund-lemma}

\begin{lemma}[Fundamental lemma]
  Let $t_1$ and $t_2$ be two terms. If $\isterm{\Ga, \Ga_1}{t_1}{T_1}$ and
  $\isterm{\Ga, \Ga_2}{t_2}{T_2}$ and $t_1 \sim t_2$ then there exists $p$ such
  that
  $\isterm{\Ga, \Pack{\Ga_1}{\Ga_2}}
          {p}
          {\Heq{\llift{\gamma}{}{T_1}}
               {\llift{\gamma}{}{t_1}}
               {\rlift{\gamma}{}{T_2}}
               {\rlift{\gamma}{}{t_2}}}$.
\end{lemma}

For readability we will abbreviate the left and right substitutions
$\llift{\gamma}{}{\_}$ and $\rlift{\gamma}{}{\_}$ by $\upharpoonleft$
and $\upharpoonright$ respectively.

\begin{proof}
  We prove it by induction on the derivation of $t_1 \sim t_2$.

  \begin{itemize}
    \item \textsc{Var}
    \[
      \infer[]
        { }
        {x \sim x}
      %
    \]
    If $x$ belongs to $\Ga$, we apply reflexivity---together with uniqueness of
    typing~\eqref{lem:uniq}---to conclude.
    Otherwise, $\ProjE{x}$ has the expected type (since
    $\llift{\gamma}{}{x} \equiv \ProjO{x}$ and $\rlift{\gamma}{}{x} \equiv \ProjT{x}$).

    \item \textsc{TransportLeft}
    \[
      \infer[]
        {t_1 \sim t_2}
        {\transpo{p}\ t_1 \sim t_2}
      %
    \]
    We have $\isterm{\Ga, \Ga_1}{\transpo{p}\ t_1}{T_1}$ and
    $\isterm{\Ga, \Ga_2}{t_2}{T_2}$.
    By inversion~\eqref{lem:inversion} we have
    $\isterm{\Ga, \Ga_1}{p}{\Eq{}{T_1'}{T_1}}$ and
    $\isterm{\Ga, \Ga_1}{t_1}{T_1'}$.
    Then by induction hypothesis we have $e$ such that
    $\isterm{\Ga, \Gp}{e}{\Heq{}{\lo{t_1}}{}{\ro{t_2}}}$.
    From transitivity and symmetry we only need to provide a proof of
    $\Heq{}{\lo{t_1}}{}{\transpo{\lo{p}}\ \lo{t_1}}$ which is inhabited by
    $\pair{\_}{\_}{\lo{p}}{\refl{} (\transpo{\lo{p}}\ \lo{t_1})}$.

    \item \textsc{TransportRight}
    \[
      \infer[]
        {t_1 \sim t_2}
        {t_1 \sim \transpo{p}\ t_2}
      %
    \]
    Similarly.

    \item \textsc{Product}
    \[
      \infer[]
        {A_1 \sim A_2 \\
         B_1 \sim B_2
        }
        {\Prod{x:A_1} B_1 \sim \Prod{x:A_2} B_2}
      %
    \]
    We have $\isterm{\Ga, \Ga_1}{\Prod{x:A_1} B_1}{T_1}$ and
    $\isterm{\Ga, \Ga_2}{\Prod{x:A_2} B_2}{T_2}$ so by
    inversion~\eqref{lem:inversion} we have $\isterm{\Ga, \Ga_1}{A_1}{s_1}$ and
    $\isterm{\Ga, \Ga_1, x:A_1}{B_1}{s'_1}$ and
    $\eqtype{\Ga, \Ga_1}{s''_1}{T_1}$ for $(s_1,s'_1,s''_1) \in \Rl$
    (and similarly with $2$s).
    By induction hypothesis we have
    $\isterm{\Ga, \Gp}{p_A}{\Heq{}{\lo{A_1}}{}{\ro{A_2}}}$ and
    $\isterm
      {\Ga, \Gp, x : \Pack{A_1}{A_2}}
      {p_B}
      {\Heq{}{\lo{B_1}}{}{\ro{B_2}}}
    $
    hence the result (using UIP and FunExt, refer to the
    formalisation and especially to the file \rpath{Quotes.v} for more details
    on how to realise this equality).

    \item \textsc{Equality}
    \[
      \infer[]
        {A_1 \sim A_2 \\
         u_1 \sim u_2 \\
         v_1 \sim v_2
        }
        {\Eq{A_1}{u_1}{v_1} \sim \Eq{A_2}{u_2}{v_2}}
      %
    \]
    We have $\isterm{\Ga, \Ga_1}{\Eq{A_1}{u_1}{v_1}}{T_1}$ and
    $\isterm{\Ga, \Ga_2}{\Eq{A_2}{u_2}{v_2}}{T_2}$ so, by
    inversion~\eqref{lem:inversion}, we have
    $\isterm{\Ga, \Ga_1}{A_1}{s_1}$ and $\isterm{\Ga, \Ga_1}{u_1}{A_1}$ and
    $\isterm{\Ga, \Ga_1}{v_1}{A_1}$ as well as $\eqtype{\Ga, \Ga_1}{s_1}{T_1}$
    (and the same with $2$s). By induction hypothesis we thus have
    $\isterm{\Ga, \Gp}{p_{A}}{\Heq{}{A_1}{}{A_2}}$ and
    $\isterm{\Ga, \Gp}{p_u}{\Heq{}{u_1}{}{u_2}}$ and
    $\isterm{\Ga, \Gp}{p_v}{\Heq{}{v_1}{}{v_2}}$.
    We can thus conclude.

    \item \textsc{Reflexivity}
    \[
      \infer[]
        { }
        {s \sim s}
      %
    \]
    This one holds by reflexivity and uniqueness of typing~\eqref{lem:uniq}
    (indeed, $\lo{s} \equiv s$ and $\ro{s} \equiv s$).

    \item \textsc{Lambda}
    \[
      \infer[]
        {A_1 \sim A_2 \\
         B_1 \sim B_2 \\
         t_1 \sim t_2
        }
        {\lam{x:A_1}{B_1} t_1 \sim \lam{x:A_2}{B_2} t_2}
      %
    \]
    We have $\isterm{\Ga, \Ga_1}{\lam{x:A_1}{B_1} t_1}{T_1}$ and
    $\isterm{\Ga, \Ga_2}{\lam{x:A_2}{B_2} t_2}{T_2}$, thus, by
    inversion~\ref{lem:inversion} the subterms are well-typed and we can
    apply induction hypothesis. The conclusion follows similarly to the $\Pi$
    case.

    \item \textsc{Application}
    \[
      \infer[]
        {t_1 \sim t_2 \\
         A_1 \sim A_2 \\
         B_1 \sim B_2 \\
         u_1 \sim u_2
        }
        {\app{t_1}{x:A_1}{B_1}{u_1} \sim \app{t_2}{x:A_2}{B_2}{u_2}}
      %
    \]
    We have $\isterm{\Ga, \Ga_1}{\app{t_1}{x:A_1}{B_1}{u_1}}{T_1}$ and
    $\isterm{\Ga, \Ga_2}{\app{t_2}{x:A_2}{B_2}{u_2}}{T_2}$ which means by
    inversion~\eqref{lem:inversion} that the subterms are well-typed.
    We apply the induction hypothesis and then conclude.

    \item \textsc{Reflexivity}
    \[
      \infer[]
        {A_1 \sim A_2 \\
         u_1 \sim u_2
        }
        {\refl{A_1} u_1 \sim \refl{A_2} u_2}
      %
    \]
    We have $\isterm{\Ga, \Ga_1}{\refl{A_1} u_1}{T_1}$ and
    $\isterm{\Ga, \Ga_2}{\refl{A_2} u_2}{T_2}$ so by
    inversion~\eqref{lem:inversion} we have $\isterm{\Ga, \Ga_1}{A_1}{s_1}$ and
    $\isterm{\Ga, \Ga_1}{u_1}{A_1}$
    (same with $2$s). By IH we have $\Heq{}{\lo{A_1}}{}{\ro{A_2}}$ and
    $\Heq{\lo{A_1}}{\lo{u_1}}{\ro{A_2}}{\ro{u_2}}$.
    The proof follows easily.

    \item \textsc{Funext}
    \[
      \infer[]
        {A_1 \sim A_2 \\
         B_1 \sim B_2 \\
         f_1 \sim f_2 \\
         g_1 \sim g_2 \\
         e_1 \sim e_2
        }
        {\funext{x:A_1}{B_1}{f_1}{g_1}{e_1}
         \sim \funext{x:A_2}{B_2}{f_2}{g_2}{e_2}
        }
      %
    \]
    Similar.

    \item \textsc{UIP}
    \[
      \infer[]
        {A_1 \sim A_2 \\
         u_1 \sim u_2 \\
         v_1 \sim v_2 \\
         p_1 \sim p_2 \\
         q_1 \sim q_2
        }
        {\uip{A_1}{u_1}{v_1}{p_1}{q_1} \sim \uip{A_2}{u_2}{v_2}{p_2}{q_2}}
      %
    \]
    Similar.

    \item \textsc{J}
    \[
      \infer[]
        {A_1 \sim A_2 \\
         u_1 \sim u_2 \\
         P_1 \sim P_2 \\
         w_1 \sim w_2 \\
         v_1 \sim v_2 \\
         p_1 \sim p_2
        }
        {\J{A_1}{u_1}{x.e.P_1}{w_1}{v_1}{p_1} \sim
         \J{A_2}{u_2}{x.e.P_2}{w_2}{v_2}{p_2}
        }
      %
    \]
    Similar.
  \end{itemize}
\end{proof}

\section{Correctness of the translation}
\label{sec:corr-transl}
\begin{theorem}[Translation]
  \leavevmode
  \begin{itemize}
    \item If $\xisctx{\Ga}$ then there exists
    $\isctx{\Gb} \in \transl{\xisctx{\Ga}}$,

    \item If $\xisterm{\Ga}{t}{T}$ then for any
    $\isctx{\Gb} \in \transl{\xisctx{\Ga}}$ there exist $\tb$ and $\Tb$ such that
    $\isterm{\Gb}{\tb}{\Tb} \in \transl{\xisterm{\Ga}{t}{T}}$,

    \item If $\xeqterm{\Ga}{u}{v}{A}$ then for any
    $\isctx{\Gb} \in \transl{\xisctx{\Ga}}$ there exist
    $A \ir \Ab, A \ir \Ab', u \ir \ub, v \ir \vb$ and $\eb$ such that
    $\isterm{\Gb}{\eb}{\Heq{\Ab}{\ub}{\Ab'}{\vb}}$.
  \end{itemize}
\end{theorem}

\begin{proof}
  We prove the theorem by induction on the derivation in the extensional
  type theory. In most cases we need to assume some $\Gb$, translation of the
  context, we will implicitly refer to $\Gb$ in such cases as the one given as
  hypothesis.
  \leavevmode
  \begin{itemize}
    \item \textsc{Empty}
    \[
      \infer[]
        { }
        {\xisctx{\ctxempty}}
      %
    \]
    We have $\isctx{\ctxempty} \in \transl{\xisctx{\ctxempty}}$.

    \item \textsc{Extend}
    \[
      \infer[]
        {\xisctx{\Ga} \\
         \xistype{\Ga}{A}
        }
        {\xisctx{\Ga, x:A}}
      (x \notin \Ga)
    \]
    By IH we have $\isctx{\Gb} \in \transl{\xisctx{\Ga}}$ and, using $\Gb$
    as well as lemma~\ref{lem:choose},
    $\isterm{\Gb}{\Ab}{s} \in \transl{\xisterm{\Ga}{A}{s}}$.
    Thus $\isctx{\Gb, x:\Ab} \in \transl{\xisctx{\Ga, x:A}}$.

    \item \textsc{Sort}
    \[
      \infer[]
        {\xisctx{\Ga}}
        {\xisterm{\Ga}{s}{s'}}
      (s,s')
    \]
    We have $\isterm{\Gb}{s}{s'} \in \transl{\xisterm{\Ga}{s}{s'}}$.

    \item \textsc{Product}
    \[
      \infer[]
        {\xisterm{\Ga}{A}{s} \\
         \xisterm{\Ga,x:A}{B}{s'}
        }
        {\xisterm{\Ga}{\Prod{x:A} B}{s''}}
      (s,s',s'')
    \]
    By IH and lemma~\ref{lem:choose} we have $\isterm{\Gb}{\Ab}{s}$,
    meaning $\isctx{\Gb,x:\Ab} \in \transl{\xisctx{\Ga, x:A}}$,
    and then $\isterm{\Gb,x:\Ab}{\Bb}{s'}$.
    We thus conclude
    $\isterm{\Gb}{\Prod{x:\Ab} \Bb}{s''} \in
    \transl{\xisterm{\Ga}{\Prod{x:A} B}{s''}}$.

    \item \textsc{Sigma}
    \[
      \infer[]
        {\xisterm{\Ga}{A}{s} \\
         \xisterm{\Ga,x:A}{B}{s'}
        }
        {\xisterm{\Ga}{\Sum{x:A} B}{s''}}
      (s,s',s'')
    \]
    Similar.

    \item \textsc{Equality}
    \[
      \infer[]
        {\xisterm{\Ga}{A}{s} \\
         \xisterm{\Ga}{u}{A} \\
         \xisterm{\Ga}{v}{A}
        }
        {\xisterm{\Ga}{\Eq{A}{u}{v}}{s}}
      %
    \]
    By IH and lemma~\ref{lem:choose} we have $\isterm{\Gb}{\Ab}{s}$,
    and---using lemma~\ref{lem:change-type}---we also have
    $\isterm{\Gb}{\ub}{\Ab}$ and $\isterm{\Gb}{\vb}{\Ab}$.
    Then
    $\isterm{\Gb}{\Eq{\Ab}{\ub}{\vb}}{s} \in
    \transl{\xisterm{\Ga}{\Eq{A}{u}{v}}{s}}$.

    \item \textsc{Variable}
    \[
      \infer[]
        {\xisctx{\Ga} \\
         (x : A) \in \Ga
        }
        {\xisterm{\Ga}{x}{A}}
      %
    \]
    We have $\isctx{\Gb} \in \transl{\xisctx{\Ga}}$ (as we assumed, this is not
    an instance of the induction hypothesis) and $(x : A) \in \Ga$.
    By definition of $\Ga \ir \Gb$ we also have some $(x : \Ab) \in \Gb$
    with $A \ir \Ab$, thus
    $\isterm{\Gb}{x}{\Ab} \in \transl{\xisterm{\Ga}{x}{A}}$.

    \item \textsc{Conversion}
    \[
      \infer[]
        {\xisterm{\Ga}{u}{A} \\
         \xeqtype{\Ga}{A}{B}
        }
        {\xisterm{\Ga}{u}{B}}
      %
    \]
    By IH and lemma~\ref{lem:uip-cong} we have
    $\isterm{\Gb}{\eb}{\Eq{}{\Ab}{\Bb}}$ which implies
    $\istype{\Gb}{\Ab} \in \transl{\xistype{\Ga}{A}}$ by
    inversion~\eqref{lem:inversion}, thus, from lemma~\ref{lem:change-type}
    and IH we get $\isterm{\Gb}{\ub}{\Ab}$, yielding
    $\isterm{\Gb}{\transpo{\eb}\ \ub}{\Bb} \in \transl{\xisterm{\Ga}{u}{B}}$.

    \item \textsc{Lambda}
    \[
      \infer[]
        {\xisterm{\Ga}{A}{s} \\
         \xisterm{\Ga,x:A}{B}{s'} \\
         \xisterm{\Ga,x:A}{t}{B}
        }
        {\xisterm{\Ga}{\lam{x:A}{B} t}{\Prod{x:A} B}}
      %
    \]
    By IH and lemma~\ref{lem:choose} we have $\isterm{\Gb}{\Ab}{s}$ and thus
    $\isctx{\Gb, x:\Ab} \in \transl{\xisctx{\Ga,x:A}}$, meaning we can apply IH
    and lemma~\ref{lem:choose} to the second hypothesis to get
    $\isterm{\Gb,x:\Ab}{\Bb}{s'} \in \transl{\xisterm{\Ga,x:A}{B}{s'}}$ and then
    IH and lemma~\ref{lem:change-type} to get
    $\isterm{\Gb,x:\Ab}{\tb}{\Bb} \in \transl{\xisterm{\Ga,x:A}{t}{B}}$.
    All of this yields
    $\isterm{\Gb}{\lam{x:\Ab}{\Bb} \tb}{\Prod{x:\Ab} \Bb}
    \in \transl{\xisterm{\Ga}{\lam{x:A}{B} t}{\Prod{x:A} B}}$.

    \item \textsc{Application}
    \[
      \infer[]
        {\xisterm{\Ga}{A}{s} \\
         \xisterm{\Ga,x:A}{B}{s'} \\
         \xisterm{\Ga}{t}{\Prod{x:A} B} \\
         \xisterm{\Ga}{u}{A}
        }
        {\xisterm{\Ga}{\app{t}{x:A}{B}{u}}{B[x \sto u]}}
      %
    \]
    Using IH together with lemmata~\ref{lem:choose} and~\ref{lem:change-type}
    we get $\isterm{\Gb}{\Ab}{s}$ and $\isterm{\Gb,x:\Ab}{\Bb}{s'}$ and
    $\isterm{\Gb}{\tb}{\Prod{x:\Ab} \Bb}$ and $\isterm{\Gb}{\ub}{\Ab}$
    meaning we can conclude
    $\isterm{\Gb}{\app{\tb}{x:\Ab}{\Bb}{\ub}}{\Bb[x \sto \ub]}
    \in \transl{\xisterm{\Ga}{\app{t}{x:A}{B}{u}}{B[x \sto u]}}$.

    \item \textsc{Pair}
    \[
      \infer[]
        {\xisterm{\Ga}{u}{A} \\
         \xisterm{\Ga}{A}{s} \\
         \xisterm{\Ga,x:A}{B}{s'} \\
         \xisterm{\Ga}{v}{B[x \sto u]}
        }
        {\xisterm{\Ga}{\pair{x:A}{B}{u}{v}}{\Sum{x:A}{B}}}
      %
    \]
    Using IH with lemmata~\ref{lem:choose} and~\ref{lem:change-type} we
    translate all the hypotheses to conclude
    $\isterm{\Gb}{\pair{x:\Ab}{\Bb}{\ub}{\vb}}{\Sum{x:\Ab}{\Bb}}
    \in \transl{\xisterm{\Ga}{\pair{x:A}{B}{u}{v}}{\Sum{x:A}{B}}}$.

    \item \textsc{Proj$_1$}
    \[
      \infer[]
        {\xisterm{\Ga}{p}{\Sum{x:A}{B}}}
        {\xisterm{\Ga}{\pio{x:A}{B}{p}}{A}}
      %
    \]
    Similar.

    \item \textsc{Proj$_2$}
    \[
      \infer[]
        {\xisterm{\Ga}{p}{\Sum{x:A}{B}}}
        {\xisterm{\Ga}{\pit{x:A}{B}{p}}{B[x \sto \pio{x:A}{B}{p}]}}
      %
    \]
    Similar.

    \item \textsc{Reflexivity}
    \[
      \infer[]
        {\xisterm{\Ga}{A}{s} \\
         \xisterm{\Ga}{u}{A}
        }
        {\xisterm{\Ga}{\refl{A} u}{\Eq{A}{u}{u}}}
      %
    \]
    By IH we have $\isterm{\Gb}{\ub}{\Ab}$ and thus
    $\isterm{\Gb}{\refl{\Ab} \ub}{\Eq{\Ab}{\ub}{\ub}} \in
    \transl{\xisterm{\Ga}{\refl{A} u}{\Eq{A}{u}{u}}}$.

    \item \textsc{J}
    \[
      \infer[]
        {\xisterm{\Ga}{A}{s} \\
         \xisterm{\Ga}{u,v}{A} \\
         \xisterm{\Ga, x:A, e:\Eq{A}{u}{x}}{P}{s'} \\
         \xisterm{\Ga}{p}{\Eq{A}{u}{v}} \\
         \xisterm{\Ga}{w}{P[x \sto u, e \sto \refl{A} u]}
        }
        {\xisterm
          {\Ga}
          {\J{A}{u}{x.e.P}{w}{v}{p}}
          {P[x \sto v, e \sto p]}
        }
      %
    \]
    By IH and lemma~\ref{lem:choose} we have $\isterm{\Gb}{\Ab}{s}$.
    From this and IH and lemma~\ref{lem:change-type} we have
    $\isterm{\Gb}{\ub,\vb}{\Ab}$. We can thus deduce
    $\isctx{\Gb,x:\Ab,e:\Eq{\Ab}{\ub}{x}} \in \transl{\Ga, x:A, e:\Eq{A}{u}{x}}$
    which in turn gives us
    $\isterm{\Gb, x:\Ab,e:\Eq{\Ab}{\ub}{x}}{\Pb}{s'}$.
    Similarly we also get $\isterm{\Gb}{\pb}{\Eq{\Ab}{\ub}{\vb}}$ and
    $\isterm{\Gb}{\wb}{\Pb[x \sto \ub, e \sto \refl{\Ab} \ub]}$.
    All of this allows us to conclude
    $\isterm
      {\Gb}
      {\J{\Ab}{\ub}{x.e.\Pb}{\wb}{\vb}{\pb}}
      {\Pb[x \sto \vb, e \sto \pb]}
    \in
    \transl{
      \xisterm
        {\Ga}
        {\J{A}{u}{x.e.P}{w}{v}{p}}
        {P[x \sto v, e \sto p]}
    }$.

    \item \textsc{Funext}
    \[
      \infer[]
        {\isterm{\Ga}{f,g}{\Prod{x:A} B} \\
         \isterm
           {\Ga}
           {e}
           {\Prod{x:A} \Eq{B}{\app{f}{x:A}{B}{x}}{\app{g}{x:A}{B}{x}}}
        }
        {\isterm{\Ga}{\funext{x:A}{B}{f}{g}{e}}{\Eq{}{f}{g}}}
      %
    \]
    Similar.

    \item \textsc{UIP}
    \[
      \infer[]
        {\xisterm{\Ga}{e_1,e_2}{\Eq{A}{u}{v}}}
        {\xisterm{\Ga}{\uip{A}{u}{v}{e_1}{e_2}}{\Eq{}{e_1}{e_2}}}
      %
    \]
    Similar.

    \item \textsc{Beta}
    \[
      \infer[]
        {\xisterm{\Ga}{A}{s} \\
         \xisterm{\Ga,x:A}{B}{s'} \\
         \xisterm{\Ga,x:A}{t}{B} \\
         \xisterm{\Ga}{u}{A}
        }
        {\xeqterm
          {\Ga}
          {\app
            {(\lam{x:A}{B} t)}
            {x:A}
            {B}
            {u}
          }
          {t[x \sto u]}
          {B[x \sto u]}
        }
      %
    \]
    From IH and the lemmata, we even get the conversion, we conclude using
    reflexivity.

    \item \textsc{Proj$_1$-Red}
    \[
      \infer[]
        {\xisterm{\Ga}{A}{s} \\
         \xisterm{\Ga}{u}{A} \\
         \xisterm{\Ga,x:A}{B}{s'} \\
         \xisterm{\Ga}{v}{B[x \sto u]}
        }
        {\xeqterm
          {\Ga}
          {\pio{x:A}{B}{\pair{x:A}{B}{u}{v}}}
          {u}
          {A}
        }
      %
    \]
    Likewise.

    \item \textsc{Proj$_2$-Red}
    \[
      \infer[]
        {\xisterm{\Ga}{A}{s} \\
         \xisterm{\Ga}{u}{A} \\
         \xisterm{\Ga,x:A}{B}{s'} \\
         \xisterm{\Ga}{v}{B[x \sto u]}
        }
        {\xeqterm
          {\Ga}
          {\pit{x:A}{B}{\pair{x:A}{B}{u}{v}}}
          {v}
          {B[x \sto u]}
        }
      %
    \]
    Likewise.

    \item \textsc{J-Red}
    \[
      \infer[]
        {\xisterm{\Ga}{A}{\Un{i}} \\
         \xisterm{\Ga}{u}{A} \\
         \xisterm{\Ga, x:A, e:\Eq{A}{u}{x}}{P}{\Un{j}} \\
         \xisterm{\Ga}{w}{P[x \sto u, e \sto \refl{A} u]}
        }
        {\xeqterm
          {\Ga}
          {\J{A}{u}{x.e.P}{w}{u}{\refl{A} u}}
          {w}
          {P[x \sto u, e \sto \refl{A} u]}
        }
      %
    \]
    Likewise.

    \item \textsc{Conv-Refl}
    \[
      \infer[]
        {\xisterm{\Ga}{u}{A}}
        {\xeqterm{\Ga}{u}{u}{A}}
      %
    \]
    We conclude from IH and reflexivity of $\Heqs$.

    \item \textsc{Conv-Sym}
    \[
      \infer[]
        {\xeqterm{\Ga}{u}{v}{A}}
        {\xeqterm{\Ga}{v}{u}{A}}
      %
    \]
    We conclude from IH and symmetry of $\Heqs$.

    \item \textsc{Conv-Trans}
    \[
      \infer[]
        {\xeqterm{\Ga}{u}{v}{A} \\
         \xeqterm{\Ga}{v}{w}{A}
        }
        {\xeqterm{\Ga}{u}{w}{A}}
      %
    \]
    We conclude from IH and transitivity of $\Heqs$.

    % \item \textsc{}
    % \[
    %   \infer[]
    %     {\xisterm{\Ga}{A}{s} \\
    %      \xisterm{\Ga, x:A}{B}{s'} \\
    %      \xisterm{\Ga}{f}{\Prod{x:A} B}
    %     }
    %     {\xeqterm
    %       {\Ga}
    %       {\lam{x:A}{B} \app{f}{x:A}{B}{x}}
    %       {f}
    %       {\Prod{x:A} B}
    %     }
    %   %
    % \]
    % This works like the computation cases.

    \item \textsc{Conv-Conv}
    \[
      \infer[]
        {\xeqterm{\Ga}{t_1}{t_2}{T_1} \\
         \xeqtype{\Ga}{T_1}{T_2}
        }
        {\xeqterm{\Ga}{t_1}{t_2}{T_2}}
      %
    \]
    By IH (and lemma~\ref{lem:uip-cong}) we have
    $\isterm{\Gb}{\eb}{\Heq{\Tb_1}{\tb_1}{\Tb'_1}{\tb_2}}$ and
    $\isterm{\Gb}{p}{\Eq{}{\Tb''_1}{\Tb_2}}$. Also from
    lemmata~\ref{lem:sim-cong} and~\ref{lem:uip-cong} we have
    $\Eq{}{\Tb'_1}{\Tb''_1}$ and $\Eq{}{\Tb_1}{\Tb''_1}$, meaning we get
    $\Eq{}{\Tb'_1}{\Tb_2}$ and $\Eq{}{\Tb_1}{\Tb_2}$.
    This allows us to conclude by transporting along the aforementioned
    equalities.

    \item \textsc{Conv-Prod}
    \[
      \infer[]
        {\xeqterm{\Ga}{A_1}{A_2}{s} \\
         \xeqterm{\Ga,x:A_1}{B_1}{B_2}{s'}
        }
        {\xeqterm{\Ga}{\Prod{x:A_1} B_1}{\Prod{x:A_2} B_2}{s''}}
      (s,s',s'')
    \]
    We conclude exactly like we did in the proof of lemma~\ref{lem:sim-cong}.

    \item All congruences hold like in proof of
    lemma~\ref{lem:sim-cong}.

    \item \textsc{Conv-Eq}
    \[
      \infer[]
        {\xisterm{\Ga}{e}{\Eq{A}{u}{v}}}
        {\xeqterm{\Ga}{u}{v}{A}}
      %
    \]
    By IH and lemma~\ref{lem:choose} we have
    $\isterm{\Gb}{\eb}{\Eq{\Ab}{\ub}{\vb}}
    \in \transl{\xisterm{\Ga}{e}{\Eq{A}{u}{v}}}$ which yields the conclusion we
    wanted.
  \end{itemize}
\end{proof}

%----------------------------------------------------------------------------------------

\backmatter % Denotes the end of the main document content
\setchapterstyle{plain} % Output plain chapters from this point onwards

%----------------------------------------------------------------------------------------
%	BIBLIOGRAPHY
%----------------------------------------------------------------------------------------

% The bibliography needs to be compiled with biber using your LaTeX editor, or on the command line with 'biber main' from the template directory

\defbibnote{bibnote}{Here are the references in citation order.\par\bigskip} % Prepend this text to the bibliography
\printbibliography[heading=bibintoc, title=Bibliography, prenote=bibnote] % Add the bibliography heading to the ToC, set the title of the bibliography and output the bibliography note

%----------------------------------------------------------------------------------------
%	NOMENCLATURE
%----------------------------------------------------------------------------------------

% The nomenclature needs to be compiled on the command line with 'makeindex main.nlo -s nomencl.ist -o main.nls' from the template directory

\nomenclature{$c$}{Speed of light in a vacuum inertial frame}
\nomenclature{$h$}{Planck constant}

\renewcommand{\nomname}{Notation} % Rename the default 'Nomenclature'
\renewcommand{\nompreamble}{The next list describes several symbols that will be later used within the body of the document.} % Prepend this text to the nomenclature

\printnomenclature % Output the nomenclature

%----------------------------------------------------------------------------------------
%	GREEK ALPHABET
% 	Originally from https://gitlab.com/jim.hefferon/linear-algebra
%----------------------------------------------------------------------------------------

\vspace{1cm}

{\usekomafont{chapter}Greek Letters with Pronounciation} \\[2ex]
\begin{center}
	\newcommand{\pronounced}[1]{\hspace*{.2em}\small\textit{#1}}
	\begin{tabular}{l l @{\hspace*{3em}} l l}
		\toprule
		Character & Name & Character & Name \\
		\midrule
		$\alpha$ & alpha \pronounced{AL-fuh} & $\nu$ & nu \pronounced{NEW} \\
		$\beta$ & beta \pronounced{BAY-tuh} & $\xi$, $\Xi$ & xi \pronounced{KSIGH} \\
		$\gamma$, $\Gamma$ & gamma \pronounced{GAM-muh} & o & omicron \pronounced{OM-uh-CRON} \\
		$\delta$, $\Delta$ & delta \pronounced{DEL-tuh} & $\pi$, $\Pi$ & pi \pronounced{PIE} \\
		$\epsilon$ & epsilon \pronounced{EP-suh-lon} & $\rho$ & rho \pronounced{ROW} \\
		$\zeta$ & zeta \pronounced{ZAY-tuh} & $\sigma$, $\Sigma$ & sigma \pronounced{SIG-muh} \\
		$\eta$ & eta \pronounced{AY-tuh} & $\tau$ & tau \pronounced{TOW (as in cow)} \\
		$\theta$, $\Theta$ & theta \pronounced{THAY-tuh} & $\upsilon$, $\Upsilon$ & upsilon \pronounced{OOP-suh-LON} \\
		$\iota$ & iota \pronounced{eye-OH-tuh} & $\phi$, $\Phi$ & phi \pronounced{FEE, or FI (as in hi)} \\
		$\kappa$ & kappa \pronounced{KAP-uh} & $\chi$ & chi \pronounced{KI (as in hi)} \\
		$\lambda$, $\Lambda$ & lambda \pronounced{LAM-duh} & $\psi$, $\Psi$ & psi \pronounced{SIGH, or PSIGH} \\
		$\mu$ & mu \pronounced{MEW} & $\omega$, $\Omega$ & omega \pronounced{oh-MAY-guh} \\
		\bottomrule
	\end{tabular} \\[1.5ex]
	Capitals shown are the ones that differ from Roman capitals.
\end{center}

%----------------------------------------------------------------------------------------
%	GLOSSARY
%----------------------------------------------------------------------------------------

% The glossary needs to be compiled on the command line with 'makeglossaries main' from the template directory

\newglossaryentry{computer}{
	name=computer,
	description={is a programmable machine that receives input, stores and manipulates data, and provides output in a useful format}
}

% Glossary entries (used in text with e.g. \acrfull{fpsLabel} or \acrshort{fpsLabel})
\newacronym[longplural={Frames per Second}]{fpsLabel}{FPS}{Frame per Second}
\newacronym[longplural={Tables of Contents}]{tocLabel}{TOC}{Table of Contents}

\setglossarystyle{listgroup} % Set the style of the glossary (see https://en.wikibooks.org/wiki/LaTeX/Glossary for a reference)
\printglossary[title=Special Terms, toctitle=List of Terms] % Output the glossary, 'title' is the chapter heading for the glossary, toctitle is the table of contents heading

%----------------------------------------------------------------------------------------
%	INDEX
%----------------------------------------------------------------------------------------

% The index needs to be compiled on the command line with 'makeindex main' from the template directory

\printindex % Output the index

%----------------------------------------------------------------------------------------
%	BACK COVER
%----------------------------------------------------------------------------------------

% If you have a PDF/image file that you want to use as a back cover, uncomment the following lines

%\clearpage
%\thispagestyle{empty}
%\null%
%\clearpage
%\includepdf{cover-back.pdf}

%----------------------------------------------------------------------------------------


\end{document}