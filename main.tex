% !TEX program = xelatex
% !TEX options = -synctex=1 -interaction=nonstopmode -file-line-error --shell-escape "%DOC%"

%----------------------------------------------------------------------------------------
%	PACKAGES AND OTHER DOCUMENT CONFIGURATIONS
%----------------------------------------------------------------------------------------

\documentclass[
  fontsize=10pt,
  twoside=false
  % overfullrule
]{kaobook}

% Choose the language
\usepackage[english]{babel} % Load characters and hyphenation
\usepackage[english=british]{csquotes}	% English quotes

% Load the bibliography package
\usepackage{styles/kaobiblio}
\addbibresource{main.bib} % Bibliography file

% Load mathematical packages for theorems and related environments. NOTE: choose only one between 'mdftheorems' and 'plaintheorems'.
\usepackage{styles/mdftheorems}
%\usepackage{styles/plaintheorems}

\graphicspath{{images/}} % Paths in which to look for images

\makeindex[columns=3, title=Alphabetical Index, intoc] % Make LaTeX produce the files required to compile the index

\makeglossaries % Make LaTeX produce the files required to compile the glossary

\makenomenclature % Make LaTeX produce the files required to compile the nomenclature

% Reset sidenote counter at chapters
%\counterwithin*{sidenote}{chapter}

%%% Then my own packages

\usepackage{fontspec}

\usepackage{mathpartir}

% Trick for current font size
\makeatletter
\newcommand{\currentfontsize}{\fontsize{\f@size}{\f@baselineskip}\selectfont}
\makeatother

% For code higlighting
\usepackage{minted}
\setminted{
	fontsize=\footnotesize,
	encoding=utf8
}
\setmintedinline{fontsize=\currentfontsize}

% HACK: Remove red boxes
% From https://tex.stackexchange.com/questions/343494/minted-red-box-around-greek-characters
\usepackage{etoolbox,xpatch}

\makeatletter
\AtBeginEnvironment{minted}{\dontdofcolorbox}
\def\dontdofcolorbox{\renewcommand\fcolorbox[4][]{##4}}
\xpatchcmd{\inputminted}{\minted@fvset}{\minted@fvset\dontdofcolorbox}{}{}
\xpatchcmd{\mintinline}{\minted@fvset}{\minted@fvset\dontdofcolorbox}{}{}
\makeatother

% Without red boxes: xcode, arduino, abap,
% \usemintedstyle{abap}

\usepackage{stmaryrd}

% Miniature document
% \usepackage{minidocument}
% \def\minidocumentscale{0.25}
% \usepackage{blindtext}

% Tikz
\usetikzlibrary{tikzmark, arrows, arrows.meta, decorations.pathmorphing}

% Tikz styles
\tikzstyle{tred} = [-{Triangle[open]}]
% \tikzstyle{arrow} = [thick,->,>=stealth]
\tikzstyle{barrow} = [->, bend angle=40]

% https://tex.stackexchange.com/questions/88949/curved-waved-lines-with-tikz
\tikzset{sim/.style={decorate, decoration={coil, aspect = 0, segment length = 0.9cm, pre = moveto, post = moveto}}}

% https://tex.stackexchange.com/questions/104119/star-next-to-arrowhead-in-tikz
\tikzset{
    to*/.style = {
        shorten >=.25em,#1-to,
        to path = {-- node[inner sep=0pt,at end,sloped] {${}^\star$} (\tikztotarget) \tikztonodes}
    },
    to*/.default=
}

\tikzset{
    *to/.style = {
        shorten >=.25em,#1-to,
        to path = {-- node[inner sep=0pt,at end,sloped] {${}_\star$} (\tikztotarget) \tikztonodes}
    },
    *to/.default=
}

% TODO ref
\newcommand{\misref}{\textcolor{orange}{\emph{?ref}}}

% Reminder box
\newcommand\reminder[3][0pt]{%
  \marginnote[#1]{
    \begin{kaobox}[
      frametitle=Reminder: #2,
      backgroundcolor=YellowGreen!25!White,
      frametitlebackgroundcolor=YellowGreen!25!White
    ]
      #3
    \end{kaobox}
  }
}

% Highlight for maths
\newcommand{\highlight}[1]{%
  \colorbox{Yellow!50}{$\displaystyle#1$}%
}

% Names
\def\name#1{\textsf{#1}\xspace}
\def\Coq{\name{Coq}}
\def\MetaCoq{\name{MetaCoq}}
\def\TemplateCoq{\name{TemplateCoq}}
\def\Equations{\name{Equations}}
\def\Andromeda{\name{Andromeda}}
\def\NuPRL{\name{NuPRL}}
\def\Agda{\name{Agda}}
\def\Epigram{\name{EPIGRAM}}
%%% General

\newcommand{\inv}{^{-1}}

% \newcommand{\eg}{e.g.\ }
% \newcommand{\ie}{\emph{i.e.,}\xspace}

\newcommand{\paradot}[1]{\paragraph{#1.}}

\newcommand{\exmark}{\ensuremath{\mathsf{x}}}

%% Proof by cases
\newenvironment{caselist}{%
  \begin{list}{{\it Case}}{}%
}{\end{list}%
}
\newenvironment{subcaselist}{%
  \begin{list}{{\it Subcase}}{}%
}{\end{list}%
}
\newenvironment{subsubcaselist}{%
  \begin{list}{{\it Subsubcase}}{}%
}{\end{list}%
}

\newcommand{\nextcase}{\item~}

%%% Type theory

% Meta
\newcommand{\Ax}{\ensuremath{\mathsf{Ax}}}
\newcommand{\Rl}{\ensuremath{\mathsf{R}}}

% Generic entities
\newcommand{\Ga}{\Gamma}   % a context
\newcommand{\D}{\Delta}   % another context
\newcommand{\E}{\Xi}      % another context
\newcommand{\sbs}{\sigma} % a substitution
\newcommand{\sbt}{\theta} % another substitution
\newcommand{\sbr}{\rho}   % a third substitution

%% Syntax
\newcommand{\bnf}{\ \mathrel{{:}{:}{=}}\ }
\newcommand{\bnfor}{\ \mid\ \ }

%% Contexts
\newcommand{\ctxempty}{\bullet} % empty context
% \newcommand{\ctxextend}[2]{#1, #2} % extended context

%\newcommand{\ctxdom}[1]{\mathsf{dom}(#1)} % the domain of a context

%% Substitution

\newcommand{\sto}{\mathop{\leftarrow}}

% Reduction
\newcommand{\red}{\rightarrowtriangle}
\newcommand{\redl}{\leftarrowtriangle}
\newcommand{\reds}{\red^{\star}}

%% Marker
\newcommand{\stt}[1]{#1^{\mathsf{s}}}
\newcommand{\fm}[1]{#1^{\mathsf{f}}}


% Type formers
\newcommand{\Prod}[1]{\mathop{\Pi(#1).\ }}                 % dependent product
\newcommand{\Sum}[1]{\mathop{\Sigma(#1).}}               % dependent sum
\newcommand{\Eq}[3]{#2 =_{#1} #3}                        % fibrant equality
\newcommand{\Eqs}[3]{#2 \overset{\mathsf{s}}{=}_{#1} #3} % strict equality
\newcommand{\Ty}[1]{\square_{#1}}
\newcommand{\Un}[1]{\mathsf{U}_{#1}}                      % strict universe
\newcommand{\F}[1]{\mathsf{F}_{#1}}                      % fibrant universe
\newcommand{\Type}{\mathsf{Type}}
\newcommand{\Prop}{\mathsf{Prop}}
\newcommand{\SProp}{\mathsf{SProp}}
\newcommand{\nat}{\mathsf{nat}}
% \newcommand{\zero}{\mathsf{zero}}
\newcommand{\zero}{0}
% \newcommand{\natsucc}{\mathsf{succ}}
\newcommand{\natsucc}{\mathbf{S}}
\newcommand{\natrec}{\mathsf{natrec}}
\newcommand{\bool}{\mathsf{bool}}
\newcommand{\unit}{\mathsf{unit}}

% Inductive types
\newcommand{\Ical}{\mathcal{I}}
\newcommand{\at}[1]{\{#1\}}

% Terms
\newcommand{\lam}[2]{\lambda (#1). #2 .}                        % abstraction
\newcommand{\app}[4]{#1\mathbin{@_{#2.#3}} #4}                  % application
\newcommand{\pair}[4]{\langle #3 ; #4 \rangle_{#1.#2}}          % pair
\newcommand{\pio}[3]{\pi_1^{#1.#2}\ #3}                         % first proj
\newcommand{\pit}[3]{\pi_2^{#1.#2}\ #3}                         % second proj
\newcommand{\idpath}[1]{{\mathsf{idpath}_{#1}}\ }               % identity path
\newcommand{\refl}[1]{{\mathsf{refl}}_{#1}\ }                   % reflexivity
\newcommand{\funexts}[5]
  {\stt{\mathsf{funext}}(#1,#2,#3,#4,#5)}                        % strict funext
\newcommand{\funext}[5]
  {\mathsf{funext}(#1,#2,#3,#4,#5)}                             % fibrant funext
\newcommand{\uip}[5]{\mathsf{uip}(#1,#2,#3,#4,#5)}              % uip
\newcommand{\J}[6]{\mathsf{J}(#1,#2,#3,#4,#5,#6)}               % the J elim
\newcommand{\Js}[6]{\mathsf{J}^{\mathsf{s}}(#1,#2,#3,#4,#5,#6)} % the J elim
\newcommand{\ttrue}{\mathsf{true}}
\newcommand{\ffalse}{\mathsf{false}}
\newcommand{\tif}[4]{\mathsf{if}\ #1\ \mathsf{return}\ #2\ \mathsf{then}\ #3\ \mathsf{else}\ #4}
\newcommand{\notfunext}{\mathsf{notfunext}}
\newcommand{\Notfunext}{\mathsf{NotFunExt}}
\newcommand{\tunit}{\mathsf{()}}
\newcommand{\inl}{\mathsf{inl}}
\newcommand{\inr}{\mathsf{inr}}
\newcommand{\mto}{\Longrightarrow}
\newcommand{\branch}[2]{\mid #1 &\mto& #2}
\newcommand{\pmatch}[3]{
  \begin{array}{lcl}
    \multicolumn{3}{l}{\mathsf{match}\ #1\ \mathsf{return}\ #2\ \mathsf{with}} \\
    #3 \\
    \multicolumn{3}{l}{\mathsf{end}} \\
  \end{array}
}
\newcommand{\ax}[1]{\mathsf{ax}(#1)}

% Sorts
\newcommand{\succs}[1]{#1+1}
\newcommand{\pisort}[2]{\mathsf{pi}(#1,#2)}

% Pattern-matching (TODO Duplicate...)
\newcommand{\matchs}{\mathsf{match}}
\newcommand{\patt}[2]{\mid #1 \Longrightarrow #2}
\newcommand{\match}[3]{%
  \begin{array}{l}%
    \matchs\ #1 \\ \mathsf{return}\ #2\ \mathsf{with} \\ #3 \\ \mathsf{end}%
  \end{array}%
}

% Vectors
\newcommand{\av}{\vec{a}}
\newcommand{\bv}{\vec{b}}
\newcommand{\iv}{\vec{i}}
\newcommand{\pv}{\vec{p}}
\newcommand{\sv}{\vec{s}}
\newcommand{\Av}{\vec{A}}
\newcommand{\Iv}{\vec{I}}
\newcommand{\Nv}{\vec{N}}
\newcommand{\Pv}{\vec{P}}
\newcommand{\Vv}{\vec{V}}
\newcommand{\Xv}{\vec{X}}
\newcommand{\Yv}{\vec{Y}}

% Judgments
\newcommand{\isctx}[1]{\vdash #1}          % well formed context
\newcommand{\istype}[2]{#1 \vdash #2}      % well formed type
\newcommand{\isterm}[3]{#1 \vdash #2 : #3} % well formed term

\newcommand{\eqtype}[3]{#1 \vdash #2 \equiv #3}      % equal types
\newcommand{\eqterm}[4]{#1 \vdash #2 \equiv #3 : #4} % equal terms

\newcommand{\isjg}[2]{#1 \vdash \mathcal{#2}} % arbitrary judgment

% Ext. judgments
\newcommand{\xisctx}[1]{\vdash_{\exmark} #1}          % well formed context
\newcommand{\xistype}[2]{#1 \vdash_{\exmark} #2}      % well formed type
\newcommand{\xisterm}[3]{#1 \vdash_{\exmark} #2 : #3} % well formed term

\newcommand{\xeqtype}[3]{#1 \vdash_{\exmark} #2 \equiv #3}      % equal types
\newcommand{\xeqterm}[4]{#1 \vdash_{\exmark} #2 \equiv #3 : #4} % equal terms

\newcommand{\xisjg}[2]{#1 \vdash_{\exmark} \mathcal{#2}} % arbitrary judgment

% Translation
\newcommand{\transl}[1]{\llbracket #1 \rrbracket}
\newcommand{\So}{\mathsf{S}}
\newcommand{\HTS}{\mathsf{HTS}}
\newcommand{\Gb}{\overline{\Ga}}
\newcommand{\Db}{\overline{\D}}
\newcommand{\Ab}{\overline{A}}
\newcommand{\Bb}{\overline{B}}
\newcommand{\Tb}{\overline{T}}
\newcommand{\Pb}{\overline{P}}
\newcommand{\eb}{\overline{e}}
\newcommand{\jg}{\mathcal{J}}
\newcommand{\jgb}{\overline{\jg}}
\newcommand{\ub}{\overline{u}}
\newcommand{\vb}{\overline{v}}
\newcommand{\wb}{\overline{w}}
\newcommand{\pb}{\overline{p}}
\newcommand{\tb}{\overline{t}}
\newcommand{\fb}{\overline{f}}
\newcommand{\gb}{\overline{g}}

\newcommand{\transpo}[1]{#1_*}
\newcommand{\otransport}[4]{\mathsf{transport}'_{#1,#2}(#3,#4)}
\newcommand{\transport}[4]{\mathsf{transport}_{#1,#2}(#3,#4)}

\newcommand{\translitivity}[2]{\mathsf{transitivity}(#1,#2)}
\newcommand{\otransitivity}[2]{\mathsf{transitivity}'(#1,#2)}

% Operations
\newcommand{\nmax}[2]{\mathsf{max}(#1,#2)}

% Notations
\newcommand{\Heqs}{\cong}
\newcommand{\Heq}[4]{#2 \mathrel{{}_{#1}{\Heqs}_{#3}} #4}
\newcommand{\ir}{\sqsubset}
\newcommand{\Pack}[2]{\mathsf{Pack}\ #1\ #2}
% \newcommand{\ProjO}[3]{\mathsf{Proj}_1\ #1\ #2\ #3}
% \newcommand{\ProjT}[3]{\mathsf{Proj}_2\ #1\ #2\ #3}
% \newcommand{\ProjE}[3]{\mathsf{Proj}_e\ #1\ #2\ #3}
\newcommand{\ProjO}[1]{\mathsf{Proj}_1\ #1}
\newcommand{\ProjT}[1]{\mathsf{Proj}_2\ #1}
\newcommand{\ProjE}[1]{\mathsf{Proj}_\mathsf{e}\ #1}
\newcommand{\llift}[3]{#3 {[#1_1]}^{#2}}
\newcommand{\rlift}[3]{#3 {[#1_2]}^{#2}}
% \newcommand{\lo}[1]{\llift{}{}{#1}}
% \newcommand{\ro}[1]{\rlift{}{}{#1}}
\newcommand{\lo}[1]{#1 \upharpoonleft}
\newcommand{\ro}[1]{#1 \upharpoonright}
\newcommand{\Gp}{\Ga_\mathsf{p}}
\newcommand{\lift}[2]{\mathord{\uparrow^{#1}_{#2}}}


% Links to the repository

% For anonymity we should remove the links
\newcommand{\repoURL}{https://github.com/TheoWinterhalter/ett-to-itt/blob/master/theories/}
\newcommand{\rpath}[1]{\href{{\repoURL #1}}{#1}}
% \newcommand{\rpath}[1]{\path{#1}}
 % TODO Integrate into macros and factorise

% Customising margin citations
\renewcommand{\formatmargincitation}[1]{%
	\color{Gray!50} \parencite{#1}: \citeauthor*{#1} (\citeyear{#1}), \citetitle{#1}\\%
}

% Font fallback
\usepackage{newunicodechar}
\newfontfamily{\fallbackmonofont}{Symbola}[Scale=MatchLowercase]
\DeclareTextFontCommand{\textfallbackmono}{\fallbackmonofont}
\newcommand{\fallbackcharmono}[2][\textfallbackmono]{%
    \newunicodechar{#2}{#1{#2}}%
}
\fallbackcharmono{⊗}
\fallbackcharmono{⊩}
\fallbackcharmono{⨶}
\fallbackcharmono{⊨}
\fallbackcharmono{⨷}
\fallbackcharmono{⊕}
% \fallbackcharmono{∘}
\fallbackcharmono{∀}

\newfontfamily{\dejavusansfont}{DejaVu Sans}[Scale=MatchLowercase]
\DeclareTextFontCommand{\textfallbackdejavusans}{\dejavusansfont}
\newcommand{\fallbackchardejavusans}[2][\textfallbackdejavusans]{%
    \newunicodechar{#2}{#1{#2}}%
}
\fallbackchardejavusans{ᵗ}
\fallbackchardejavusans{∘}

%----------------------------------------------------------------------------------------

\begin{document}

% XeLaTeX tikzmark fix
% Coutesy of https://tex.stackexchange.com/questions/364498/tikz-offsets-origin-when-compiled-in-xelatex-as-compared-to-pdflatex
\makeatletter
\def\pgfsys@hboxsynced#1{%
  {%
    \pgfsys@beginscope%
    \setbox\pgf@hbox=\hbox{%
      \hskip\pgf@pt@x%
      \raise\pgf@pt@y\hbox{%
        \pgf@pt@x=0pt%
        \pgf@pt@y=0pt%
        \special{pdf: content q}%
        \pgflowlevelsynccm%
        \pgfsys@invoke{q -1 0 0 -1 0 0 cm}%
        \special{pdf: content -1 0 0 -1 0 0 cm q}% translate to original coordinate system
        \pgfsys@invoke{0 J [] 0 d}% reset line cap and dash
        \wd#1=0pt%
        \ht#1=0pt%
        \dp#1=0pt%
        \box#1%
        \pgfsys@invoke{n Q Q Q}%
      }%
      \hss%
    }%
    \wd\pgf@hbox=0pt%
    \ht\pgf@hbox=0pt%
    \dp\pgf@hbox=0pt%
    \pgfsys@hbox\pgf@hbox%
    \pgfsys@endscope%
  }%
}
\makeatother

\setmainfont[
	Ligatures=TeX,
	% UprightFont = ,
	ItalicFont = Fira Sans Light Italic,
	% SmallCapsFont = ,
	BoldFont = Fira Sans,
	BoldItalicFont = Fira Sans Italic
]{Fira Sans Light}
% \setsansfont[Ligatures=TeX]{Fira Sans}
\setmonofont[Ligatures=TeX]{Fira Mono}

%----------------------------------------------------------------------------------------
%	BOOK INFORMATION
%----------------------------------------------------------------------------------------

\titlehead{Some text}
\subject{PhD Thesis}

\title{Formalisation and Meta-Theory of Type Theory}
\subtitle{Customise this page according to your needs}

\author{Théo Winterhalter}

\date{\today}

\publishers{An Awesome Publisher}

%----------------------------------------------------------------------------------------

\frontmatter % Denotes the start of the pre-document content, uses roman numerals

%----------------------------------------------------------------------------------------
%	OPENING PAGE
%----------------------------------------------------------------------------------------

%\makeatletter
%\extratitle{
%	% In the title page, the title is vspaced by 9.5\baselineskip
%	\vspace*{9\baselineskip}
%	\vspace*{\parskip}
%	\begin{center}
%		% In the title page, \huge is set after the komafont for title
%		\usekomafont{title}\huge\@title
%	\end{center}
%}
%\makeatother

%----------------------------------------------------------------------------------------
%	COPYRIGHT PAGE
%----------------------------------------------------------------------------------------

\makeatletter
\uppertitleback{\@titlehead} % Header

\lowertitleback{
	\textbf{Disclaimer}\\
	You can edit this page to suit your needs. For instance, here we have a no copyright statement, a colophon and some other information. This page is based on the corresponding page of Ken Arroyo Ohori's thesis, with minimal changes.

	\medskip

	\textbf{No copyright}\\
	\cczero\ This book is released into the public domain using the CC0 code. To the extent possible under law, I waive all copyright and related or neighbouring rights to this work.

	To view a copy of the CC0 code, visit: \\\url{http://creativecommons.org/publicdomain/zero/1.0/}

	\medskip

	\textbf{Colophon} \\
	This document was typeset with the help of \href{https://sourceforge.net/projects/koma-script/}{\KOMAScript} and \href{https://www.latex-project.org/}{\LaTeX} using the \href{https://github.com/fmarotta/kaobook/}{kaobook} class.

	The source code of this book is available at:\\\url{https://github.com/fmarotta/kaobook}

	(You are welcome to contribute!)

	\medskip

	\textbf{Publisher} \\
	First printed in May 2019 by \@publishers
}
\makeatother

%----------------------------------------------------------------------------------------
%	DEDICATION
%----------------------------------------------------------------------------------------

\dedication{
	Figure something to put here or remove it altogether.\\
	\flushright -- Myself
}

%----------------------------------------------------------------------------------------
%	OUTPUT TITLE PAGE AND PREVIOUS
%----------------------------------------------------------------------------------------

% Note that \maketitle outputs the pages before here

% If twoside=false, \uppertitleback and \lowertitleback are not printed
% To overcome this issue, we set twoside=semi just before printing the title pages, and set it back to false just after the title pages
\KOMAoptions{twoside=semi}
\maketitle
\KOMAoptions{twoside=false}

%----------------------------------------------------------------------------------------
%	PREFACE
%----------------------------------------------------------------------------------------

%\input{chapters/preface.tex}
% \setchapterpreamble[u]{\margintoc}
\chapter{Acknowledgements}
\labch{ack}

% % From CPP
% I would like to thank Andrej Bauer and Philipp Haselwarter with whom I had
% fruitful discussions on the subject, prior to this work.
% I also would like to thank the attendees of the Aarhus EUTypes 2018 meeting
% for their insightful feedback on the plugin stemming from the translation.
\pagelayout{margin} % Restore margins
% \setchapterpreamble[u]{\margintoc}
\chapter{How to read this thesis}
\labch{how-to-read}

I believe that a document of this size will more often viewed on computer than
on paper. This how I personally view most papers I read anyway.
As such I believe that the document should be adapted for such a medium, unlike
most \LaTeX{} documents.
While trying to make my own kind of document I stumbled upon the great
\href{https://github.com/fmarotta/kaobook/}{kaobook} class which corresponds
almost exactly to what I was looking for: citations, notes, reminders can be put
in a large margin.

Citations like~\sidecite{boulier17:next-syntac-model-type-theor} will go in the
margin, as well as in the \refbib.
The margin also contain sidenotes\sidenote{
  No more going back and forth between the end and top of the page.
}, replacing the usual footnotes.

I will also use margin notes that are not anchored in the main document but are
usually relevant to the paragraph or figure on their left.
\marginnote[-0.4cm]{
  They are not placed entirely automatically, I tend to adjust the height so it
  is at the right place.
}

\begin{theorem}
  Theorems and definitions will be in these yellow boxes.
\end{theorem}

\reminder[-0.7cm]{of something}{
  Body of the reminder.
}
I will also use the margin to set up reminders in green boxes. Their purpose
is to avoid the reader having to go back to a previous chapter they might not
have even read and still grasp the meaning of the statements at hand.

\sidedef[-0.7cm]{of something else}{
  This is supposedly a definition
}
On the other hand, I will sometimes refer to concepts for the first time without
taking the time to introduce them because they are secondary to the main point.
In some cases those will be accompagnied by a short definition on the side, in
a yellowish box.
\pagelayout{wide} % No margins

%----------------------------------------------------------------------------------------
%	TABLE OF CONTENTS & LIST OF FIGURES/TABLES
%----------------------------------------------------------------------------------------

\begingroup % Local scope for the following commands

% Define the style for the TOC, LOF, and LOT
%\setstretch{1} % Uncomment to modify line spacing in the ToC
%\hypersetup{linkcolor=blue} % Uncomment to set the colour of links in the ToC
\setlength{\textheight}{23cm} % Manually adjust the height of the ToC pages

% Turn on compatibility mode for the etoc package
\etocstandarddisplaystyle % "toc display" as if etoc was not loaded
\etocstandardlines % "toc lines as if etoc was not loaded

\tableofcontents % Output the table of contents

\listoffigures % Output the list of figures

% Comment both of the following lines to have the LOF and the LOT on different pages
\let\cleardoublepage\bigskip
\let\clearpage\bigskip

\listoftables % Output the list of tables

\endgroup

%----------------------------------------------------------------------------------------
%	MAIN BODY
%----------------------------------------------------------------------------------------

\mainmatter % Denotes the start of the main document content, resets page numbering and uses arabic numbers

\setchapterstyle{kao} % Choose the default chapter heading style

% \setchapterpreamble[u]{\margintoc}
\chapter{Introduction}
\labch{intro}

\todo{Nicolas: Speak about ``derivation checking''. Where? How?}

\todo{Have a section on contributions, in particular to point to card model
and formal type theory that aren't outlined as such}

\pagelayout{wide} % No margins
\addpart{Proofs, Types and Programs}
\pagelayout{margin} % Restore margins

% \setchapterpreamble[u]{\margintoc}
\chapter{Proof theory}
\labch{proof-theory}

My work is done in the wide domain of proof theory. Proof theorists are
interested in the way to prove things, in the `proof' object itself.
This allows us to understand more about proof reuse, about transposition of a
property to another system, or about some intrinsinc properties of the proof
itself. The study of proofs also allows to define clear systems outlining
formally what a proof is and when it is valid. This leads to the notion of
certificate that one can check independently. The epitome of this is the ability
to write proofs that are checkable by a computer, shifting the trust one needs
to put within every proof, to the system which validates them all.

\section{How to prove something}

\subsection{A social construct?}

Before we start proving something, we must know percisely what it is we want to
prove. In informal mathematics, the statement will be a sentence, involving
concepts that the writer and reader agree on. The proof then consists in a
sequence of sentences and argument that convince the same readers.

With this definition, a proof is a subjective concept, it depends on the
reader's capacity to understand and potentially fill the gap themselves about
understood statements and properties. It also involves a fair bit of
\emph{trusting} usually: you may not understand a proof, but will believe in
the common effort of the community to verify the proof or, even better,
reproduce it.
As such, \emph{consensus} seems key in the scientific community.

One way to reach consensus much faster is to have statements and proofs really
precise and unambiguous, described in formal systems. This approach still has
shortcomings, will all readers check every tiny detail of the proof once it's
laid out extensively? This poses the risk of having the \emph{idea} lost in a
sea of information.
To me this calls for computer-verified proofs---and maybe even automated or
computer aided proofs---so that the reader can focus on the interesting part of
the proof while trusting only the tool and not the human that used it.

Even there is room for question on whether this really constitutes a proof.
For instance, how \emph{hard} is it for the computer to \emph{see} that the
proof is indeed correct? One definition would be to say, as long as it takes
a finite amount of time, it's good, but if it takes ages, we won't have any
certainty. In \sidecite{de1991plea}, de Bruijn suggests that a proof should be
self-evident. You shouldn't have to think for hours before seeing that is indeed
correct, and the same holds for computers.

In the remainder of the section I will address the way we \emph{write}
statements and proofs.

\subsection{Formal statements}

\marginnote[0.2cm]{
  Since I want to be as general as possible, this talk about formal statements
  will be pretty informal.
}
What do formal statements look like? Probably something like
\[
  \forall n \in \mathbb{N}. \exists m \in \mathbb{R}.
  f(n) = \mathsf{e}^{\phi(m)} \wedge g(n) > m
\]
It involves defined symbols, and logical connectives to make something precise.
In particular you will not the universal (\(\forall\)) and existential
(\(\exists\)) quantifiers, equality (\(=\)), logical conjunction (\(\wedge\))
and a comparison operator (\(>\)).
We can assume \(\mathbb{N}\), \(\mathbb{R}\), \(f\), \(g\), \(h\), \(\phi\) and
\(e\) to be defined prior to the statement (using similar formalism).

To define this, we give a syntax of propositions, mutually with a syntax of
sets on which we want to quantify. Because equality can mention elements however
we also have to provide a syntax for those, and maybe one for function symbols.
\[
  \begin{array}{rrl}
    P, Q &\bnf& \top \bnfor \bot \bnfor P \wedge Q \bnfor P \vee Q \\
    &\bnfor& \forall x \in E. P \bnfor \exists x \in E. P \bnfor u = v \\
    E &\bnf& \mathbb{N} \bnfor \mathbb{R} \bnfor \dots \\
    u, v &\bnf& x \bnfor \mathsf{e}^u \bnfor u + v \bnfor f(u) \bnfor \dots \\
    f, g, h &\bnf& \dots
  \end{array}
\]

Coming up with a correct syntax like those can be pretty painful so formalisms
tend to be as minimal as possible, the other advantage being that is much easier
to reason on the statements when there aren't hundreds of syntactical
constructs.

Of course, giving a syntax of statements is not enough. We must give these
symbols a semantics to know what it means to prove them.

\subsection{Inference rules}

\todo{Talk about the three implications, \(\to\), \(\vdash\) and inference
rules}

\section{Proof frameworks}

\todo{Different logics, Constructiveness, classical logic, linear logic}
\todo{mechanised proofs, proof assistants, automated theorem provers}
% \setchapterpreamble[u]{\margintoc}
\chapter{Simple type theory}
\labch{simple-types}
% \setchapterpreamble[u]{\margintoc}
\chapter{Dependent types}
\labch{dependent-types}

\todo{
  For programming,
  For verification,
  For maths,
  Constructiveness by computation (forward ref canonicity)
}

The idea between dependent types is that they now can \emph{depend} on terms,
\ie terms can appear in types. This is very interesting because we can talk
about things like \(P\ n\) for a property on a natural number \(n\),
equality \(x = y\), and with it we can support quantifiers.

It is not only useful on the logic side, but also on the programming language
side: with it programs can be given much more precise types.
For instance, you might want to specify that the division operator doesn't
accept \(0\) for the denominator, or that the operation returning the tail of
list only applies to non-empty lists.
Finally, we can take advantage of both and write \emph{proofs} about the
programs we wrote, using the very same language.

\section{A minimal dependent type theory}

Let me describe a very basic type theory with dependent types.

\paradot{\(\Pi\)-types}

The simplest way to get dependent types is to extend simply type theory---which
only features arrow or function types---with dependent function types or
\(\Pi\)-types.
\[
  \infer
    {\Ga, x :A \vdash t : B}
    {\Ga \vdash \lambda (x:A).t : \Pi (x:A).\ B}
  %
\]
As you can see, we keep the \(\lambda\)-terms of earlier but now they represent
dependent functions. Now, not only \(t\) can mention \(x\), but \(B\) also.
For instance we can write the polymorphic identity function as follows
\marginnote[0.7cm]{
  I will explain later what the \(\Type\) here means, but it should be
  intuitive: I am taking a \emph{type} \(A\) as argument and then an element
  \(a\) of that type.
}
\[
  \lambda (A : \Type).\ \lambda (a : A).\ a
\]
and it has type
\[
  \Pi (A : \Type).\ \Pi (a : A).\ A
\]
which will write more concisely as
\[
  \lambda\ (A : \Type)\ (a : A).\ a
  : \Pi\ (A : \Type)\ (a : A).\ A
\]
or even as
\marginnote[0.5cm]{
  \(A \to B\) is no longer a definition, but a notation for the dependent
  function type which is not actually dependent \(\Pi (\_ : A).\ B\).
}
\[
  \lambda\ (A : \Type)\ (a : A).\ a
  : \Pi\ (A : \Type).\ A \to A
\]
since \(a\) is not mentioned.
Now we should be able to see the dependency on \(A\).
One can argue that \(A\) is a \emph{type} and not a term, but in this setting,
types are just a special kind of terms that happen to be of type \(\Type\).

If we have some \(B : \Type\) we might want to apply our polymorphic identity
function to it to get the identity function on \(B\), \ie
\[
  \lambda (a:B).\ a : B \to B
\]

For this we have to rely on substitutions again, not only in the terms after
\(\beta\)-reduction, but also in types.
This can be seen in the application rule:
\[
  \infer
    {
      \Ga \vdash u : \Pi (x:A).\ B \\
      \Ga \vdash v : A
    }
    {\Ga \vdash u\ v : B[x \sto v]}
  %
\]
Once again it is pretty similar to the application rule of simple type theory
except we have to account for the dependency. In our example---writing \(\cid\)
for the polymorphic identity function---we have
\marginnote[0.7cm]{
  What happens is the type of the application is
  \[
    (A \to A)[A \sto B] = B \to B
  \]
}
\[
  \infer
    {
      \Ga \vdash \cid : \Pi (A:\Type).\ A \to A \\
      \Ga \vdash B : \Type
    }
    {\Ga \vdash \cid\ B : B \to B}
  %
\]

\paradot{Universes}

In the example above there is the peculiar type \(\Type\).
This can be thought of as the type of types. If you know about Russel's paradox
stating that there can be no set of all sets, you might be skeptical and indeed
having \(\Type\) of type \(\Type\) is inconsistent.
We will address this in more details in \refsubsec{coq-univ} in
\nrefch{flavours}.
In this case, \(\Type\) will be a special type---which we call universe as it
is inhabited solely by types---that doesn't have a type itself.
Having it is mainly to allow for quantifying over types, but we could also of
course define some base types like the natural numbers \(\nat\) and put it in
\[
  \nat : \Type
\]
We will more example of types in \nrefch{usual-defs}. For now we only have
\(\Pi\)-types in them:
\[
  \infer
    {
      \Ga \vdash A : \Type \\
      \Ga, x : A \vdash B : \Type
    }
    {\Ga \vdash \Pi (x:A).\ B : \Type}
  %
\]
Here we evidence the fact \(B\) is indeed dependent over \(x : A\).
This kind of universe is called a Russel universe, and there are also Tarski
universes, the difference will be explained in \refsubsec{univ-and-types}
of \nrefch{formalisation}.

I will not put the syntax and the rules together for clarity, at the risk of
repeating myself.
\[
  \begin{array}{rcl}
    A, B, t, u &\bnf& x \bnfor \lambda (x:A).t \bnfor t\ u \bnfor \Pi (x:A). B
    \bnfor \Type \\
    \Ga, \D &\bnfor& \ctxempty \bnfor \Ga, x:A
  \end{array}
\]

\begin{mathpar}
  \infer
    {(x : A) \in \Ga}
    {\Ga \vdash x : A}
  %

  \infer
    {
      \Ga \vdash A : \Type \\
      \Ga, x : A \vdash B : \Type
    }
    {\Ga \vdash \Pi (x:A).\ B : \Type}
  %

  \infer
    {
      \Ga, x :A \vdash t : B \\
      \Ga \vdash \Pi (x:A).\ B : \Type
    }
    {\Ga \vdash \lambda (x:A).t : \Pi (x:A).\ B}
  %

  \infer
    {
      \Ga \vdash u : \Pi (x:A).\ B \\
      \Ga \vdash v : A
    }
    {\Ga \vdash u\ v : B[x \sto v]}
  %
\end{mathpar}
\marginnote[-3cm]{
  Here I changed a bit the typing rule for \(\lambda\)-abstraction to also ask
  for its type to be well-formed. This is necessary to ensure we put legitimate
  types in the context.
}

We usually also add a definition of well-formed context to ensure it is
comprised of types that make sense and that the dependencies are in order:
\marginnote[1cm]{
  As you can see, each type might depend on the previous variables.
}
\begin{mathpar}
  \infer
    { }
    {\vdash \ctxempty}
  %

  \infer
    {
      \vdash \Ga \\
      \Ga \vdash A : \Type
    }
    {\vdash \Ga, x:A}
  %
\end{mathpar}

\section{Dependent types in \Coq}

\todo{Maybe present \Agda a bit too?}
% \setchapterpreamble[u]{\margintoc}
\chapter{Usual definitions in type theory}
\labch{usual-defs}

Dependent type theory as presented in \nrefch{dependent-types} is rather barren.
It really shines when extended with some interesting principles and datatypes.
I will give an overview of these features---with the \Coq proof assistant in
mind---and focus mainly on those that are relevant to this thesis.

\section{Inductive types}
\labsec{inductive-types}

Inductive types are probably the most emblematic feature of dependent type
theory. They are an extension of the variant datatypes present in \ocaml like
the type of lists.
\marginnote[0.6cm]{
  A list, say of integers, is either empty (\mintinline{ocaml}{nil}) or some
  head \mintinline{ocaml}{h : int} and some tail
  \mintinline{ocaml}{t : int list}, written \mintinline{ocaml}{cons h t}.
}
\begin{minted}{ocaml}
type 'a list =
| nil
| cons of 'a * 'a list
\end{minted}
They come in different flavours which I will try to explain.

\subsection{Variants}

The simplest case of inductive types is that of variants. They consist in a list
of different options.

\paradot{Booleans}

\(\bool\) is the type inhabited by \(\ttrue\) and \(\ffalse\).
\begin{mathpar}
  \infer
    { }
    {\Ga \vdash \bool}
  %

  \infer
    { }
    {\Ga \vdash \ttrue : \bool}
  %

  \infer
    { }
    {\Ga \vdash \ffalse : \bool}
  %
\end{mathpar}

In \Coq you would write it as follows:
\begin{minted}{coq}
Inductive bool : Type :=
| true
| false.
\end{minted}
Of course, having those is not nearly enough without the usual
\(\mathsf{if}\) construct.
For instance \mintinline{coq}{if b then 0 else 1} will return
\mintinline{coq}{0} if \mintinline{coq}{b} is \mintinline{coq}{true}
and \mintinline{coq}{1} if it is \mintinline{coq}{false}.
This is already something that makes sense in the simple
case\sidenote{See \nrefch{simple-types}}, but with dependent types the case
analysis is also dependent on the scrutinee.
\begin{mathpar}
  \infer
    {
      \Ga \vdash b : \bool \\
      \Ga, x : \bool \vdash P \\
      \Ga \vdash u : P[x \sto \ttrue] \\
      \Ga \vdash v : P[x \sto \ffalse]
    }
    {\Ga \vdash \tif{b}{x.P}{u}{v}}
  %
\end{mathpar}
Before we break down the typing rule, let me show you the computational
behaviour of \(\mathsf{if}\).
\begin{mathpar}
  \begin{array}{lcl}
    \tif{\ttrue}{x.P}{u}{v} &\red& u \\
    \tif{\ffalse}{x.P}{u}{v} &\red& v
  \end{array}
\end{mathpar}
It is still the same as the well-known \(\mathsf{if}\), except that we are more
liberal in the types given to the two branches: they don't have to match as they
can now depend on the boolean. The \(x.P\) notation means that \(P\) lives in a
context extended by \(x\) (of type \(\bool\)).

The \(\mathsf{if}\) is actually just a notation for a more generic construction
called \emph{pattern-matching}. \(\tif{b}{x.P}{u}{v}\) is in fact the term
\[
  \pmatch{b}{x.P}{
    \branch{\ttrue}{u} \\
    \branch{\ffalse}{v}
  }
\]
It describes the case analysis by saying which constructor is sent to which
term. If the scrutinee---here \(b\)---\emph{matches} one of the branches on
left-hand side of \(\mto\), the whole expression will reduce to the
corresponding right-hand side.

\paradot{Unit}

The \(\unit\) type is similar to \(\bool\) but has only one constructor written
\(\tunit\).
\begin{mathpar}
  \infer
    { }
    {\Ga \vdash \unit}
  %

  \infer
    { }
    {\Ga \vdash \tunit : \unit}
  %
\end{mathpar}

In \Coq it is defined as:
\begin{minted}{coq}
Inductive unit : Type :=
| tt.
\end{minted}
And the notation mechanism can help use write \mintinline{coq}{tt}
as \mintinline{coq}{()}.
Once again, pattern-matching allows us to inspect a proof of \(\unit\):
\begin{mathpar}
  \infer
    {
      \Ga \vdash u : \unit \\
      \Ga, x:\unit \vdash P \\
      \Ga \vdash v : P[x \sto \tunit]
    }
    {
      \Ga \vdash
      \pmatch{u}{x.P}{
        \branch{\tunit}{v}
      }
      : P[x \sto u]
    }
  %
\end{mathpar}
This might seem a bit useless, but essentially it means that to prove anything
involving a dependency on \(\unit\), like \(P[x \sto u]\), it suffices to prove
it assuming it is \(\tunit\): \(P[x \sto \tunit]\).

Sometimes this type is called \(\top\) as in the logical triviality.

\paradot{Empty type}

The empty (or false) type, \(\bot\) is the dual of the unit type. This time it
has no constructors \emph{at all}.
\begin{mathpar}
  \infer
    { }
    {\Ga \vdash \bot}
  %
\end{mathpar}

In \Coq, it is written in a rather queer manner.
\begin{minted}{coq}
Inductive False :=.
\end{minted}

It represents the data that should never exist, so any term of type \(\bot\)
is a \emph{contradiction} with the hyptheses at hand.
Even though it does not have constructors, pattern-matching still makes sense on
such terms.
\marginnote[1.6cm]{
  The pattern-matching does not have any branches, hence the empty space.
}
\begin{mathpar}
  \infer
    {
      \Ga \vdash t : \bot \\
      \Ga, x:\bot \vdash P
    }
    {\Ga \vdash \pmatch{t}{x.P}{} : P[x \sto t]}
  %
\end{mathpar}
This is the essence of the \emph{principle of explosion}:
\emph{ex falso quodlibet}, from falsehood, anything follows.
Here we are able to conjure some inhabitant of \(P[x \sto t]\) from thin air.
The \(P\) is typically not dependent on the proof of \(\bot\), meaning that from
an inhabitant of \(\bot\) we can get an inhabitant of \emph{any} type.

\subsection{Parameterised inductive types}


\todo{nat, lists, sigmas, prods and prods as sigmas}

\subsection{Indexed inductive types}
\todo{vectors}
\todo{Lead to equality}

\todo{Inductive Inductive/Recusrive? Probably irrelevant here}

\section{Coinductive types and records}

\section{Equality}
\labsec{equality-def}

\todo{This is not a debate about how to define it, we will show the Coq one
and refer to the next chapter to the discussion about its definition.}
\todo{UIP, funext, transport, heterogenous equality}

\acrfull{JMeq} introduced by
\sidecite{mcbride2000dependently} has axioms
instead better
\[ \Heq{T}{t}{U}{u} := \Sum{p:\Eq{}{T}{U}} \Eq{}{\transpo{p}\ t}{u}. \]
% \setchapterpreamble[u]{\margintoc}
\chapter{Flavours of type theory}
\labch{flavours}

\todo{Cumulativity?}

Type theory comes in many different flavours and shapes, different formulations
and properties. I will not try to be exhaustive but I will try to cover the
main kind of dependent type theories I have encountered.
I will not attempt to define properly the notion of \emph{type theory},
there is work on this~\misref{} but it is still a bit early to grasp the
concept fully.

\section{Computation and type theory}

The first prism in which to see type theory through can be that of computation.
Indeed, not all type theories feature it to the same extent. Though for some
people in computation resides the essence of type theory, it is still worth
it to investigate theories where conversion is defined differently.

\subsection{Intensional Type Theory}
\labsubsec{pts-itt}

\acrfull{ITT} is the name given to a wide range of type theories actually.
Those could be described in the setting of \acrlongpl{PTS}.
The theories behind the proof assistans \Coq and \Agda---respectively
\acrfull{PCUIC} and \acrfull{MLTT}\sidenote{Note that \acrshort{MLTT} also
have various forms.}---are variants of \acrshort{ITT}.

A \acrshort{PTS} is a pretty basic type theory, it is parametrised by a
collection of sorts \cS, with so called \emph{rules}
(\Rl) and \emph{axioms} (\Ax).
Its syntax features \(\lambda\)-abstractions, applications, variables,
\(\Pi\)-types and sorts.
%
\[
  \begin{array}{lrl}
    s &\in& \mathcal{S} \\
    T,A,B,t,u,v &\bnf& x \bnfor \lambda (x:A). t \bnfor t\ u
    \bnfor \Pi (x : A). B \bnfor s \\
    \Ga, \D &\bnf& \ctxempty \bnfor \Ga, x:A
  \end{array}
\]

Their computational behaviour is defined by a reduction relation (\(\red\))
which is the contextual closure of the \(\beta\)-reduction.
\[
  (\lambda (x:A). t)\ u \red_\beta t[x \sto u]
\]
\marginnote[-0.9cm]{
  For instance
  \(\lambda (x : A). (\lambda (y : B) y x)\ t \red \lambda (x : A). t x\)
}

The typing rules involve the rules and axioms we mentioned earlier.
%
\begin{mathpar}
  \infer
    {(s,s') \in \Ax}
    {\isterm{\Ga}{s}{s'}}
  %

  \infer
    {
      \isterm{\Ga}{A}{s_1} \\
      \isterm{\Ga, x : A}{B}{s_2} \\
      (s_1, s_2, s_3) \in \Rl
    }
    {\isterm{\Ga}{\Pi (x:A). B}{s_3}}
  %

  \infer
    {(x : A) \in \Ga}
    {\isterm{\Ga}{x}{A}}
  %

  \infer
    {
      \isterm{\Ga, x:A}{t}{B} \\
      \isterm{\Ga}{\Pi (x:A).B}{s}
    }
    {\isterm{\Ga}{\lambda (x:A).t}{\Pi (x:A).B}}
  %

  \infer
    {
      \isterm{\Ga}{t}{\Pi (x:A). B} \\
      \isterm{\Ga}{u}{A}
    }
    {\isterm{\Ga}{t\ u}{B[x \sto u]}}
  %

  \infer
    {
      \isterm{\Ga}{t}{A} \\
      A \equiv B \\
      \isterm{\Ga}{B}{s}
    }
    {\isterm{\Ga}{t}{B}}
  %
\end{mathpar}
%
Axioms determine typing of sorts, and rules what dependent products are allowed.
The last rule is the conversion rule, it is the rule that involves computation:
basically you can exchange two computationally equal types in a typing
judgement. The \(\isterm{\Ga}{B}{s}\) bit is to make sure the type we want
to substitute still makes sense.
In this case, conversion (\(\equiv\)) is defined as the reflexivie,
symmetric, transitive closure of reduction.
\marginnote[-0.4cm]{
  \(t \equiv u\) is defined as \(t \mathop{(\redl . \red)^\star} u\)
}

When talking about \acrshort{ITT} we usually mean an extension of this with
more concepts like some base types~\misref{} and computation rules on their
eliminators (pattern-matching). The conversion rule is also not always strictly
derived from the reduction alone, it often includes \(\eta\)-rules, the most
common being \(\eta\)-expansion of functions.
\[
  f \equiv_\eta \lambda (x:A). f\ x
\]
For it to make sense expansion has to be limited to functions which requires
type information\sidenote{Also, the \(A\)---domain of the function---has to
be sumrised somehow.}, this is why is certain contexts---like \Agda---the
conversion is also typed.
The relation between typed and untyped conversion has been explored at several
occasions~\misref.
\Coq manages to verify \(\eta\)-conversion for functions and records without
relying on a typed-conversion as we will see later~\misref.
\marginnote[-0.5cm]{
  For instance, \(\eta\) for pairs is \(p \equiv (p.1, p.2)\) where
  \(p.1\) and \(p.2\) are the first and second projections of \(p\).
}

\subsection{Extensional Type Theory}

\acrfull{ETT} is an extension of \acrshort{ITT} where conversion is extended
to capture all provable equalities, this principle is called \emph{reflection
(of equality)}.
This of course implies that the considered \acrshort{ITT} is equipped with
an equality type.

\begin{definition}{Reflection Rule}
  \labdef{reflection}
  \begin{mathpar}
    \infer[]
      {\xisterm{\Ga}{e}{\Eq{A}{u}{v}}}
      {\xeqterm{\Ga}{u}{v}{A}}
    %
  \end{mathpar}
\end{definition}

As you can see, this time I opted for a typed conversion, I think it makes more
sense since the conversion is more semantical than syntactical.
Also, you can note the little \(\exmark\) subscript, its purpose is to mark
the judgement as beeing \emph{ex}tensional.
\Andromeda and \NuPRL implement variants of \acrlongpl{ETT}~\misref.
To see its usefulness, we're going to look at the definition of reversal of
vectors in \Coq, using an accumulator for the definiton to be tail-recursive.
%
\begin{minted}{coq}
Definition vrev {A n m} (v : vec A n) (acc : vec A m)
  : vec A (n + m) :=
  match v with
  | vnil => acc
  | vcons a n v => vrev v (vcons a m acc)
  end.
\end{minted}
%
The recursive call of \mintinline{coq}{vrev} returns a vector of length
\mintinline{coq}|n + S m| where the context expects one of length
\mintinline{coq}|S n + m|. In \acrshort{ITT} and \Coq, thesec types are not
convertible, and thus the definition isn't accepted, even though it feels like
it is the right definition. \acrshort{ETT} solves this problem by exploiting
the fact that \mintinline{coq}|n + S m = S n + m| is provable.
You can still define it in \Coq, but you have to explicitely transport along
the abovementioned equality which can result in some problems while reasoning
on the resulting function and inconveniences overall.
\todo{Perhaps give concrete defs for ETT/ITT/WTT so that we can rely on them
in the second part.}

\acrshort{ETT} isn't the ultimate solution however and suffers from many
drawbacks the main of which being that type-checking is not decidable as we
shall see later~\misref. We will also explore the relation between
\acrshort{ETT} and \acrshort{ITT}~\misref.

\subsection{Weak Type Theory}
\labsubsec{wtt}

A \acrfull{WTT} is on the other end of the spectrum: instead of extending
conversion with everything that can be proven equal, conversion is removed
altogether. Computation (like \(\beta\)-reduction) is now handled by
propositional equality alone, and conversion of types is done using transports
of said equality.

This time it's a bit hard to advertise it for practical use in a proof
assistant, it's nonetheless interesting. For one, its meta-theory is that much
simpler, an even more attractive fact once combined with a translation from
\acrshort{ITT} (or \acrshort{ETT}) to \acrshort{WTT} as is the object of
\refpart{elim-reflection}.
Another point worth mentioning is that it really crystallises the notion that
proofs are really just terms and do not require extra machinery to make sure
they are indeed proofs (even when conversion is decidable, it might takes
eons before you get this knowlegde)~\misref.

One might also be tempted to call it minimal, but in order to simulate the
congruence aspect of conversion, we have to extend the theory with principles
to allow equalities under binders, and this has to be done for each binder
(once for \(\lambda\)-abstractions---the usual functional extensionality---and
once for \(\Pi\)-types---much less standard---at least).

\section{Focus on the theory behind \Coq}

\Coq is originally based on the \acrfull{CoC}, or the \acrfull{CIC}, though it
is nowadays rather called the \acrfull{PCUIC}.
\todo{What to say?}

\section{Equality in type theories}

\todo{Re-present the inductive eq then HoTT, talk about alternatives like OTT
and cubical, then present two-level type theories and HTS. Also point towards
quotients, QIT}
% \setchapterpreamble[u]{\margintoc}
\chapter{Desirable properties of type theories}
\labch{desirable-props}

To compare different type theories there are several measures we can use in
form of usual or desirable properties that they migh satisfy or not.
After a brief presentation of the main ones I will summarise which of the
theories of \vrefch{flavours} has which properties in a table.

\section{Properties}

\subsection{Weakening and substitutivity}

Variables and binders are essential to type theory and as such we have to treat
them with care, in particular we want our theories to be \emph{compositional}
meaning that different blocks that make sense can be assembled into something
that still makes sense.
This is embodied in the two following properties.

\begin{definition}[Weakening]
  A type theory enjoys weakening when for any \(\Ga, \Xi \vdash t : A\) and
  \(\vdash \Ga, \Delta\) we have \(\Ga, \Delta, \Xi \vdash t : A\).
\end{definition}
\marginnote[-1.2cm]{
  Note that in more generality you might have to rename variables when
  weakening, so for this we usually introduce a \emph{lifting} operator
  \(\lift{}{}\) such that we have
  \(\Ga, \Delta, \Xi \vdash \lift{n}{k}\ t : \lift{n}{k}\ A\)
  where \(n = |\Delta|\) and \(k = |\Xi|\).
}

Weakening means that you can plug a term into a larger context.

Susbstitutions are the way to instantiate the variables that are bound.
This happens for instance after a \(\beta\)-reduction.
\reminder[-0.9cm]{\(\beta\)-reduction}{
  \[
    (\lambda (x:A). t)\ u \red_\beta t[x \sto u]
  \]
}
Substitutions are typed using two contexts: \(\sigma : \Ga \to \D\) basically
states that \(\sigma\) maps variables of \(\D\) to terms typed in \(\Ga\).
\begin{mathpar}
  \infer
    {\forall (x : A) \in \D,\ \Ga \vdash \sigma(x) : A\sigma}
    {\sigma : \Ga \to \D}
\end{mathpar}
This is sometimes written \(\Ga \vdash \sigma : \D\) instead, and typing
definitons vary a little depending on the definition of substitution but this
is the basic idea.

\begin{definition}[Substitutivity]
  A type theory is substitutive when for any \(\D \vdash t : A\)
  and any substitution \(\sigma : \Ga \to \D\), we have
  \(\Ga \vdash t\sigma : A\sigma\).
\end{definition}

More often than not, weakening and substitutivity also hold for reduction and
conversion.

\subsection{Inversion of typing}

Inversion of typing is not really a measure as it is always present in some way.
It is nonetheless a very useful property to state when reasoning on a type
theory.
It is saying than when we have \(\Ga \vdash t : A\), by analysing the shape of
\(t\) we can get information on \(A\) (and sometimes even \(\Ga\)).

\reminder[-0.6cm]{Application rule}{
  \[
    \infer
      {
        \Ga \vdash \Pi (x:A).\ B : s \\
        \Ga \vdash t : \Pi (x:A).\ B \\
        \Ga \vdash u : A
      }
      {\Ga \vdash t\ u : B[x \sto u]}
  \]
}
For instance, if we have \(\Ga \vdash t\ u : T\), then by inversion of typing
we know that there must exist \(A\) and \(B\) such that
\begin{mathpar}
  \Ga \vdash \Pi (x:A).\ B : s

  \Ga \vdash t : \Pi (x:A).\ B

  \Ga \vdash u : A

  \Ga \vdash T \equiv B[x \sto u]
\end{mathpar}

This can be proved by seeing that only two typing rules can be concluded with
\(\Ga \vdash t\ u : T\): the application rule and the conversion rule. The
result follows from a simple induction.

Now, it won't always be stated this way depending on the premises of the
application rule, but also depending on the presence or not of a conversion
rule. In the case of \acrshort{WTT} for instance, there is no conversion so
instead of \(\Ga \vdash T \equiv B[x \sto u]\) we will have syntactic equality
\(T =_{\alpha} B[x \sto u]\).

You usually prove inversion of typing for every term constructor, but I won't
do it here.

\subsection{Validity}

The term \emph{validity} might be a bit overloaded, and maybe not the norm
when it comes to type theory, but I will use it to be consistent with the
notion for \Coq.
This property states that the type on the right-hand side of the colon is indeed
a type.

\begin{definition}[Validity]
  A type theory enjoys validity when from \(\Ga \vdash t : A\) one can deduce
  \(\Ga \vdash A\) (\ie \(\Ga \vdash A : s\) for some sort \(s\)).
\end{definition}

Depending on the theory, we can often prove a similar property regarding
contexts: namely that \(\Ga \vdash t : A\) implies that \(\Ga\) is well-formed
(\(\vdash \Ga\)). Having this property mainly depends on whether the typing
rules of things like sorts and variables ask for the context to be well-formed.
%
\begin{mathpar}
  \infer
    {(x : A) \in \Ga}
    {\Ga \vdash x : A}
  %

  \text{vs}

  \infer
    {
      \vdash \Ga \\
      (x : A) \in \Ga
    }
    {\Ga \vdash x : A}
  %
\end{mathpar}
%
In a theory which does not have this requirement / property, many lemmata will
only apply assuming the contexts involved are well-formed.
The difference of presentation like this will be studied in
\nrefch{formalisation}.

\subsection{Unique / principal typing}

\subsection{Properties of reduction}
\todo{Confluence, termination}

\subsection{Canonicity}
\todo{With reduction or weaker with conversion (for ETT)}

\subsection{Decidability of type checking}

\subsection{Consistency}

\section{Summarising table}

\todo{Fill}
\begin{table*}
  \rowcolors{1}{SkyBlue!10!White}{}
  \caption[Properties of type theories]{Properties of type theories.}
  \begin{tabular}{l|c|c|c|c|c|c|c|c|c|}
    \cline{2-10}
    & \acrshort{WTT} & \acrshort{ITT} & \acrshort{ETT} & \acrshort{MLTT}
    & \acrshort{PCUIC} & ?HoTT & ?CubicalTT & ?2TT & ?HTS \\
    \hline
    \multicolumn{1}{ |c|  }{Weakening / substitutivity} &
    \multicolumn{9}{c|}{Yes} \\
    \hline
    \multicolumn{1}{ |c|  }{Inversion of typing} &
    \multicolumn{9}{c|}{Yes} \\
    \hline
    \multicolumn{1}{ |c|  }{Unique (U) / Principal type (P)} &
    U & U/P & No? & U & P & ? & ? & ? & ? \\
    \hline
  \end{tabular}
\end{table*}
% \setchapterpreamble[u]{\margintoc}
\chapter{Models of type theory}
\labch{models}

Justifying a logic is often achieved using \emph{models}. A model consists
in giving an interpretation to all constructs of the logic we want to study,
such that its rules are still verified.
There are several ways to get models of type theory, I will present some of the
most common ones in this chapter, though my means of choice will presented in
depth in \nrefch{translations}.

\section{What is a model?}

A model of type theory is an interpretation of the concepts of a type theory
into another theory or object, both living in the same meta-theory.
\begin{figure}[hb]
  \includegraphics[width=0.9\textwidth]{model}
\end{figure}
To be more precise, a model is given by a class of objects to interpret
contexts, one for terms and one for types; but a model also provides an
interpretation to judgments in such a way that the interpretation is coherent.

\subsection{What can be proved using models}

\paragraph{Consistency.}

I already briefly mentioned this but the main point of models is to prove
consistency of a theory, relying on the already \emph{known} consistency of the
theory in which lives the model... At least in principle. It is however very
rare\sidenote{Impossible?} to \emph{know} that a theory is consistent, instead
we should see it as a theory we \emph{trust}, typically a theory that's widely
accepted as beeing consistent by the community of mathematicians.
This also applies to the meta-theory in which we show that the interpretation is
correct. As such it is best to keep it as simple as possible to avoid relying on
the consistency of too complicated objects.

\paragraph{Independence.}

Another interesing application of models is showing \emph{independence} of a
proposition.

\begin{definition}[Independent proposition]
  A proposition \(P\) is said to be independent from a theory \cT when neither
  \(P\) nor \(\neg P\) can be proven within \cT.
\end{definition}

A way to prove that some \(P\) is independent from \cT is to give a model of \cT
which validates \(P\) and another model which invalidates it (or validates
\(\neg P\)). Indeed if any one of \(P\) or \(\neg P\) %, let's say \(P\),
could be proven in \cT, then it would be valid in both models, leading to
at least one of them being inconsistent.

The fact that a proposition can neither be validated or invalidated in a theory
can come as surprising for some, especially in a classical mindset.
\reminder[-1.8cm]{Classical logic}{
  Classical logic is often characterised by the presence of the \acrshort{LEM}
  which consists in a proof of \(A \vee \neg A\).
}
Gödel's incompleteness theorems have to do with this.

\subsection{Gödel's incompleteness theorems}

\todo{Is it really the place? Should it even go in the proof theory chapter?}

\section{Set-theoretic models}

Set-theoretic models are models of type-theory where types are interpreted as
sets. For instance the type \(A \to B\) will be interpreted as the set of
set-theoretic functions from the interpretation of \(A\) to the interpretation
of \(B\):
\marginnote[0.2cm]{
  It is usual to write \(\transl{A}\) for the interpretation of \(A\).
}
\[
  \transl{A \to B} \coloneqq \setfun{\transl{A}}{\transl{B}})
\]
\todo{I'm missing a ref here on how to do this...}

\section{Categorical models}

Categorical models are one of the most used way of defining models for type
theories. There are several notions of categorical models:
contextual categories~\sidecite[-0.9cm]{cartmell1986generalised} from which stem
categories with attributes~\sidecite[0.2cm]{hofmann1994interpretation} and
C-systems~\sidecite[0.9cm]{voevodsky2016subsystems},
\acrfullpl{CwF}~\sidecite[1.7cm]{dybjer1995internal}, and more\dots
I will focus on \acrshortpl{CwF} in this document.

\subsection{Categories with families}

A \acrshort{CwF} is given by
\begin{enumerate}
  \item a collection of objects (or contexts) \(\Con\);
  \item morphisms (or substitutions) \(\sigma : \Ga \to \D\) between contexts
  \(\Ga, \D : \Con\);
  \item for each context \(\Ga : \Con\), a collection \(\CTy\ \Ga\) of types in
  that context;
  \item for each context \(\Ga : \Con\) and type \(A : \CTy\ \Ga\), a collection
  \(\CTm\ \Ga\ A\) of terms of type \(A\);
  \item an operation of substitution for types taking \(\sigma : \Ga \to \D\)
  and a type \(A : \CTy\ \D\) to type \(A[\sigma] : \CTy\ \Ga\);
  \item \label{item:term-subst} a similar substitution operation on terms taking
  \(\sigma : \Ga \to \D\) and a term \(t : \CTm\ \D\ A\) to term
  \(t[\sigma] : \CTm\ \Ga\ A[\sigma]\);
  \item a terminal object representing the empty context \(\ctxempty : \Con\);
  \item an extension operation taking \(\Ga : \Con\) and \(A : \CTy\ \Ga\)
  to a new object \(\Ga, A : \Con\) together with a morphism
  \(\pp : \Ga, A \to \Ga\) and a term \(\pq : \CTm\ (\Ga, A)\ A[\pp]\);
  \item an operation taking \(\sigma : \Ga \to \D\) and term
  \(a : \CTm\ \Ga\ A[\sigma]\) to substitution \(\sigma, a : \Ga \to \D, A\)
  such that \(\pp \circ (\sigma, a) = \sigma\) and \(\pq[\sigma, a] = a\).
\end{enumerate}
The two first have to give the structure of category, this in particular means
that there is an identity substitution, and associative composition of those.

The category \(\Set\) of sets is a \acrshort{CwF}:
\marginnote[3.8cm]{
  \ref{item:term-subst} holds because for every \(\gamma \in \Ga\),
  \[
    (t \circ \sigma)(\gamma) = t(\sigma(\gamma)) \in A_{\sigma(\gamma)}
    = A[\sigma]_\gamma
  \]
}
\begin{enumerate}
  \item objects are sets;
  \item substitutions are given as set-theoretic functions, composed as
  functions;
  \item types in \(\CTy\ \Ga\) are sets indexed by \(\Ga\);
  \item terms of \(\CTm\ \Ga\ A\) are choice functions
  \(t \in \setfun{\Ga}{\bigcup_{\gamma \in \Ga} A_\gamma}\) such that
  \(t(\gamma) \in A_\gamma\) for every \(\gamma \in \Ga\);
  \item given \(\sigma : \Ga \to \D\) and \(A : \CTy\ \D\), \(A[\sigma]\)
  is defined as the \(\Ga\)-index family \((A_{\sigma(\gamma)})_\gamma\);
  \item given \(\sigma : \Ga \to \D\) and \(t : \CTm\ \D\ A\), \(t[\sigma]\)
  is defined as the choice function \(t \circ \sigma\);
  \item the empty context is given as the singleton \(\{ \emptyset \}\);
  \item given \(\Ga : \Con\) and \(A : \CTy\ \Ga\), \(\Ga, A\) is defined as
  the disjoint union \(\amalg_{\gamma \in \Ga} A_\gamma\), \(\pp\) and \(\pq\)
  are defined as the first and second projection respectively;
  \item given \(\sigma : \Ga \to \D\) and \(a : \CTm\ \Ga\ A[\sigma]\),
  \(\sigma, a\) is defined as the function
  \(\gamma \mapsto (\sigma(\gamma), a(\gamma))\) which verifies the conditions.
\end{enumerate}

Right now this only interprets a type theory with no type or term constructors.
I will give a few example of those.

\paradot{Dependent products}

A \acrshort{CwF} is said to have dependent products when it has the following
elements:
\begin{itemize}
  \item for any \(A : \CTy\ \Ga\) and \(B : \CTy\ (\Ga, A)\) there exists
  \(\Pi\ A\ B : \CTy\ \Ga\);
  \item for \(b : \CTm\ (\Ga, A)\ B\) there exists
  \(\lambda b : \CTm\ \Ga\ (\Pi\ A\ B)\);
  \item for \(f : \CTm\ \Ga\ (\Pi\ A\ B)\) and \(u : \CTm\ \Ga\ A\) there exists
  \(\capp(f,u) : \CTm\ \Ga\ B[\cid, u]\)
\end{itemize}
such that for any \(\sigma : \D \to \Ga\) and terms and types as above the
following equations hold:
\[
  \begin{array}{rcl}
    (\Pi\ A\ B)[\sigma] &=& \Pi\ A[\sigma]\ B[(\sigma \circ \pp), \pq] \\
    (\lambda b)[\sigma] &=& \lambda b[(\sigma \circ \pp), \pq] \\
    (\capp(f,u))[\sigma] &=& \capp(f[\sigma],u[\sigma]) \\
    \capp(\lambda b, u) &=& b[\cid, u] \\
    \lambda \capp(f \circ \pp, \pq) &=& f
  \end{array}
\]

\sidedef[-0.7cm]{\(\eta\)-expansion}{
  \(\eta\)-expansion is defined as follows
  \[
    t \red_\eta \lambda x.\ t\ x
  \]
  \(\eta\)-contraction is the opposite relation.
}
The four first equations are simply there to attest to the good behaviour of
substitution. The last two are more interesting, corresponding to
\(\beta\)-reduction and \(\eta\)-contraction respectively.

\paradot{Dependent sums}

A \acrshort{CwF} has (negative\sidenote{See \nrefch{flavours}.}) dependent sums
when:
\begin{itemize}
  \item for any \(A : \CTy\ \Ga\) and \(B : \CTy\ (\Ga, A)\) there exists
  \(\Sigma\ A\ B : \CTy\ \Ga\);
  \item for \(a : \CTm\ \Ga\ A\) and \(b : \CTm\ \Ga\ B[\cid,a]\) there exists
  the dependent pair \(\dpair{a,b} : \CTm\ \Ga\ (\Sigma\ A\ B)\);
  \item for \(p : \CTm\ \Ga\ (\Sigma\ A\ B)\) there exists
  \(p.1 : \CTm\ \Ga\ A\);
  \item for \(p : \CTm\ \Ga\ (\Sigma\ A\ B)\) there exists
  \(p.2 : \CTm\ \Ga\ B[\cid, p.1]\)
\end{itemize}
such that
\[
  \begin{array}{rcl}
    (\Sigma\ A\ B)[\sigma] &=& \Sigma\ A[\sigma]\ B[(\sigma \circ \pp), \pq] \\
    \dpair{a,b}[\sigma] &=& \dpair{a[\sigma], b[\sigma]} \\
    (p.1)[\sigma] &=& p[\sigma].1 \\
    (p.2)[\sigma] &=& p[\sigma].2 \\
    \dpair{a,b}.1 &=& a \\
    \dpair{a,b}.2 &=& b
  \end{array}
\]

\paradot{Identity types}

A \acrshort{CwF} has identity types---another name for the equality type---when:
\begin{itemize}
  \item for \(A : \CTy\ \Ga\) and \(u, v : \CTm\ \Ga\ A\) there exists
  \(\CId_A\ u\ v : \CTy\ \Ga\);
  \item for \(A : \CTy\ \Ga\) and \(u : \CTm\ \Ga\ A\) there exists
  \(\crefl_u : \CTm\ \Ga\ (\CId_A\ u\ u)\)
\end{itemize}
such that
\[
  \begin{array}{rcl}
    (\CId_A\ u\ v)[\sigma] &=& \CId_{A[\sigma]}\ u[\sigma]\ v[\sigma] \\
    (\crefl_a)[\sigma] &=& \crefl_{a[\sigma]}
  \end{array}
\]
In a lot of models, equality end up being interpreted as the equality of the
model meaning the eliminator is not necessary, hence my not requiring it.

\paradot{Booleans}

A \acrshort{CwF} has booleans when for any \(\Ga : \Con\) it has
\begin{itemize}
  \item \(\bool : \CTy\ \Ga\);
  \item \(\ttrue : \CTm\ \Ga\ \bool\);
  \item \(\ffalse : \CTm\ \Ga\ \bool\);
  \item given \(b : \CTm\ \Ga\ \bool\), \(C : \CTy\ (\Ga, \bool)\),
  \(t : \CTm\ \Ga\ C[\cid, \ttrue]\) and \(f : \CTm\ \Ga\ C[\cid, \ffalse]\),
  some \(\tif{b}{C}{t}{f} : \CTm\ \Ga\ C[\cid, b]\)
\end{itemize}
such that
\[
  \begin{array}{rcl}
    \bool[\sigma] &=& \bool \\
    \ttrue[\sigma] &=& \ttrue \\
    \ffalse[\sigma] &=& \ffalse \\
    \left(\fitif{b}{C}{t}{f}\right)[\sigma] &=&
    \fitif{b[\sigma]}{C[(\sigma \circ \pp), \pq]}{t[\sigma]}{f[\sigma]} \\
    \tif{\ttrue}{C}{t}{f} &=& t \\
    \tif{\ffalse}{C}{t}{f} &=& f
  \end{array}
\]

\paradot{Natural numbers}

A \acrshort{CwF} has natural numbers when, given \(\Ga : \Con\), it has
\marginnote[0.1cm]{
  For the successor I use \(\Pi\)-types when I could have instead put things
  in the context. Both are valid and I'm often favorable to the idea that things
  should be kept separate as much as possible, but the reliance on application
  and everything allows us to avoid proving the same things several times,
  especially since manipulating contexts is so tedious with categories.
}
\begin{itemize}
  \item \(\nat : \CTy\ \Ga\);
  \item \(\czero : \CTm\ \Ga\ \nat\);
  \item for \(n : \CTm\ \Ga\ \nat\), \(\csucc\ n : \CTm\ \Ga\ \nat\);
  \item for \(n : \CTm\ \Ga\ \nat\), \(C : \CTy\ (\Ga, \nat)\),
  \(z : \CTm\ \Ga\ C[\cid, \czero]\) and
  \(s : \CTm\ \Ga\ (\Pi\ \nat\ \Pi\ C\ C[(\pp, \csucc\ \pq) \circ \pp])\),
  there exists \(\natrec_C\ n\ z\ s : \CTm\ \Ga\ C[\cid, n]\)
\end{itemize}
such that
\[
  \begin{array}{rcl}
    \nat[\sigma] &=& \nat \\
    \czero[\sigma] &=& \czero \\
    (\csucc\ n)[\sigma] &=& \csucc\ n[\sigma] \\
    (\natrec_C\ n\ z\ s) &=&
    \natrec_{C[(\sigma \circ \pp), \pq]}\ n[\sigma]\ z[\sigma]\ s[\sigma] \\
    \natrec_C\ \czero\ z\ s &=& z \\
    \natrec_C\ (\csucc\ n)\ z\ s &=& \capp(\capp(s,n), \natrec_C\ n\ z\ s)
  \end{array}
\]

\paradot{Tarski universes}

A Tarski universe is given by
\begin{itemize}
  \item a type \(\CU : \CTy\ \Ga\);
  \item for \(a : \CTm\ \Ga\ \CU\), a type \(\CEl(a) : \CTy\ \Ga\);
  \item for \(a : \CTm\ \Ga\ \CU\), \(b : \CTm\ \Ga\ (\Pi\ \CEl(a)\ \CU)\),
  a term \(\pi(a,b) : \CTm\ \Ga\ \CU\)
\end{itemize}
such that
\[
  \begin{array}{rcl}
    \CU[\sigma] &=& \CU \\
    \CEl(a)[\sigma] &=& \CEl(a[\sigma]) \\
    \CEl(\pi(a,b)) &=& \Pi\ \CEl(a)\ \CEl(\capp(b[\pp], \pq))
  \end{array}
\]
This makes the universe closed under dependent types, we can of course extend
this notion to include other constructions.

All of these can be defined in the \(\Set\) model above. They are however all
simpler cases than the \(\Card\) model I will give in the next section so I
won't detail them.

\subsection{Cardinal model}
\labsubsec{card-model}

In the previous section I briefly presented the \(\Set\) model. Here I will
detail the cardinal model of type theory that we initially wrote with Andrej
Bauer. It teaches us interesting things about syntax and especially about
\acrshort{ETT}.

For every set \(X\) there is a unique cardinal \(\card{X}\) such that
\(X \cong \card{X}\). We choose a bijection \(\beta_X : \setfun{X}{\card{X}}\).
I will write \(\beta\) for \(\beta_X\) when the \(X\) is understood.

The \(\Card\) \acrshort{CwF} is defined very similarly to \(\Set\):
\begin{enumerate}
  \item objects are cardinals;
  \item substitutions are given as set-theoretic functions, composed as
  functions;
  \item types in \(\CTy\ \Ga\) are cardinals indexed by \(\Ga\);
  \item terms of \(\CTm\ \Ga\ A\) are choice functions
  \(t \in \setfun{\Ga}{\bigcup_{\gamma \in \Ga} A_\gamma}\) such that
  \(t(\gamma) \in A_\gamma\) for every \(\gamma \in \Ga\);
  \item given \(\sigma : \Ga \to \D\) and \(A : \CTy\ \D\), \(A[\sigma]\)
  is defined as the \(\Ga\)-index family \((A_{\sigma(\gamma)})_\gamma\);
  \item given \(\sigma : \Ga \to \D\) and \(t : \CTm\ \D\ A\), \(t[\sigma]\)
  is defined as the choice function \(t \circ \sigma\);
  \item the empty context is given as the cardinal \(1\);
  \item given \(\Ga : \Con\) and \(A : \CTy\ \Ga\), \(\Ga, A\) is defined as
  the cardinal of the disjoint union
  \(\card{\amalg_{\gamma \in \Ga} A_\gamma}\), \(\pp\) is defined as
  \(\pi_1 \circ \beta^{-1}\) where \(\pi_1\) is the first projection of the
  disjoint union, \(\pq\) is defined as \(\pi_2 \circ \beta^{-1}\);
  \item given \(\sigma : \Ga \to \D\) and \(a : \CTm\ \Ga\ A[\sigma]\),
  \(\sigma, a\) is defined as the function
  \(\gamma \mapsto \beta(\sigma(\gamma), a(\gamma))\).
\end{enumerate}

Let us verify the coherence conditions:
\[
  \begin{array}{rcl}
    \pp \circ (\sigma, a) &=& \sigma \\
    \pq [\sigma, a] &=& a
  \end{array}
\]
\marginnote[0.8cm]{
  Notice how I verify equality of functions extensionally, that is because we
  are considering set-theoretic functions here.
}
Given \(\sigma : \Ga \to \D\), \(a : \CTm\ \Ga\ A[\sigma]\) and
\(\gamma \in \Ga\) we have
\[
  \begin{array}{lclcl}
    (\pp \circ (\sigma, a))(\gamma) &=&
    \pp((\sigma, a)(\gamma)) \\
    &=& \pp(\beta(\sigma(\gamma),a(\gamma))) \\
    &=& (\pi_1 \circ \beta^{-1})(\beta(\sigma(\gamma),a(\gamma))) \\
    &=& \pi_1(\beta^{-1}(\beta(\sigma(\gamma),a(\gamma)))) \\
    &=& \pi_1(\sigma(\gamma), a(\gamma)) &=& \sigma(\gamma)
  \end{array}
\]
and
\[
  \begin{array}{lclcl}
    (\pq [\sigma, a])(\gamma) &=&
    (\pq \circ (\sigma, a))(\gamma) \\
    &=& \pq((\sigma,a)(\gamma)) \\
    &=& \pq(\beta(\sigma(\gamma),a(\gamma))) \\
    &=& (\pi_2 \circ \beta^{-1})(\beta(\sigma(\gamma),a(\gamma))) \\
    &=& \pi_2(\beta^{-1}(\beta(\sigma(\gamma),a(\gamma)))) \\
    &=& \pi_2(\sigma(\gamma),a(\gamma)) &=& a(\gamma)
  \end{array}
\]

I will now echo the previous section and show that \(\Card\) admits all the
constructions I introduced then.

\paradot{Dependent products}

\sidedef[-0.7cm]{Dependent product of sets}{
  \(\prod_{x \in A} B(x)\) is defined as the set of functions \(f\) from \(A\)
  to \(\bigcup_{x \in A} B(x)\) such that \(f(x) \in B(x)\).
}
Given \(A : \CTy\ \Ga\) and \(B : \CTy\ (\Ga, A)\) we define
\(\Pi\ A\ B : \CTy\ \Ga\) as
\[
  (\Pi\ A\ B)(\gamma) \coloneqq \card{\prod_{x \in A(\gamma)} B(\beta(\gamma,x))}
\]
We define abstraction and application as follows:
\marginnote[0.7cm]{
  With \(b : \CTm\ (\Ga,A)\ B\), \(f : \CTm\ \Ga\ (\Pi\ A\ B)\) and
  \(u : \CTm\ \Ga\ A\).
}
\[
  \begin{array}{lcl}
    (\lambda b)(\gamma) &\coloneqq& \beta(x \mapsto b(\beta(\gamma,x))) \\
    \capp(f,u)(\gamma) &\coloneqq& \beta^{-1}(f(\gamma))(u(\gamma))
  \end{array}
\]
Now we show
\[
  \begin{array}{rcl}
    (\Pi\ A\ B)[\sigma] &=& \Pi\ A[\sigma]\ B[(\sigma \circ \pp), \pq] \\
    (\lambda b)[\sigma] &=& \lambda b[(\sigma \circ \pp), \pq] \\
    (\capp(f,u))[\sigma] &=& \capp(f[\sigma],u[\sigma]) \\
    \capp(\lambda b, u) &=& b[\cid, u] \\
    \lambda \capp(f \circ \pp, \pq) &=& f
  \end{array}
\]

\[
  \begin{array}{rcl}
    (\Pi\ A[\sigma]\ B[(\sigma \circ \pp), \pq])(\gamma)
    &=& \card{
      \prod_{x \in A[\sigma](\gamma)}
      B[(\sigma \circ \pp), \pq](\beta(\gamma, x))
    } \\
    &=& \card{
      \prod_{x \in A(\sigma(\gamma))}
      B(((\sigma \circ \pp), \pq)(\beta(\gamma, x)))
    } \\
    &=& \card{\prod_{x \in A(\sigma(\gamma))} B(\beta(\sigma(\gamma),x))} \\
    &=& (\Pi\ A\ B)(\sigma(\gamma)) \\
    &=& (\Pi\ A\ B)[\sigma](\gamma) \\
    \\
    (\lambda b[(\sigma \circ \pp), \pq])(\gamma)
    &=& \beta(x \mapsto b[(\sigma \circ \pp), \pq](\beta(\gamma,x))) \\
    &=& \beta(x \mapsto b(((\sigma \circ \pp), \pq)(\beta(\gamma,x)))) \\
    &=& \beta(x \mapsto b(\beta(\sigma(\gamma), x))) \\
    &=& (\lambda b)(\sigma(\gamma)) \\
    &=& (\lambda b)[\sigma](\gamma) \\
    \\
    (\capp(f,u))[\sigma](\gamma)
    &=& \capp(f,u)(\sigma(\gamma)) \\
    &=& \beta^{-1}(f(\sigma(\gamma)))(u(\sigma(\gamma))) \\
    &=& \beta^{-1}(f[\sigma](\gamma))(u[\sigma](\gamma)) \\
    &=& \capp(f[\sigma], u[\sigma])(\gamma) \\
    \\
    \capp(\lambda b, u)(\gamma)
    &=& \beta^{-1}((\lambda b)(\gamma))(u(\gamma)) \\
    &=& \beta^{-1}(\beta(x \mapsto b(\beta(\gamma,x))))(u(\gamma)) \\
    &=& (x \mapsto b(\beta(\gamma,x))(u(\gamma)) \\
    &=& b(\beta(\gamma, u(\gamma))) \\
    &=& b[\cid, u](\gamma) \\
    \\
    (\lambda \capp(f \circ \pp, \pq))(\gamma)
    &=& \beta(x \mapsto \capp(f \circ \pp, \pq)(\beta(\gamma,x))) \\
    &=& \beta(x \mapsto \beta^{-1}((f \circ \pp)(\beta(\gamma,x)))(\pq(\beta(\gamma,x))))) \\
    &=& \beta(x \mapsto \beta^{-1}(f(\gamma))(x))) \\
    &=& \beta(\beta^{-1}(f(\gamma))) \\
    &=& f(\gamma)
  \end{array}
\]
\marginnote[-13.2cm]{
  In the argument of \(B\) what happens is
  \[
    \begin{array}{ll}
      \multicolumn{2}{l}{((\sigma \circ \pp), \pq)(\beta(\gamma, x))} \\
      =& \beta((\sigma \circ \pp)(\beta(\gamma, x)), \pq(\beta(\gamma, x))) \\
      =& \beta((\sigma(\pp(\beta(\gamma, x)))), \pq(\beta(\gamma, x))) \\
      =& \beta((\sigma(\pi_1(\gamma, x))), \pi_2(\gamma, x)) \\
      =& \beta(\sigma(\gamma), x)
    \end{array}
  \]
}


\paradot{Dependent sums}

\(\Card\) has dependent sums, as defined as
\[
  (\Sigma\ A\ B)(\gamma) \coloneqq
  \card{\amalg_{x \in A(\gamma)} B(\beta(\gamma,x))}
\]
with constructors and destructors:
\[
  \begin{array}{lcl}
    \dpair{a,b}(\gamma) &\coloneqq& \beta(a(\gamma), b(\gamma)) \\
    p.1 &\coloneqq& \pi_1 \circ \beta^{-1} \circ p \\
    p.2 &\coloneqq& \pi_2 \circ \beta^{-1} \circ p
  \end{array}
\]

We now show
\[
  \begin{array}{rcl}
    (\Sigma\ A\ B)[\sigma] &=& \Sigma\ A[\sigma]\ B[(\sigma \circ \pp), \pq] \\
    \dpair{a,b}[\sigma] &=& \dpair{a[\sigma], b[\sigma]} \\
    (p.1)[\sigma] &=& p[\sigma].1 \\
    (p.2)[\sigma] &=& p[\sigma].2 \\
    \dpair{a,b}.1 &=& a \\
    \dpair{a,b}.2 &=& b
  \end{array}
\]

\[
  \begin{array}{rcl}
    (\Sigma\ A[\sigma]\ B[(\sigma \circ \pp), \pq])(\gamma)
    &=& \card{\amalg_{x \in A[\sigma](\gamma)} B[(\sigma \circ \pp), \pq](\beta(\gamma,x))} \\
    &=& \card{\amalg_{x \in A(\sigma(\gamma))} B(\beta(\sigma(\gamma), x))} \\
    &=& (\Sigma\ A\ B)(\sigma(\gamma)) \\
    &=& (\Sigma\ A\ B)[\sigma](\gamma) \\
    \\
    \dpair{a[\sigma], b[\sigma]}(\gamma)
    &=& \beta(a(\sigma(\gamma)), b(\sigma(\gamma))) \\
    &=& \dpair{a,b}[\sigma](\gamma) \\
    \\
    (p[\sigma].1)(\gamma)
    &=& \pi_1(\beta^{-1}(p(\sigma(\gamma)))) \\
    &=& (p.1)[\sigma](\gamma) \\
    \\
    (p[\sigma].2)(\gamma)
    &=& \pi_2(\beta^{-1}(p(\sigma(\gamma)))) \\
    &=& (p.2)[\sigma](\gamma) \\
    \\
    (\dpair{a,b}.1)(\gamma)
    &=& \pi_1(\beta^{-1}(\dpair{a,b}(\gamma))) \\
    &=& \pi_1(\beta^{-1}(\beta(a(\gamma), b(\gamma)))) \\
    &=& a(\gamma) \\
    \\
    (\dpair{a,b}.2)(\gamma)
    &=& \pi_2(\beta^{-1}(\dpair{a,b}(\gamma))) \\
    &=& \pi_2(\beta^{-1}(\beta(a(\gamma), b(\gamma)))) \\
    &=& b(\gamma)
  \end{array}
\]

\paradot{Identity types}

Crucially, \(\Card\) has identity types.
Given \(A : \CTy\ \Ga\) and \(u, v : \CTm\ \Ga\ A\), we define
\(\CId_A\ u\ v : \CTy\ \Ga\) to be the type family
\[
  (\CId_A\ u\ v)(\gamma) \coloneqq \{ 0 \mid u(\gamma) = v(\gamma) \}
\]
The set \(\{ 0 \mid u(\gamma) = v(\gamma) \}\) is indeed a cardinal number,
namely \(0\) when \(u(\gamma) \not= v(\gamma)\) and \(1\) when
\(u(\gamma) = v(\gamma)\).
In fact, \(u(\gamma) = v(\gamma)\) is equivalent to
\(0 \in (\CId_A\ u\ v)(\gamma)\).
Thus, we define \(\crefl_u : \CTm\ \Ga\ (\CId_A\ u\ u)\) as
\[
  (\crefl_u)(\gamma) \coloneqq 0
\]
which is the only possible function.

We check
\[
  \begin{array}{rcl}
    (\CId_A\ u\ v)[\sigma] &=& \CId_{A[\sigma]}\ u[\sigma]\ v[\sigma] \\
    (\crefl_a)[\sigma] &=& \crefl_{a[\sigma]}
  \end{array}
\]

\[
  \begin{array}{rcl}
    \CId_{A[\sigma]}\ u[\sigma]\ v[\sigma]
    &=& \{ 0 \mid u[\sigma](\gamma) = v[\sigma](\gamma) \} \\
    &=& \{ 0 \mid u(\sigma(\gamma)) = v(\sigma(\gamma)) \} \\
    &=& (\CId_A\ u\ v)(\sigma(\gamma)) \\
    &=& (\CId_A\ u\ v)[\sigma](\gamma) \\
    \\
    \crefl_{a[\sigma]}(\gamma)
    &=& 0 \\
    &=& (\crefl_a)(\sigma(\gamma)) \\
    &=& (\crefl_a)[\sigma](\gamma)
  \end{array}
\]

Now that we have equality I will why we don't need an eliminator in this
particular case.

\begin{theorem}
  The cardinal model validates equality reflection.
\end{theorem}
\reminder[-1.6cm]{Reflection rule}{
  \begin{equation*}
    \infer[]
      {\isterm{\Ga}{e}{\Eq{A}{u}{v}}}
      {\eqterm{\Ga}{u}{v}{A}}
    %
  \end{equation*}
  See \refdef{reflection}
}%
%
\begin{proof}
  Assuming that we have \(p : \CTm\ \Ga\ (\CId_A\ u\ v)\) we want to show that
  \(u\) and \(v\) are equal choice functions, \ie that \(u(gamma) = v(\gamma)\)
  for all \(\gamma \in \Ga\).
  Because \(0 = p(\gamma) \in (\CId_A\ u\ v)(\gamma)\), it follows that
  \(u(\gamma) = v(\gamma)\).
\end{proof}

The cardinal model is even stronger than that in that it validates isomorphism
reflection.
An isomorphism \(A \cong B\) is given by two functions \(f : A \to B\)
and \(g : B \to A\) that are inverses of each other (for the type-theoretic
equality).

\begin{definition}[Isomorphism reflection]
  Isomorphism reflection states that isomorphic types are \emph{convertible}:
  \[
    \infer
      {\Ga \vdash e : A \cong B}
      {\Ga \vdash A \equiv B}
    %
  \]
\end{definition}

This consists in a strong invariance principle stating that isomorphic types
cannot be distinguished. Another invariance principle is the univalence axiom.

\begin{theorem}
  The cardinal model validates isomorphism reflection.
\end{theorem}

\begin{proof}
  Assume we are given isomorphic type \(A, B : \CTy\ \Ga\), that is two
  functions \(f : \CTm\ \Ga\ (\Pi\ A\ B[\pp])\) and
  \(g : \CTm\ \Ga\ (\Pi\ B\ A[\pp])\) as well as two equalities showing they
  are mutual inverses:
  \[
    \begin{array}{l}
      p :
      \CTm\ \Ga\
      (\Pi\ A\ \CId_{A[\pp]}\ \capp(g[\pp], \capp(f[\pp], \pq))\ \pq) \\
      q :
      \CTm\ \Ga\
      (\Pi\ B\ \CId_{B[\pp]}\ \capp(f[\pp], \capp(g[\pp], \pq))\ \pq)
    \end{array}
  \]

  We want to show equality of \(A\) and \(B\) that is \(A(\gamma) = B(\gamma)\)
  for every \(\gamma \in \Ga\). Given such \(\gamma\), we have
  \(f(\gamma) \in \card{\prod_{x \in A(\gamma)} B[\pp](\beta(\gamma,x))}\)
  which is to say \(f(\gamma) \in \card{\setfun{A(\gamma)}{B(\gamma)}}\),
  similarly \(g(\gamma) \in \card{\setfun{B(\gamma)}{A(\gamma)}}\).
  We thus have two functions
  \[
    \begin{array}{rcl}
      \beta^{-1}(f(\gamma)) &\in& \setfun{A(\gamma)}{B(\gamma)} \\
      \beta^{-1}(g(\gamma)) &\in& \setfun{B(\gamma)}{A(\gamma)}
    \end{array}
  \]
  We show they are in bijection using \(p\) and \(q\):
  \[
    \begin{array}{rcl}
      \beta^{-1}(p(\gamma)) &\in&
      \prod_{a \in A(\gamma)}
      \{ 0 \mid \beta^{-1}(g(\beta^{-1}(f(\gamma)))) = a \} \\
      \beta^{-1}(q(\gamma)) &\in&
      \prod_{b \in B(\gamma)}
      \{ 0 \mid \beta^{-1}(f(\beta^{-1}(g(\gamma)))) = b \}
    \end{array}
  \]
  Since \(A(\gamma)\) and \(B(\gamma)\) are cardinals in bijection, they are
  necessarily equal: \(A(\gamma) = B(\gamma)\).
\end{proof}

\reminder[-0.7cm]{Injectivity of \(\Pi\)-types}{
  Whenever \(\Pi (x:A).B \equiv \Pi (x:A').B'\)
  then we have both \(A \equiv A'\) and \(B \equiv B'\).
}
This means that the cardinal model validates strange equalities like that of
Cantor space and Baire space \(2^{\mathbb{N}} = \mathbb{N}^{\mathbb{N}}\)
which violates injectivity of \(\Pi\)-types.

All this proves that in type theory, one cannot distinguish isomorphic types.

\paradot{Booleans}

\(\Card\) has booleans, defined as
\[
  \begin{array}{rcl}
    \bool(\gamma) &=& 2 \\
    \ttrue(\gamma) &=& 1 \\
    \ffalse(\gamma) &=& 0 \\
    (\tif{b}{C}{t}{f})(\gamma) &=&
    \left\{
    \begin{array}{ll}
      t(\gamma) & \text{if \(b(\gamma) = 1\)} \\
      f(\gamma) & \text{otherwise}
    \end{array}
    \right.
  \end{array}
\]

The equations are
\[
  \begin{array}{rcl}
    \bool[\sigma] &=& \bool \\
    \ttrue[\sigma] &=& \ttrue \\
    \ffalse[\sigma] &=& \ffalse \\
    \left(\fitif{b}{C}{t}{f}\right)[\sigma] &=&
    \fitif{b[\sigma]}{C[(\sigma \circ \pp), \pq]}{t[\sigma]}{f[\sigma]} \\
    \tif{\ttrue}{C}{t}{f} &=& t \\
    \tif{\ffalse}{C}{t}{f} &=& f
  \end{array}
\]

The first three hold trivially. I prove the remaining below.
\[
  \begin{array}{rcl}
    \left(
      \fitif{b[\sigma]}{C[(\sigma \circ \pp), \pq]}{t[\sigma]}{f[\sigma]}
    \right)(\gamma)
    &=&
    \left\{
    \begin{array}{ll}
      t(\sigma(\gamma)) & \text{if \(b(\sigma(\gamma)) = 1\)} \\
      f(\sigma(\gamma)) & \text{otherwise}
    \end{array}
    \right. \\
    &=& \left(\fitif{b}{C}{t}{f}\right)[\sigma](\gamma) \\
    \\
    \tif{\ttrue}{C}{t}{f}
    &=&
    \left\{
    \begin{array}{ll}
      t(\gamma) & \text{if \(\ttrue(\gamma) = 1\)} \\
      f(\gamma) & \text{otherwise}
    \end{array}
    \right. \\
    &=& t(\gamma) \\
    \\
    \tif{\ffalse}{C}{t}{f}
    &=&
    \left\{
    \begin{array}{ll}
      t(\gamma) & \text{if \(\ffalse(\gamma) = 1\)} \\
      f(\gamma) & \text{otherwise}
    \end{array}
    \right.\\
    &=& f(\gamma)
  \end{array}
\]

\paradot{Natural numbers}

We define natural numbers as the cardinal of \(\mathbb{N}\):
\[
  \nat(\gamma) \coloneqq \omega
\]
Zero is the constant function
\[
  \czero(\gamma) \coloneqq 0
\]
while the successor of a natural number is defined as
\[
  (\csucc\ n)(\gamma) \coloneqq n(\gamma) + 1
\]
The eliminator is a bit more complex but not too surprising:
% \[
%   (\natrec_C\ n\ z\ s)(\gamma) \coloneqq
%   \left\{
%   \begin{array}{ll}
%     z(\gamma) & \text{when \(n(\gamma) = 0\)} \\
%     \capp(\capp(s, (x \mapsto m)), \natrec_C\ (x \mapsto m)\ z\ s)(\gamma) &
%     \text{when \(n(\gamma) = m + 1\)}
%   \end{array}
%   \right.
% \]
\[
  \begin{array}{l}
    (\natrec_C\ n\ z\ s)(\gamma) \coloneqq
    \left\{
    \begin{array}{ll}
      z(\gamma) & \text{when \(n(\gamma) = 0\)} \\
      f(\gamma) &
      \text{when \(n(\gamma) = m + 1\)}
    \end{array}
    \right. \\
    \text{where } f \coloneqq
    \capp(\capp(s, (x \mapsto m)), \natrec_C\ (x \mapsto m)\ z\ s)
  \end{array}
\]

The equalities are fairly straightforward.
\[
  \begin{array}{rcl}
    \nat[\sigma] &=& \nat \\
    \czero[\sigma] &=& \czero \\
    (\csucc\ n)[\sigma] &=& \csucc\ n[\sigma] \\
    (\natrec_C\ n\ z\ s)[\sigma] &=&
    \natrec_{C[(\sigma \circ \pp), \pq]}\ n[\sigma]\ z[\sigma]\ s[\sigma] \\
    \natrec_C\ \czero\ z\ s &=& z \\
    \natrec_C\ (\csucc\ n)\ z\ s &=& \capp(\capp(s,n), \natrec_C\ n\ z\ s)
  \end{array}
\]

\paradot{Tarski universes}

\sidedef[-0.7cm]{Inaccessible cardianl}{
  A cardinal \(\kappa\) is inaccessible if it is uncountable, it is not a sum
  of fewer than \(\kappa\) cardinals that are smaller than \(\kappa\), and is
  closed under exponentials (or power sets).
}
If \(\kappa\) is an inaccessible cardinal then we get a universe by taking
\[
  \begin{array}{lcl}
    \CU(\gamma) &\coloneqq& \kappa \\
    \CEl(a)(\gamma) &\coloneqq& \card{a(\gamma)} \\
    \pi(a,b)(\gamma) &\coloneqq&
    \prod_{x \in \card{a(\gamma)}} \card{\beta^{-1}(b(\gamma))(x)}
  \end{array}
\]

\marginnote[2cm]{
  This definition works because inaccessible cardinals are fixed-points of
  the \(\aleph\) function.
}
A perhaps more interesting universe is given by using the \(\aleph\) function:
\[
  \begin{array}{lcl}
    \CU(\gamma) &\coloneqq& \kappa \\
    \CEl(a)(\gamma) &\coloneqq& \aleph_{a(\gamma)} \\
    \pi(a,b) &\coloneqq& \mu
  \end{array}
\]
where
\(\aleph_\mu = \prod_{x \in \CEl{a(\gamma)}} \CEl{\beta^{-1}(b(\gamma))(x)}\).

I won't really go into detail about cardinals and everything here, or as to
why this universe also supports natural numbers, equality and such.

\paradot{A warning about syntax}

Together all these show that the cardinal model of type theory is indeed a model
of type theory with some quirks, teaching us that isomorphic objects can't be
distinguished.
There is another important teaching it brings us about syntax.

Once we have reflection, we have to be careful how we deal with
\(\beta\)-reduction. In the cardinal model we indeed have that
\(\nat \to \bool\) and \(\nat \to \nat\) are \emph{equal} types.
\marginnote[0.1cm]{
  \(A \to B\) will be a shorthand for \(\Pi\ A\ B[\pp]\).
}
As such the identity function \(\lambda \pq : \CTm\ \ctxempty\ (\nat \to \nat)\)
can be typed \(\lambda \pq : \CTm\ \ctxempty\ (\nat \to \bool)\).
Since \(\ctwo \coloneqq \csucc\ (\csucc\ \czero) : \CTm\ \ctxempty\ \nat\)
we have
\[
  \capp(\lambda \pq, \csucc\ (\csucc\ \czero)) : \CTm\ \ctxempty\ \bool
\]
If we allow it to \(\beta\)-reduce we end up with
\(\ctwo : \CTm\ \ctxempty\ \bool\) which does not hold as
\[
  \ctwo(\gamma) = 2 \not\in 2 = \bool(\gamma)
\]
for any \(\gamma \in \ctxempty\).
\marginnote[0cm]{
  In fact there is only one such \(\gamma\): \(0\).
}
Didn't we prove that \(\beta\)-reduction held however?
Let us replay the proof of \(\beta\)-reduction above in our specific case:
\[
  \begin{array}{rcl}
    \capp(\lambda \pq, \ctwo)(\gamma)
    &=& \beta^{-1}((\lambda \pq)(\gamma))(\ctwo(\gamma)) \\
    &=& \beta^{-1}(\beta(x \mapsto \pq(\beta(\gamma,x))))(\ctwo(\gamma)) \\
    &=& (x \mapsto \pq(\beta(\gamma,x))(\ctwo(\gamma)) \\
    &=& \pq(\beta(\gamma, \ctwo(\gamma))) \\
    &=& \pq[\cid, \ctwo](\gamma)
  \end{array}
\]
Simplifying away \(\pq\) and \(\ctwo\) this becomes
\[
  \begin{array}{rcl}
    \capp(\lambda \pq, \ctwo)(\gamma)
    &=& \beta^{-1}((\lambda \pq)(\gamma))(2) \\
    &=& \beta^{-1}(\beta(x \mapsto x))(2) \\
    &=& (x \mapsto x)(2) \\
    &=& 2 \\
    &=& \pq[\cid, \ctwo](\gamma)
  \end{array}
\]
The problematic part is the equality
\[
  \beta^{-1}(\beta(x \mapsto x))(2) = (x \mapsto x)(2)
\]
where simplify the expression \(\beta^{-1} \circ \beta\) to the identity
function. As usual I left the subscripts implicit, but that's where the error
is coming from: we actually have
\[
  \beta_{2^\omega}^{-1} \circ \beta_{\omega^\omega}
\]
which do not cancel each other out.

This is case for using type annotations on \(\lambda\)-abstractions and
applications to block \(\beta\)-reduction when the types don't match.

\section{Type-theoretic models}

\todo{Cat models fall in this too}
\todo{translations, syntactical models, standard model}
% \setchapterpreamble[u]{\margintoc}
\chapter{Syntax and formalisation of type theory}
\labch{formalisation}

An interesting fact of type theory (and perhaps one its main selling points) is
that it is a suitable framework in which to reason about type theory.
Representing type theory in itself isn't entirely straightforward, and some care
must be taken. There are actually several choices to be made when representing
type theory and they are not all equivalent or with the same pros and cons.
I will detail some of them, spending more time on those I ended up choosing
and will try to motivate my choice.
% \setchapterpreamble[u]{\margintoc}
\chapter{Translations}
\labch{translations}

Syntactical translations are a special case of program transformations suited
for type theory, and a method of choice to get models of type theories.
This is better studied in Simon Boulier's thesis and next 700~\misref but in
order to keep this document self contained, I will do my best to give a
meaningful excerpt here.

\pagelayout{wide} % No margins
\addpart{Elimination of Reflection}
\labpart{elim-reflection}
\pagelayout{margin} % Restore margins

% \setchapterpreamble[u]{\margintoc}
\chapter{What I mean by elimination of reflection}
\labch{elim-reflection-intro}

We presented earlier \acrshort{ETT} and its defining reflection rule.
%
\reminder[-1.0cm]{Reflection rule}{
  \begin{equation*}
    \infer[]
      {\xisterm{\Ga}{e}{\Eq{A}{u}{v}}}
      {\xeqterm{\Ga}{u}{v}{A}}
    %
  \end{equation*}
}
%
The next few chapters are going to be dedicated to its elimination from type
theory: that is how to make a translation from \acrshort{ETT} to type theories
that do not feature reflection.

This work gave rise to a publication~\sidecite{winterhalter:hal-01849166} that
focused on translating \acrshort{ETT} to \acrshort{ITT}.
However, with Simon Boulier we worked on another version translating directly to
\acrshort{WTT} which I'm going to present here.

\section{No syntactical translation}

\todo{Have reminders clickable to refer to actual definition.}
\todo{Use the notation table to include the notation of typing etc. especially
the ETT mark.}

First of all we have to wonder about what kind of translation is possible.
We presented earlier~\misref{} the notion of syntactical translation.
Unfortunately it is not possible to devise a syntactical translation to
eliminate reflection from type theory.

Assume we have such a translation given by \(\transl{.}\) and \([.]\) such that
whenever \(\xisterm{\Gamma}{t}{A}\) we have
\(\isterm{\transl{\Gamma}}{[t]}{\transl{A}}\) in the target type theory.
Now let's say we have an inconsistent context in \acrshort{ETT}: \(\Gamma_\bot\)
(one can for instance assume \(0 = 1\) or \(\forall A, A\)), in such a context
anything can have any type because conversion has become trivial.
%
\marginnote[1.4cm]{
  Since \(\Gamma_\bot\) is inconsistent, anything can be proved from it,
  including the equality between the two types \(\mathbb{N}\) and \(\bot\).
  We then use reflection and conversion.
}
\begin{mathpar}
  \infer
    {
      \infer
        {\vdots}
        {\xisterm{\Gamma_\bot}{0}{\mathbb{N}}}
      \\
      \infer
        {
          \infer
            {\vdots}
            {\xisterm{\Gamma_\bot}{\_}{\mathbb{N} = \bot}}
          %
        }
        {\xeqterm{\Gamma_\bot}{\mathbb{N}}{\bot}{\Type}}
    }
    {\xisterm{\Gamma_\bot}{0}{\bot}}
  %
\end{mathpar}
%
It thus follows that in the target you have
\( \isterm{\transl{\Gamma_\bot}}{[0]}{\transl{\bot}} \).
Similarly you would have
\( \isterm{\transl{\Gamma_\bot}}{[0]}{\transl{\mathbb{N}}} \). This means
that both \(\bot\) and \(\mathbb{N}\) should be translated to similar things
(to convertible types in case the target theory has uniqueness of type), without
being able to exploit the knowledge that \(\Gamma_\bot\) is inconsistent
because of the syntactical nature of the translation.

\todo{Make things clearer}
Even worse, translations should preserve falsehood meaning in particular that
the translation of \(0\) should imply a proof of \(\bot\) in the target.
This is not a concrete proof that it is impossible but rather an argument
to see that such a translation would not behave well. One of the reasons is
that it would translate terms, types and contexts independently when it cannot.
Another is that an \acrshort{ETT} term does not contain any hints with respect
to the uses of reflection.
% \setchapterpreamble[u]{\margintoc}
\chapter{Framework}
\labch{elim-reflection-framework}
% \setchapterpreamble[u]{\margintoc}
\chapter{Relating translated expressions}
\labch{elim-rel}

We want to define a relation on terms that equates two terms that are
the same up to transport.
%
This begs the question of what notion of transport is going to be
used.
%
Transport can be defined from elimination of equality as in \refch{usual-defs}.
However, in order not to confuse the transports added by the
translation with the transports that were already present in the
source, we consider $\transpo{p}$---\ie the transports added by the
translation---as part of the syntax in the reasoning. It will be unfolded to its
definition only after the complete translation is performed.
%
This idea is not novel as Hofmann already had a $\mathsf{Subst}$ operator that
was part of his \acrshort{ITT} (written TT\textsubscript{I} in his
paper~\sidecite{hofmann1995conservativity}).

\section{Relating terms and their translation}

We first define the (purely syntactic) relation $\ir$ between \acrshort{ETT}
terms and target terms by saying the translated term must be a decoration of the
first term by transports.
Its purpose is to state how close to the original term its translation is.
% In the definition, the most important rule is highlighted.

\marginnote[1cm]{
  As you can see, this relation does not talk about \acrshort{ITT} or
  \acrshort{WTT} specific terms.
}
\begin{mathpar}

  \highlight{
    \infer[]
      {t_1 \ir t_2}
      {t_1 \ir \transpo{p}\ t_2}
    %
  }

  \infer[]
    { }
    {x \ir x}
  %

  \infer[]
    {A_1 \ir A_2 \\
     B_1 \ir B_2
    }
    {\Prod{x:A_1} B_1 \ir \Prod{x:A_2} B_2}
  %

  \infer[]
    {A_1 \ir A_2 \\
     B_1 \ir B_2
    }
    {\Sum{x:A_1} B_1 \ir \Sum{x:A_2} B_2}
  %

  \infer[]
    {A_1 \ir A_2 \\
     u_1 \ir u_2 \\
     v_1 \ir v_2
    }
    {\Eq{A_1}{u_1}{v_1} \ir \Eq{A_2}{u_2}{v_2}}
  %

  \infer[]
    { }
    {\ax{n} \ir \ax{n}}
  %

  \infer[]
    { }
    {s \ir s}
  %

  \infer[]
    {A_1 \ir A_2 \\
     B_1 \ir B_2 \\
     t_1 \ir t_2
    }
    {\lam{x:A_1}{B_1} t_1 \ir \lam{x:A_2}{B_2} t_2}
  %

  \infer[]
    {t_1 \ir t_2 \\
     A_1 \ir A_2 \\
     B_1 \ir B_2 \\
     u_1 \ir u_2
    }
    {\app{t_1}{x:A_1}{B_1}{u_1} \ir \app{t_2}{x:A_2}{B_2}{u_2}}
  %

  \infer[]
    {A_1 \ir A_2 \\
     B_1 \ir B_2 \\
     t_1 \ir t_2 \\
     u_1 \ir u_2
    }
    {\pair{x:A_1}{B_1}{t_1}{u_1} \ir \pair{x:A_2}{B_2}{t_2}{u_2}}
  %

  \infer[]
    {A_1 \ir A_2 \\
     B_1 \ir B_2 \\
     p_1 \ir p_2
    }
    {\pio{x:A_1}{B_1}{p_1} \ir \pio{x:A_2}{B_1}{p_2}}
  %

  \infer[]
    {A_1 \ir A_2 \\
     B_1 \ir B_2 \\
     p_1 \ir p_2
    }
    {\pit{x:A_1}{B_1}{p_1} \ir \pit{x:A_2}{B_2}{p_2}}
  %

  \infer[]
    {A_1 \ir A_2 \\
     u_1 \ir u_2
    }
    {\refl{A_1} u_1 \ir \refl{A_2} u_2}
  %

  \infer
    {
      A_1 \ir A_2 \\
      u_1 \ir u_2 \\
      P_1 \ir P_2 \\
      w_1 \ir w_2 \\
      v_1 \ir v_2 \\
      p_1 \ir p_2
    }
    {
      \J{A_1}{u_1}{x.e.P_1}{w_1}{v_1}{p_1} \ir
      \J{A_2}{u_2}{x.e.P_2}{w_2}{v_2}{p_2}
    }
  %
\end{mathpar}

From this relation we build a new one (\(\sim\)) between translated terms.
\(u \sim v\) basically means that \(u\) and \(v\) are both decorations of the
same term.

\[
  \sim\ \coloneqq\ \sqsupset^+ . \ir^+
\]

In order to better reason about it however, we actually define it inductively
again.

\begin{mathpar}
  \highlight{
    \infer[]
      {t_1 \sim t_2}
      {\transpo{p}\ t_1 \sim t_2}
    %
  }

  \highlight{
    \infer[]
      {t_1 \sim t_2}
      {t_1 \sim \transpo{p}\ t_2}
    %
  }

  \infer[]
    {A_1 \sim A_2 \\
     B_1 \sim B_2
    }
    {\Prod{x:A_1} B_1 \sim \Prod{x:A_2} B_2}
  %

  \infer[]
    {A_1 \sim A_2 \\
     B_1 \sim B_2
    }
    {\Sum{x:A_1} B_1 \sim \Sum{x:A_2} B_2}
  %

  \infer[]
    {A_1 \sim A_2 \\
     u_1 \sim u_2 \\
     v_1 \sim v_2
    }
    {\Eq{A_1}{u_1}{v_1} \sim \Eq{A_2}{u_2}{v_2}}
  %

  \infer[]
    { }
    {\ax{n} \sim \ax{n}}
  %

  \infer[]
    { }
    {s \sim s}
  %

  \infer[]
    {A_1 \sim A_2 \\
     B_1 \sim B_2 \\
     t_1 \sim t_2
    }
    {\lam{x:A_1}{B_1} t_1 \sim \lam{x:A_2}{B_2} t_2}
  %

  \infer[]
    {t_1 \sim t_2 \\
     A_1 \sim A_2 \\
     B_1 \sim B_2 \\
     u_1 \sim u_2
    }
    {\app{t_1}{x:A_1}{B_1}{u_1} \sim \app{t_2}{x:A_2}{B_2}{u_2}}
  %

  \infer[]
    {A_1 \sim A_2 \\
     B_1 \sim B_2 \\
     t_1 \sim t_2 \\
     u_1 \sim u_2
    }
    {\pair{x:A_1}{B_1}{t_1}{u_1} \sim \pair{x:A_2}{B_2}{t_2}{u_2}}
  %

  \infer[]
    {A_1 \sim A_2 \\
     B_1 \sim B_2 \\
     p_1 \sim p_2
    }
    {\pio{x:A_1}{B_1}{p_1} \sim \pio{x:A_2}{B_1}{p_2}}
  %

  \infer[]
    {A_1 \sim A_2 \\
     B_1 \sim B_2 \\
     p_1 \sim p_2
    }
    {\pit{x:A_1}{B_1}{p_1} \sim \pit{x:A_2}{B_2}{p_2}}
  %

  \infer[]
    {A_1 \sim A_2 \\
     u_1 \sim u_2
    }
    {\refl{A_1} u_1 \sim \refl{A_2} u_2}
  %

  \infer
    {
      A_1 \sim A_2 \\
      u_1 \sim u_2 \\
      P_1 \sim P_2 \\
      w_1 \sim w_2 \\
      v_1 \sim v_2 \\
      p_1 \sim p_2
    }
    {
      \J{A_1}{u_1}{x.e.P_1}{w_1}{v_1}{p_1} \sim
      \J{A_2}{u_2}{x.e.P_2}{w_2}{v_2}{p_2}
    }
  %
\end{mathpar}

Once more, constructions specific to \acrshort{ITT} or \acrshort{WTT} are not
related with \(\sim\): these will only appear in the equalities along which
we transport (the \(p\) in the highlighted rules).

\section{Properties of the relation}

As I just remarked, the \(\sim\) relation is not reflexive; but only in that
respect does it fall short of being an equivalence relation.

\begin{lemma}[$\sim$ is a partial equivalence relation]
  \lablemma{sim-er}
  $\sim$ is symmetric and transitive.
\end{lemma}

% \begin{proof}
%   For reflexivity we proceed by induction on the term.
% \end{proof}

The goal is to prove that two terms in this relation, that are well-typed in the
target type theory, are heterogeneously equal.

Heterogeneous equality, as seen in \arefsubsec{hetero-eq-def} of
\refch{usual-defs}, is reflexive, symmetric and transitive.
Thanks to \acrshort{UIP}, heterogeneous equality collapses to regular equality
when taken on the same type on both sides.

\begin{remark}
  As a corollary, $\Heqs$ on types corresponds to equality.
  Indeed when we have $\isterm{\Ga}{e}{\Heq{s}{A}{s'}{B}}$ we have
  that $\Eq{}{s}{s'}$, which implies that $s$ and $s'$ have the same sort
  and thus are syntactically the same (by an inversion argument).
\end{remark}

For heterogeneous equality to be a congurence however we have to rely on
\acrshort{funext} in \acrshort{ITT}, and on the several equality constructors
we introduce in \acrshort{WTT} like \(\overline{\lambda}\), \(\overline{\Pi}\),
etc.

Before we can prove the fundamental lemma stating that two terms in relation
are heterogeneously equal, we need to consider another construction.
%
As explained earlier, when proving the property by induction on terms, we
introduce variables in the context that are equal only up to heterogeneous
equality.
%
This phenomenon is similar to what happens in the parametricity
translation~\sidecite{bernardy2012proofs}.
%
Our fundamental lemma on the decoration relation $\sim$ assumes two
related terms of potentially different types $T1$ and $T2$ to produce an
heterogeneous equality between them. For induction to go through under
binders (e.g. for dependent products and abstractions), we hence need to
consider the two terms under different, but heterogeneously equal
contexts.
%
Therefore, the context we produce will not only be a telescope of
variables, but rather a telescope of triples consisting of two variables
of possibly different types, and a witness that they are heterogeneously
equal.
%
To make this precise, we define the following macro:
%
\[
\Pack{A_1}{A_2} := \Sum{x:A_1} \Sum{y:A_2} \Heq{}{x}{}{y}
\]
together with its projections
\begin{mathpar}
  \ProjO{p} := \pio{}{}{p}

  \ProjT{p} := \pio{}{}{\pit{}{}{p}}

  \ProjE{p} := \pit{}{}{\pit{}{}{p}}.
\end{mathpar}
%
We can then extend this notion canonically to contexts of the same
length that are well formed using the same sorts:
\marginnote[-0,5cm]{
  That is \(\Ga_1 \coloneqq x_1 : A_1, \dots, x_n : A_n\),
  \(\Ga_2 \coloneqq y_1 : B_1, \dots, y_n : B_n\) where
  there exists \(s_i\) such that \(A_i : s_i\) and \(B_i : s_i\) for each \(i\).
}
%
\[
\begin{array}{l}
    \Pack{(\Ga_1, x:A_1)}{(\Ga_2, x:A_2)} := \\
    (\Pack{\Ga_1}{\Ga_2}),
    x : \Pack{(\llift{\gamma}{}{A_1})}{(\rlift{\gamma}{}{A_2})} \\
    \\
    \Pack{\ctxempty}{\ctxempty} := \ctxempty.
\end{array}
\]
%
When we pack contexts, we also need to apply the correct projections for
the types in that context to still make sense. Assuming two contexts
$\Ga_1$ and $\Ga_2$ of the same length, we can define left and right
substitutions:
\[
\begin{array}{ll}
  \gamma_1 &:= [ x \leftarrow \ProjO{x}\ |\ (x : \_) \in \Ga_1 ] \\
  \gamma_2 &:= [ x \leftarrow \ProjT{x}\ |\ (x : \_) \in \Ga_2 ].
\end{array}
\]
These substitutions implement lifting of terms to packed contexts:
we have
$\isterm{\Ga, \Pack{\Ga_1}{\Ga_2}}{\llift{\gamma}{}{t}}{\llift{\gamma}{}{A}}$
whenever $\isterm{\Ga, \Ga_1}{t}{A}$
and
$\isterm{\Ga, \Pack{\Ga_1}{\Ga_2}}{\rlift{\gamma}{}{t}}{\rlift{\gamma}{}{A}}$
whenever $\isterm{\Ga, \Ga_2}{t}{A}$.

For readability, when $\Ga_1$ and $\Ga_2$ are understood we will write $\Gp$ for
$\Pack{\Ga_1}{\Ga_2}$.

Implicitly, whenever we use the notation $\Pack{\Ga_1}{\Ga_2}$ it means that
the two contexts are of the same length and well-formed with the same
sorts.
%
We can now tackle the fundamental lemma.

\marginnote[1cm]{
  I use \(\vdash\) and not \(\vdash_\exmark\) in the judgements to indicate that
  these judgements are in the target, \ie \acrshort{ITT} or \acrshort{WTT}.
}
\begin{lemma}[Fundamental lemma]
  \lablemma{sim-cong}
  Let $t_1$ and $t_2$ be two terms such that \(t_1 \sim t_2\).
  For all contexts \(\Ga\), \(\Ga_1\) and \(\Ga_2\) we can \emph{construct}
  another term \(p\) such that whenever $\isterm{\Ga, \Ga_1}{t_1}{T_1}$ and
  $\isterm{\Ga, \Ga_2}{t_2}{T_2}$ we have
  $\isterm{\Ga, \Pack{\Ga_1}{\Ga_2}}
          {p}
          {\Heq{\llift{\gamma}{}{T_1}}
               {\llift{\gamma}{}{t_1}}
               {\rlift{\gamma}{}{T_2}}
               {\rlift{\gamma}{}{t_2}}}$.
\end{lemma}

\begin{proof}
  The proof is by induction on the derivation of $t_1 \sim t_2$. We show
  the three most interesting cases:

  \begin{itemize}
  \item \textsc{Var}
    \[
      \infer[]
        { }
        {x \sim x}
      %
    \]
    If $x$ belongs to $\Ga$, we apply reflexivity---together with unique
    typing---to conclude.
    Otherwise, $\ProjE{x}$ has the expected type (since
    $\llift{\gamma}{}{x} \equiv \ProjO{x}$ and $\rlift{\gamma}{}{x} \equiv \ProjT{x}$).

  \item \textsc{Application}
    \[
      \infer[]
        {t_1 \sim t_2 \\
         A_1 \sim A_2 \\
         B_1 \sim B_2 \\
         u_1 \sim u_2
        }
        {\app{t_1}{x:A_1}{B_1}{u_1} \sim \app{t_2}{x:A_2}{B_2}{u_2}}
      %
    \]
    We have $\isterm{\Ga, \Ga_1}{\app{t_1}{x:A_1}{B_1}{u_1}}{T_1}$ and
    $\isterm{\Ga, \Ga_2}{\app{t_2}{x:A_2}{B_2}{u_2}}{T_2}$ which means by
    inversion that the subterms are well-typed.
    We apply the induction hypothesis and then conclude.
  \item \textsc{TransportLeft}
    \[
      \infer[]
        {t_1 \sim t_2}
        {\transpo{p}\ t_1 \sim t_2}
      %
    \]
    We have $\isterm{\Ga, \Ga_1}{\transpo{p}\ t_1}{T_1}$ and
    $\isterm{\Ga, \Ga_2}{t_2}{T_2}$.
    By inversion we have
    $\isterm{\Ga, \Ga_1}{p}{\Eq{}{T_1'}{T_1}}$ and
    $\isterm{\Ga, \Ga_1}{t_1}{T_1'}$.
    By induction hypothesis we have $e$ such that
    $\isterm{\Ga, \Gp}{e}{\Heq{}{\llift{\gamma}{}{t_1}}{}{\rlift{\gamma}{}{t_2}}}$.
    From transitivity and symmetry we only need to provide a proof of
    $\Heq{}{\llift{\gamma}{}{t_1}}{}{\transpo{\llift{\gamma}{}{p}}\ \llift{\gamma}{}{t_1}}$ which is inhabited by
    $\pair{\_}{\_}{\llift{\gamma}{}{p}}{\refl{} (\transpo{\llift{\gamma}{}{p}}\ \llift{\gamma}{}{t_1})}$.
  \end{itemize}
\end{proof}

We can also prove that $\sim$ preserves substitution.

\begin{lemma}
  If $t_1 \sim t_2$ and $u_1 \sim u_2$ then
  $t_1[x \sto u_1] \sim t_2[x \sto u_2]$.
\end{lemma}

\begin{proof}
  We proceed by induction on the derivation of $t_1 \sim t_2$.
\end{proof}

The fundamental lemma, as the name suggests, is the main ingredient to proving
the translation correct (and actually building the translation). Now that we
have it, we can proceed with the translation.
% \setchapterpreamble[u]{\margintoc}
\chapter{Translation from \acrshort{ETT} to \acrshort{ITT} and \acrshort{WTT}}
\labch{elim-trans}
% \setchapterpreamble[u]{\margintoc}
\chapter{Reflection and homotopy}
\labch{elim-hott}

\marginnote[0.5cm]{
  Refer to \nrefch{flavours} for more on this.
}
Homotopy and reflection seem contradictory at first since \acrshort{UIP}
kind of comes with reflection and is negated by univalence, but as I have
exposed earlier, using \acrlongpl{2TT} we can make two equalities---one with
\acrshort{UIP} and one which is homotopic---cohabit consistently.

Now, in \nrefch{elim-reflection-framework} I showed how the proof was
(almost) agnostic with respect to universes, thanks to abstract universe
constructors, one of which was (crucially) the sort of an identity type.

This is the key to instantiate our translation into one going from
\acrshort{HTS} to \acrshort{2TT} or \acrshort{2WTT}.
We can recover those theories in our setting by taking $\F{i}$ and $\Un{i}$ as
respectively the fibrant and strict universes
(for $i \in \mathbb{N}$), along with the following \acrshort{PTS} rules:
%
\[
\begin{array}{l@{~}c@{~}l@{\qquad}l@{~}c@{~}l}
  (\F{i}, \F{i+1}) &\in& \Ax &
  (\Un{i}, \Un{i+1}) &\in& \Ax \\
  (\F{i}, \F{j}, \F{\nmax{i}{j}}) &\in& \Rl &
  (\F{i}, \Un{j}, \Un{\nmax{i}{j}}) &\in& \Rl \\
  (\Un{i}, \F{j}, \Un{\nmax{i}{j}}) &\in& \Rl &
  (\Un{i}, \Un{j}, \Un{\nmax{i}{j}}) &\in& \Rl \\
\end{array}
\]
%
and the fact that the sort of the (strict) identity type on $A : s$ is
the \emph{strictified} version of $s$, \ie $\Un{i}$ for $s = \Un{i}$ or
$s = \F{i}$.
\marginnote[-0.2cm]{
  Of course, just extending the proof to two equalities is a possibility as
  well. As one can imagine it doesn't change the proof much.
}
The fibrant equality can be recovered using axioms as exposed in
\nrefsec{trans-syntaxes}.

Note that this is just one of many presentations of \acrlongpl{2TT}: instead
of being type-based, fibration is sometimes dealt with using a specific
judgment and this is not covered by our formalisation.
\marginnote[-1cm]{
  Type-based approaches are usually much better suited to translations
  (because they are type-preserving).
}

In short, the translation from \acrshort{HTS} to \acrshort{2TT} or
\acrshort{2WTT} is \emph{exactly} the same as the one
from \acrshort{ETT} to \acrshort{ITT} or \acrshort{WTT} that I present in this
thesis. This fact is factorised through our formalisation.
% \setchapterpreamble[u]{\margintoc}
\chapter{Formalisation of the translation}
\labch{elim-formalised}

The formalisation is inspired from that of \Coq in \MetaCoq. It is actually
defined \emph{besides} \MetaCoq to allow for some interoperability, bringing
about fairly realistic examples.
This provides evidence that the translation is constructive \emph{and} computes!
Note that we also rely on the
\Equations~\sidecite{DBLP:conf/itp/Sozeau10,sozeau2019equations} plugin to
derive nice dependent induction principles.

Our formalisation takes full advantage of its easy interfacing with \MetaCoq:
we define two theories, namely \acrshort{ETT} and \acrshort{ITT}
(or \acrshort{WTT}), but the target features a lot of syntactic
sugar by having things such as transport, heterogeneous equality and packing as
part of the syntax. The operations regarding these constructors---in particular
the tedious ones---are written in \Coq and then quoted to finally be
\emph{realised} in a translation from \acrshort{ITT} to \MetaCoq.
\marginnote[1cm]{
  Some work for the future\dots
}
For \acrshort{WTT} the story is a bit different because there is no \emph{weak}
\Coq or \MetaCoq so for now there is no translation to a purer \acrshort{WTT}
though this is work that we started but put on hold by coming to realise the
target would not be that much simpler.

\paragraph{Interoperability with \MetaCoq.}
The translation we define from \acrshort{ITT} to \MetaCoq is not proven
correct, but it is not really important as it can just be seen as a
feature to observe the produced terms in a nicer setting.
\marginnote{
  We will discuss about \MetaCoq and its theory in greater detail in
  \arefpart{coq-in-coq}.
}
In any case, \MetaCoq does not yet provide a complete formalisation of
\acrshort{CIC} rules, as guard checking of recursive definitions and strict
positivity of inductive type declarations are not formalised yet.

Our formalised theorems however do not depend on \MetaCoq itself and as such
there is no need to \emph{trust} the plugin or the formalisation of \Coq inside
it.

We also provide a translation from \MetaCoq (and thus \Coq!) to \acrshort{ETT}
that we will describe more extensively with the examples in
\nrefsec{ett-flavoured}.

\section{Quick Overview of the Formalisation}

The formalisation can be found at
\href{https://github.com/TheoWinterhalter/ett-to-itt}{github.com/TheoWinterhalter/ett-to-itt}
and
\href{https://github.com/TheoWinterhalter/ett-to-itt/tree/weak}{github.com/TheoWinterhalter/ett-to-itt/tree/weak}
for the translations from \acrshort{ETT} to \acrshort{ITT} and from
\acrshort{ETT} to \acrshort{WTT} respectively.
Let me describe quickly the formalisation in the case of \acrshort{ETT} to
\acrshort{ITT}, the other formalisation is pretty similar.

\marginnote[0.1cm]{
  In \nrefch{elim-reflection-framework} I showed two different syntaxes but to
  make things simpler I use a common type of terms, only some terms will only
  have a typing rule in \acrshort{ITT}.
}
The file \rpath{SAst.v} contains the definition of the (common) abstract syntax
of \acrshort{ETT} and \acrshort{ITT} in the form of an inductive definition with
de Bruijn indices for variables.
Sorts are defined separately in \rpath{Sorts.v} and we will address them later
in \nrefsec{elim-trans-sorts}.

\begin{minted}{coq}
Inductive sterm : Type :=
| sRel (n : nat)
| sSort (s : sort)
| sProd (nx : name) (A B : sterm)
| sLambda (nx : name) (A B t : sterm)
| sApp (u : sterm) (nx : name) (A B v : sterm)
| sEq (A u v : sterm)
| sRefl (A u : sterm)
| (* ... *) .
\end{minted}

The files \rpath{ITyping.v} and \rpath{XTyping.v} define respectively the
typing judgments for \acrshort{ITT} and \acrshort{ETT}.
The first is defined with first a notion of reduction from which is deduced
conversion, while the latter has typed conversion and typing defined using
mutual inductive types.
\marginnote[0.3cm]{
  In the formalisation of the translation to \acrshort{WTT}, these meta-theory
  files are kept to a minimum!
}
Then, most of the files are focused on the meta-theory of \acrshort{ITT} and can
be ignored by readers who do not need to see yet another proof of subject
reduction.

The most interesting files are obviously those where the fundamental lemma
and the translation are formalised: \rpath{FundamentalLemma.v} and
\rpath{Translation.v}.
For instance, here is the main theorem, as stated in our formalisation:
%
\begin{minted}{coq}
Theorem complete_translation {Σ} :
  type_glob Σ ->
  (forall {Γ t A} (h : Σ ;;; Γ |-x t : A)
     {Γ'} (hΓ : Σ |--i Γ' ∈ ⟦ Γ ⟧),
      ∑ A' t', Σ ;;;; Γ' ⊢ [t'] : A' ∈ ⟦ Γ ⊢ [t] : A ⟧) *
  (forall {Γ u v A} (h : Σ ;;; Γ |-x u ≡ v : A)
     {Γ'} (hΓ : Σ |--i Γ' ∈ ⟦ Γ ⟧),
      ∑ A' A'' u' v' p',
        eqtrans Σ Γ A u v Γ' A' A'' u' v' p').
\end{minted}
%
Herein \mintinline{coq}{type_glob Σ} refers to the fact that the global
context is well-typed---thing which we mainly ignored in the paper translation.
The fact that the theorem holds in \Coq ensures we can actually
compute a translated term and type out of a derivation in \acrshort{ETT}.

%%%%% Seems like this is covered in the translation itself.
% \section{Inductive Types and Recursion}
% \label{sec:inductives}

% In the proof of Section~\ref{sec:translation}, we did not mention
% anything about inductive types, pattern-matching or recursion as it is
% a bit technical on paper.  In the formalisation, we offer a way to
% still be able to use them, and we will even show how it works in
% practice with the examples (Section\ref{sec:examples}).

% The main guiding principle is that inductive types and induction are orthogonal
% to the translation, they should more or less be translated to
% themselves.
% %
% To realise that easily, we just treat an inductive definition as a way
% to introduce new constants in the theory, one for the type, one for
% each constructor, one for its elimination principle, and one equality
% per computation rule.
% %
% For instance, the natural numbers can be represented by having the following
% constants in the context:
% %
% \[
% \begin{array}{l@{~}c@{~}l}
%   \nat &:& \Ty{0} \\
%   \zero &:& \nat \\
%   \natsucc &:& \nat \to \nat \\
%   \natrec &:& \forall P,\
%   P\ \zero \to (\forall m,\ P\ m \to P\ (\natsucc\ m)) \to
%   \forall n,\ P\ n \\
%   \natrec_\zero &:& \forall P\ P_z\ P_s,\ \natrec\ P\ P_z\ P_s\ \zero = P_z \\
%   \natrec_\natsucc &:& \forall P\ P_z\ P_s\ n,\\
%   &&\natrec\ P\ P_z\ P_s\ (\natsucc\ n) = P_s\ n\ (\natrec\ P\ P_z\ P_s\ n)
% \end{array}
% \]
% %
% Here we rely on the reflection rule to obtain the computational behaviour of the
% eliminator $\natrec$.

% This means for instance that we do not consider inductive types that would only
% make sense in ETT, but we deem this not to be a restriction and to the best of
% our knowledge is not something that is usually considered in the literature.
% %
% With that in mind, our translation features a global context of typed constants
% with the restriction that the types of those constants should be well-formed
% in ITT. Those constants are thus used as black boxes inside ETT.

% With this we are able to recover what we were missing from
% \Coq, without having to deal with the trouble of proving that the translation
% does not break the guard condition of fixed points, and we are instead relying on
% a more type-based approach.

\section{About Universes}
\labsec{elim-trans-sorts}

As I mentioned earlier in \nrefch{elim-reflection-framework}, sorts are treated
abstractly in the translation. In the formalisation sorts are defined using
the following class:
%
\begin{minted}{coq}
Class Sorts.notion := {
  sort : Type ;
  succ : sort -> sort ;
  prod_sort : sort -> sort -> sort ;
  sum_sort : sort -> sort -> sort ;
  eq_sort : sort -> sort ;
  eq_dec : forall s z : sort, {s = z} + {s <> z} ;
  succ_inj : forall s z, succ s = succ z -> s = z
}.
\end{minted}
%
From the notion of sorts, we require functions to get the sort of a sort,
the sort of a product from the sorts of its arguments, and the sort of an
identity type.
We also require some measure of decidable equality and injectivity on those.
From this we ensure the \acrshort{PTS} is \emph{functional} (in particular
without cumulativity) so that we have unique typing.

We are using classes here because they come with some automation in \Coq which
allows us to assume globally a notion of sort without having to specify it at
each use.

This allows us to instantiate this by a lot of different notions like a natural
number based hierarchy of \(\Type_i\) and even an extension of it with a
universe $\Prop$ of propositions as in \acrshort{CIC}.
We also provide an instance corresponding to \acrshort{2TT} as explained in
\nrefch{elim-hott}.

\marginnote[1.2cm]{
  For inconsistency of \(\Type\) \emph{in} \(\Type\), see \arefsubsec{coq-univ}.
}
In order to deal with examples in a simpler manner (making up for our lack of
universe polymorphism and cumulativity) by interacting with \Coq (thanks to
\MetaCoq), one of the instances we provide comes with only one universe $\Type$
and the inconsistent typing rule $\Type : \Type$.

\section{ETT-flavoured \Coq: Examples}
\labsec{ett-flavoured}

In this section I demonstrate how our translation can bring extensionality to
the world of \Coq in action. The examples can be found in
\rpath{plugin\_demo.v}.
Again, since we do not have \emph{weak} \Coq or \MetaCoq, these are in the case
of the translation to \acrshort{ITT} only.

\paragraph{First, a pedestrian approach.}
%
I would like to begin by showing how one can write an example step by step
before we show how it can be instrumented and automated as a plugin.
For this I use a self-contained example without any inductive
types or recursion, illustrating a very simple case of reflection.
The term we want to translate is the identity coercion:
\[
  \lambda\ A\ B\ e\ x.\ x : \Pi\ A\ B.\ A = B \to
  A \to B
\]
which relies on the equality \(e : A = B\) and reflection to convert  \(x : A\)
to \(x : B\).
%
Of course, this definition is not accepted in \Coq because this
conversion is not valid in \acrshort{ITT}.
%
\begin{minted}{coq}
Fail Definition pseudoid (A B : Type) (e : A = B) (x : A) : B
  := x.
\end{minted}
%
However, we still want to be able to write it \emph{in some way}, in order to
avoid manipulating de Bruijn indices directly.
\marginnote[0.1cm]{
  When I say ill-typed term, I mean a subterm that is typed but of the wrong
  type.
}%
For this, we use a little trick by first defining a \Coq axiom to represent
an ill-typed term:
%
\begin{minted}{coq}
Axiom candidate : forall A B (t : A), B.
\end{minted}
%
\mintinline{coq}|candidate A B t| is a candidate \mintinline{coq}|t| of type
\mintinline{coq}|A| to inhabit type \mintinline{coq}|B| in the fashion of
\ocaml's \mintinline{ocaml}|Obj.magic|.
We complete this by adding a notation that is reminiscent to \Agda's hole
mechanism.
%
\begin{minted}{coq}
Notation "'{!' t '!}'" := (candidate _ _ t).
\end{minted}

We can now write the \acrshort{ETT} function within \Coq.
%
\begin{minted}{coq}
Definition pseudoid (A B : Type) (e : A = B) (x : A) : B :=
  {! x !}.
\end{minted}
%
We can then quote the term and its type to \MetaCoq thanks to the
\mintinline{coq}|Quote Definition| command provided by the plugin.
%
\begin{minted}{coq}
Quote Definition pseudoid_term :=
  ltac:(let t := eval compute in pseudoid in exact t).
Quote Definition pseudoid_type :=
  ltac:(let T := type of pseudoid in exact T).
\end{minted}
\marginnote[-1.5cm]{
  The syntax is a bit heavy because of a call to \ltac used to quote the
  normal form of the term.
}
%
The terms that we get are now \MetaCoq terms, representing \Coq syntax.
We need to put them in \acrshort{ETT}, meaning adding the annotations, and also
removing the \mintinline{coq}|candidate| axiom.
This is the purpose of the \mintinline{coq}|fullquote| function that we provide
in our formalisation.
%
\marginnote[1cm]{
  \mintinline{coq}|fullquote| is given a \emph{big} number corresponding to the
  the number of recursive calls it is allowed to do. This \emph{fuel} technique
  allows us to circumvent the termination checker of \Coq.
}
\begin{minted}{coq}
Definition pretm_pseudoid :=
  Eval lazy in
  fullquote (2^18) Σ [] pseudoid_term empty empty nomap.

Definition tm_pseudoid :=
  Eval lazy in
  match pretm_pseudoid with
  | Success t => t
  | Error _ => sRel 0
  end.


Definition prety_pseudoid :=
  Eval lazy in
  fullquote (2^18) Σ [] pseudoid_type empty empty nomap.

Definition ty_pseudoid :=
  Eval lazy in
  match prety_pseudoid with
  | Success t => t
  | Error _ => sRel 0
  end.
\end{minted}
%
\mintinline{coq}|tm_pseudoid| and \mintinline{coq}|ty_pseudoid| correspond
respectively to the \acrshort{ETT} representation of \mintinline{coq}|pseudoid|
and its type.
We then produce, using our home-brewed \ltac type-checking tactic, the
corresponding \acrshort{ETT} typing derivation.
%
\begin{minted}{coq}
Lemma type_pseudoid : Σi ;;; [] |-x tm_pseudoid : ty_pseudoid.
Proof.
  unfold tm_pseudoid, ty_pseudoid.
  ettcheck. cbn.
  eapply reflection with (e := sRel 1).
  ettcheck.
Defined.
\end{minted}
%
Notice the use of \mintinline{coq}{reflection} which is a constructor of the
inductive type of typing derivations in \acrshort{ETT}. Its type is given by
\begin{minted}{coq}
reflection :
  forall A u v e,
    Σ ;;; Γ |-x e : sEq A u v ->
    Σ ;;; Γ |-x u ≡ v : A
\end{minted}
and corresponds to a use of the reflection rule.
We can then translate this derivation, obtain the translated term and then
convert it to \MetaCoq.
%
\begin{minted}{coq}
Definition itt_pseudoid : sterm :=
  Eval lazy in
  let '(_ ; t ; _) :=
    type_translation type_pseudoid istrans_nil
  in t.

Definition tc_pseudoid : tsl_result term :=
  Eval lazy in
  tsl_rec (2 ^ 18) Σ [] itt_pseudoid empty.
\end{minted}
%
Once we have it, we \emph{unquote} the term to obtain a \Coq term
(notice that the only use of reflection has been replaced by a transport).
%
\begin{minted}{coq}
fun (A B : Type) (e : A = B) (x : A) => transport e x
     : forall A B : Type, A = B -> A -> B
\end{minted}

\paragraph{Making a Plugin with \MetaCoq.}
%
All of this work is pretty systematic. Fortunately for us,
\MetaCoq also features a monad to reify \Coq commands which we can
use to \emph{program} the translation steps.
As such we have written a complete procedure, relying on type-checkers we
wrote for \acrshort{ITT} and \acrshort{ETT}, which can generate equality
obligations.

Thanks to this, the user does not have to know about the details of
implementation of the translation, and can stay within the \Coq ecosystem.

For instance, our previous example now becomes:
%
\begin{minted}{coq}
Definition pseudoid (A B : Type) (e : A = B) (x : A) : B :=
  {! x !}.

Run TemplateProgram (Translate ε "pseudoid").
\end{minted}
%
\marginnote[-0.6cm]{
  \mintinline{coq}{ε} is the empty translation context, see the next example to
  understand the need for a translation context
}
This produces a \Coq term \mintinline{coq}{pseudoid'} corresponding to the
translation.
Notice how the user does not even have to provide any proof of equality or
derivations of any sort. The derivation part is handled by our own typechecker
while the obligation part is solved automatically by the \Coq obligation mechanism.

\paragraph{About inductive types.}
%
As we promised, our translation is able to handle inductive types.
For this consider the inductive type of vectors (or length-indexed lists) below,
together with a simple definition (we will remain in \acrshort{ITT} for
simplicity).
%
\begin{minted}{coq}
Inductive vec A : nat -> Type :=
| vnil : vec A 0
| vcons : A -> forall n, vec A n -> vec A (S n).

Arguments vnil {_}.
Arguments vcons {_} _ _ _.

Definition vv := vcons 1 _ vnil.
\end{minted}
%
This time, in order to apply the translation we need to extend the translation
context with \mintinline{coq}|nat| and \mintinline{coq}|vec|.
%
\marginnote[1cm]{
  The \mintinline{coq}{_ <- _ ;; _} notation corresponds to a monadic bind.
}
\begin{minted}{coq}
Run TemplateProgram (
  Θ <- TranslateConstant ε "nat" ;;
  Θ <- TranslateConstant Θ "vec" ;;
  Translate Θ "vv"
).
\end{minted}
%
The command \mintinline{coq}|TranslateConstant| enriches the current
translation context with the types of the inductive type and of its
constructors. The translation context then also contains associative
tables between our own representation of constants and those of \Coq.
Unsurprisingly\sidenote{It is thanks to all the effort that has gone into
optimising the translation}, the translated \Coq term is the same as the
original term.

\paragraph{Reversal of vectors.}
%
Next, we tackle a motivating example: reversal on vectors.
Indeed, implementing this operation the same way it can
be done on lists ends up in the following conversion problem:
%
\begin{minted}{coq}
Fail Definition vrev {A n m} (v : vec A n) (acc : vec A m)
: vec A (n + m) :=
  vec_rect
    A (fun n _ => forall m, vec A m -> vec A (n + m))
    (fun m acc => acc)
    (fun a n _ rv m acc => rv _ (vcons a m acc))
    n v m acc.
\end{minted}
%
The recursive call returns a vector of length \mintinline{coq}|n + S m|
where the context expects one of length \mintinline{coq}|S n + m|. In
\acrshort{ITT}, these types are not convertible. This example is thus a perfect
fit for \acrshort{ETT} where we can use the fact that these two expressions
always compute to the same thing when instantiated with concrete numbers.
%
\begin{minted}{coq}
Definition vrev {A n m} (v : vec A n) (acc : vec A m)
: vec A (n + m) :=
  vec_rect
    A (fun n _ => forall m, vec A m -> vec A (n + m))
    (fun m acc => acc)
    (fun a n _ rv m acc => {! rv _ (vcons a m acc) !})
    n v m acc.

Run TemplateProgram (
  Θ <- TranslateConstant ε "nat" ;;
  Θ <- TranslateConstant Θ "vec" ;;
  Θ <- TranslateConstant Θ "Nat.add" ;;
  Θ <- TranslateConstant Θ "vec_rect" ;;
  Translate Θ "vrev"
).
\end{minted}
%
This generates four obligations that are all solved automatically. One of
them contains a proof of \mintinline{coq}|S n + m = n + S m| while the remaining
three correspond to the computation rules of addition (as mentioned before,
\mintinline{coq}|add| is simply a constant and does not compute in our
representation, hence the need for equalities).
%
The returned term is the following, with only one transport remaining
(remember our interpretation map removes unnecessary transports).
\begin{minted}{coq}
fun (A : Type) (n m : nat) (v : vec A n) (acc : vec A m) =>
vec_rect A
  (fun n _ => forall m, vec A m -> vec A (n + m))
  (fun m acc => acc)
  (fun a n₀ v₀ rv m₀ acc₀ =>
    transport
      (vrev_obligation_3 A n m v acc a n₀ v₀ rv m₀ acc₀)
      (rv (S m₀) (vcons a m₀ acc₀))
  ) n v m acc
: forall A n m, vec A n -> vec A m -> vec A (n + m)
\end{minted}

\section{Towards an Interfacing between \Andromeda and \Coq}

\Andromeda~\sidecite{andromeda} is a proof assistant implementing \acrshort{ETT}
in a sense that is really close to our formalisation. Aside from a concise
nucleus consisting in a very basic type theory, \Andromeda features an interface
in which the user can declare constants with given types. Definitions and
computational behaviour are defined using constants which inhabit equalities.

Here is for instance the definition of natural numbers and their eliminator
with computation rules.
\begin{minted}{text}
constant nat : Type
constant O : nat
constant S : nat -> nat

constant natrec :
  ∏ (P : nat -> Type),
    P O ->
    (∏ (n : nat), P n -> P (S n)) ->
    ∏ (n : nat), P n.

constant natrec_O :
  ∏ P Pz Ps, natrec P Pz Ps O ≡ Pz.

constant natrec_S :
  ∏ P Pz Ps n, natrec P Pz Ps (S n) ≡ Ps n (natrec P Pz Ps n).
\end{minted}
This is essentially what we do in our formalisation.
Furthermore, their theory relies on $\Type : \Type$, meaning, our modular
handling of universes can accommodate for this as well.

All in all, it should be possible in the future to use our translation
(or a similar one) to produce \Coq terms out of \Andromeda developments.
\Andromeda would be generating the typing derivations for us, which is
particularly interesting because \Andromeda's system is much more practical
than the small type-checker I wrote in \Coq.

\section{Composition with other Translations}

This translation also enables the formalisation of translations that
target \acrshort{ETT} rather than \acrshort{ITT} and still get mechanised proofs
of (relative) consistency by composition with this \acrshort{ETT} to
\acrshort{ITT} translation.
This could also be used to implement plugins based on the composition of
translations. In particular, supposing we have a theory which forms a
subset of \acrshort{ETT} and whose conversion is decidable. Using this
translation, we could formalise it as an embedded domain-specific type theory
and provide an automatic translation of well-typed terms into witnesses in
\Coq. This would make it possible to extend conversion with the theory
of lists for example.

This would provide a simple way to justify the consistency of
\acrshort{CoqMT}~\sidecite{DBLP:conf/lpar/JouannaudS17} for example, seeing it
as an extensional type theory where reflection is restricted to equalities on a
specific domain whose theory is decidable.
% \setchapterpreamble[u]{\margintoc}
\chapter{Conclusions regarding elimination of reflection}
\labch{elim-conclusion}

\todo{Copied}

\section{Limitations and Axioms}
\label{sec:axioms}

Currently, the representation of terms and derivations and the
computational content of the proof only allow us to deal with the
translation of relatively small terms but we hope to improve that in
the future. As we have seen, the actual translation involves the
computational content of lemmata of inversion, substitution, weakening
and equational reasoning and thus cannot be presented as a simple
recursive definition on derivations.


As we already mentioned, the axioms K and FunExt are both
necessary in ITT if we want the translation to be conservative as they are
provable in ETT~\sidecite{hofmann1995conservativity}.
However, one might still be concerned about having axioms
as they can for instance hinder canonicity of the system.
In that respect, K isn't really a restriction since it preserves canonicity.
The best proof of that is probably \Agda itself which natively features K---in
fact, one needs to explicitly deactivate it with a flag if they wish to work
without.

The case of FunExt is trickier. It should be possible to realise
the axiom by composing our translation with a setoid
interpretation~\sidecite{altenkirch99} which validates it, or by going into a
system featuring it, for instance by implementing Observational Type
Theory~\sidecite{altenkirch2007observational} like
\Epigram~\sidecite{mcbride2004epigram}.

However, these two axioms are not used to define the translation itself,
but only to witness UIP and function extensionality in the translation to
\Coq.
The translation only relies on one axiom, called
\mintinline{coq}|conv_trans_AXIOM| in the formalisation, stating that conversion
of ITT is transitive.
%
The proof of this property basically sums up to the confluence of the
reduction rules of ITT which is out of scope for this paper and has
recently been formalised in Agda~\sidecite{Abel:2017:DCT:3177123.3158111} (in a
simpler setting with only one universe).
Regardless, this axiom inhabits a proposition (the type of conversion is in
\mintinline{coq}|Prop|) and is thus irrelevant for computation. Actually no
information about the derivation leaks to the production of the ITT term.

On a different note, the \mintinline{coq}|candidate| axiom allows us to derive
\mintinline{coq}|False| but is merely used to write ill-typed terms in \Coq.
The translated term will never make us of it and one can always check if a
term is relying on unsafe assumptions thanks to the
\mintinline{coq}|Print Assumptions| command.

\section{Related Works and Conclusion}
\label{sec:related-works}

The seminal works on the precise connection between ETT and ITT go
back to \sidecite{streicher1993investigations} and
\sidecite{hofmann1995conservativity,HofmannPhD}.
%
In particular, the work of Hofmann provides a categorical answer to
the question of consistency and conservativity of ETT over ITT with
UIP and FunExt.
%
Ten years later, \sidecite{oury2005extensionality,Oury2006} provided
a translation from ETT to ITT with
UIP and FunExt and other axioms (mainly due to
technical difficulties).
%
Although a first step towards a move from categorical semantics to a
syntactic translation, his work does not stress any constructive
aspect of the proof and shows that there merely exist translations in
ITT to a typed term in ETT.

\sidecite{van2013explicit} have later proposed and
formalised a similar translation between a PTS with and without explicit
conversion. This does not entail anything about ETT to ITT but we can
find similarities in that there is a witness of conversion between any
term and itself under an explicit conversion, which internalises
irrelevance of explicit conversions. This morally corresponds to a
Uniqueness of Conversions principle.

The Program \sidecite{sozeau:icfp07} extension of \Coq performs a
related coercion insertion algorithm, between objects in subsets on the
same carrier or in different instances of the same inductive family,
assuming a proof-irrelevance axiom. Inserting coercions locally is not
as general as the present translation from ETT to ITT which can insert
transports in any context.

In this paper we provide the first effective translation from ETT to ITT
with UIP and FunExt. The translation has been
formalised in \Coq using \TemplateCoq, a meta-programming plugin of
\Coq. This translation is also effective in the sense that we can
produce in the end a \Coq term using the \TemplateCoq denotation
machinery.
%
With ongoing work to extend the translation to the inductive fragment
of \Coq, we are paving the way to an extensional version of the \Coq
proof assistant which could be translated back to its intensional
version, allowing the user to navigate between the two modes, and in
the end produce a proof term checkable in the intensional fragment.

% \begin{acks}
  We would like to thank Andrej Bauer and Philipp Haselwarter with whom we had
  fruitful discussions on the subject, prior to this work.
  We also would like to thank the attendees of the Aarhus EUTypes 2018 meeting
  for their insightful feedback on the plugin stemming from the translation.
% \end{acks}

\pagelayout{wide} % No margins
\addpart{A verified type-checker for \Coq, in \Coq}
\labpart{coq-in-coq}
\pagelayout{margin} % Restore margins

% \setchapterpreamble[u]{\margintoc}
\chapter{Overview}
\labch{coq-overview}
% \setchapterpreamble[u]{\margintoc}
\chapter{A specification of \Coq}
\labch{coq-spec}
% \setchapterpreamble[u]{\margintoc}
\chapter{Meta-theoretical properties}
\labch{coq-meta-theory}

\todo{
  Give an overview of what is proven without dwelling on the how.
  Maybe wait before things settle?
}

\todo{
  Put SN here?
  This is where fixguard and co make sense.
}
% \setchapterpreamble[u]{\margintoc}
\chapter{Well-founded induction and well-orders}
\labch{coq-orders}
% \setchapterpreamble[u]{\margintoc}
\chapter{Term positions and stacks}
\labch{coq-positions}

As we will see in \nrefch{coq-reduction} and \nrefch{coq-conversion}, when
manipulating terms we sometimes have to go deep withing subterms.
Positions point you to a specific subterm of a term while stacks operate as
some sort of terms with a hole or equivalently some evaluation environments.

\section{Positions}

In a general setting, positions in trees are given by sequences of choices or
directions. The empty sequence corresponds to the root of the tree, and at each
branching you have to say which branch you want to take.

\marginnote[1cm]{
  Sequences such as \(0.1.0\) are read from left to right, and correspond to
  directions starting from the root.
  In black is the subtree as the given position.
}
\begin{figure}[hb]
  \includegraphics[width=0.9\textwidth]{tree-position.pdf}
\end{figure}

Now, terms are a special kind of tree so we can do something similar.
There are many ways to represent positions: for instance in the example above
the position \(0.3\) doesn't correspond to anything, so is it still considered
a position, only an \emph{invalid} one? Or should all expressable positions
be valid?

My approach is a bit in between, I constrain the syntax of choices a bit more
than this, but they are not necessarily valid.
Basically choices are defined inductively, with several constructors for each
of the constructs of the syntax: for instance, applications will have two
corresponding choices, one for the applicant, one for the argument.
\marginnote[1cm]{
  \mintinline{coq}{app_l} corresponds to going left under an application
  while \mintinline{coq}{app_r} corresponds to going right.
}
\begin{minted}{coq}
Inductive choice :=
| app_l
| app_r
| case_p
| case_c
| case_brs (n : nat)
| proj_c
| fix_mfix_ty (n : nat)
| fix_mfix_bd (n : nat)
| lam_ty
| lam_tm
| prod_l
| prod_r
| let_bd
| let_ty
| let_in.
\end{minted}

A position is just a list of choices.
\begin{minted}{coq}
Definition position := list choice.
\end{minted}

Now as I already said, these are not necessarily valid positions, for this
we define a function which verifies if a given position is valid in a given
term.
\marginnote[1cm]{
  It shouldn't feel too surprising, the empty position is always valid,
  and otherwise, the head choice should match the structure of the term.
  There are some trickier cases for pattern-matching and fixed-points because
  they involve lists of terms but it's still pretty natural.
}
\begin{minted}{coq}
Fixpoint validpos t (p : position) {struct p} :=
  match p with
  | [] => true
  | c :: p =>
    match c, t with
    | app_l, tApp u v => validpos u p
    | app_r, tApp u v => validpos v p
    | case_p, tCase indn pr c brs => validpos pr p
    | case_c, tCase indn pr c brs => validpos c p
    | case_brs n, tCase indn pr c brs =>
        match nth_error brs n with
        | Some (_, br) => validpos br p
        | None => false
        end
    | proj_c, tProj pr c => validpos c p
    | fix_mfix_ty n, tFix mfix idx =>
        match nth_error mfix n with
        | Some d => validpos d.(dtype) p
        | None => false
        end
    | fix_mfix_bd n, tFix mfix idx =>
        match nth_error mfix n with
        | Some d => validpos d.(dbody) p
        | None => false
        end
    | lam_ty, tLambda na A t => validpos A p
    | lam_tm, tLambda na A t => validpos t p
    | prod_l, tProd na A B => validpos A p
    | prod_r, tProd na A B => validpos B p
    | let_bd, tLetIn na b B t => validpos b p
    | let_ty, tLetIn na b B t => validpos B p
    | let_in, tLetIn na b B t => validpos t p
    | _, _ => false
    end
  end.
\end{minted}
This function might serve as a specification for the positions.

Finally we can define a type of valid positions in a term using a subset type.
\begin{minted}{coq}
Definition pos (t : term) :=
  { p : position | validpos t p = true }.
\end{minted}

For instance \mintinline{coq}{[ app_l ; let_in ]} is valid position in term
\begin{minted}{coq}
  tApp (tLetIn na b B t) u
\end{minted}
which represents the term \mintinline{coq}{(let na := b : B in t) u}
and points to subterm \mintinline{coq}{t}.

We can also define a function to access the subterm at a given position.
\begin{minted}{coq}
Fixpoint atpos t (p : position) {struct p} : term :=
  match p with
  | [] => t
  | c :: p =>
    match c, t with
    | app_l, tApp u v => atpos u p
    | app_r, tApp u v => atpos v p
    | case_p, tCase indn pr c brs => atpos pr p
    | case_c, tCase indn pr c brs => atpos c p
    | case_brs n, tCase indn pr c brs =>
        match nth_error brs n with
        | Some (_, br) => atpos br p
        | None => tRel 0
        end
    | proj_c, tProj pr c => atpos c p
    | fix_mfix_ty n, tFix mfix idx =>
        match nth_error mfix n with
        | Some d => atpos d.(dtype) p
        | None => tRel 0
        end
    | fix_mfix_bd n, tFix mfix idx =>
        match nth_error mfix n with
        | Some d => atpos d.(dbody) p
        | None => tRel 0
        end
    | lam_ty, tLambda na A t => atpos A p
    | lam_tm, tLambda na A t => atpos t p
    | prod_l, tProd na A B => atpos A p
    | prod_r, tProd na A B => atpos B p
    | let_bd, tLetIn na b B t => atpos b p
    | let_ty, tLetIn na b B t => atpos B p
    | let_in, tLetIn na b B t => atpos t p
    | _, _ => tRel 0
    end
  end.
\end{minted}
\marginnote[-2.5cm]{
  The \mintinline{coq}{tRel 0} case is in an impossible branch when the position
  is valid, but for simplicity, the function is defined for any position.
  Otherwise we would have to carry the proof that it is valid everywhere.
}

Positions let you go deep inside a term, forgetting about its surrounding;
surrounding which can be recorded using stacks.

\section{Stacks}

My use of the term \emph{stack} might be an abuse as it's probably a
generalisation of it and might be better called an evaluation environment
or context; I will stick to \emph{stack} anyway.
The main reason behind the name is that it's not presented as a term with a hole
but rather as a succession of terms with a hole that stack on top of each other.

If you take the following example,
\[
  \stack{f\ \stack{\stack{(\lambda x. \stack{t})}\ u}}
\]
you are considering term \(t\) against the stack
\[
  \stack{f\ \stack{\stack{(\lambda x. \shole)}\ u}}
\]

It can be decomposed into
\marginnote[1cm]{
  \(\varepsilon\) represents the empty stack.
}
\[
  \stack{\lambda x. \shole} :: \stack{\shole\ u} :: \stack{f\ \shole}
  :: \varepsilon
\]
meaning that the term will first be put under an abstraction, the result of this
applied to \(u\) and the whole given as an argument to \(f\).
This notion will prove particularly useful when considering the reduction
machine in \nrefch{coq-reduction}, indeed the stack is way to remember the
surrounding term when focusing on a subterm, once it has reached a normal form,
we can use the stack as a \emph{continuation}.

You can see the process step by step in the following example.
\marginnote[2cm]{
  I use \(\red_\beta\) to denote the cases where a \emph{real} reduction
  happens and the focused term (a \(\lambda\)-abstraction) consumes its
  argument; the other cases are \emph{focusing}, \ie pushing some term with
  hole on the stack.
}
\[
  \begin{array}{lc}
    \stack{
      (
        (\lambda x.\ x\ u)\
        (\lambda y.\ y)
      ) \ v
    } & \red \\[0.2cm]
    \stack{
      \stack{(
        (\lambda x.\ x\ u)\
        (\lambda y.\ y)
      )} \ v
    } & \red \\[0.3cm]
    \stack{
      \stack{(
        \stack{(\lambda x.\ x\ u)}\
        (\lambda y.\ y)
      )} \ v
    } & \red_\beta \\[0.4cm]
    \stack{
      \stack{(
        (\lambda y.\ y)\ u
      )} \ v
    } & \red \\[0.3cm]
    \stack{
      \stack{(
        \stack{(\lambda y.\ y)}\ u
      )} \ v
    } & \red_\beta \\[0.4cm]
    \stack{
      \stack{u} \ v
    }
  \end{array}
\]

\marginnote{
  In the formalism above, \(\vscmd{t}{\stack{\shole\ u} :: \varepsilon}\)
  corresponds to \(\stack{\stack{t}\ u}\). I am simply now making explicit
  the separation between the focused term and the stack.
}
The intesting bit is that the focused term still interacts with the stack.
To see clearer we can use the notation \(\vscmd{t}{\pi}\) representing
term \(t\) \emph{against} stack \(\pi\). From this we can write the following
reduction rules that together correspond to \(\beta\)-reduction.

\marginnote[0.5cm]{
  \(\beta\)-reduction now happens in two steps:
  \(
    \vscmd{(\lambda x.\ t)\ u}{\pi} \red
    \vscmd{\lambda x.\ t}{\stack{\shole\ u} :: \pi} \red
    \vscmd{t[x \sto u]}{\pi}
  \)
}
\[
  \begin{array}{lcl}
    \vscmd{u \ v}{\pi} &\red& \vscmd{u}{\stack{\shole\ v} :: \pi} \\
    \vscmd{\lambda x.\ t}{\stack{\shole\ u} :: \pi}
    &\red& \vscmd{t[x \sto u]}{\pi}
  \end{array}
\]

In \Coq I define stacks as follows.
\begin{minted}{coq}
Inductive stack : Type :=
| Empty
| App (t : term) (π : stack)
| Fix (f : mfixpoint term) (n : nat) (args : list term)
      (π : stack)
| Fix_mfix_ty (na : name) (bo : term) (ra : nat)
              (mfix1 mfix2 : mfixpoint term) (id : nat)
              (π : stack)
| Fix_mfix_bd (na : name) (ty : term) (ra : nat)
              (mfix1 mfix2 : mfixpoint term) (id : nat)
              (π : stack)
| CoFix (f : mfixpoint term) (n : nat) (args : list term)
        (π : stack)
| Case_p (indn : inductive * nat) (c : term)
         (brs : list (nat * term)) (π : stack)
| Case (indn : inductive * nat) (p : term)
       (brs : list (nat * term)) (π : stack)
| Case_brs (indn : inductive * nat) (p c : term) (m : nat)
           (brs1 brs2 : list (nat * term)) (π : stack)
| Proj (p : projection) (π : stack)
| Prod_l (na : name) (B : term) (π : stack)
| Prod_r (na : name) (A : term) (π : stack)
| Lambda_ty (na : name) (b : term) (π : stack)
| Lambda_tm (na : name) (A : term) (π : stack)
| LetIn_bd (na : name) (B t : term) (π : stack)
| LetIn_ty (na : name) (b t : term) (π : stack)
| LetIn_in (na : name) (b B : term) (π : stack)
| coApp (t : term) (π : stack).

Notation "'ε'" := (Empty).
\end{minted}

\section{Positions induced by stacks}

The reason I present stacks and positions together is because they are related.
If you consider the term \(t\) against some stack \(\pi\) as one term, there
exists a position in that term that points to \(t\).
For instance if you consider
\[
  \stack{u\ \stack{\stack{t}\ v}}
\]
then the position that first chooses the right-hand term of the top-level
application and then the left-hand term will indeed point to \(t\).
In \Coq syntax as earlier this would be
\begin{minted}{coq}
[ app_r ; app_l ]
\end{minted}

In fact the position can be computed from the stack itself, regardless of the
term we want to plug in. In the example above, the stack is
\[
  \stack{\shole\ v} :: \stack{u\ \shole} :: \varepsilon
\]
We read it from right to left\sidenote{This is because the leftmost term with
hole corresponds to the innermost element in the resulting term.} to reconstruct
the position and simply give the position of the hole.

\begin{minted}{coq}
Fixpoint stack_position π : position :=
  match π with
  | ε => []
  | App u ρ => stack_position ρ ++ [ app_l ]
  | Fix f n args ρ => stack_position ρ ++ [ app_r ]
  | Fix_mfix_ty na bo ra mfix1 mfix2 idx ρ =>
      stack_position ρ ++ [ fix_mfix_ty #|mfix1| ]
  | Fix_mfix_bd na ty ra mfix1 mfix2 idx ρ =>
      stack_position ρ ++ [ fix_mfix_bd #|mfix1| ]
  | CoFix f n args ρ => stack_position ρ ++ [ app_r ]
  | Case_p indn c brs ρ => stack_position ρ ++ [ case_p ]
  | Case indn pred brs ρ => stack_position ρ ++ [ case_c ]
  | Case_brs indn pred c m brs1 brs2 ρ =>
      stack_position ρ ++ [ case_brs #|brs1| ]
  | Proj pr ρ => stack_position ρ ++ [ proj_c ]
  | Prod_l na B ρ => stack_position ρ ++ [ prod_l ]
  | Prod_r na A ρ => stack_position ρ ++ [ prod_r ]
  | Lambda_ty na u ρ => stack_position ρ ++ [ lam_ty ]
  | Lambda_tm na A ρ => stack_position ρ ++ [ lam_tm ]
  | LetIn_bd na B u ρ => stack_position ρ ++ [ let_bd ]
  | LetIn_ty na b u ρ => stack_position ρ ++ [ let_ty ]
  | LetIn_in na b B ρ => stack_position ρ ++ [ let_in ]
  | coApp u ρ => stack_position ρ ++ [ app_r ]
  end.
\end{minted}

As I said multiple times, we will use stacks for reduction machines, to show the
termination of such machines it is nice to have an order on stacks. This order
will be achieved by going from stacks to positions.

\section{Ordering positions}

If you consider the valid positions in a given term (\ie expressions of type
\mintinline{coq}{pos t} for some \mintinline{coq}{t}) there exists a
well-founded order on them. There are actually several due to some degree of
liberty as we shall see.

Since positions are defined as lists, the natural order on them would the
structural which says that any (strict) prefix of a position is smaller.
This order isn't really of interest to us because it would correspond to
\emph{unfocusing}.
Conversely, by extending a position with a new choice you are going deeper into
the term, something which you cannot do indefinitely, you are bound to reach a
leaf at some (finite) point.

If you think about it dually, I'm merely explaining that the structural order on
terms is well-founded, so why bother?
The interesting part is that the order on positions can be more flexible than
the subterm relation in that it allows us to compare `going left' with
`going right'.
Say you have the application \(t \coloneqq u\ v\), both \(u\) and \(v\) are
subterms of \(u\) and \(v\) but there is no way to relate them reliably and
have \(v\) \emph{always} smaller than \(u\) also because they can be arbitrary.
If instead we were to consider positions in \(t\) we would have the empty one
corresponding to \(t\) itself, \mintinline{coq}{[ app_l ]} corresponding to
\(u\) and \mintinline{coq}{[ app_r ]} for \(v\). This means we abstract over the
content of both \(u\) and \(v\). The positions can thus be ordered as follows:
\begin{minted}{coq}
[ app_r ] < [ app_l ] < []
\end{minted}

I am going to explain why this is desirable in \nrefch{coq-reduction}.

In \Coq, the definition of the order is done in two steps: first an order on
raw positions, from which we then derive an order on valid positions.
\marginnote[4.8cm]{
  The \mintinline{coq}{(` p)} notation corresponds to the first projection of
  a subset type.
}
\begin{minted}{coq}
Inductive positionR : position -> position -> Prop :=
| positionR_app_lr p q :
    positionR (app_r :: p) (app_l :: q)

| positionR_deep c p q :
    positionR p q ->
    positionR (c :: p) (c :: q)

| positionR_root c p :
    positionR (c :: p) [].

Definition posR {t} (p q : pos t) : Prop :=
  positionR (` p) (` q).
\end{minted}

While the first order is not well-founded, the second one is, as crystallised
by the following lemma.
\begin{minted}{coq}
Lemma posR_Acc :
  forall t p, Acc (@posR t) p.
\end{minted}
% \setchapterpreamble[u]{\margintoc}
\chapter{Reduction}
\labch{coq-reduction}

A key element to defining a type checker is to have a conversion checker.
Conversion in turn relies on reduction. The reduction I describe in
\nrefch{coq-spec} is just a relation and is non-deterministic. In this chapter
I will present an algorithmic implementation of reduction.
This implementation relies on the axiom of strong normalisation, but even then
it requires some more work to show the process is terminating.

\section{Weak head normalisation}

Since reduction is non deterministic, we have to pick a strategy to implement.
When checking terms for conversion, one thing we want to do is verify that they
have the same head constructor: if two terms are \(\lambda\)-abstractions
\(\lambda (x:A).t\) and \(\lambda (x:A').t'\) we need only compare \(A\) and
\(A'\), as well as \(t\) and \(t'\) for conversion.
Weak head reduction is a strategy that allows us to access the head of a term
rather efficiently.

\subsection{Characterisation}

The idea is to only deal with redexes (\ie left-hand side of reduction rules)
that appear at the top-level and might be hiding the head constructor. As such
when considering \(\lambda (x:A).t\), neither \(A\) nor \(t\) will be reduced
because we already know the head, or the shape, of the term: it is a
\(\lambda\). However, if we have an application \(t\ u\), we will first weak
head reduce \(t\) to see if it reaches a \(\lambda\): if it does we substitute
\(u\) and repeat the process, if it does not, then we have reached a weak head
normal form.

Weak head normal forms are by definition the terms that cannot be reduced
further by weak head reduction.

\begin{definition}[Normal form]
  A term \(t\) is a normal form for a reduction \(\red\) if it doesn't reduce
  for \(\red\).
  \[
    t \not \red
  \]
\end{definition}

It is however possible to characterise weak
head normal forms syntactically. For this we need to talk about \emph{neutral}
forms.
The intuition is that a neutral term is a \emph{stuck} term.

\begin{definition}[Neutral forms]
  A term \(t\) is neutral for a reduction \(\red\) if substituting it inside
  another term doesn't introduce redexes. Equivalently substituting a neutral
  term \(t\) in a normal form \(u\) will yield another normal form.
  \[
    u[x \sto t] \not \red
  \]
\end{definition}

A neutral form is in particular a normal form.

Focusing on a small fragment of \acrshort{PCUIC} without inductive types and so
on, we can characterise weak head neutral (\(\whne\)) and normal forms \(\whnf\)
with the following mutual judgements:
\marginnote[1cm]{
  The first rule is there to make sure that the variable doesn't correspond to
  local definition because those reduce.
}
\begin{mathpar}
  \infer
    {(x : A) \in \Ga}
    {\Ga \vdash x\ \whne}
  %

  \infer
    {\Ga \vdash u\ \whne}
    {\Ga \vdash u\ v\ \whne}
  %

  \infer
    {\Ga \vdash u\ \whne}
    {\Ga \vdash u\ \whnf}
  %

  \infer
    { }
    {\Ga \vdash \Type\ \whnf}
  %

  \infer
    { }
    {\Ga \vdash \Pi (x.A).B\ \whnf}
  %

  \infer
    { }
    {\Ga \vdash \lambda (x.A).t\ \whnf}
  %
\end{mathpar}
As you can see, terms like \((\lambda (x.A).t)\ u\) or
\(\tlet\ x := u\ \tin\ t\) are not weak head normal forms, and a fortiori not
weak head neutral forms.
Indeed they both reduce to \(t[x \sto u]\) for weak head reduction.

Before I show how we deal with rest of the constructions in the formalisation
I need to talk about how we parameterise the weak head reduction.

\subsection{Reduction flags}

In order to be able to fine-tune our reduction strategy we introduce the notion
of reduction flags. Those mimic the flags that are used by \Coq internally.

\begin{minted}{coq}
Module RedFlags.

  Record t := mk {
    beta   : bool ;
    iota   : bool ;
    zeta   : bool ;
    delta  : bool ;
    fix_   : bool ;
    cofix_ : bool
  }.

  Definition default := mk true true true true true true.

  Definition nodelta := mk true true true false true true.

End RedFlags.
\end{minted}

The names refer to the reduction rules that they activate or not.
Setting \mintinline{coq}{beta} to \mintinline{coq}{true} means that we will
reduce \(\beta\)-redexes.
\mintinline{coq}{iota} is for the reduction of pattern-matching,
\mintinline{coq}{zeta} for the reduction of let-bindings and unfolding of local
definitions, \mintinline{coq}{delta} is for the unfolding of global definitions,
\mintinline{coq}{fix_} is for the unfolding of fixed-points and
\mintinline{coq}{cofix_} for that of cofixed-points.

The default is to reduce all those, but later we will also use a version which
doesn't do \(\delta\)-reductions to speed up conversion.

\subsection{\Coq specification}

The reduction flags parameterise the weak head reduction and as such they
affect the definition of weak head normal and neutral forms.

\begin{minted}{coq}
Context (flags : RedFlags.t).
Context (Σ : global_env).

Inductive whnf Γ : term -> Prop :=
| whnf_ne t            : whne Γ t -> whnf Γ t
| whnf_sort s          : whnf Γ (tSort s)
| whnf_prod na A B     : whnf Γ (tProd na A B)
| whnf_lam na A B      : whnf Γ (tLambda na A B)
| whnf_cstrapp i n u v : whnf Γ (mkApps (tConstruct i n u) v)
| whnf_indapp i u v    : whnf Γ (mkApps (tInd i u) v)
| whnf_fix mfix idx    : whnf Γ (tFix mfix idx)
| whnf_cofix mfix idx  : whnf Γ (tCoFix mfix idx)

with whne Γ : term -> Prop :=
| whne_rel i :
    option_map decl_body (nth_error Γ i) = Some None ->
    whne Γ (tRel i)

| whne_rel_nozeta i :
    RedFlags.zeta flags = false ->
    whne Γ (tRel i)

| whne_letin_nozeta na B b t :
    RedFlags.zeta flags = false ->
    whne Γ (tLetIn na B b t)

| whne_const c u decl :
    lookup_env Σ c = Some (ConstantDecl decl) ->
    decl.(cst_body) = None ->
    whne Γ (tConst c u)

| whne_const_nodelta c u :
    RedFlags.delta flags = false ->
    whne Γ (tConst c u)

| whne_app f v :
    whne Γ f ->
    whne Γ (tApp f v)

| whne_case i p c brs :
    whne Γ c ->
    whne Γ (tCase i p c brs)

| whne_case_noiota i p c brs :
    RedFlags.iota flags = false ->
    whne Γ (tCase i p c brs)

| whne_proj p c :
    whne Γ c ->
    whne Γ (tProj p c)

| whne_proj_noiota p c :
    RedFlags.iota flags = false ->
    whne Γ (tProj p c).
\end{minted}

Applied constructors and inductive types are normal, same as fixed- and
cofixed-points.

Variables are neutral if they refer to local definitions when \(\zeta\) is
deactivated, or if they refer to assumptions.
Similarly, constants are neutral is they refer to global definitions when
\(\delta\) is off, or if they refer to an axiom.
let-bindings are only neutral when \(\zeta\) is off.
A pattern-matching expression is neutral when the scrutiny is neutral or if the
\(\iota\) flag is off. The same goes for projections.
% \setchapterpreamble[u]{\margintoc}
\chapter{Conversion}
\labch{coq-conversion}
% \setchapterpreamble[u]{\margintoc}
\chapter{Type inference and checking}
\labch{coq-inference}
% \setchapterpreamble[u]{\margintoc}
\chapter{Towards completeness}
\labch{coq-completeness}

\todo{Move in conclusion as there is not much done?}
% \setchapterpreamble[u]{\margintoc}
\chapter{Conclusion}
\labch{coq-conclusion}

We have formalised an almost feature-complete type checker for \Coq in \Coq
and proven it correct. Thanks to the extraction mechanism of \Coq, this can
actually be turned into an independent type checker program.
For instance it can be run within Coq in the manner of a plugin:
\begin{minted}{coq}
MetaCoq SafeCheck nat.
\end{minted}

Because we do not deal with template polymorphism, the module system and
\(\eta\)-expansion, we are not yet able to typecheck the standard library
or the formalisation itself (as it relies on the standard library).
We were however able to test it on reasonable proof terms coming from
the \acrshort{HoTT} library~\sidecite{bauer2017hott}.
\marginnote[1.1cm]{
  Examples of the checker in action, including the \acrshort{HoTT} theorems
  can be found in the file
  \href{https://github.com/MetaCoq/metacoq/blob/popl-artifact-eval/test-suite/safechecker_test.v}{test-suite/safechecker\_test.v}.
}%
For instance, we have been able to typecheck the proof that an isomorphism can
be turned into an equivalence.
It currently is about one order of magnitude slower than the \Coq
implementation (0.015s vs 0.002s averaged over 10 runs for each checking).
This can in particular be attributed to our very ineficient representation of
global environments as association lists indexed by character lists, where \Coq
uses efficient hash-maps on strings.

Another point I would like to address is the lack of completeness proof.
Our tests suggest that we are close to completeness if not there yet for the
fragment we are considering. It would be however really interesting to have a
proof that conversion and inference are complete with respect to their
specifications.
At the moment, dealing with \(\eta\)-conversion seems like a much more pressing
matter as it would make the checker usable in practice at least as a standalone
independent checker.

There is also the long term dream of having a \Coq kernel fully verified and
running in \Coq.

\appendix % From here onwards, chapters are numbered with letters, as is the appendix convention

\pagelayout{wide} % No margins
\addpart{Appendix}
\pagelayout{margin} % Restore margins

\section{Complementary rules}
\label{sec:more-rules}

\begin{figure*}[htbp]
  \flushleft
  \hrulefill
  \paradot{Equivalence relation}

  \begin{mathpar}
    \infer[]
      {\isterm{\Ga}{u}{A}}
      {\eqterm{\Ga}{u}{u}{A}}
    %

    \infer[]
      {\eqterm{\Ga}{u}{v}{A}}
      {\eqterm{\Ga}{v}{u}{A}}
    %

    \infer[]
      {\eqterm{\Ga}{u}{v}{A} \\
       \eqterm{\Ga}{v}{w}{A}
      }
      {\eqterm{\Ga}{u}{w}{A}}
    %
  \end{mathpar}

  \paradot{Congruence of type constructors}

  \begin{mathpar}
    \infer[]
      {\eqterm{\Ga}{A_1}{A_2}{s} \\
       \eqterm{\Ga,x:A_1}{B_1}{B_2}{s'}
      }
      {\eqterm{\Ga}{\Prod{x:A_1} B_1}{\Prod{x:A_2} B_2}{s''}}
    (s,s',s'')

    \infer[]
      {\eqterm{\Ga}{A_1}{A_2}{s} \\
       \eqterm{\Ga,x:A_1}{B_1}{B_2}{s'}
      }
      {\eqterm{\Ga}{\Sum{x:A_1} B_1}{\Sum{x:A_2} B_2}{s''}}
    (s,s',s'')

    \infer[]
      {\eqterm{\Ga}{A_1}{A_2}{s} \\
       \eqterm{\Ga}{u_1}{u_2}{A_1} \\
       \eqterm{\Ga}{v_1}{v_2}{A_1}
      }
      {\eqterm{\Ga}{\Eq{A_1}{u_1}{v_1}}{\Eq{A_2}{u_2}{v_2}}{s}}
    %
  \end{mathpar}

  \paradot{Congruence of $\lambda$-calculus terms}

  \begin{mathpar}
    \infer[]
      {\eqterm{\Ga}{A_1}{A_2}{s} \\
       \eqterm{\Ga,x:A_1}{B_1}{B_2}{s'} \\
       \eqterm{\Ga,x:A_1}{t_1}{t_2}{B_1}
      }
      {\eqterm{\Ga}{\lam{x:A_1}{B_1} t_1}{\lam{x:A_2}{B_2} t_2}{\Prod{x:A_1} B_1}}
    %

    \infer[]
      {\eqterm{\Ga}{A_1}{A_2}{s} \\
       \eqterm{\Ga,x:A_1}{B_1}{B_2}{s'} \\
       \eqterm{\Ga}{t_1}{t_2}{\Prod{x:A_1} B_1} \\
       \eqterm{\Ga}{u_1}{u_2}{A_1}
      }
      {\eqterm
        {\Ga}
        {\app{t_1}{x:A_1}{B_1}{u_1}}
        {\app{t_1}{x:A_1}{B_1}{u_1}}
        {B_1[x \sto u_1]}
      }
    %

    \infer[]
      {\eqterm{\Ga}{A_1}{A_2}{s} \\
       \eqterm{\Ga}{u_1}{u_2}{A_1} \\
       \eqterm{\Ga,x:A_1}{B_1}{B_2}{s'} \\
       \eqterm{\Ga}{v_1}{v_2}{B_1[x \sto u_1]}
      }
      {\eqterm
        {\Ga}
        {\pair{x:A_1}{B_1}{u_1}{v_1}}
        {\pair{x:A_2}{B_2}{u_2}{v_2}}
        {\Sum{x:A_1}{B_1}}
      }
    %

    \infer[]
      {\eqterm{\Ga}{A_1}{A_2}{s} \\
       \eqterm{\Ga,x:A_1}{B_1}{B_2}{s'} \\
       \eqterm{\Ga}{p_1}{p_2}{\Sum{x:A_1}{B_1}}
      }
      {\eqterm{\Ga}{\pio{x:A_1}{B_1}{p_1}}{\pio{x:A_2}{B_2}{p_2}}{A_1}}
    %

    \infer[]
      {\eqterm{\Ga}{A_1}{A_2}{s} \\
       \eqterm{\Ga,x:A_1}{B_1}{B_2}{s'} \\
       \eqterm{\Ga}{p_1}{p_2}{\Sum{x:A_1}{B_1}}
      }
      {\eqterm
        {\Ga}
        {\pit{x:A_1}{B_1}{p_1}}
        {\pit{x:A_2}{B_2}{p_2}}
        {B_1[x \sto \pio{x:A_1}{B_1}{p_1}]}
      }
    %
  \end{mathpar}

  \paradot{Congruence of equality terms}

  \begin{mathpar}
    \infer[]
      {\eqterm{\Ga}{A_1}{A_2}{s} \\
       \eqterm{\Ga}{u_1}{u_2}{A}
      }
      {\eqterm{\Ga}{\refl{A_1} u_1}{\refl{A_2} u_2}{\Eq{A_1}{u_1}{u_1}}}
    %

    \infer[]
      {\eqterm{\Ga}{A_1}{A_2}{s} \\
       \eqterm{\Ga}{u_1}{u_2}{A_1} \\
       \eqterm{\Ga}{v_1}{v_2}{A_1} \\
       \eqterm{\Ga, x:A_1, e:\Eq{A_1}{u_1}{x}}{P_1}{P_2}{s'} \\
       \eqterm{\Ga}{p_1}{p_2}{\Eq{A_1}{u_1}{v_1}} \\
       \eqterm{\Ga}{w_1}{w_2}{P_1[x \sto u_1, e \sto \refl{A_1} u_1]}
      }
      {\eqterm
        {\Ga}
        {\J{A_1}{u_1}{x.e.P_1}{w_1}{v_1}{p_1}}
        {\J{A_2}{u_2}{x.e.P_2}{w_2}{v_2}{p_2}}
        {P[x \sto v_1, e \sto p_1]}
      }
    %

    \infer[]
      {\eqterm{\Ga}{A_1}{A_2}{s} \\
       \eqterm{\Ga,x:A_1}{B_1}{B_2}{s'} \\
       \eqterm{\Ga}{f_1}{f_2}{\Prod{x:A_1} B_1} \\
       \eqterm{\Ga}{g_1}{g_2}{\Prod{x:A_1} B_1} \\
       \eqterm
         {\Ga}
         {e_1}
         {e_2}
         {\Prod{x:A_1}
          \Eq{B_1}{\app{f_1}{x:A_1}{B_1}{x}}{\app{g_1}{x:A_1}{B_1}{x}}}
      }
      {\eqterm
        {\Ga}
        {\funext{x:A_1}{B_1}{f_1}{g_1}{e_1}}
        {\funext{x:A_2}{B_2}{f_2}{g_2}{e_2}}
        {\Eq{}{f_1}{g_1}}
      }
    %

    \infer[]
      {\eqtype{\Ga}{A_1}{A_2} \\
       \eqterm{\Ga}{u_1}{u_2}{A_1} \\
       \eqterm{\Ga}{v_1}{v_2}{A_2} \\
       \eqterm{\Ga}{p_1}{p_2}{\Eq{A_1}{u_1}{v_1}} \\
       \eqterm{\Ga}{q_1}{q_2}{\Eq{A_1}{u_1}{v_1}} \\
      }
      {\eqterm
        {\Ga}
        {\uip{A_1}{u_1}{v_1}{p_1}{q_1}}
        {\uip{A_2}{u_2}{v_2}{p_2}{q_2}}
        {\Eq{}{p_1}{q_1}}
      }
    %
  \end{mathpar}
  \hrulefill
  \vspace{-2ex}
  \caption{Congruence rules}
  \label{fig:cong-rules}
\end{figure*}



\section{Proof of the fundamental lemma}
\label{sec:proof-fund-lemma}

\begin{lemma}[Fundamental lemma]
  Let $t_1$ and $t_2$ be two terms. If $\isterm{\Ga, \Ga_1}{t_1}{T_1}$ and
  $\isterm{\Ga, \Ga_2}{t_2}{T_2}$ and $t_1 \sim t_2$ then there exists $p$ such
  that
  $\isterm{\Ga, \Pack{\Ga_1}{\Ga_2}}
          {p}
          {\Heq{\llift{\gamma}{}{T_1}}
               {\llift{\gamma}{}{t_1}}
               {\rlift{\gamma}{}{T_2}}
               {\rlift{\gamma}{}{t_2}}}$.
\end{lemma}

For readability we will abbreviate the left and right substitutions
$\llift{\gamma}{}{\_}$ and $\rlift{\gamma}{}{\_}$ by $\upharpoonleft$
and $\upharpoonright$ respectively.

\begin{proof}
  We prove it by induction on the derivation of $t_1 \sim t_2$.

  \begin{itemize}
    \item \textsc{Var}
    \[
      \infer[]
        { }
        {x \sim x}
      %
    \]
    If $x$ belongs to $\Ga$, we apply reflexivity---together with uniqueness of
    typing~\eqref{lem:uniq}---to conclude.
    Otherwise, $\ProjE{x}$ has the expected type (since
    $\llift{\gamma}{}{x} \equiv \ProjO{x}$ and $\rlift{\gamma}{}{x} \equiv \ProjT{x}$).

    \item \textsc{TransportLeft}
    \[
      \infer[]
        {t_1 \sim t_2}
        {\translpo{p}\ t_1 \sim t_2}
      %
    \]
    We have $\isterm{\Ga, \Ga_1}{\translpo{p}\ t_1}{T_1}$ and
    $\isterm{\Ga, \Ga_2}{t_2}{T_2}$.
    By inversion~\eqref{lem:inversion} we have
    $\isterm{\Ga, \Ga_1}{p}{\Eq{}{T_1'}{T_1}}$ and
    $\isterm{\Ga, \Ga_1}{t_1}{T_1'}$.
    Then by induction hypothesis we have $e$ such that
    $\isterm{\Ga, \Gp}{e}{\Heq{}{\lo{t_1}}{}{\ro{t_2}}}$.
    From transitivity and symmetry we only need to provide a proof of
    $\Heq{}{\lo{t_1}}{}{\translpo{\lo{p}}\ \lo{t_1}}$ which is inhabited by
    $\pair{\_}{\_}{\lo{p}}{\refl{} (\translpo{\lo{p}}\ \lo{t_1})}$.

    \item \textsc{TransportRight}
    \[
      \infer[]
        {t_1 \sim t_2}
        {t_1 \sim \translpo{p}\ t_2}
      %
    \]
    Similarly.

    \item \textsc{Product}
    \[
      \infer[]
        {A_1 \sim A_2 \\
         B_1 \sim B_2
        }
        {\Prod{x:A_1} B_1 \sim \Prod{x:A_2} B_2}
      %
    \]
    We have $\isterm{\Ga, \Ga_1}{\Prod{x:A_1} B_1}{T_1}$ and
    $\isterm{\Ga, \Ga_2}{\Prod{x:A_2} B_2}{T_2}$ so by
    inversion~\eqref{lem:inversion} we have $\isterm{\Ga, \Ga_1}{A_1}{s_1}$ and
    $\isterm{\Ga, \Ga_1, x:A_1}{B_1}{s'_1}$ and
    $\eqtype{\Ga, \Ga_1}{s''_1}{T_1}$ for $(s_1,s'_1,s''_1) \in \Rl$
    (and similarly with $2$s).
    By induction hypothesis we have
    $\isterm{\Ga, \Gp}{p_A}{\Heq{}{\lo{A_1}}{}{\ro{A_2}}}$ and
    $\isterm
      {\Ga, \Gp, x : \Pack{A_1}{A_2}}
      {p_B}
      {\Heq{}{\lo{B_1}}{}{\ro{B_2}}}
    $
    hence the result (using UIP and FunExt, refer to the
    formalisation and especially to the file \rpath{Quotes.v} for more details
    on how to realise this equality).

    \item \textsc{Equality}
    \[
      \infer[]
        {A_1 \sim A_2 \\
         u_1 \sim u_2 \\
         v_1 \sim v_2
        }
        {\Eq{A_1}{u_1}{v_1} \sim \Eq{A_2}{u_2}{v_2}}
      %
    \]
    We have $\isterm{\Ga, \Ga_1}{\Eq{A_1}{u_1}{v_1}}{T_1}$ and
    $\isterm{\Ga, \Ga_2}{\Eq{A_2}{u_2}{v_2}}{T_2}$ so, by
    inversion~\eqref{lem:inversion}, we have
    $\isterm{\Ga, \Ga_1}{A_1}{s_1}$ and $\isterm{\Ga, \Ga_1}{u_1}{A_1}$ and
    $\isterm{\Ga, \Ga_1}{v_1}{A_1}$ as well as $\eqtype{\Ga, \Ga_1}{s_1}{T_1}$
    (and the same with $2$s). By induction hypothesis we thus have
    $\isterm{\Ga, \Gp}{p_{A}}{\Heq{}{A_1}{}{A_2}}$ and
    $\isterm{\Ga, \Gp}{p_u}{\Heq{}{u_1}{}{u_2}}$ and
    $\isterm{\Ga, \Gp}{p_v}{\Heq{}{v_1}{}{v_2}}$.
    We can thus conclude.

    \item \textsc{Reflexivity}
    \[
      \infer[]
        { }
        {s \sim s}
      %
    \]
    This one holds by reflexivity and uniqueness of typing~\eqref{lem:uniq}
    (indeed, $\lo{s} \equiv s$ and $\ro{s} \equiv s$).

    \item \textsc{Lambda}
    \[
      \infer[]
        {A_1 \sim A_2 \\
         B_1 \sim B_2 \\
         t_1 \sim t_2
        }
        {\lam{x:A_1}{B_1} t_1 \sim \lam{x:A_2}{B_2} t_2}
      %
    \]
    We have $\isterm{\Ga, \Ga_1}{\lam{x:A_1}{B_1} t_1}{T_1}$ and
    $\isterm{\Ga, \Ga_2}{\lam{x:A_2}{B_2} t_2}{T_2}$, thus, by
    inversion~\ref{lem:inversion} the subterms are well-typed and we can
    apply induction hypothesis. The conclusion follows similarly to the $\Pi$
    case.

    \item \textsc{Application}
    \[
      \infer[]
        {t_1 \sim t_2 \\
         A_1 \sim A_2 \\
         B_1 \sim B_2 \\
         u_1 \sim u_2
        }
        {\app{t_1}{x:A_1}{B_1}{u_1} \sim \app{t_2}{x:A_2}{B_2}{u_2}}
      %
    \]
    We have $\isterm{\Ga, \Ga_1}{\app{t_1}{x:A_1}{B_1}{u_1}}{T_1}$ and
    $\isterm{\Ga, \Ga_2}{\app{t_2}{x:A_2}{B_2}{u_2}}{T_2}$ which means by
    inversion~\eqref{lem:inversion} that the subterms are well-typed.
    We apply the induction hypothesis and then conclude.

    \item \textsc{Reflexivity}
    \[
      \infer[]
        {A_1 \sim A_2 \\
         u_1 \sim u_2
        }
        {\refl{A_1} u_1 \sim \refl{A_2} u_2}
      %
    \]
    We have $\isterm{\Ga, \Ga_1}{\refl{A_1} u_1}{T_1}$ and
    $\isterm{\Ga, \Ga_2}{\refl{A_2} u_2}{T_2}$ so by
    inversion~\eqref{lem:inversion} we have $\isterm{\Ga, \Ga_1}{A_1}{s_1}$ and
    $\isterm{\Ga, \Ga_1}{u_1}{A_1}$
    (same with $2$s). By IH we have $\Heq{}{\lo{A_1}}{}{\ro{A_2}}$ and
    $\Heq{\lo{A_1}}{\lo{u_1}}{\ro{A_2}}{\ro{u_2}}$.
    The proof follows easily.

    \item \textsc{Funext}
    \[
      \infer[]
        {A_1 \sim A_2 \\
         B_1 \sim B_2 \\
         f_1 \sim f_2 \\
         g_1 \sim g_2 \\
         e_1 \sim e_2
        }
        {\funext{x:A_1}{B_1}{f_1}{g_1}{e_1}
         \sim \funext{x:A_2}{B_2}{f_2}{g_2}{e_2}
        }
      %
    \]
    Similar.

    \item \textsc{UIP}
    \[
      \infer[]
        {A_1 \sim A_2 \\
         u_1 \sim u_2 \\
         v_1 \sim v_2 \\
         p_1 \sim p_2 \\
         q_1 \sim q_2
        }
        {\uip{A_1}{u_1}{v_1}{p_1}{q_1} \sim \uip{A_2}{u_2}{v_2}{p_2}{q_2}}
      %
    \]
    Similar.

    \item \textsc{J}
    \[
      \infer[]
        {A_1 \sim A_2 \\
         u_1 \sim u_2 \\
         P_1 \sim P_2 \\
         w_1 \sim w_2 \\
         v_1 \sim v_2 \\
         p_1 \sim p_2
        }
        {\J{A_1}{u_1}{x.e.P_1}{w_1}{v_1}{p_1} \sim
         \J{A_2}{u_2}{x.e.P_2}{w_2}{v_2}{p_2}
        }
      %
    \]
    Similar.
  \end{itemize}
\end{proof}

\section{Correctness of the translation}
\label{sec:corr-transl}
\begin{theorem}[Translation]
  \leavevmode
  \begin{itemize}
    \item If $\xisctx{\Ga}$ then there exists
    $\isctx{\Gb} \in \transl{\xisctx{\Ga}}$,

    \item If $\xisterm{\Ga}{t}{T}$ then for any
    $\isctx{\Gb} \in \transl{\xisctx{\Ga}}$ there exist $\tb$ and $\Tb$ such that
    $\isterm{\Gb}{\tb}{\Tb} \in \transl{\xisterm{\Ga}{t}{T}}$,

    \item If $\xeqterm{\Ga}{u}{v}{A}$ then for any
    $\isctx{\Gb} \in \transl{\xisctx{\Ga}}$ there exist
    $A \ir \Ab, A \ir \Ab', u \ir \ub, v \ir \vb$ and $\eb$ such that
    $\isterm{\Gb}{\eb}{\Heq{\Ab}{\ub}{\Ab'}{\vb}}$.
  \end{itemize}
\end{theorem}

\begin{proof}
  We prove the theorem by induction on the derivation in the extensional
  type theory. In most cases we need to assume some $\Gb$, translation of the
  context, we will implicitly refer to $\Gb$ in such cases as the one given as
  hypothesis.
  \leavevmode
  \begin{itemize}
    \item \textsc{Empty}
    \[
      \infer[]
        { }
        {\xisctx{\ctxempty}}
      %
    \]
    We have $\isctx{\ctxempty} \in \transl{\xisctx{\ctxempty}}$.

    \item \textsc{Extend}
    \[
      \infer[]
        {\xisctx{\Ga} \\
         \xistype{\Ga}{A}
        }
        {\xisctx{\Ga, x:A}}
      (x \notin \Ga)
    \]
    By IH we have $\isctx{\Gb} \in \transl{\xisctx{\Ga}}$ and, using $\Gb$
    as well as lemma~\ref{lem:choose},
    $\isterm{\Gb}{\Ab}{s} \in \transl{\xisterm{\Ga}{A}{s}}$.
    Thus $\isctx{\Gb, x:\Ab} \in \transl{\xisctx{\Ga, x:A}}$.

    \item \textsc{Sort}
    \[
      \infer[]
        {\xisctx{\Ga}}
        {\xisterm{\Ga}{s}{s'}}
      (s,s')
    \]
    We have $\isterm{\Gb}{s}{s'} \in \transl{\xisterm{\Ga}{s}{s'}}$.

    \item \textsc{Product}
    \[
      \infer[]
        {\xisterm{\Ga}{A}{s} \\
         \xisterm{\Ga,x:A}{B}{s'}
        }
        {\xisterm{\Ga}{\Prod{x:A} B}{s''}}
      (s,s',s'')
    \]
    By IH and lemma~\ref{lem:choose} we have $\isterm{\Gb}{\Ab}{s}$,
    meaning $\isctx{\Gb,x:\Ab} \in \transl{\xisctx{\Ga, x:A}}$,
    and then $\isterm{\Gb,x:\Ab}{\Bb}{s'}$.
    We thus conclude
    $\isterm{\Gb}{\Prod{x:\Ab} \Bb}{s''} \in
    \transl{\xisterm{\Ga}{\Prod{x:A} B}{s''}}$.

    \item \textsc{Sigma}
    \[
      \infer[]
        {\xisterm{\Ga}{A}{s} \\
         \xisterm{\Ga,x:A}{B}{s'}
        }
        {\xisterm{\Ga}{\Sum{x:A} B}{s''}}
      (s,s',s'')
    \]
    Similar.

    \item \textsc{Equality}
    \[
      \infer[]
        {\xisterm{\Ga}{A}{s} \\
         \xisterm{\Ga}{u}{A} \\
         \xisterm{\Ga}{v}{A}
        }
        {\xisterm{\Ga}{\Eq{A}{u}{v}}{s}}
      %
    \]
    By IH and lemma~\ref{lem:choose} we have $\isterm{\Gb}{\Ab}{s}$,
    and---using lemma~\ref{lem:change-type}---we also have
    $\isterm{\Gb}{\ub}{\Ab}$ and $\isterm{\Gb}{\vb}{\Ab}$.
    Then
    $\isterm{\Gb}{\Eq{\Ab}{\ub}{\vb}}{s} \in
    \transl{\xisterm{\Ga}{\Eq{A}{u}{v}}{s}}$.

    \item \textsc{Variable}
    \[
      \infer[]
        {\xisctx{\Ga} \\
         (x : A) \in \Ga
        }
        {\xisterm{\Ga}{x}{A}}
      %
    \]
    We have $\isctx{\Gb} \in \transl{\xisctx{\Ga}}$ (as we assumed, this is not
    an instance of the induction hypothesis) and $(x : A) \in \Ga$.
    By definition of $\Ga \ir \Gb$ we also have some $(x : \Ab) \in \Gb$
    with $A \ir \Ab$, thus
    $\isterm{\Gb}{x}{\Ab} \in \transl{\xisterm{\Ga}{x}{A}}$.

    \item \textsc{Conversion}
    \[
      \infer[]
        {\xisterm{\Ga}{u}{A} \\
         \xeqtype{\Ga}{A}{B}
        }
        {\xisterm{\Ga}{u}{B}}
      %
    \]
    By IH and lemma~\ref{lem:uip-cong} we have
    $\isterm{\Gb}{\eb}{\Eq{}{\Ab}{\Bb}}$ which implies
    $\istype{\Gb}{\Ab} \in \transl{\xistype{\Ga}{A}}$ by
    inversion~\eqref{lem:inversion}, thus, from lemma~\ref{lem:change-type}
    and IH we get $\isterm{\Gb}{\ub}{\Ab}$, yielding
    $\isterm{\Gb}{\translpo{\eb}\ \ub}{\Bb} \in \transl{\xisterm{\Ga}{u}{B}}$.

    \item \textsc{Lambda}
    \[
      \infer[]
        {\xisterm{\Ga}{A}{s} \\
         \xisterm{\Ga,x:A}{B}{s'} \\
         \xisterm{\Ga,x:A}{t}{B}
        }
        {\xisterm{\Ga}{\lam{x:A}{B} t}{\Prod{x:A} B}}
      %
    \]
    By IH and lemma~\ref{lem:choose} we have $\isterm{\Gb}{\Ab}{s}$ and thus
    $\isctx{\Gb, x:\Ab} \in \transl{\xisctx{\Ga,x:A}}$, meaning we can apply IH
    and lemma~\ref{lem:choose} to the second hypothesis to get
    $\isterm{\Gb,x:\Ab}{\Bb}{s'} \in \transl{\xisterm{\Ga,x:A}{B}{s'}}$ and then
    IH and lemma~\ref{lem:change-type} to get
    $\isterm{\Gb,x:\Ab}{\tb}{\Bb} \in \transl{\xisterm{\Ga,x:A}{t}{B}}$.
    All of this yields
    $\isterm{\Gb}{\lam{x:\Ab}{\Bb} \tb}{\Prod{x:\Ab} \Bb}
    \in \transl{\xisterm{\Ga}{\lam{x:A}{B} t}{\Prod{x:A} B}}$.

    \item \textsc{Application}
    \[
      \infer[]
        {\xisterm{\Ga}{A}{s} \\
         \xisterm{\Ga,x:A}{B}{s'} \\
         \xisterm{\Ga}{t}{\Prod{x:A} B} \\
         \xisterm{\Ga}{u}{A}
        }
        {\xisterm{\Ga}{\app{t}{x:A}{B}{u}}{B[x \sto u]}}
      %
    \]
    Using IH together with lemmata~\ref{lem:choose} and~\ref{lem:change-type}
    we get $\isterm{\Gb}{\Ab}{s}$ and $\isterm{\Gb,x:\Ab}{\Bb}{s'}$ and
    $\isterm{\Gb}{\tb}{\Prod{x:\Ab} \Bb}$ and $\isterm{\Gb}{\ub}{\Ab}$
    meaning we can conclude
    $\isterm{\Gb}{\app{\tb}{x:\Ab}{\Bb}{\ub}}{\Bb[x \sto \ub]}
    \in \transl{\xisterm{\Ga}{\app{t}{x:A}{B}{u}}{B[x \sto u]}}$.

    \item \textsc{Pair}
    \[
      \infer[]
        {\xisterm{\Ga}{u}{A} \\
         \xisterm{\Ga}{A}{s} \\
         \xisterm{\Ga,x:A}{B}{s'} \\
         \xisterm{\Ga}{v}{B[x \sto u]}
        }
        {\xisterm{\Ga}{\pair{x:A}{B}{u}{v}}{\Sum{x:A}{B}}}
      %
    \]
    Using IH with lemmata~\ref{lem:choose} and~\ref{lem:change-type} we
    translate all the hypotheses to conclude
    $\isterm{\Gb}{\pair{x:\Ab}{\Bb}{\ub}{\vb}}{\Sum{x:\Ab}{\Bb}}
    \in \transl{\xisterm{\Ga}{\pair{x:A}{B}{u}{v}}{\Sum{x:A}{B}}}$.

    \item \textsc{Proj$_1$}
    \[
      \infer[]
        {\xisterm{\Ga}{p}{\Sum{x:A}{B}}}
        {\xisterm{\Ga}{\pio{x:A}{B}{p}}{A}}
      %
    \]
    Similar.

    \item \textsc{Proj$_2$}
    \[
      \infer[]
        {\xisterm{\Ga}{p}{\Sum{x:A}{B}}}
        {\xisterm{\Ga}{\pit{x:A}{B}{p}}{B[x \sto \pio{x:A}{B}{p}]}}
      %
    \]
    Similar.

    \item \textsc{Reflexivity}
    \[
      \infer[]
        {\xisterm{\Ga}{A}{s} \\
         \xisterm{\Ga}{u}{A}
        }
        {\xisterm{\Ga}{\refl{A} u}{\Eq{A}{u}{u}}}
      %
    \]
    By IH we have $\isterm{\Gb}{\ub}{\Ab}$ and thus
    $\isterm{\Gb}{\refl{\Ab} \ub}{\Eq{\Ab}{\ub}{\ub}} \in
    \transl{\xisterm{\Ga}{\refl{A} u}{\Eq{A}{u}{u}}}$.

    \item \textsc{J}
    \[
      \infer[]
        {\xisterm{\Ga}{A}{s} \\
         \xisterm{\Ga}{u,v}{A} \\
         \xisterm{\Ga, x:A, e:\Eq{A}{u}{x}}{P}{s'} \\
         \xisterm{\Ga}{p}{\Eq{A}{u}{v}} \\
         \xisterm{\Ga}{w}{P[x \sto u, e \sto \refl{A} u]}
        }
        {\xisterm
          {\Ga}
          {\J{A}{u}{x.e.P}{w}{v}{p}}
          {P[x \sto v, e \sto p]}
        }
      %
    \]
    By IH and lemma~\ref{lem:choose} we have $\isterm{\Gb}{\Ab}{s}$.
    From this and IH and lemma~\ref{lem:change-type} we have
    $\isterm{\Gb}{\ub,\vb}{\Ab}$. We can thus deduce
    $\isctx{\Gb,x:\Ab,e:\Eq{\Ab}{\ub}{x}} \in \transl{\Ga, x:A, e:\Eq{A}{u}{x}}$
    which in turn gives us
    $\isterm{\Gb, x:\Ab,e:\Eq{\Ab}{\ub}{x}}{\Pb}{s'}$.
    Similarly we also get $\isterm{\Gb}{\pb}{\Eq{\Ab}{\ub}{\vb}}$ and
    $\isterm{\Gb}{\wb}{\Pb[x \sto \ub, e \sto \refl{\Ab} \ub]}$.
    All of this allows us to conclude
    $\isterm
      {\Gb}
      {\J{\Ab}{\ub}{x.e.\Pb}{\wb}{\vb}{\pb}}
      {\Pb[x \sto \vb, e \sto \pb]}
    \in
    \transl{
      \xisterm
        {\Ga}
        {\J{A}{u}{x.e.P}{w}{v}{p}}
        {P[x \sto v, e \sto p]}
    }$.

    \item \textsc{Funext}
    \[
      \infer[]
        {\isterm{\Ga}{f,g}{\Prod{x:A} B} \\
         \isterm
           {\Ga}
           {e}
           {\Prod{x:A} \Eq{B}{\app{f}{x:A}{B}{x}}{\app{g}{x:A}{B}{x}}}
        }
        {\isterm{\Ga}{\funext{x:A}{B}{f}{g}{e}}{\Eq{}{f}{g}}}
      %
    \]
    Similar.

    \item \textsc{UIP}
    \[
      \infer[]
        {\xisterm{\Ga}{e_1,e_2}{\Eq{A}{u}{v}}}
        {\xisterm{\Ga}{\uip{A}{u}{v}{e_1}{e_2}}{\Eq{}{e_1}{e_2}}}
      %
    \]
    Similar.

    \item \textsc{Beta}
    \[
      \infer[]
        {\xisterm{\Ga}{A}{s} \\
         \xisterm{\Ga,x:A}{B}{s'} \\
         \xisterm{\Ga,x:A}{t}{B} \\
         \xisterm{\Ga}{u}{A}
        }
        {\xeqterm
          {\Ga}
          {\app
            {(\lam{x:A}{B} t)}
            {x:A}
            {B}
            {u}
          }
          {t[x \sto u]}
          {B[x \sto u]}
        }
      %
    \]
    From IH and the lemmata, we even get the conversion, we conclude using
    reflexivity.

    \item \textsc{Proj$_1$-Red}
    \[
      \infer[]
        {\xisterm{\Ga}{A}{s} \\
         \xisterm{\Ga}{u}{A} \\
         \xisterm{\Ga,x:A}{B}{s'} \\
         \xisterm{\Ga}{v}{B[x \sto u]}
        }
        {\xeqterm
          {\Ga}
          {\pio{x:A}{B}{\pair{x:A}{B}{u}{v}}}
          {u}
          {A}
        }
      %
    \]
    Likewise.

    \item \textsc{Proj$_2$-Red}
    \[
      \infer[]
        {\xisterm{\Ga}{A}{s} \\
         \xisterm{\Ga}{u}{A} \\
         \xisterm{\Ga,x:A}{B}{s'} \\
         \xisterm{\Ga}{v}{B[x \sto u]}
        }
        {\xeqterm
          {\Ga}
          {\pit{x:A}{B}{\pair{x:A}{B}{u}{v}}}
          {v}
          {B[x \sto u]}
        }
      %
    \]
    Likewise.

    \item \textsc{J-Red}
    \[
      \infer[]
        {\xisterm{\Ga}{A}{\Un{i}} \\
         \xisterm{\Ga}{u}{A} \\
         \xisterm{\Ga, x:A, e:\Eq{A}{u}{x}}{P}{\Un{j}} \\
         \xisterm{\Ga}{w}{P[x \sto u, e \sto \refl{A} u]}
        }
        {\xeqterm
          {\Ga}
          {\J{A}{u}{x.e.P}{w}{u}{\refl{A} u}}
          {w}
          {P[x \sto u, e \sto \refl{A} u]}
        }
      %
    \]
    Likewise.

    \item \textsc{Conv-Refl}
    \[
      \infer[]
        {\xisterm{\Ga}{u}{A}}
        {\xeqterm{\Ga}{u}{u}{A}}
      %
    \]
    We conclude from IH and reflexivity of $\Heqs$.

    \item \textsc{Conv-Sym}
    \[
      \infer[]
        {\xeqterm{\Ga}{u}{v}{A}}
        {\xeqterm{\Ga}{v}{u}{A}}
      %
    \]
    We conclude from IH and symmetry of $\Heqs$.

    \item \textsc{Conv-Trans}
    \[
      \infer[]
        {\xeqterm{\Ga}{u}{v}{A} \\
         \xeqterm{\Ga}{v}{w}{A}
        }
        {\xeqterm{\Ga}{u}{w}{A}}
      %
    \]
    We conclude from IH and transitivity of $\Heqs$.

    % \item \textsc{}
    % \[
    %   \infer[]
    %     {\xisterm{\Ga}{A}{s} \\
    %      \xisterm{\Ga, x:A}{B}{s'} \\
    %      \xisterm{\Ga}{f}{\Prod{x:A} B}
    %     }
    %     {\xeqterm
    %       {\Ga}
    %       {\lam{x:A}{B} \app{f}{x:A}{B}{x}}
    %       {f}
    %       {\Prod{x:A} B}
    %     }
    %   %
    % \]
    % This works like the computation cases.

    \item \textsc{Conv-Conv}
    \[
      \infer[]
        {\xeqterm{\Ga}{t_1}{t_2}{T_1} \\
         \xeqtype{\Ga}{T_1}{T_2}
        }
        {\xeqterm{\Ga}{t_1}{t_2}{T_2}}
      %
    \]
    By IH (and lemma~\ref{lem:uip-cong}) we have
    $\isterm{\Gb}{\eb}{\Heq{\Tb_1}{\tb_1}{\Tb'_1}{\tb_2}}$ and
    $\isterm{\Gb}{p}{\Eq{}{\Tb''_1}{\Tb_2}}$. Also from
    lemmata~\ref{lem:sim-cong} and~\ref{lem:uip-cong} we have
    $\Eq{}{\Tb'_1}{\Tb''_1}$ and $\Eq{}{\Tb_1}{\Tb''_1}$, meaning we get
    $\Eq{}{\Tb'_1}{\Tb_2}$ and $\Eq{}{\Tb_1}{\Tb_2}$.
    This allows us to conclude by transporting along the aforementioned
    equalities.

    \item \textsc{Conv-Prod}
    \[
      \infer[]
        {\xeqterm{\Ga}{A_1}{A_2}{s} \\
         \xeqterm{\Ga,x:A_1}{B_1}{B_2}{s'}
        }
        {\xeqterm{\Ga}{\Prod{x:A_1} B_1}{\Prod{x:A_2} B_2}{s''}}
      (s,s',s'')
    \]
    We conclude exactly like we did in the proof of lemma~\ref{lem:sim-cong}.

    \item All congruences hold like in proof of
    lemma~\ref{lem:sim-cong}.

    \item \textsc{Conv-Eq}
    \[
      \infer[]
        {\xisterm{\Ga}{e}{\Eq{A}{u}{v}}}
        {\xeqterm{\Ga}{u}{v}{A}}
      %
    \]
    By IH and lemma~\ref{lem:choose} we have
    $\isterm{\Gb}{\eb}{\Eq{\Ab}{\ub}{\vb}}
    \in \transl{\xisterm{\Ga}{e}{\Eq{A}{u}{v}}}$ which yields the conclusion we
    wanted.
  \end{itemize}
\end{proof}

%----------------------------------------------------------------------------------------

\backmatter % Denotes the end of the main document content
\setchapterstyle{plain} % Output plain chapters from this point onwards

%----------------------------------------------------------------------------------------
%	BIBLIOGRAPHY
%----------------------------------------------------------------------------------------

% The bibliography needs to be compiled with biber using your LaTeX editor, or on the command line with 'biber main' from the template directory

\defbibnote{bibnote}{\label{bib}Here are the references in citation order.\par\bigskip} % Prepend this text to the bibliography
\printbibliography[heading=bibintoc, title=Bibliography, prenote=bibnote] % Add the bibliography heading to the ToC, set the title of the bibliography and output the bibliography note

%----------------------------------------------------------------------------------------
%	NOMENCLATURE
%----------------------------------------------------------------------------------------

% The nomenclature needs to be compiled on the command line with 'makeindex main.nlo -s nomencl.ist -o main.nls' from the template directory

\nomenclature{$\isjg{\Gamma}{J}$}{Judgment in some type theory (often \acrshort{ITT} or \Coq)}
\nomenclature{$\xisjg{\Gamma}{J}$}{Judgment in \acrshort{ETT}}

\renewcommand{\nomname}{Notation} % Rename the default 'Nomenclature'
\renewcommand{\nompreamble}{The next list describes several symbols that will be later used within the body of the document.} % Prepend this text to the nomenclature

\printnomenclature % Output the nomenclature

%----------------------------------------------------------------------------------------
%	GLOSSARY
%----------------------------------------------------------------------------------------

% The glossary needs to be compiled on the command line with 'makeglossaries main' from the template directory

\newglossaryentry{computer}{
	name=computer,
	description={is a programmable machine that receives input, stores and manipulates data, and provides output in a useful format}
}

% Glossary entries (used in text with e.g. \acrfull{fpsLabel} or \acrshort{fpsLabel})
\newacronym[longplural={Frames per Second}]{fpsLabel}{FPS}{Frame per Second}
\newacronym[longplural={Tables of Contents}]{tocLabel}{TOC}{Table of Contents}
\newacronym[longplural={Extensional Type Theories}]{ETT}{ETT}{Extensional Type Theory}
\newacronym{ITT}{ITT}{Intensional Type Theory}
\newacronym{WTT}{WTT}{Weak Type Theory}
\newacronym[longplural={Pure Type Systems}]{PTS}{PTS}{Pure Type System}
\newacronym{PCUIC}{PCUIC}{Predicative Calculus of Cumulative Inductive Constructions}
\newacronym{CoC}{CoC}{Calculus of Constructions}
\newacronym{CIC}{CIC}{Calculus of Inductive Constructions}
\newacronym{MLTT}{MLTT}{Martin-Löf Type Theory}
\newacronym{UIP}{UIP}{Uniqueness of Identity Proofs}
\newacronym{funext}{funext}{functional extensionality}
\newacronym{JMeq}{JMeq}{John Major equality}
\newacronym{LEM}{LEM}{Law of Excluded Middle}
\newacronym{HoTT}{HoTT}{Homotopy Type Theory}
\newacronym{CubicalTT}{CubicalTT}{Cubical Type Theory}
\newacronym[longplural={2-level Type Theories}]{2TT}{2TT}{2-level Type Theory}
\newacronym{2WTT}{2TT}{2-level Weak Type Theory}
\newacronym{HTS}{HTS}{Homotopy Type System}
\newacronym{CoqMT}{CoqMT}{Coq Modulo Theory}
\newacronym{OTT}{OTT}{Observational Type Theory}
\newacronym{STT}{STT}{Setoïd Type Theory}
\newacronym{TCB}{TCB}{Trusted Code Base}
\newacronym{TTB}{TTB}{Trusted Theory Base}
\newacronym{CbV}{CbV}{Call by Value}
\newacronym{STL}{STL}{Simply Typed \(\lambda\)-calculus}
\newacronym[longplural={Quotient Inductive Types}]{QIT}{QIT}{Quotient Inductive Type}
\newacronym[longplural={Categories with Families}]{CwF}{CwF}{Category with Families}
\newacronym{HOAS}{HOAS}{Higher Order Abstract Syntax}

\setglossarystyle{listgroup} % Set the style of the glossary (see https://en.wikibooks.org/wiki/LaTeX/Glossary for a reference)
\printglossary[title=Special Terms, toctitle=List of Terms] % Output the glossary, 'title' is the chapter heading for the glossary, toctitle is the table of contents heading

%----------------------------------------------------------------------------------------
%	INDEX
%----------------------------------------------------------------------------------------

% The index needs to be compiled on the command line with 'makeindex main' from the template directory

\printindex % Output the index

%----------------------------------------------------------------------------------------
%	BACK COVER
%----------------------------------------------------------------------------------------

% If you have a PDF/image file that you want to use as a back cover, uncomment the following lines

%\clearpage
%\thispagestyle{empty}
%\null%
%\clearpage
%\includepdf{cover-back.pdf}

%----------------------------------------------------------------------------------------


\end{document}