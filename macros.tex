% TODO ref
\newcommand{\misref}{\textcolor{orange}{\emph{[??]}}}

% Reference to bibliography
\newcommand{\refbib}{\hyperref[bib]{Bibliography}}

% Reminder box
\newcommand{\reminder}[3][0pt]{%
  \marginnote[#1]{
    \begin{kaobox}[
      frametitle=Reminder: #2,
      backgroundcolor=YellowGreen!25!White,
      frametitlebackgroundcolor=YellowGreen!25!White
    ]
      #3
    \end{kaobox}
  }
}

% Definition box
\newcommand{\sidedef}[3][0pt]{%
  \marginnote[#1]{
    \begin{kaobox}[
      frametitle=Definition: #2,
      backgroundcolor=GreenYellow!25!White,
      frametitlebackgroundcolor=GreenYellow!25!White
    ]
      #3
    \end{kaobox}
  }
}

% Warning box
\newcommand{\warn}[1]{%
  \begin{kaobox}[
    % frametitle=,
    backgroundcolor=OrangeRed!25!White,
    % frametitlebackgroundcolor=GreenYellow!25!White
  ]
    #1
  \end{kaobox}
}

% Highlight for maths
\colorlet{highlightcol}{Yellow!50}
\colorlet{bluelightcol}{CornflowerBlue!50}
\newcommand{\highlight}[2][highlightcol]{%
  \colorbox{#1}{$\displaystyle#2$}%
}

% French refs
\newcommand{\partnamefr}{Partie}
\newcommand{\chapternamefr}{Chapitre}
\newcommand{\refchfr}[1]{\hyperref[ch:#1]{\chapternamefr\xspace\ref{ch:#1}}}
\newcommand{\nrefchfr}[1]{\hyperref[ch:#1]{\chapternamefr\xspace\ref{ch:#1} (\nameref{ch:#1})}}
\newcommand{\subsectionnamefr}{Sous-section}
\newcommand{\arefsubsecfr}[1]{\hyperref[subsec:#1]{\subsectionnamefr\xspace`\nameref{subsec:#1}'}}
\newcommand{\avrefpartfr}[1]{\hyperref[part:#1]{\partnamefr}\xspace`\nameref{part:#1}', \vpageref{part:#1}} % Part `Name of the Part' on page 84

% Faded text
\colorlet{fadedcol}{Gray!50}
\newcommand{\faded}[1]{\textcolor{fadedcol}{#1}}

% Stack highlight
\newcounter{stack}
\colorlet{stackcol}{Green}
\newcommand{\stack}[1]{%
  % \stepcounter{stack}%
  \addtocounter{stack}{20}%
  \colorbox{white!\the\value{stack}!stackcol}{$\displaystyle#1$}%
  \addtocounter{stack}{-20}%
}
\newcommand{\shole}{%
  \colorbox{white}{\text{\phantom{X}}}
}

% Stack command
\newcommand{\vscmd}[2]{\langle #1 \mid #2 \rangle}
\newcommand{\zip}{\mathsf{zip}}

% Stack shortcuts
\def\coapp#1{\stack{\shole\ #1}}

% Names
\def\name#1{\textsf{#1}\xspace}
\def\Coq{\name{Coq}}
\def\CoqMT{\name{CoqMT}}
\def\ocaml{\name{ocaml}}
\def\MetaCoq{\name{MetaCoq}}
\def\TemplateCoq{\name{TemplateCoq}}
\def\ltac{\name{LTac}}
\def\Program{\name{Program}}
\def\Equations{\name{Equations}}
\def\Andromeda{\name{Andromeda}}
\def\NuPRL{\name{NuPRL}}
\def\Agda{\name{Agda}}
\def\Epigram{\name{EPIGRAM}}
\def\LK{\name{LK}}
\def\LJ{\name{LJ}}
\def\NK{\name{NK}}
\def\NJ{\name{NJ}}
\def\IsaHOL{\name{Isabelle/HOL}}

% Repos
\def\repo#1{\texttt{#1}\xspace}
\def\ftt{\repo{formal-type-theory}}

% Often used expressions
\def\cS{\ensuremath{\mathcal{S}}\xspace}
\def\cT{\ensuremath{\mathcal{T}}\xspace}

% Relations
\def\cumul{\preceq}

% Notations
\newcommand{\dpair}[1]{\langle #1 \rangle}

% CwF
\newcommand{\setfun}[2]{\mathcal{F}(#1,#2)}
\def\Con{\mathsf{Con}}
\def\CTy{\mathsf{Ty}}
\def\CTm{\mathsf{Tm}}
\def\pp{\mathsf{p}}
\def\pq{\mathsf{q}}
\def\Set{\mathsf{Set}}
\def\Card{\mathsf{Card}}
\def\capp{\mathsf{app}}
\def\cid{\mathsf{id}}
\def\CId{\mathsf{Id}}
\def\crefl{\mathsf{refl}}
\newcommand{\czero}{\mathsf{zero}}
\newcommand{\csucc}{\mathsf{succ}}
\def\CU{\mathsf{U}}
\def\CEl{\mathsf{El}}
% \newcommand{\card}[1]{\left\|#1\right\|}
\newcommand{\card}[1]{\left|#1\right|}
\def\ctwo{\mathsf{two}}

% de Bruijn indices
\newcommand{\db}[1]{\underline{#1}}

% formal-type-theory
\newcommand{\sbzero}[2]{\{#2\}_{#1}}     % substitition of the zeroth index
\newcommand{\sbweak}[1]{\mathsf{w}_{#1}} % weakening substitution
\newcommand{\sbshift}[2]{(#2 {\mid} #1)} % shifted substitution
\newcommand{\sbid}{\mathsf{id}}          % identity substitution
% \newcommand{\sbcomp}[2]{#1 \circ #2}     % composition of substitutions
\newcommand{\sbterminal}{\triangleright} % terminal substitution

% Neutral / normal
\newcommand{\whne}{\mathsf{whne}}
\newcommand{\whnf}{\mathsf{whnf}}

% Machine
\newcommand{\WM}{\mathsf{W}}

% Conversion machine
\newcommand{\cred}{\mathsf{R}}
\newcommand{\cterm}{\mathsf{T}}
\newcommand{\cargs}{\mathsf{A}}
\newcommand{\cfall}{\mathsf{F}}
\newcommand{\mconv}{\overset{?}{\equiv}}
\newcommand{\isconv}[7][]{
  #2 \vdash \vscmd{#3}{#4} \mconv_{#1} \vscmd{#5}{#6} \rightrightarrows #7
}
\newcommand{\convred}[6]{\isconv[\cred]{#1}{#2}{#3}{#4}{#5}{#6}}
\newcommand{\convterm}[6]{\isconv[\cterm]{#1}{#2}{#3}{#4}{#5}{#6}}
\newcommand{\convargs}[6]{\isconv[\cargs]{#1}{#2}{#3}{#4}{#5}{#6}}
\newcommand{\convfall}[6]{\isconv[\cfall]{#1}{#2}{#3}{#4}{#5}{#6}}
\newcommand{\ayes}{\text{yes}}
\newcommand{\stackctx}{\mathsf{stack\_context}}