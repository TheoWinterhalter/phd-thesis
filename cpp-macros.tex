%%% General

\newcommand{\inv}{^{-1}}

% \newcommand{\eg}{e.g.\ }
% \newcommand{\ie}{\emph{i.e.,}\xspace}

\newcommand{\paradot}[1]{\paragraph{#1.}}

\newcommand{\exmark}{\ensuremath{\mathsf{x}}}

%% Proof by cases
\newenvironment{caselist}{%
  \begin{list}{{\it Case}}{}%
}{\end{list}%
}
\newenvironment{subcaselist}{%
  \begin{list}{{\it Subcase}}{}%
}{\end{list}%
}
\newenvironment{subsubcaselist}{%
  \begin{list}{{\it Subsubcase}}{}%
}{\end{list}%
}

\newcommand{\nextcase}{\item~}

%%% Type theory

% Meta
\newcommand{\Ax}{\ensuremath{\mathsf{Ax}}}
\newcommand{\Rl}{\ensuremath{\mathsf{R}}}

% Generic entities
\newcommand{\Ga}{\Gamma}   % a context
\newcommand{\D}{\Delta}   % another context
\newcommand{\E}{\Xi}      % another context
\newcommand{\sbs}{\sigma} % a substitution
\newcommand{\sbt}{\theta} % another substitution
\newcommand{\sbr}{\rho}   % a third substitution

%% Syntax
\newcommand{\bnf}{\ \mathrel{{:}{:}{=}}\ }
\newcommand{\bnfor}{\ \mid\ \ }

%% Contexts
\newcommand{\ctxempty}{\bullet} % empty context
% \newcommand{\ctxextend}[2]{#1, #2} % extended context

%\newcommand{\ctxdom}[1]{\mathsf{dom}(#1)} % the domain of a context

%% Substitution

\newcommand{\sto}{\mathop{\leftarrow}}

% Reduction
\newcommand{\red}{\rightarrowtriangle}
\newcommand{\redl}{\leftarrowtriangle}
\newcommand{\reds}{\red^{\star}}

%% Marker
\newcommand{\stt}[1]{#1^{\mathsf{s}}}
\newcommand{\fm}[1]{#1^{\mathsf{f}}}


% Type formers
\newcommand{\Prod}[1]{\mathop{\Pi(#1).\ }}                 % dependent product
\newcommand{\Sum}[1]{\mathop{\Sigma(#1).}}               % dependent sum
\newcommand{\Eq}[3]{#2 =_{#1} #3}                        % fibrant equality
\newcommand{\Eqs}[3]{#2 \overset{\mathsf{s}}{=}_{#1} #3} % strict equality
\newcommand{\Ty}[1]{\square_{#1}}
\newcommand{\Un}[1]{\mathsf{U}_{#1}}                      % strict universe
\newcommand{\F}[1]{\mathsf{F}_{#1}}                      % fibrant universe
\newcommand{\Type}{\mathsf{Type}}
\newcommand{\Prop}{\mathsf{Prop}}
\newcommand{\SProp}{\mathsf{SProp}}
\newcommand{\nat}{\mathsf{nat}}
% \newcommand{\zero}{\mathsf{zero}}
\newcommand{\zero}{0}
% \newcommand{\natsucc}{\mathsf{succ}}
\newcommand{\natsucc}{\mathbf{S}}
\newcommand{\natrec}{\mathsf{natrec}}
\newcommand{\bool}{\mathsf{bool}}
\newcommand{\unit}{\mathsf{unit}}

% Inductive types
\newcommand{\Ical}{\mathcal{I}}
\newcommand{\at}[1]{\{#1\}}

% Terms
\newcommand{\lam}[2]{\lambda (#1). #2 .}                        % abstraction
\newcommand{\app}[4]{#1\mathbin{@_{#2.#3}} #4}                  % application
\newcommand{\pair}[4]{\langle #3 ; #4 \rangle_{#1.#2}}          % pair
\newcommand{\pio}[3]{\pi_1^{#1.#2}\ #3}                         % first proj
\newcommand{\pit}[3]{\pi_2^{#1.#2}\ #3}                         % second proj
\newcommand{\idpath}[1]{{\mathsf{idpath}_{#1}}\ }               % identity path
\newcommand{\refl}[1]{{\mathsf{refl}}_{#1}\ }                   % reflexivity
\newcommand{\funexts}[5]
  {\stt{\mathsf{funext}}(#1,#2,#3,#4,#5)}                        % strict funext
\newcommand{\funext}[5]
  {\mathsf{funext}(#1,#2,#3,#4,#5)}                             % fibrant funext
\newcommand{\uip}[5]{\mathsf{uip}(#1,#2,#3,#4,#5)}              % uip
\newcommand{\J}[6]{\mathsf{J}(#1,#2,#3,#4,#5,#6)}               % the J elim
\newcommand{\Js}[6]{\mathsf{J}^{\mathsf{s}}(#1,#2,#3,#4,#5,#6)} % the J elim
\newcommand{\ttrue}{\mathsf{true}}
\newcommand{\ffalse}{\mathsf{false}}
\newcommand{\tif}[4]{\mathsf{if}\ #1\ \mathsf{return}\ #2\ \mathsf{then}\ #3\ \mathsf{else}\ #4}
\newcommand{\notfunext}{\mathsf{notfunext}}
\newcommand{\Notfunext}{\mathsf{NotFunExt}}
\newcommand{\tunit}{\mathsf{()}}
\newcommand{\inl}{\mathsf{inl}}
\newcommand{\inr}{\mathsf{inr}}
\newcommand{\mto}{\Longrightarrow}
\newcommand{\branch}[2]{\mid #1 &\mto& #2}
\newcommand{\pmatch}[3]{
  \begin{array}{lcl}
    \multicolumn{3}{l}{\mathsf{match}\ #1\ \mathsf{return}\ #2\ \mathsf{with}} \\
    #3 \\
    \multicolumn{3}{l}{\mathsf{end}} \\
  \end{array}
}
\newcommand{\ax}[1]{\mathsf{ax}(#1)}
\newcommand{\lamb}[2]{\overline{\lambda}(#1). #2.}
\newcommand{\appb}[4]{#1\mathbin{\overline{@}_{#2.#3}} #4}
\newcommand{\pairb}[4]{\overline{\langle} #3 ; #4 \overline{\rangle}_{#1.#2}}
\newcommand{\piob}[3]{\overline{\pi}_1^{#1.#2}\ #3}
\newcommand{\pitb}[3]{\overline{\pi}_2^{#1.#2}\ #3}
\newcommand{\Jb}{\overline{\mathsf{J}}}
\newcommand{\pib}{\overline{\Pi}}
\newcommand{\sumb}{\overline{\Sigma}}
\newcommand{\redJ}{\beta_{\mathsf{J}}}
\newcommand{\redpio}{\beta_{\pi_1}}
\newcommand{\redpit}{\beta_{\pi_2}}
\newcommand{\fixp}{\mathsf{fix}}

% Sorts
\newcommand{\succs}[1]{#1+1}
\newcommand{\pisort}[2]{\mathsf{pi}(#1,#2)}
\newcommand{\sumsort}[2]{\mathsf{sig}(#1,#2)}
\newcommand{\eqsort}[1]{\mathsf{eq}(#1)}

% Pattern-matching (TODO Duplicate...)
\newcommand{\matchs}{\mathsf{match}}
\newcommand{\patt}[2]{\mid #1 \Longrightarrow #2}
\newcommand{\match}[3]{%
  \begin{array}{l}%
    \matchs\ #1 \\ \mathsf{return}\ #2\ \mathsf{with} \\ #3 \\ \mathsf{end}%
  \end{array}%
}

% Vectors
\newcommand{\av}{\vec{a}}
\newcommand{\bv}{\vec{b}}
\newcommand{\iv}{\vec{i}}
\newcommand{\pv}{\vec{p}}
\newcommand{\sv}{\vec{s}}
\newcommand{\Av}{\vec{A}}
\newcommand{\Iv}{\vec{I}}
\newcommand{\Nv}{\vec{N}}
\newcommand{\Pv}{\vec{P}}
\newcommand{\Vv}{\vec{V}}
\newcommand{\Xv}{\vec{X}}
\newcommand{\Yv}{\vec{Y}}

% Judgments
\newcommand{\isctx}[1]{\vdash #1}          % well formed context
\newcommand{\istype}[2]{#1 \vdash #2}      % well formed type
\newcommand{\isterm}[3]{#1 \vdash #2 : #3} % well formed term

\newcommand{\eqtype}[3]{#1 \vdash #2 \equiv #3}      % equal types
\newcommand{\eqterm}[4]{#1 \vdash #2 \equiv #3 : #4} % equal terms

\newcommand{\isjg}[2]{#1 \vdash \mathcal{#2}} % arbitrary judgment

% Ext. judgments
\newcommand{\xisctx}[1]{\vdash_{\exmark} #1}          % well formed context
\newcommand{\xistype}[2]{#1 \vdash_{\exmark} #2}      % well formed type
\newcommand{\xisterm}[3]{#1 \vdash_{\exmark} #2 : #3} % well formed term

\newcommand{\xeqtype}[3]{#1 \vdash_{\exmark} #2 \equiv #3}      % equal types
\newcommand{\xeqterm}[4]{#1 \vdash_{\exmark} #2 \equiv #3 : #4} % equal terms

\newcommand{\xisjg}[2]{#1 \vdash_{\exmark} \mathcal{#2}} % arbitrary judgment

% Translation
\newcommand{\transl}[1]{\llbracket #1 \rrbracket}
\newcommand{\So}{\mathsf{S}}
\newcommand{\HTS}{\mathsf{HTS}}
\newcommand{\Gb}{\overline{\Ga}}
\newcommand{\Db}{\overline{\D}}
\newcommand{\Ab}{\overline{A}}
\newcommand{\Bb}{\overline{B}}
\newcommand{\Tb}{\overline{T}}
\newcommand{\Pb}{\overline{P}}
\newcommand{\eb}{\overline{e}}
\newcommand{\jg}{\mathcal{J}}
\newcommand{\jgb}{\overline{\jg}}
\newcommand{\ub}{\overline{u}}
\newcommand{\vb}{\overline{v}}
\newcommand{\wb}{\overline{w}}
\newcommand{\pb}{\overline{p}}
\newcommand{\tb}{\overline{t}}
\newcommand{\fb}{\overline{f}}
\newcommand{\gb}{\overline{g}}

\newcommand{\transpo}[1]{#1_*}
\newcommand{\otransport}[4]{\mathsf{transport}'_{#1,#2}(#3,#4)}
\newcommand{\transport}[4]{\mathsf{transport}_{#1,#2}(#3,#4)}

\newcommand{\translitivity}[2]{\mathsf{transitivity}(#1,#2)}
\newcommand{\otransitivity}[2]{\mathsf{transitivity}'(#1,#2)}

% Operations
\newcommand{\nmax}[2]{\mathsf{max}(#1,#2)}

% Notations
\newcommand{\Heqs}{\cong}
\newcommand{\Heq}[4]{#2 \mathrel{{}_{#1}{\Heqs}_{#3}} #4}
\newcommand{\ir}{\sqsubset}
\newcommand{\Pack}[2]{\mathsf{Pack}\ #1\ #2}
% \newcommand{\ProjO}[3]{\mathsf{Proj}_1\ #1\ #2\ #3}
% \newcommand{\ProjT}[3]{\mathsf{Proj}_2\ #1\ #2\ #3}
% \newcommand{\ProjE}[3]{\mathsf{Proj}_e\ #1\ #2\ #3}
\newcommand{\ProjO}[1]{\mathsf{Proj}_1\ #1}
\newcommand{\ProjT}[1]{\mathsf{Proj}_2\ #1}
\newcommand{\ProjE}[1]{\mathsf{Proj}_\mathsf{e}\ #1}
\newcommand{\llift}[3]{#3 {[#1_1]}^{#2}}
\newcommand{\rlift}[3]{#3 {[#1_2]}^{#2}}
% \newcommand{\lo}[1]{\llift{}{}{#1}}
% \newcommand{\ro}[1]{\rlift{}{}{#1}}
\newcommand{\lo}[1]{#1 \upharpoonleft}
\newcommand{\ro}[1]{#1 \upharpoonright}
\newcommand{\Gp}{\Ga_\mathsf{p}}
\newcommand{\lift}[2]{\mathord{\uparrow^{#1}_{#2}}}


% Links to the repository

% For anonymity we should remove the links
\newcommand{\repoURL}{https://github.com/TheoWinterhalter/ett-to-itt/blob/master/theories/}
\newcommand{\rpath}[1]{\href{{\repoURL #1}}{#1}}
% \newcommand{\rpath}[1]{\path{#1}}
