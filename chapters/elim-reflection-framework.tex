% \setchapterpreamble[u]{\margintoc}
\chapter{Framework}
\labch{elim-reflection-framework}

Now that the problem is stated, I will clearly define the type theories I will
use as source and target. This is a bit redundant with \nrefch{flavours} but I
think it is worthwhile to make it clear what the translation is operating on.

\section{Syntax(es)}
\labsec{trans-syntaxes}

To make things simple I will consider the syntax of \acrshort{ITT} and
\acrshort{WTT} as extensions of the syntax of \acrshort{ETT}.
Actually, \acrshort{WTT} will also extend \acrshort{ITT}.

\subsection{Syntax of \acrshort{ETT}}

First, comes the syntax of \acrshort{ETT}.

\[
  \begin{array}{l@{~\,}r@{~\,}l}
    s &\in& \cS \\
    T,A,B,t,u,v &\bnf& x \bnfor \lam{x:A}{B} t \bnfor \app{t}{x:A}{B}{u} \\
    &\bnfor& \pair{x:A}{B}{u}{v} \bnfor \pio{x:A}{B}{p} \bnfor \pit{x:A}{B}{p} \\
    &\bnfor& \refl{A} u \bnfor \J{A}{u}{x.e.P}{w}{v}{p} \\
    &\bnfor& \ax{n} \\
    &\bnfor& s \bnfor \Prod{x:A} B \bnfor \Sum{x:A} B \bnfor \Eq{A}{u}{v} \\
    \Ga, \D &\bnf& \ctxempty \bnfor \Ga, x:A \\
    \Sigma &\bnf& \ctxempty \bnfor \Sigma, n:A
  \end{array}
\]

\paragraph{Sorts.}

Sorts are kept abstract (as represented by the generic \cS) but are not
unrestricted. With the sorts should come the sort of a sort (or successor
sort) \(\succs{s}\), the sort of a \(\Pi\)-type \(\pisort{s_1}{s_2}\)
where \(s_1\) and \(s_2\) are the sorts of the domain and codomain respectively,
and likewise for each constructor.
\sidedef[-0.9cm]{Functional \acrshort{PTS}}{
  A \acrshort{PTS} is \emph{functional} when \(\Ax\) and \(\Rl\) are functional,
  \ie for each \(s\) there is at most one \(s'\) such that \((s, s') \in \Ax\)
  and given \(s\) and \(s'\) there is at most one \(s''\) such that
  \((s,s',s'') \in \Rl\).
}
These are functions, so the underlying \acrshort{PTS} is functional.
Additionally, we ask that equality of sorts is decidable and that the successor
function is injective:
\[
  \succs{s_1} = \succs{s_2} \longrightarrow s_1 = s_2.
\]
Keeping the sorts abstract means that the proof can be instantiated with a lot
of different hierarchies, as long as they do not feature cumulativity
unfortunately. This will help us apply our result in the homotopy framework.

\paragraph{Annotations.}

Although it may look like a technical detail, the use of annotation is more
fundamental in \acrshort{ETT} than it is in \acrshort{ITT}/\acrshort{WTT}
where it is irrelevant and does not affect the theory.
This is actually one of the main differences between our work
(and that of Martin Hofmann~\sidecite{hofmann1995conservativity} who has a
similar presentation) and the work of Nicolas
Oury~\sidecite[0.7cm]{oury2005extensionality}.

\marginnote[1cm]{
  See \arefsubsec{card-model} of \nrefch{models} for more on this.
}
Indeed, using the cardinal model, it is possible to see that the equality
$\nat \to \nat = \nat \to \bool$ is independent from the theory, it is
thus possible to assume it (as an axiom, or for those that would still
not be convinced, simply under a $\lambda$ that would introduce this
equality).  In that context, the identity map $\lambda(x : \nat).\ x$
can be given the type $\nat \to \bool$ and we thus type
$(\lambda(x : \nat).\ x)\ \zero : \bool$.  Moreover, the
$\beta$-reduction of the non-annotated system used by Oury concludes
that this expression reduces to $\zero$, but cannot be given the type
$\bool$ (as we said, the equality $\nat \to \nat = \nat \to \bool$ is
independent from the theory, so the context is consistent). This means
we lack subject reduction in this case (or unique typing,
depending on how we see the issue).  Our presentation has a blocked
$\beta$-reduction limited to matching annotations:
$\app{(\lam{x:A}{B}\ t)}{x:A}{B}{u} = t[x \sto u]$, from which subject
reduction and unique typing follow.

\marginnote{
  \[
    \infer[JMAPP]
      {
        \Heq{\forall (x : U_1). V_1}{f_1}{\forall (x : U_2). V_2}{f_2} \\
        \Heq{U_1}{u_1}{U_2}{u_2}
      }
      {\Heq{V_1[x \sto u_1]}{f_1\ u_1}{V_2[x \sto u_2]}{f_2\ u_2}}
    %
  \]
}
Although subtle, this difference is responsible for Oury's need for an
extra axiom. Indeed, to treat the case of equality of applications in
his proof, he needs to assume the congruence rule for heterogeneous
equality of applications, which is not provable when formulated with
\acrshort{JMeq}. Thanks to annotations and our notion of heterogeneous equality,
we can prove this congruence rule for applications.

\paragraph{Axioms.}

Another interesting bit is the presence of the \(\ax{n}\) term, as well as
that of the \(\Sigma\) context. Indeed, in order for our \acrshort{ETT} to be
a bit extensible, we add a context \(\Sigma\) of axioms \(n : A\) that can be
referenced using \(\ax{n}\) but cannot be bound (they are global).
For a theory like \Coq, having just axioms is bit weak, but in \acrshort{ETT}
this is enough to simulate constants and even things such as inductive types:
any computation rule can be stated propositionally and still be definitional
thanks to reflection.

For instance to simulate the constant
\[
  n : A \coloneqq t
\]
we can use the following context
\[
  n : A,\ n_{\mathit{def}} : n = t
\]

Similarly, booleans can be encoded as follows.
\marginnote[1cm]{
  Note that the sort \(s\) is fixed in the eliminator so it still is not that
  convenient, but it will serve for a few examples.
}
\[
  \begin{array}{lcl}
    B &:& s, \\
    t &:& B, \\
    f &:& B, \\
    \mathit{if} &:& \Pi (P : B \to s).\ P\ t \to P\ f \to \Pi (b : B).\ P\ b, \\
    \mathit{if_t} &:& \Pi P\ u\ v.\ \mathit{if}\ P\ u\ v\ t = u, \\
    \mathit{if_f} &:& \Pi P\ u\ v.\ \mathit{if}\ P\ u\ v\ f = v
  \end{array}
\]

\subsection{Syntax of \acrshort{ITT} and \acrshort{WTT}}

\acrshort{ITT} is not really far from \acrshort{ETT}, we only extend the syntax
of terms with \acrshort{funext} and \acrshort{UIP}:

\[
  \begin{array}{l@{~\,}r@{~\,}l}
    T,A,B,t,u,v &\bnf& \dots \bnfor \funext{x:A}{B}{f}{g}{e} \bnfor
    \uip{A}{u}{v}{p}{q}
  \end{array}
\]

We build \acrshort{WTT} over \acrshort{ITT} by adding computation rules
and some missing congruence rules (in the likeness of \acrshort{funext}).
Indeed the elimination of equality provided by \(\mathsf{J}\) or by Leibniz'
principle does not account for equalities under binders; conversion however does
and as such in \acrshort{WTT} we need to add this to the theory.
In \acrshort{ITT} it is enough to use the congruence of conversion and
\acrshort{funext} to recover all of those, unfortunately in \acrshort{WTT} I
could not yet find a way to factorise them. As such we add to the syntax
equality constructors for each binder.

\marginnote[1cm]{
  The first rules (with \(\beta\)) are the computation rules.
  The other are congruence rules going under binders.
  They might be easier to understand with their typing rules in
  \arefsubsec{wtt-typing}.
}
\[
  \begin{array}{l@{~\,}r@{~\,}l}
    T,A,B,t,u,v &\bnf& \dots \bnfor \beta(t,u) \\
    &\bnfor& \redpio(u,v) \bnfor \redpit(u,v) \\
    &\bnfor& \redJ(u,x.e.P,w) \\
    &\bnfor& \lamb{x:A}{e_B} e_t
    \bnfor \appb{e_u}{x:A}{e_B}{v} \\
    &\bnfor& \pairb{x:A}{e_B}{u}{e_v}
    \bnfor \piob{x:A}{e_B}{e_p} \bnfor \pitb{x:A}{e_B}{e_p} \\
    &\bnfor& \Jb(A,u,x.e.e_P,e_w,v,p) \\
    &\bnfor& \pib (x:A). e_B \bnfor \sumb (x:A). e_B
  \end{array}
\]

Note most of these congruence symbols would no longer be necessary if we were to
type things such as \(J\) of \(\Sigma\) using \(\Pi\)-types so that \(\Pi\)
and \(\lambda\) are the only binders. Computation rules would still be required
however.

\subsection{Can \acrshort{ETT} still be called a conservative extension?}

\reminder[-0.7cm]{Conservative extension}{
  An extension \(\cT_2\) of a type theory \(\cT_1\) is \emph{conservative} if
  for every judgement \(\vdash_2 t : A\) such that \(\vdash_1 A\)
  (\ie \(A\) is a type in \(\cT_1\)), there exists \(t'\) such that
  \(\vdash_1 t' : A\).
}
As I said earlier one of the goals is to show that \acrshort{ETT} is a
conservative extension of both \acrshort{ITT} and \acrshort{WTT}. However this
cannot stricly be the case if \acrshort{ETT} is not first an \emph{extension}
of those.

This is not really an issue because the terms we add to the syntax of
\acrshort{ITT} and \acrshort{WTT} can be encoded in \acrshort{ETT} as
definitions. The notion of conservativity can thus be adapted to work in that
case: let us write \(\iota\) the translation that is homorphically the identity
but replaces extra symbols like \acrshort{funext} and \acrshort{UIP} with their
respective proofs in \acrshort{ETT}, \acrshort{ETT} is conservative over
\acrshort{ITT}/\acrshort{WTT} in the sense that every \(\vdash A\) such that
\(\vdash_\exmark t : \iota(A)\) there exists a term \(t'\) such that
\(\vdash t' : A\).

\section{Typing rules}

Similarly to the syntax I will present first the common rules of all involved
theories and then detail the specificities of each.

Note that I will differenciate \acrshort{ETT} judgements from those of the
target by using \(\exmark\) as a subscript to the turnstile: \(\vdash_\exmark\).

\subsection{Common typing rules}

For all systems, the typing rules are the same for the most part. Conversion and
the conversion rule will be the main point of divergence, besides the new
terms in \acrshort{ITT} and \acrshort{WTT} of course.

\paradot{Well-formedness of contexts}

\begin{mathpar}
  \infer[]
    { }
    {\isctx{\ctxempty}}
  %

  \infer[]
    {\isctx{\Ga} \\
      \istype{\Ga}{A}
    }
    {\isctx{\Ga, x:A}}
  (x \notin \Ga)
\end{mathpar}

\paradot{Types}

\marginnote[1cm]{
  The sort constructors here are those mentioned in \nrefsec{trans-syntaxes},
  though not all were given explicit names.
}
\begin{mathpar}
  \infer[]
    {\isctx{\Ga}}
    {\isterm{\Ga}{s}{\succs{s}}}
  %

  \infer[]
    {\isterm{\Ga}{A}{s_1} \\
      \isterm{\Ga,x:A}{B}{s_2}
    }
    {\isterm{\Ga}{\Prod{x:A} B}{\pisort{s_1}{s_2}}}
  %

  \infer[]
    {\isterm{\Ga}{A}{s_1} \\
      \isterm{\Ga,x:A}{B}{s_2}
    }
    {\isterm{\Ga}{\Sum{x:A} B}{\sumsort{s_1}{s_2}}}
  (s,s',s'')

  \infer[]
    {\isterm{\Ga}{A}{s} \\
      \isterm{\Ga}{u}{A} \\
      \isterm{\Ga}{v}{A}
    }
    {\isterm{\Ga}{\Eq{A}{u}{v}}{\eqsort{s}}}
  %
\end{mathpar}

\paradot{$\lambda$-calculus terms}

\begin{mathpar}
  \infer[]
    {\isctx{\Ga} \\
      (x : A) \in \Ga
    }
    {\isterm{\Ga}{x}{A}}
  %

  \infer[]
    {\isterm{\Ga}{A}{s} \\
      \isterm{\Ga,x:A}{B}{s'} \\
      \isterm{\Ga,x:A}{t}{B}
    }
    {\isterm{\Ga}{\lam{x:A}{B} t}{\Prod{x:A} B}}
  %

  \infer[]
    {\isterm{\Ga}{A}{s} \\
      \isterm{\Ga,x:A}{B}{s'} \\
      \isterm{\Ga}{t}{\Prod{x:A} B} \\
      \isterm{\Ga}{u}{A}
    }
    {\isterm{\Ga}{\app{t}{x:A}{B}{u}}{B[x \sto u]}}
  %

  \infer[]
    {\isterm{\Ga}{u}{A} \\
      \isterm{\Ga}{A}{s} \\
      \isterm{\Ga,x:A}{B}{s'} \\
      \isterm{\Ga}{v}{B[x \sto u]}
    }
    {\isterm{\Ga}{\pair{x:A}{B}{u}{v}}{\Sum{x:A}{B}}}
  %

  \infer[]
    {\isterm{\Ga}{p}{\Sum{x:A}{B}}}
    {\isterm{\Ga}{\pio{x:A}{B}{p}}{A}}
  %

  \infer[]
    {\isterm{\Ga}{p}{\Sum{x:A}{B}}}
    {\isterm{\Ga}{\pit{x:A}{B}{p}}{B[x \sto \pio{x:A}{B}{p}]}}
  %
\end{mathpar}

\paradot{Equality terms}

\begin{mathpar}
  \infer[]
    {\isterm{\Ga}{A}{s} \\
      \isterm{\Ga}{u}{A}
    }
    {\isterm{\Ga}{\refl{A} u}{\Eq{A}{u}{u}}}
  %

  \infer[]
    {\isterm{\Ga}{A}{s} \\
      \isterm{\Ga}{u,v}{A} \\
      \isterm{\Ga, x:A, e:\Eq{A}{u}{x}}{P}{s'} \\
      \isterm{\Ga}{p}{\Eq{A}{u}{v}} \\
      \isterm{\Ga}{w}{P[x \sto u, e \sto \refl{A} u]}
    }
    {\isterm
      {\Ga}
      {\J{A}{u}{x.e.P}{w}{v}{p}}
      {P[x \sto v, e \sto p]}
    }
  %
\end{mathpar}

\paradot{Axioms}

Although I will omit the global context \(\Sigma\) most of the time (it is
global after all), you can still imagine that it is present. The axiom typing
rule is the only interaction with it. In particular we assume that it is
well-formed.
\marginnote[-0.6cm]{
  Well-formedness of global context \(\Sigma\) is defined similarly to that of
  regular contexts.
}

\[
  \infer
    {(n : A) \in \Sigma}
    {\Ga \vdash \ax{n} : A}
\]

\subsection{Typing of \acrshort{ETT}}

As already mentioned several times, the key feature of \acrshort{ETT} is its
reflection rule, allowing us to go from propositional equality to conversion.
This means we need to define conversion in this setting.
\marginnote[0.3cm]{
  See \nrefch{formalisation}.
}
Here, I will define conversion without relying on reduction, but instead in a
typed way.

\paradot{Computation}

\begin{mathpar}
  \infer[]
    {
      \xisterm{\Ga}{A}{s} \\
      \xisterm{\Ga,x:A}{B}{s'} \\
      \xisterm{\Ga,x:A}{t}{B} \\
      \xisterm{\Ga}{u}{A}
    }
    {\xeqterm
      {\Ga}
      {\app
        {(\lam{x:A}{B} t)}
        {x:A}
        {B}
        {u}
      }
      {t[x \sto u]}
      {B[x \sto u]}
    }
  %

  \infer[]
    {
      \xisterm{\Ga}{A}{s} \\
      \xisterm{\Ga}{u}{A} \\
      \xisterm{\Ga, x:A, e:\Eq{A}{u}{x}}{P}{s'} \\
      \xisterm{\Ga}{w}{P[x \sto u, e \sto \refl{A} u]}
    }
    {\xeqterm
      {\Ga}
      {\J{A}{u}{x.e.P}{w}{u}{\refl{A} u}}
      {w}
      {P[x \sto u, e \sto \refl{A} u]}
    }
  %

  \infer[]
    {
      \xisterm{\Ga}{A}{s} \\
      \xisterm{\Ga}{u}{A} \\
      \xisterm{\Ga,x:A}{B}{s'} \\
      \xisterm{\Ga}{v}{B[x \sto u]}
    }
    {\xeqterm
      {\Ga}
      {\pio{x:A}{B}{\pair{x:A}{B}{u}{v}}}
      {u}
      {A}
    }
  %

  \infer[]
    {
      \xisterm{\Ga}{A}{s} \\
      \xisterm{\Ga}{u}{A} \\
      \xisterm{\Ga,x:A}{B}{s'} \\
      \xisterm{\Ga}{v}{B[x \sto u]}
    }
    {\xeqterm
      {\Ga}
      {\pit{x:A}{B}{\pair{x:A}{B}{u}{v}}}
      {v}
      {B[x \sto u]}
    }
  %
\end{mathpar}

% \paradot{$\eta$-rule}
%
% \begin{mathpar}
%   \infer[]
%     {\isterm{\Ga}{A}{s} \\
%      \isterm{\Ga, x:A}{B}{s'} \\
%      \isterm{\Ga}{f}{\Prod{x:A} B}
%     }
%     {\eqterm
%       {\Ga}
%       {\lam{x:A}{B} \app{f}{x:A}{B}{x}}
%       {f}
%       {\Prod{x:A} B}
%     }
%   %
% \end{mathpar}

\paradot{Conversion}

\marginnote[0.8cm]{
  Notice that I use conversion and not cumulativity as I need uniqueness of
  types to proceed.
}
\begin{mathpar}
  \infer[]
    {
      \xeqterm{\Ga}{t_1}{t_2}{T_1} \\
      \xeqtype{\Ga}{T_1}{T_2}
    }
    {\xeqterm{\Ga}{t_1}{t_2}{T_2}}
  %

  \infer[]
    {
      \xisterm{\Ga}{u}{A} \\
      \xeqtype{\Ga}{A}{B} : s
    }
    {\xisterm{\Ga}{u}{B}}
  %
\end{mathpar}

\paradot{Reflection}

\begin{mathpar}
  \highlight{
    \infer[]
      {\xisterm{\Ga}{e}{\Eq{A}{u}{v}}}
      {\xeqterm{\Ga}{u}{v}{A}}
    %
  }
\end{mathpar}

I omit the congruence rules as they are quite heavy and boring.
Here is for instance the congruence rule for application:
\[
  \infer[]
    {
      \xeqterm{\Ga}{A_1}{A_2}{s} \\
      \xeqterm{\Ga,x:A_1}{B_1}{B_2}{s'} \\
      \xeqterm{\Ga}{t_1}{t_2}{\Prod{x:A_1} B_1} \\
      \xeqterm{\Ga}{u_1}{u_2}{A_1}
    }
    {\xeqterm
      {\Ga}
      {\app{t_1}{x:A_1}{B_1}{u_1}}
      {\app{t_1}{x:A_1}{B_1}{u_1}}
      {B_1[x \sto u_1]}
    }
  %
\]

\subsection{Typing of \acrshort{ITT}}

\acrshort{ITT} comes with a conversion deduced from reduction (of which we take
the congruent closure). Because of this, the conversion rule will be slightly
different to account for the untyped style of conversion.

\paradot{Reduction}

\marginnote[0.8cm]{
  Notice that in the typed conversion for \(\mathsf{J}\), \(v\) and \(u'\) were
  both \(u\), and \(A'\) was \(A\). This will still be the case for well-typed
  terms, but the reduction rule should not have to decide conversion to apply.
}
\begin{mathpar}
  \app{(\lam{x:A}{B} t)}{x:A}{B}{u} \red t[x \sto u]

  \J{A}{u}{x.e.P}{w}{v}{\refl{A'} u'} \red w

  \pio{x:A}{B}{\pair{x:A}{B}{u}{v}} \red u

  \pit{x:A}{B}{\pair{x:A}{B}{u}{v}} \red v
\end{mathpar}

\(\red\) is also extended to reduce in subterms with rules like
\[
  \infer
    {t \red t'}
    {\lam{x:A}{B} t \red \lam{x:A}{B} t'}
  %
\]

\paradot{Conversion}

As I said, conversion is symply the reflexive symmetric transitive closure of
reduction.

\marginnote[0.8cm]{
  Granted, this presentation does not \emph{immediately} corresponds to
  reflexive symmetric transitive closure, but it is equivalent.
}
\begin{mathpar}
  \infer
    { }
    {u \equiv u}
  %

  \infer
    {
      u \red w \\
      w \equiv v
    }
    {u \equiv v}
  %

  \infer
    {
      u \equiv v \\
      v \red w
    }
    {u \equiv v}
  %
\end{mathpar}

The idea is that two terms are convertible if and only if they reduce to the
same term. This property only follows from confluence and termination of the
reduction however (see \nrefch{desirable-props}).

\paradot{Typing}

\begin{mathpar}
  \infer
    {
      \Ga \vdash t : A \\
      A \equiv B \\
      \Ga \vdash B : s
    }
    {\Ga \vdash t : B}
  %

  \infer[]
    {\isterm{\Ga}{e_1,e_2}{\Eq{A}{u}{v}}}
    {\isterm{\Ga}{\uip{A}{u}{v}{e_1}{e_2}}{\Eq{}{e_1}{e_2}}}
  %

  \infer[]
    {\isterm{\Ga}{f,g}{\Prod{x:A} B} \\
      \isterm
        {\Ga}
        {e}
        {\Prod{x:A} \Eq{B}{\app{f}{x:A}{B}{x}}{\app{g}{x:A}{B}{x}}}
    }
    {\isterm{\Ga}{\funext{x:A}{B}{f}{g}{e}}{\Eq{}{f}{g}}}
  %
\end{mathpar}

\subsection{Typing of \acrshort{WTT}}
\labsubsec{wtt-typing}

For \acrshort{WTT}, there is no notion of reduction or conversion, hence the
absence of a conversion rule. This leaves us with the typing rules of the extra
symbols we added.
Two of them are in common with \acrshort{ITT} but I will repeat them nonetheless
for clarity.

\begin{mathpar}
  \infer[]
    {\isterm{\Ga}{e_1,e_2}{\Eq{A}{u}{v}}}
    {\isterm{\Ga}{\uip{A}{u}{v}{e_1}{e_2}}{\Eq{}{e_1}{e_2}}}
  %

  \infer[]
    {\isterm{\Ga}{f,g}{\Prod{x:A} B} \\
      \isterm
        {\Ga}
        {e}
        {\Prod{x:A} \Eq{B}{\app{f}{x:A}{B}{x}}{\app{g}{x:A}{B}{x}}}
    }
    {\isterm{\Ga}{\funext{x:A}{B}{f}{g}{e}}{\Eq{}{f}{g}}}
  %

  \infer
    {
      \Ga, x:A \vdash t : B \\
      \Ga \vdash u : A
    }
    {\Ga \vdash \beta(t,u) : \app{(\lam{x:A}{B} t)}{x:A}{B}{u} = t[x \sto u]}
  %

  \infer
    {
      \Ga, x:A \vdash B : s \\
      \Ga \vdash u : A \\
      \Ga \vdash v : B[x \sto u]
    }
    {\Ga \vdash \redpio(u,v) : \pio{x:A}{B}{\pair{x:A}{B}{u}{v}} = u}
  %

  \infer
    {
      \Ga, x:A \vdash B : s \\
      \Ga \vdash u : A \\
      \Ga \vdash v : B[x \sto u]
    }
    {\Ga \vdash \redpit(u,v) : \pit{x:A}{B}{\pair{x:A}{B}{u}{v}} = v}
  %

  \infer
    {
      \Ga \vdash u : A \\
      \Ga, x:A, e:u=x \vdash P : s \\
      \Ga \vdash w : P[x \sto u, e \sto \refl{A} u]
    }
    {\Ga \vdash \redJ(u,x.e.P,w) : \J{A}{u}{x.e.P}{w}{u}{\refl{A} u} = w}
  %

  \infer
    {
      \Ga, x:A \vdash e_B : B_1 = B_2 \\
      \Ga, x:A \vdash e_t : \Heq{B_1}{t_1}{B_2}{t_2}
    }
    {
      \lamb{x:A}{e_B} e_t :
      \Heq
        {\Pi (x:A). B_1}
        {\lambda (x:A).B_1\ t_1}
        {\Pi (x:A). B_2}
        {\lambda (x:A).B_2\ t_2}
    }
  %

  \infer
    {
      \Ga, x:A \vdash e_B : B_1 = B_2 \\
      \Ga \vdash e_u : \Heq{\Pi(x:A).B_1}{u_1}{\Pi(x:A).B_2}{u_2} \\
      \Ga \vdash v : A
    }
    {
      \Ga \vdash \appb{e_u}{x:A}{e_B}{v} :
      \Heq
        {B_1[x \sto v]}
        {\app{u_1}{x:A}{B_1}{v}}
        {B_2[x \sto v]}
        {\app{u_2}{x:A}{B_2}{v}}
    }
  %

  \infer
    {
      \Ga \vdash u : A \\
      \Ga, x:A \vdash e_B : B_1 = B_2 \\
      \Ga \vdash e_v : \Heq{B_1[x \sto u]}{v_1}{B_2[x \sto u]}{v_2}
    }
    {
      \Ga \vdash \pairb{x:A}{e_B}{u}{e_v} :
      \Heq
        {\Sigma(x:A).B_1}
        {\pair{x:A}{B_1}{u}{v_1}}
        {\Sigma(x:A).B_2}
        {\pair{x:A}{B_2}{u}{v_2}}
    }
  %

  \infer
    {
      \Ga, x:A \vdash e_B : B_1 = B_2 \\
      \Ga \vdash e_p : \Heq{\Sigma(x:A).B_1}{p_1}{\Sigma(x:A).B_2}{p_2}
    }
    {
      \Ga \vdash \piob{x:A}{e_B}{e_p} :
      \pio{x:A}{B_1}{p_1} = \pio{x:A}{B_2}{p_2}
    }
  %

  \infer
    {
      \Ga, x:A \vdash e_B : B_1 = B_2 \\
      \Ga \vdash e_p : \Heq{\Sigma(x:A).B_1}{p_1}{\Sigma(x:A).B_2}{p_2}
    }
    {
      \Ga \vdash \pitb{x:A}{e_B}{e_p} :
      \Heq
        {B_1[x \sto \pio{x:A}{B_1}{p_1}]}
        {\pit{x:A}{B_1}{p_1}}
        {B_2[x \sto \pio{x:A}{B_2}{p_2}]}
        {\pit{x:A}{B_2}{p_2}}
    }
  %

  {
    \footnotesize
    \infer
      {
        \Ga \vdash u : A \\
        \Ga, x:A, e : u = x \vdash e_P : P_1 = P_2 \\
        \Ga \vdash e_w :
        \Heq
          {P_1[x \sto u, e \sto \refl{} u]}
          {w_1}
          {P_2[x \sto u, e \sto \refl{} u]}
          {w_2} \\
        \Ga \vdash v : A \\
        \Ga \vdash p : u = v
      }
      {
        \Ga \vdash \Jb(A,u,x.e.e_P,e_w,v,p) :
        \Heq
          % {P_1[x \sto v, e \sto p]}
          {}
          {\J{A}{u}{x.e.P_1}{w_1}{v}{p}}
          % {P_2[x \sto v, e \sto p]}
          {}
          {\J{A}{u}{x.e.P_2}{w_2}{v}{p}}
      }
  %
  }

  \infer
    {
      \Ga \vdash A : s \\
      \Ga, x:A \vdash B_1 = B_2
    }
    {\Ga \vdash \pib (x:A). e_B : \Pi(x:A).B_1 = \Pi(x:A).B_2}
  %

  \infer
    {
      \Ga \vdash A : s \\
      \Ga, x:A \vdash B_1 = B_2
    }
    {\Ga \vdash \sumb (x:A). e_B : \Sigma(x:A).B_1 = \Sigma(x:A).B_2}
  %
\end{mathpar}

You can notice that I'm cheating a little bit here. Indeed, I am already using
heterogeneous equality (\(\cong\)) in the typing rules. If you would indulge me,
you can imagine that the definition for heterogeneous equality is \emph{inlined}
in the typing rules. That way it is a much more convenient read (and it is
already a bit rough).