% \setchapterpreamble[u]{\margintoc}
\chapter{A specification of \Coq}
\labch{coq-spec}

In \nrefsec{coq-theory} I already talked about the type theory of \Coq we call
\acrshort{PCUIC} while in \nrefch{formalisation} you caught a glimpse of its
representation inside \Coq itself. In this chapter I will recall a few things
focusing on the points that were not discussed in those chapters like
representation of fixed-points and inductive types, as well as typing.

\section{Syntax of \acrshort{PCUIC}}

In \MetaCoq, the syntax of \acrshort{PCUIC} is defined as the following
inductive type.
\begin{minted}{coq}
Inductive term :=
| tRel (n : nat)
| tSort (u : Universe.t)
| tProd (na : name) (A B : term)
| tLambda (na : name) (A t : term)
| tLetIn (na : name) (b B t : term)
| tApp (u v : term)
| tConst (k : kername) (ui : Instance.t)
| tInd (ind : inductive) (ui : Instance.t)
| tConstruct (ind : inductive) (n : nat) (ui : Instance.t)
| tCase
    (indn : inductive * nat)
    (p c : term)
    (brs : list (nat * term))
| tProj (p : projection) (c : term)
| tFix (mfix : mfixpoint term) (idx : nat)
| tCoFix (mfix : mfixpoint term) (idx : nat).
\end{minted}

I won't present again variables, \(\Pi\)-types, etc. and will focus on
the other constructors, for this I need to introduce the global environment.
% \mintinline{coq}{tConst}, \mintinline{coq}{tInd}, \mintinline{coq}{tConstruct},
% \mintinline{coq}{tCase}, \mintinline{coq}{tProj}, \mintinline{coq}{tFix}
% and \mintinline{coq}{tCoFix}.

\paradot{Global environment}

Just as for the translation presented in \arefpart{elim-reflection}, we have
a global environment \(\Sigma\) containing the axioms, definitions and
declarations of inductive types. Instead of using de Bruijn indices we use
strings to refer to these declarations. In fact, \mintinline{coq}{kername}
is simply a notation for \mintinline{coq}{string} which we use to lift
ambiguity as to its purpose.


\paradot{Universes}

Besides the global environment we have a set of universes with their
constraints. Indeed, although we usually present the hierarchy of universes in
\Coq using \(\Type_i\) with \(i \in \mathbb{N}\), the implementation is a bit
different in that universes aren't fixed but floating.
\marginnote[0.65cm]{
  In \Coq, \(\Type_i\) is written \mintinline{coq}{Type@{i}}.
}
\begin{minted}{coq}
Type@{i} : Type@{j}
\end{minted}
holds in \Coq as long as the constraint \(i < j\) is satisfied.

\section{Typing of \acrshort{PCUIC}}

\todo{
  Give definition of term and typing, while focusing on ind, match and fix as
  they have not been really explained earlier.
  Talk about universes and constraints.
  Reduction before typing?
  Arities.
  Spec of guards.
  Discussion on eta, stating it is WIP, mention here how it is dealt with in the
  type checker?
}