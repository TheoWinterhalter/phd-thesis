% \setchapterpreamble[u]{\margintoc}
\chapter{A specification of \Coq}
\labch{coq-spec}

In \nrefsec{coq-theory} I already talked about the type theory of \Coq we call
\acrshort{PCUIC} while in \nrefch{formalisation} you caught a glimpse of its
representation inside \Coq itself. In this chapter I will recall a few things
focusing on the points that were not discussed in those chapters like
representation of fixed-points and inductive types, as well as typing.

All this is the subject of~\sidecite{sozeau2019coq}.

\section{Syntax of \acrshort{PCUIC}}

In \MetaCoq, the syntax of \acrshort{PCUIC} is defined as the following
inductive type.
\begin{minted}{coq}
Inductive term :=
| tRel (n : nat)
| tSort (u : Universe.t)
| tProd (na : name) (A B : term)
| tLambda (na : name) (A t : term)
| tLetIn (na : name) (b B t : term)
| tApp (u v : term)
| tConst (k : kername) (ui : Instance.t)
| tInd (ind : inductive) (ui : Instance.t)
| tConstruct (ind : inductive) (n : nat) (ui : Instance.t)
| tCase
    (indn : inductive * nat)
    (p c : term)
    (brs : list (nat * term))
| tProj (p : projection) (c : term)
| tFix (mfix : mfixpoint term) (idx : nat)
| tCoFix (mfix : mfixpoint term) (idx : nat).
\end{minted}

I won't present again variables, \(\Pi\)-types, etc. and will focus on
the other constructors, for this I need to introduce the global environment.
% \mintinline{coq}{tConst}, \mintinline{coq}{tInd}, \mintinline{coq}{tConstruct},
% \mintinline{coq}{tCase}, \mintinline{coq}{tProj}, \mintinline{coq}{tFix}
% and \mintinline{coq}{tCoFix}.

\paradot{Global environment}

Just as for the translation presented in \arefpart{elim-reflection}, we have
a global environment \(\Sigma\) containing the axioms, definitions and
declarations of inductive types. Instead of using de Bruijn indices we use
strings to refer to these declarations. In fact, \mintinline{coq}{kername}
is simply a notation for \mintinline{coq}{string} which we use to lift
ambiguity as to its purpose.


\paradot{Universes}

Besides the global environment we have a set of universes with their
constraints. Indeed, although we usually present the hierarchy of universes in
\Coq using \(\Type_i\) with \(i \in \mathbb{N}\), the implementation is a bit
different in that universes aren't fixed but floating.
\marginnote[0.65cm]{
  In \Coq, \(\Type_i\) is written \mintinline{coq}{Type@{i}}.
}
\begin{minted}{coq}
Type@{i} : Type@{j}
\end{minted}
holds in \Coq as long as the constraint \(i < j\) is satisfied but neither \(i\)
nor \(j\) correspond to natural numbers. Satisfying the constraints means that
there exist a valuation in the natural numbers preserving the constraints.
We also have local universes and constraints for polymorphic definitions;
universe instances (of type \mintinline{coq}{Instance.t}) are there to
substitute those for other universes.
Aside from \mintinline{coq}{Type}, we also represent \mintinline{coq}{Set}
and \mintinline{coq}{Prop}. We do not deal with \mintinline{coq}{SProp} yet.

\paradot{Constants}

\mintinline{coq}{tConst k ui} represents a definition or axiom declared in the
environment under name \mintinline{coq}{k}. Constant declarations are
inhabitants of the following record type.
\begin{minted}{coq}
Record constant_body := {
  cst_type : term ;
  cst_body : option term ;
  cst_universes : universes_decl
}.
\end{minted}
A constant comes with its type and an optional definition. Constants without
definitions are simply axioms.
The \mintinline{coq}{universes_decl} is there in case the constant happens to
be universe polymorphic.

\paradot{Inductive types}

\mintinline{coq}{tInd ind ui} represents an inductive type defined in the global
context. The type inductive is
\begin{minted}{coq}
Record inductive : Set := mkInd {
  inductive_mind : kername ;
  inductive_ind : nat
}.
\end{minted}
It is not just \mintinline{coq}{kername} because inductive types can be mutually
defined: the kername points to the mutual block, \mintinline{coq}{inductive_ind}
points to one inductive in the block. Note that this term refers to the
inductive type, without its parameters or indices. For vectors this would be
\begin{minted}{coq}
vec : Type -> nat -> Type
\end{minted}
In the global environment an inductive declaration provides the following
information.
\begin{minted}{coq}
Record mutual_inductive_body := {
  ind_finite    : recursivity_kind ;
  ind_npars     : nat ;
  ind_params    : context ;
  ind_bodies    : list one_inductive_body ;
  ind_universes : universes_decl ;
  ind_variance  : option (list Universes.Variance.t)
}.
\end{minted}
These informations include the number of parameters, the type of these
parameters in the form of a context and the universe declarations.
\mintinline{coq}{ind_variance} deals with variance of universes in the case
of cumulative inductive types.
The \mintinline{coq}{recursivity_kind} determines what kind of inductive type
we are building, this in fact include the usual inductive types, but also
record types and coinductive types.
Finally, \mintinline{coq}{ind_bodies} is a list of declarations for each of
the inductive types in the mutual block.
\begin{minted}{coq}
Record one_inductive_body := {
  ind_name  : ident ;
  ind_type  : term ;
  ind_kelim : sort_family ;
  ind_ctors : list (ident * term * nat) ;
  ind_projs : list (ident * term)
}.
\end{minted}
Each of those comes with its name and type.
\mintinline{coq}{ind_kelim} states in which univeres the definition can be
eliminated (in most cases inductive types living in \mintinline{coq}{Prop}
cannot be scrutinised to build terms in \mintinline{coq}{Type}).
Then we either have a list of constructors for (co)inductive types, or a list
of projections for (negative) records.

\paradot{Constructors}

\mintinline{coq}{tConstruct ind n ui} has \mintinline{coq}{ind} to point to the
right inductive type while \mintinline{coq}{n} tells us which constructor it is.
For instance it would be \mintinline{coq}{0} for \mintinline{coq}{true} and
\mintinline{coq}{1} for \mintinline{coq}{false}.
Each constructor is given by a term of type
\begin{minted}{coq}
ident * term * nat
\end{minted}
as we saw in the previous paragraph. It consists in a name, a type and an arity.

\paradot{Pattern-matching}
\mintinline{coq}{tCase (ind, n) p c brs} represents pattern-matching on
\mintinline{coq}{c} of inductive type \mintinline{coq}{ind} with
\mintinline{coq}{n} parameters, return predicate \mintinline{coq}{p}
and branches \mintinline{coq}{brs}.
There is one branch for each constructor and it contains the arity of the
constructor and the term corresponding to the branch.

\paradot{Projections}
\mintinline{coq}{tProj p c} represents the projection \mintinline{coq}{p}
applied to term \mintinline{coq}{c}.
\begin{minted}{coq}
Definition projection :=
  inductive * nat * nat.
\end{minted}
A projection is described by the inductive (or rather record) to which it
belongs, the number of parameters of the record, and the index of the projected
argument. In the case of dependent pairs, the latter will be \mintinline{coq}{0}
for the first projection and \mintinline{coq}{1} for the second.

\paradot{Fixed-points}
Finally we have \mintinline{coq}{tFix mfix idx} and
\mintinline{coq}{tCoFix mfix idx} to represent (co)fixed-points.
Like inductive types, fixed-points can be defined mutually, as such
\mintinline{coq}{mfix} is a list of (mutual) definitions:
\begin{minted}{coq}
Definition mfixpoint term :=
  list (def term).
\end{minted}
Definitions which are given as
\begin{minted}{coq}
Record def term := mkdef {
  dname : name ;
  dtype : term ;
  dbody : term ;
  rarg  : nat
}.
\end{minted}
that is with a name, a type and a body (the definition itself).
\mintinline{coq}{rarg} points to the recursive argument of the fixed-point.
For instance in
\begin{minted}{coq}
Fixpoint map {A B : Type} (f : A -> B) (l : list A) {struct l}
  : list B :=
  match l with
  | [] => []
  | x :: l => f x :: map f l
  end.
\end{minted}
the \mintinline{coq}{{struct l}} means that \mintinline{coq}{l} is the recursive
argument. That is all recursive calls must be structually decreasing on
\mintinline{coq}{l}. It is part of the syntax because \Coq uses it as a
syntactic guard to avoid unfolding fixed-points indefinitely: only when the
recursive argument is a constructor can the fixed-point be unfolded.

Now that the syntax is out of the way we can move to the semantics.

\section{Semantics of \acrshort{PCUIC}}

There are two important properties defined on the syntax: reduction and typing.
The former being necessary for the latter as it is the base for conversion.

\subsection{Reduction}

We define reduction as a relation on terms, however, unlike for \acrshort{STL}
or many systems, because of constants and let-bindings which can be unfolded
to their definitions, the relation has to mention the global and local
environment. Note also that the relation is not functional, that is that the
reduction we define is not deterministic.

The inductive type of \(1\)-step reduction is
\begin{minted}{coq}
Inductive red1 (Σ : global_env) (Γ : context) :
  term -> term -> Type
\end{minted}
I will give the different constructors one by one, while explaining them,
focusing on the computation rules.

\paradot{\(\beta\)-reduction}
\begin{minted}{coq}
red_beta na t b a :
  red1 Σ Γ (tApp (tLambda na t b) a) (subst10 a b)
\end{minted}
This is the usual \(\beta\)-reduction rule, \mintinline{coq}{subst10 a b}
is just the term \mintinline{coq}{b} where the \(0\)-th variable is substituted
by \mintinline{coq}{a}.

\paradot{\(\zeta\)-reduction}
\begin{minted}{coq}
red_zeta na b t b' :
  red1 Σ Γ (tLetIn na b t b') (subst10 b b')
\end{minted}
This rule is similar to \(\beta\)-reduction, it states that the expression
\begin{minted}{coq}
let x := b in t x
\end{minted}
reduces to \mintinline{coq}{t b}.

\paradot{Local definition unfolding}
\begin{minted}{coq}
red_rel i body :
  option_map decl_body (nth_error Γ i) = Some (Some body) ->
  red1 Σ Γ (tRel i) (lift0 (S i) body)
\end{minted}
If the context contains some local definition \(x := t\) then the variable
\(x\) \emph{reduces} to \(t\). Because the definition makes sense in a
smaller context, it needs to be weakened, hence the \mintinline{coq}{lift}.

\paradot{\(\iota\)-reduction}
\marginnote[1.5cm]{
  \mintinline{coq}{mkApps} is a shortcut to apply a term to a list of arguments.
}
\begin{minted}{coq}
red_iota ind pars c u args p brs :
  red1 Σ Γ
    (tCase (ind, pars) p (mkApps (tConstruct ind c u) args) brs)
    (iota_red pars c args brs)
\end{minted}
Pattern-matching reduces when the term under scrutiny is a constructor
(potentially applied to some arguments).
\mintinline{coq}{iota_red} is defined as follows
\begin{minted}{coq}
Definition iota_red npar c args brs :=
  mkApps (snd (List.nth c brs (0, tDummy))) (List.skipn npar args).
\end{minted}
\mintinline{coq}{List.nth} takes a default value, so
\mintinline{coq}{(0, tDummy)} can be safely ignored. \mintinline{coq}{iota_red}
basically picks the branch corresponding to the constructor giving it the list
of indices \mintinline{coq}{List.skipn npar args}.

\paradot{Fixed-point unfolding}
\todo{TODO}

\paradot{Cofixed-point unfolding}
\todo{TODO}

\paradot{\(\delta\)-reduction}
\todo{TODO}

\paradot{Projection}

\paradot{Congruence rules}
All remaining rules are congruence rules we have for instance the two congruence
rules for \mintinline{coq}{tLambda}:
\begin{minted}{coq}
| abs_red_l na M M' N :
    red1 Σ Γ M M' ->
    red1 Σ Γ (tLambda na M N) (tLambda na M' N)

| abs_red_r na M M' N :
    red1 Σ (Γ ,, vass na N) M M' ->
    red1 Σ Γ (tLambda na N M) (tLambda na N M')
\end{minted}
They state that you can reduce either subterm.

\subsection{Conversion}

\todo{
  Discussion on eta, stating it is WIP, mention here how it is dealt with in the
  type checker?
}

\subsection{Typing}

\todo{
  Talk about universes and constraints.
  Arities.
  Spec of guards.
}