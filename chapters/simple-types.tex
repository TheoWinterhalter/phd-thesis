% \setchapterpreamble[u]{\margintoc}
\chapter{Simple type theory}
\labch{simple-types}

When studying programming languages theoretically, \(\lambda\)-calculus imposes
itself as the prototypical example. A model of computation much simpler than
Turing machines, it becomes extermely useful in combination with a so-called
type-system. This combination culminates into the Curry-Howard isomorphism that
relates programming and logic in profound ways, serving as a foundation for
type theory and modern logic.

I will not try and do some thorough analysis and history of the subject but I
will give an account of my understanding of it, limiting myself to points that
I find relevant to my thesis.

\section{\(\lambda\)-calculus}

\(\lambda\)-calculus can be reasonably called the simplest programming language.
It consists basically of functions, variables and applications.
If you are familiar with \ocaml these constructs are summarised in the example
below.
\begin{minted}{ocaml}
(fun x -> x) u
\end{minted}
Here, we have the identity function \mintinline{ocaml}{fun x -> x}---\ie the
function which maps \mintinline{ocaml}{x} to \mintinline{ocaml}{x}---applied to
some expression \mintinline{ocaml}{u}.
In mathematical textbooks we would define the identity function (for natural
numbers) as follows.
\[
  \mathit{id} :
  \left(
  \begin{array}{lcl}
    \mathbb{N} &\to& \mathbb{N} \\
    x &\mapsto& x
  \end{array}
  \right)
\]
The whole expression written \(\mathit{id}(u)\).

\(\lambda\)-calculus is a third way of writing the same thing.
\[
  (\lambda x.\ x)\ u
\]
The little \(\lambda\) corresponds to the declaration of a function, in this
case \emph{binding} the \(x\) before the dot, in the expression after it.
Application of a function is marked with a space, such as \(u\ v\) meaning
\(u\) applied to argument \(v\).
Formally, the grammar of \(\lambda\)-calculus is:
\marginnote[0.55cm]{
  \(x\) is a placeholder for any variable name, typicaly in the range of \(x\),
  \(y\) and \(z\).
}
\[
  t, u, v \coloneqq x \mid \lambda x.\ t \mid t\ u
\]
and that's it.

In a term, the variable names are considered irrelevant, for instance, the
\(\lambda\)-terms \(\lambda x.\ x\) and \(\lambda y.\ y\) are deemed equivalent.
Of course, it only works, if all occurences of \(x\) are replaced consistently
with \(y\). This operation is called \(\alpha\)-renaming, we thus talk about
\(\alpha\)-equality:
\[
  \lambda x.\ x =_\alpha \lambda y.\ y
\]

Of course, that alone is not sufficient to describe a programming language,
it's missing a key compenent: evaluation of programs.
Indeed, without it, variables are meaningless.

We want to say that the application of a function should always yield some
result. For instance, the identity function \(\lambda x.\ x\) when applied to
\(u\) should naturally reduce to \(u\) itself.

The purpose of a variable is to be \emph{instantiated}.
Before we talk about this, we need to talk about \emph{bound} and \emph{free}
variables.
If you take the expression \((\lambda x.\ x)\ y\) you can see two variables
\(x\) and \(y\); the two do not behave the same: \(x\) is below a
\(\lambda\)-abstraction that \emph{introduces} (or \emph{binds}) it whereas
\(y\) is \emph{free}, no \(\lambda\)-abstraction constrains it.
Of course, this status changes if the term is put inside another:
\[
  \lambda y.\ (\lambda x.\ x)\ y
\]
This times both \(y\) and \(x\) are \emph{bound}.

A term with free variables is called \emph{open}, while a term with only bound
variables is called \emph{closed}.

\marginnote[0.2cm]{
  Sometimes this fact is summarised in the term \emph{capture-avoiding} meaning
  that bound variables aren't touched. I will refrain from using a qualifier
  that basically means that it's not a nonsensical definition.
}
The only variables that can be instantiated are the \emph{free} ones.
For example, it wouldn't make sense to replace the \(x\) in the identity
\(\lambda x.\ x\) function, it would somehow break it. \emph{Substitution} is
the operation that replaces free variables with other expressions.
We will write
\[
  t[x \sto u]
\]
to mean the term \(t\) were all \emph{free} occurences of the variable \(x\)
are replaced by the term \(u\).

For example,
\[
  ((\lambda x.\ x)\ y)[y \sto u] = (\lambda x.\ x)\ u
\]

Now that we have substitution, we are armed to deal with reduction.
Reduction is defined from the so-called \(\beta\)-reduction
\[
  (\lambda x.\ t)\ u \red_\beta t[x \sto u]
\]
It is essentially saying that when applied to an argument, a function
reduces to its body were the argument replaces the variable it was binding.
Reduction \(\red\) is then obtained by taking the contextual closure of
\(\red_\beta\), \ie allowing reduction in each subterm:
\begin{mathpar}
  \infer
    {u \red u'}
    {u\ v \red u'\ v}
  %

  \infer
    {v \red v'}
    {u\ v \red u\ v'}
  %

  \infer
    {t \red t'}
    {\lambda x.\ t \red \lambda x.\ t'}
  %

  \infer
    { }
    {(\lambda x.\ t)\ u \red t[x \sto u]}
  %
\end{mathpar}

There is a lot more to say on the \emph{untyped} \(\lambda\)-calculus, but this
out of scope of this document.

\section{Types for programs}

Not all programs make sense. Consider the term
\[
  \delta \coloneqq \lambda x.\ x\ x
\]
giving \(x\) as argument to \(x\) should already feel wrong somehow, but if you
give \(\delta\) as argument to itself you get
\[
  \Omega \coloneqq \delta\ \delta
\]
Now let's have a look at its computational behaviour:
\[
  \begin{array}{lcl}
    \Omega &\coloneqq& \delta\ \delta \\
    &=& (\lambda x.\ x\ x)\ \delta \\
    &\red_\beta& \delta\ \delta \\
    &=& \Omega
  \end{array}
\]
The problem should become apparent: \(\Omega\) reduces to itself!
Generally, that is not something you want to have.
In case you have more structure in your language, such as in \ocaml where you
have integers, you also want to restrict some operations like addition to
expressions that are of the right \emph{kind}---in our case, integers.
\ocaml expressions like \mintinline{ocaml}{true + 3} or
\mintinline{ocaml}{1 + fun x -> 2} should be---and are!---rejected.

The way we prevent such problematic terms---\ie those that don't have
semantics---is by using \emph{types}. The \(\lambda\)-calculus is pretty simple,
the only things we can do is define functions and apply them, as such we
introduce a type of functions \(A \to B\).
\(A \to B\) is the type of functions that take an argument of type \(A\)
and produce a value of type \(B\).
If we go back to \ocaml for a bit, a function like
\mintinline{ocaml}{fun x -> x + x} will have type \mintinline{ocaml}{int -> int}
while another making use of different data structures like
\mintinline{ocaml}{fun b -> if b then "Yes" else "No"} will be of type
\mintinline{ocaml}{bool -> string}.
Going back to \(\lambda\)-calculus, the important rule is that when you have
a function \(f\) of type \(A \to B\)---which we will write \(f : A \to B\)---and
an argument \(u : A\), the result of the application will have type \(B\):
\[
  f\ u : B
\]

The usefulness of types is best summarised in the famous quote:
\begin{quote}
  ``Well-typed programs cannot go wrong.''

  \hspace*{\fill} --- Robin Milner~\sidecite{milner1978theory}
\end{quote}

As I mentioned earlier, \(\lambda\)-terms can sometimes be open terms---\ie
contain free variables---so in order to type them, we need to know the type of
those variables. This information is called an environment or \emph{context}.
We can thus give the syntax of types and contexts.
\marginnote[1cm]{
  \(o\) stands for base types. A context is a list of type ascriptions for the
  variables, implicitly the variables are distinct.
}
\[
  \begin{array}{rcl}
    A, B &\bnf& o \bnfor A \to B \\
    \Ga, \D &\bnf& \ctxempty \bnfor \Ga, x : A
  \end{array}
\]

The typing rules of \acrlong{STL} are the following.
\begin{mathpar}
  \infer
    {\Ga, x:A \vdash t : B}
    {\Ga \vdash \lambda x.\ t : A \to B}
  %

  \infer
    {
      \Ga \vdash u : A \to B \\
      \Ga \vdash v : A
    }
    {\Ga \vdash u\ v : B}
  %

  \infer
    {(x : A) \in \Ga}
    {\Ga \vdash x : A}
  %
\end{mathpar}

\section{The Curry-Howard isomorphism}