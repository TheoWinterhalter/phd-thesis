% \setchapterpreamble[u]{\margintoc}
\chapter{Conclusions}
\labch{conclusions}

\section*{Elimination of reflection}

In this document (and in the orginal paper~\sidecite{winterhalter:hal-01849166})
we provide the first effective translation from \acrshort{ETT} to \acrshort{ITT}
with \acrshort{UIP} and \acrshort{funext}. The translation has been formalised
in \Coq.
This translation is also effective in the sense that we can produce in the end a
\Coq term using the \MetaCoq denotation machinery.
%
With ongoing work to extend the translation to the inductive fragment
of \Coq, we are paving the way to an extensional version of the \Coq
proof assistant which could be translated back to its intensional
version, allowing the user to navigate between the two modes, and in
the end produce a proof term checkable in the intensional fragment.

This thesis also introduces a translation from \acrshort{ETT} to \acrshort{WTT}
which thus yields a translation from \acrshort{ITT} to \acrshort{WTT}. This
shows that computation is not \emph{essential} to the logical power of a type
theory.

\subsection*{Perspectives}

The translation from \acrshort{ETT} is effective, but not particularly
efficient. This is mainly due to the fact that derivations in \acrshort{ETT}
are really big, due to a lot of redundant information. I think this calls for
an intermediary between the term and its typing derivation on which to perform
the translation.

On another note, I believe we can improve \acrshort{ETT} so that it is more
practical, by reconciling it with computation. I believe (at least hope) that
this can be achieved by using a weaker notion of irrelevant reflection that
would only allow reflection to happen in specific places like the indices of
inductive types, provided said indices are irrelevant for computation (one
should not be able to get the length of a vector by simply matching the head
but needs to recurse in the tail, same as for lists).
It would be interesting how it would affect the translation I presented.

\section*{A verified type-checker for \Coq, in \Coq}

We have formalised an almost feature-complete type-checker for \Coq in \Coq
and proven it correct. Thanks to the extraction mechanism of \Coq, this can
actually be turned into an independent type-checker program.
For instance it can be run within Coq in the manner of a plugin:
\begin{minted}{coq}
MetaCoq SafeCheck nat.
\end{minted}

Because we do not deal with template polymorphism, the module system and
\(\eta\)-expansion, we are not yet able to typecheck the standard library
or the formalisation itself (as it relies on the standard library).
We were however able to test it on reasonable proof terms coming from
the \acrshort{HoTT} library~\sidecite{bauer2017hott}.
\marginnote[1.1cm]{
  Examples of the checker in action, including the \acrshort{HoTT} theorems
  can be found in the file
  \href{https://github.com/MetaCoq/metacoq/blob/popl-artifact-eval/test-suite/safechecker_test.v}{test-suite/safechecker\_test.v}.
}%
For instance, we have been able to typecheck the proof that an isomorphism can
be turned into an equivalence.
It currently is about one order of magnitude slower than the \Coq
implementation (0.015s vs 0.002s averaged over 10 runs for each checking).
This can in particular be attributed to our very ineficient representation of
global environments as association lists indexed by character lists, where \Coq
uses efficient hash-maps on strings.

Another point I would like to address is the lack of completeness proof.
Our tests suggest that we are close to completeness if not there yet for the
fragment we are considering. It would be however really interesting to have a
proof that conversion and inference are complete with respect to their
specifications.
At the moment, dealing with \(\eta\)-conversion seems like a much more pressing
matter as it would make the checker usable in practice at least as a standalone
independent checker.

There is also the long term dream of having a \Coq kernel fully verified and
running in \Coq.