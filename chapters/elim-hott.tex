% \setchapterpreamble[u]{\margintoc}
\chapter{Reflection and homotopy}
\labch{elim-hott}

\marginnote[0.5cm]{
  Refer to \nrefch{flavours} for more on this.
}
Homotopy and reflection seem contradictory at first since \acrshort{UIP}
kind of comes with reflection and is negated by univalence, but as I have
exposed earlier, using \acrlongpl{2TT} we can make two equalities---one with
\acrshort{UIP} and one which is homotopic---cohabit consistently.

Now, in \nrefch{elim-reflection-framework} I showed how the proof was
(almost) agnostic with respect to universes, thanks to abstract universe
constructors, one of which was (crucially) the sort of an identity type.

This is the key to instantiate our translation into one going \acrshort{HTS}
to \acrshort{2TT} or \acrshort{2WTT}.
We can recover those theories in our setting by taking $\F{i}$ and $\Un{i}$ as
respectively the fibrant and strict universes
(for $i \in \mathbb{N}$), along with the following \acrshort{PTS} rules:
%
\[
\begin{array}{l@{~}c@{~}l@{\qquad}l@{~}c@{~}l}
  (\F{i}, \F{i+1}) &\in& \Ax &
  (\Un{i}, \Un{i+1}) &\in& \Ax \\
  (\F{i}, \F{j}, \F{\nmax{i}{j}}) &\in& \Rl &
  (\F{i}, \Un{j}, \Un{\nmax{i}{j}}) &\in& \Rl \\
  (\Un{i}, \F{j}, \Un{\nmax{i}{j}}) &\in& \Rl &
  (\Un{i}, \Un{j}, \Un{\nmax{i}{j}}) &\in& \Rl \\
\end{array}
\]
%
and the fact that the sort of the (strict) identity type on $A : s$ is
the \emph{strictified} version of $s$, \ie $\Un{i}$ for $s = \Un{i}$ or
$s = \F{i}$.
\marginnote[-0.2cm]{
  Of course, just extending the proof to two equalities is a possibility as
  well. As one can imagine it doesn't change the proof much.
}
The fibrant equality can be recovered using axioms as exposed in
\nrefsec{trans-syntaxes}.

Note that this is just one of many presentations of \acrlongpl{2TT}: instead
of being type-based, fibration is sometimes dealt with using a specific
judgment and this is not covered by our formalisation.
\marginnote[-1cm]{
  Type-based approaches are usually much better suited to translations
  (because they are type-preserving).
}

In short, the translation from \acrshort{HTS} to \acrshort{2TT} or
\acrshort{2WTT} is \emph{exactly} the same as the one
from \acrshort{ETT} to \acrshort{ITT} or \acrshort{WTT} that I presnet in this
thesis. This fact is factorised through our formalisation.