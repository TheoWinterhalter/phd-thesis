% \setchapterpreamble[u]{\margintoc}
\chapter{Meta-theoretical properties}
\labch{coq-meta-theory}

\marginnote[0.5cm]{
  Most of these properties are defined in \nrefch{desirable-props}.
}
Before we can formalise the type checker, we need to develop some of the
meta-theory. In particular the type checker relies on subject reduction,
confluence and strong normalisation of the reduction.
As I already said in \nrefch{coq-overview}, these properties can be part of
the trusted base (and some of them like strong normalisation have to be in it
lest we prove consistency of \Coq whithin \Coq). Moreover they are not the
subject of this work. I will thus focus on their statements and not their
potential proofs, those are explained in~\sidecite{sozeau2019coq} or will be
in upcoming publications regarding the \MetaCoq project.

\paradot{Substitutivity and weakening}

Weakening and substitution preserve typing, a fact which we prove rather early
in the development. Such statements cannot really be assumed anyway since it is
really easy to make mistakes while manipulating de Bruijn indices.

The weakening theorem is rather simple:
\begin{minted}{coq}
Lemma weakening :
  forall Σ Γ Γ' (t : term) T,
    wf Σ ->
    wf_local Σ (Γ ,,, Γ') ->
    Σ ;;; Γ |- t : T ->
    Σ ;;; Γ ,,, Γ' |- lift0 #|Γ'| t : lift0 #|Γ'| T.
\end{minted}
It asks for the global environment and the extended local environment to make
sense as a precondition.

Substitution is slightly more complex
\begin{minted}{coq}
Lemma substitution :
  forall Σ Γ Γ' s Δ (t : term) T,
    wf Σ ->
    subslet Σ Γ s Γ' ->
    Σ ;;; Γ ,,, Γ' ,,, Δ |- t : T ->
    Σ ;;; Γ ,,, subst_context s 0 Δ |-
      subst s #|Δ| t : subst s #|Δ| T.
\end{minted}
This time we are replacing a bunch of variables at once using parallel
substitutions. Not only the term and type, but also the context that was built
on top of the substituted variables are substituted.
Unlike weakening, the substitution itself needs to be well-typed, this is
defined as follows.
\begin{minted}{coq}
Inductive subslet {cf:checker_flags} Σ (Γ : context)
  : list term -> context -> Type :=

| emptyslet :
    subslet Σ Γ [] []

| cons_let_ass Δ s na t T :
    subslet Σ Γ s Δ ->
    Σ ;;; Γ |- t : subst0 s T ->
    subslet Σ Γ (t :: s) (Δ ,, vass na T)

| cons_let_def Δ s na t T :
    subslet Σ Γ s Δ ->
    Σ ;;; Γ |- subst0 s t : subst0 s T ->
    subslet Σ Γ (subst0 s t :: s) (Δ ,, vdef na t T).
\end{minted}
Unlike substitution typing I presented earlier this needs to account for local
definitions (\ie let-bindings) in the environment.
In a more understable language this becomes
\marginnote[1cm]{
  This time I write \(\Ga \vdash \sigma : \D\) were earlier I wrote
  \(\sigma : \Ga \to \D\). This is because in this case \(\sigma\) is a list
  of terms.
}
\begin{mathpar}
  \infer
    { }
    {\Sigma ; \Ga \vdash \bullet : \ctxempty}
  %

  \infer
    {
      \Sigma ; \Ga \vdash \sigma : \D \\
      \Sigma ; \Ga \vdash t : A[\sigma]
    }
    {\Sigma ; \Ga \vdash \sigma, x \sto t : \D, x:A}
  %

  \infer
    {
      \Sigma ; \Ga \vdash \sigma : \D \\
      \Sigma ; \Ga \vdash t[\sigma] : A[\sigma]
    }
    {\Sigma ; \Ga \vdash \sigma, x \sto t[\sigma] : \D, x : A \coloneqq t}
  %
\end{mathpar}
Notice how the term of the substitution is forced when it targets a definition,
this is reassuring because a substitution should not overwrite definitions.

\paradot{Confluence}

Confluence is an import result that we have on \Coq's theory. In particular it
allows us to simplify greatly what it means to be convertible.
Indeed, right now, conversion involves reduction going in both directions so
that \(u\) and \(v\) can be convertible by something like the following picture.
\begin{center}
  \begin{tikzpicture}[baseline=(u.base), node distance=1cm]
    \node (u) { \(u\) } ;
    \node (d1) [above right = of u] {} ;
    \node (d2) [below right = of d1] {} ;
    \node (d3) [above right = of d2] {} ;
    \node (v) [below right = of d3] { \(v\) } ;
    \path (d1.center) edge[to*, tred] (u) ;
    \path (d1.center) edge[to*, tred] (d2) ;
    \path (d3.center) edge[to*, tred] (d2) ;
    \path (d3.center) edge[to*, tred] (v) ;
  \end{tikzpicture}
\end{center}
Thanks to confluence this picture can be completed into
\begin{center}
  \begin{tikzpicture}[baseline=(u.base), node distance=1cm]
    \node (u) { \(u\) } ;
    \node (d1) [above right = of u] {} ;
    \node (d2) [below right = of d1] {} ;
    \node (d3) [above right = of d2] {} ;
    \node (v) [below right = of d3] { \(v\) } ;
    \node (a) [below right = of u] {} ;
    \node (b) [below right = of d2] {} ;
    \node (c) [below right = of a] {} ;
    \path (d1.center) edge[to*, tred] (u) ;
    \path (d1.center) edge[to*, tred] (d2) ;
    \path (d3.center) edge[to*, tred] (d2) ;
    \path (d3.center) edge[to*, tred] (v) ;
    \path (u) edge[to*, tred, dashed] (a) ;
    \path (d2.center) edge[to*, tred, dashed] (a) ;
    \path (d2.center) edge[to*, tred, dashed] (b) ;
    \path (v) edge[to*, tred, dashed] (b) ;
    \path (a.center) edge[to*, tred, dashed] (c) ;
    \path (b.center) edge[to*, tred, dashed] (c) ;
  \end{tikzpicture}
\end{center}

\paradot{Context conversion}

\paradot{Subject reduction}

\paradot{Principality}

\paradot{Strong normalisation}

\todo{
  Put SN here
  This is where fixguard and co make sense.
}