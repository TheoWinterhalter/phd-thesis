% \setchapterpreamble[u]{\margintoc}
\chapter{Meta-theoretical properties}
\labch{coq-meta-theory}

\marginnote[0.5cm]{
  Most of these properties are defined in \nrefch{desirable-props}.
}
Before we can formalise the type checker, we need to develop some of the
meta-theory. In particular the type checker relies on subject reduction,
confluence and strong normalisation of the reduction.
As I already said in \nrefch{coq-overview}, these properties can be part of
the trusted base (and some of them like strong normalisation have to be in it
lest we prove consistency of \Coq whithin \Coq). Moreover they are not the
subject of this work. I will thus focus on their statements and not their
potential proofs, those are explained in~\sidecite{sozeau2019coq} or will be
in upcoming publications regarding the \MetaCoq project.

\paradot{Substitutivity and weakening}

Weakening and substitution preserve typing, a fact which we prove rather early
in the development. Such statements cannot really be assumed anyway since it is
really easy to make mistakes while manipulating de Bruijn indices.

The weakening theorem is rather simple:
\begin{minted}{coq}
Lemma weakening :
  forall Σ Γ Γ' (t : term) T,
    wf Σ ->
    wf_local Σ (Γ ,,, Γ') ->
    Σ ;;; Γ |- t : T ->
    Σ ;;; Γ ,,, Γ' |- lift0 #|Γ'| t : lift0 #|Γ'| T.
\end{minted}
It asks for the global environment and the extended local environment to make
sense as a precondition.

Substitution is slightly more complex
\begin{minted}{coq}
Lemma substitution :
  forall Σ Γ Γ' s Δ (t : term) T,
    wf Σ ->
    subslet Σ Γ s Γ' ->
    Σ ;;; Γ ,,, Γ' ,,, Δ |- t : T ->
    Σ ;;; Γ ,,, subst_context s 0 Δ |-
      subst s #|Δ| t : subst s #|Δ| T.
\end{minted}
This time we are replacing a bunch of variables at once using parallel
substitutions. Not only the term and type, but also the context that was built
on top of the substituted variables are substituted.
Unlike weakening, the substitution itself needs to be well-typed, this is
defined as follows.
\begin{minted}{coq}
Inductive subslet {cf:checker_flags} Σ (Γ : context)
  : list term -> context -> Type :=

| emptyslet :
    subslet Σ Γ [] []

| cons_let_ass Δ s na t T :
    subslet Σ Γ s Δ ->
    Σ ;;; Γ |- t : subst0 s T ->
    subslet Σ Γ (t :: s) (Δ ,, vass na T)

| cons_let_def Δ s na t T :
    subslet Σ Γ s Δ ->
    Σ ;;; Γ |- subst0 s t : subst0 s T ->
    subslet Σ Γ (subst0 s t :: s) (Δ ,, vdef na t T).
\end{minted}
Unlike substitution typing I presented earlier this needs to account for local
definitions (\ie let-bindings) in the environment.
In a more understable language this becomes
\marginnote[1cm]{
  This time I write \(\Ga \vdash \sigma : \D\) were earlier I wrote
  \(\sigma : \Ga \to \D\). This is because in this case \(\sigma\) is a list
  of terms.
}
\begin{mathpar}
  \infer
    { }
    {\Sigma ; \Ga \vdash \bullet : \ctxempty}
  %

  \infer
    {
      \Sigma ; \Ga \vdash \sigma : \D \\
      \Sigma ; \Ga \vdash t : A[\sigma]
    }
    {\Sigma ; \Ga \vdash \sigma, x \sto t : \D, x:A}
  %

  \infer
    {
      \Sigma ; \Ga \vdash \sigma : \D \\
      \Sigma ; \Ga \vdash t[\sigma] : A[\sigma]
    }
    {\Sigma ; \Ga \vdash \sigma, x \sto t[\sigma] : \D, x : A \coloneqq t}
  %
\end{mathpar}
Notice how the term of the substitution is forced when it targets a definition,
this is reassuring because a substitution should not overwrite definitions.

\paradot{Confluence}

Confluence is an important result that we have on \Coq's theory. In particular
it allows us to simplify greatly what it means to be convertible.
Indeed, right now, conversion involves reduction going in both directions so
that \(u\) and \(v\) can be convertible by something like the following picture.
\begin{center}
  \begin{tikzpicture}[baseline=(u.base), node distance=1cm]
    \node (u) { \(u\) } ;
    \node (d1) [above right = of u] {} ;
    \node (d2) [below right = of d1] {} ;
    \node (d3) [above right = of d2] {} ;
    \node (v) [below right = of d3] { \(v\) } ;
    \path (d1.center) edge[to*, tred] (u) ;
    \path (d1.center) edge[to*, tred] (d2) ;
    \path (d3.center) edge[to*, tred] (d2) ;
    \path (d3.center) edge[to*, tred] (v) ;
  \end{tikzpicture}
\end{center}
Thanks to confluence this picture can be completed into
\begin{center}
  \begin{tikzpicture}[baseline=(u.base), node distance=1cm]
    \node (u) { \(u\) } ;
    \node (d1) [above right = of u.center] {} ;
    \node (d2) [below right = of d1.center] {} ;
    \node (d3) [above right = of d2.center] {} ;
    \node (v) [below right = of d3.center] { \(v\) } ;
    \node (a) [below right = of u.center] {} ;
    \node (b) [below right = of d2.center] {} ;
    \node (c) [below right = of a.center] {} ;
    \path (d1.center) edge[to*, tred] (u) ;
    \path (d1.center) edge[to*, tred] (d2) ;
    \path (d3.center) edge[to*, tred] (d2) ;
    \path (d3.center) edge[to*, tred] (v) ;
    \path (u) edge[to*, tred, dashed] (a) ;
    \path (d2.center) edge[to*, tred, dashed] (a) ;
    \path (d2.center) edge[to*, tred, dashed] (b) ;
    \path (v) edge[to*, tred, dashed] (b) ;
    \path (a.center) edge[to*, tred, dashed] (c) ;
    \path (b.center) edge[to*, tred, dashed] (c) ;
  \end{tikzpicture}
\end{center}
This means that we can only regard conversion as \(u\) and \(v\) reducing to
the same term (up to names and universes).

\paradot{Context conversion}

This is property that I did not mention before: context conversion states that
you can replace a context in a judgment by another context where all types
and definitions are convertible to the original one.
This will often be used to replace \(\Ga, x:A\) by \(\Ga, x:A'\) with
\(A \equiv A'\) but is proven in full generality.

\paradot{Validity}

In \acrshort{PCUIC}, validity differs from the usual because of arities.
\begin{minted}{coq}
Lemma validity_term :
  forall Σ Γ t T,
    wf Σ ->
    Σ ;;; Γ |- t : T ->
    isWfArity_or_Type Σ Γ T.
\end{minted}
As you can see, \mintinline{coq}{T} is guaranteed to either be a type (as in
be typed in a universe) or a well-formed arity.

\paradot{Subject reduction}

At the time of writing, subject reduction is not entierly proven.
\begin{minted}{coq}
Conjecture subject_reduction :
  forall Σ Γ t A B,
    wf Σ ->
    Σ ;;; Γ |- t : A ->
    red Σ Γ A B ->
    Σ ;;; Γ |- t : B.
\end{minted}

\paradot{Principality}

Because of cumulativity, \acrshort{PCUIC} cannot have uniqueness of type, it
however has principal types which we express as follows.
\begin{minted}{coq}
Conjecture principal_typing :
  forall Σ Γ u A B,
    Σ ;;; Γ |- u : A ->
    Σ ;;; Γ |- u : B ->
    ∑ C,
      Σ ;;; Γ |- C <= A ×
      Σ ;;; Γ |- C <= B ×
      Σ ;;; Γ |- u : C.
\end{minted}
If a term \(u\) has two types, then there is a smaller type typing \(u\).
That property is also work-in-progress.

\paradot{Strong normalisation}

Unlike for principality and subject reduction, there is no hope of proving
strong normalisation of \acrshort{PCUIC} inside \Coq. As such it will not be
stated as a conjecture but rather as an axiom.

\marginnote[1cm]{
  An infinite decreasing sequence for the opposite of reduction is simply an
  infinite reduction sequence, so we are indeed talking about the same concept.
}
The usual presentation of strong normalisation stating that there is no infinite
reduction sequence is ill-suited in a constructive setting so we instead show
that the opposite of reduction is well-founded.
This opposite, which I call coreduction, is defined as follows:
\begin{minted}{coq}
Inductive cored Σ Γ : term -> term -> Prop :=
| cored1 :
    forall u v,
      red1 Σ Γ u v ->
      cored Σ Γ v u

| cored_trans :
    forall u v w,
      cored Σ Γ v u ->
      red1 Σ Γ v w ->
      cored Σ Γ w u.
\end{minted}
It is the transitive closure of the symmetric of \(1\)-step reduction.

As I will explain in more depth in \nrefch{coq-orders}, we use accessiblity
predicates to state that a relation is well-founded constructively.
As such the axiom of strong normalisation is
\begin{minted}{coq}
Axiom normalisation :
  forall Γ t,
    wf Σ ->
    wellformed Σ Γ t ->
    Acc (cored (fst Σ) Γ) t.
\end{minted}
Herein \mintinline{coq}{wellformed Σ Γ t} states that \(t\) is either a
well-typed term or a well-formed arity. \mintinline{coq}{wellformed} also lands
in \mintinline{coq}{Prop} so that it gets erased during extraction to \ocaml.