% \setchapterpreamble[u]{\margintoc}
\chapter{Translation from \acrshort{ETT} to \acrshort{ITT} and \acrshort{WTT}}
\labch{elim-trans}

\section{The Translation}
\labsec{the-translation}

We now define the translations (let us stress the plural here) of an
extensional judgment. We extend $\ir$ canonically to contexts.
\marginnote[-0.5cm]{
  $\Ga \ir \Gb$ holds when they bind the same variables and the types are in
  relation for $\ir$.
}

Before defining the translation, we define a set
$\transl{\xisterm{\Ga}{t}{A}}$ of typing judgments in the target associated to a
typing judgment $\xisterm{\Ga}{t}{A}$ in \acrshort{ETT}.
\marginnote[-0.5cm]{
  This is rather a predicate. The formalisation is clearer on that fact.
}
%
The idea is that this set describes all the possible translations that
lead to the expected property. When
$\isterm{\Gb}{\tb}{\Ab} \in \transl{\xisterm{\Ga}{t}{A}}$, we say that
$\isterm{\Gb}{\tb}{\Ab}$ realises $\xisterm{\Ga}{t}{A}$. The
translation will be given by showing that this set is inhabited by
induction on the derivation.

\begin{definition}[Characterisation of possible translations]
  \leavevmode
  \begin{itemize}
    \item For any $\xisctx{\Ga}$ we define $\transl{\xisctx{\Ga}}$ as a set of
    valid judgments (in the target) such that
    $\isctx{\Gb} \in \transl{\xisctx{\Ga}}$ if and only if $\Ga \ir \Gb$.

    \item Similarly, $\isterm{\Gb}{\tb}{\Ab} \in \transl{\xisterm{\Ga}{t}{A}}$
    iff $\isctx{\Gb} \in \transl{\xisctx{\Ga}}$ and $A \ir \Ab$ and $t \ir \tb$.
  \end{itemize}
\end{definition}

In order to better master the shape of the produced realiser, we state the
following lemma which states that we can chose the translated term such that
the type has the same head type constructor as the type it realises.
%
This is important for instance for the case of an application, where we
do not know \emph{a priori} if the translated function has a dependent product
type, which is required to be able to use the typing rule for application.

\begin{lemma}
  \lablemma{choose}
  We can always \emph{choose} types $\Tb$ that have the same head constructor
  as $T$.
\end{lemma}

\begin{proof}
  Assume we have $\isterm{\Gb}{\tb}{\Tb} \in \transl{\xisterm{\Ga}{t}{T}}$.
  By definition of $\ir$,
  $T \ir \Tb$ means that $\Tb$ is shaped
  $\transpo{p}\ \transpo{q}\ ...\ \transpo{r}\ \Tb'$ with $\Tb'$ having
  the same head constructor as $T$. By inversion~\eqref{lemma:inversion}, the
  subterms are typable, including $\Tb'$. Actually, from inversion, we
  even get that the type of $\Tb'$ is a universe. Then,
  using \reflemma{sim-cong} and \reflemma{uip-cong}, we get
  $\isterm{\Gb}{e}{\Eq{}{\Tb}{\Tb'}}$.
  We conclude with
  $\isterm{\Gb}{\transpo{e}\ \tb}{\Tb'} \in \transl{\xisterm{\Ga}{t}{T}}$.
\end{proof}

Finally, in order for the induction to go through, we need to know
that when we have a realiser of a derivation $\xisterm{\Ga}{t}{T}$, we can
pick an arbitrary other type realising $\xistype{\Ga}{T}$ and still
get a new derivation realising $\xisterm{\Ga}{t}{T}$ with that type.
%
This is important for instance for the case of an application, where
the type of the domain of the translated function may differ from the
type of the translated argument. So we need to be able to change it \textit{a
posteriori}.


\begin{lemma}
  \lablemma{change-type}
  When we have $\isterm{\Gb}{\tb}{\Tb} \in \transl{\xisterm{\Ga}{t}{T}}$
  and $\istype{\Gb}{\Tb'} \in \transl{\xistype{\Ga}{T}}$ then we also have
  $\isterm{\Gb}{\tb'}{\Tb'} \in \transl{\xisterm{\Ga}{t}{T}}$ for some $\tb'$.
\end{lemma}

\begin{proof}
  By definition we have $T \ir \Tb$ and $T \ir \Tb'$ and thus $\Tb \sim \Tb'$.
  By \reflemma{sim-cong}
  (in the case $\Ga_1 \equiv \Ga_2 \equiv \ctxempty$) we get
  $\isterm{\Gb}{p}{\Heq{}{\Tb}{}{\Tb'}}$ for some $p$.
  By \reflemma{uip-cong} (and \reflemma{choose} to give
  universes as types to $\Tb$ and $\Tb'$) we can assume
  $\isterm{\Gb}{p}{\Eq{}{\Tb}{\Tb'}}$. Then
  $\isterm{\Gb}{\transpo{p}\ \tb}{\Tb'}$ is still a translation since $\ir$
  ignores transports.
\end{proof}

We can now define the translation. This is done by mutual induction on
context well-formedness, typing and conversion derivations. Indeed,
in order to be able to produce a realiser by induction, we need to show
that every conversion in \acrshort{ETT} is translated as an heterogeneous
equality.

\begin{theorem}[Translation]
  \label{thm:translation}
  \leavevmode
  \begin{itemize}
    \item If\,\,\,$\xisctx{\Ga}$ then there exists
    $\isctx{\Gb} \in \transl{\xisctx{\Ga}}$,

    \item If\,\,\,$\xisterm{\Ga}{t}{T}$ then for any
    $\isctx{\Gb} \in \transl{\xisctx{\Ga}}$ there exist $\tb$ and $\Tb$ such
    that $\isterm{\Gb}{\tb}{\Tb} \in \transl{\xisterm{\Ga}{t}{T}}$,

    \item If\,\,\,$\xeqterm{\Ga}{u}{v}{A}$ then for any
    $\isctx{\Gb} \in \transl{\xisctx{\Ga}}$ there exist
    $A \ir \Ab, A \ir \Ab', u \ir \ub, v \ir \vb$ and $\eb$ such that
    $\isterm{\Gb}{\eb}{\Heq{\Ab}{\ub}{\Ab'}{\vb}}$.
  \end{itemize}
\end{theorem}

\begin{proof}
  We prove the theorem by induction on the derivation in \acrshort{ETT}.
  We only show the two most interesting cases of application and conversion.
  The complete proof is given in \frefsec{corr-transl}.
  \todo{Again, not clear I want to keep it... Isn't the formalisation or the
  paper enough?}

  \begin{itemize}
    \item \textsc{Application}
    \[
      \infer[]
        {\xisterm{\Ga}{A}{s} \\
         \xisterm{\Ga,x:A}{B}{s'} \\
         \xisterm{\Ga}{t}{\Prod{x:A} B} \\
         \xisterm{\Ga}{u}{A}
        }
        {\xisterm{\Ga}{\app{t}{x:A}{B}{u}}{B[x \sto u]}}
      %
    \]
    Using IH together with \reflemma{choose} and \reflemma{change-type}
    we get $\isterm{\Gb}{\Ab}{s}$ and $\isterm{\Gb,x:\Ab}{\Bb}{s'}$ and
    $\isterm{\Gb}{\tb}{\Prod{x:\Ab} \Bb}$ and $\isterm{\Gb}{\ub}{\Ab}$
    meaning we can conclude
    $\isterm{\Gb}{\app{\tb}{x:\Ab}{\Bb}{\ub}}{\Bb[x \sto \ub]}
    \in \transl{\xisterm{\Ga}{\app{t}{x:A}{B}{u}}{B[x \sto u]}}$.

    \item \textsc{Conversion}
    \[
      \infer[]
        {\xisterm{\Ga}{u}{A} \\
         \xeqtype{\Ga}{A}{B}
        }
        {\xisterm{\Ga}{u}{B}}
      %
    \]
    By IH and \reflemma{uip-cong} we have
    $\isterm{\Gb}{\eb}{\Eq{}{\Ab}{\Bb}}$ which implies
    $\istype{\Gb}{\Ab} \in \transl{\xistype{\Ga}{A}}$ by
    inversion~\eqref{lemma:inversion}, thus, from \reflemma{change-type}
    and IH we get $\isterm{\Gb}{\ub}{\Ab}$, yielding
    $\isterm{\Gb}{\transpo{\eb}\ \ub}{\Bb} \in \transl{\xisterm{\Ga}{u}{B}}$.
  \end{itemize}

\end{proof}


\section{Meta-theoretical Consequences}
\labsec{meta-consequences}

We can check that all \acrshort{ETT} theorems whose type are typable in
\acrshort{ITT} have proofs in \acrshort{ITT} as well.
This also holds for \acrshort{WTT}.

The reason I did not differentiate the two target theories is simply because
there was no need. The proof is exactly the same, indeed, the proof never uses
the conversion in the target since every conversion from \acrshort{ETT} is
translated to an equality.

\begin{corollary}[Conservativity]
  \labcor{preservation}
  If $\xisterm{}{t}{T}$ and $\istype{}{T}$ then there exist $\tb$ such that
    $\isterm{}{\tb}{T} \in \transl{\xisterm{}{t}{T}}$.
\end{corollary}

\begin{proof}
  Since $\isctx{\ctxempty} \in \transl{\xisctx{\ctxempty}}$, by
  Theorem~\eqref{thm:translation}, there exists $\tb$ and $\Tb$ such
  that
  $\isterm{}{\tb}{\Tb} \in \transl{\xisterm{}{t}{T}}$
  But as $\istype{}{T}$, we have
  $\istype{}{T} \in \transl{\xistype{}{T}}$, and,
  using \reflemma{change-type}, we obtain
  $\isterm{}{\tb}{T} \in \transl{\xisterm{}{t}{T}}$.
\end{proof}

\begin{corollary}[Relative consistency]
  \labcor{consistency}
  Assuming the target (\acrshort{ITT} or \acrshort{WTT} depending on you taste)
  is consistent, there is no term $t$ such that
  $\xisterm{}{t}{\Prod{A:\Ty{0}}{A}}$.
\end{corollary}

\begin{proof}
  Assume such a $t$ exists.
  We have $\istype{}{\Prod{A:\Ty{0}}{A}}$, so by \refcor{preservation}
  there exists $\tb$ such that $\isterm{}{\tb}{\Prod{A:\Ty{0}}{A}}$ which
  contradicts the assumed consistency of the target.
\end{proof}

\section{Optimisations}
\labsec{optim}

Up until now, we remained silent about one thing: the size of the
translated terms. Indeed, the translated term is a decoration of the
initial one by transports which appear in many locations. For example,
at each application we use a transport by \reflemma{choose} to
ensure that the term in function position is given a function type. In
most cases---in particular when translating ITT terms---this produces
unnecessary transports (often by reflexivity) that we wish to avoid.

In order to limit the size explosion, in the above we use a different version of
transport, namely $\mathsf{transport}'$ such that
%
\begin{align*}
  \otransport{A_1}{A_2}{p}{t} &= t &\text{ when } A_1 =_\alpha A_2 \\
  &= \transpo{p}{t} &\text{ otherwise.}
\end{align*}
%
The idea is that we avoid \emph{trivially} unnecessary transports (we do not
deal with $\beta$-conversion for instance).
We extend this technique to the different constructors of equality (symmetry,
transitivity, \dots) so that they reduce to reflexivity whenever possible.
Take transitivity for instance:
%
\begin{align*}
  \otransitivity{\refl{} u}{q} &= q \\
  \otransitivity{p}{\refl{} u} &= p \\
  \otransitivity{p}{q} &= \translitivity{p}{q}.
\end{align*}
%
We show these \emph{defined terms} enjoy the same typing rules as their
counterparts and use them instead.
In practice it is enough to recover the exact same term when it is typed in
\acrshort{ITT}.
In \acrshort{ITT}, this can even be pushed a bit further to test some
conversions, but in \acrshort{WTT} it has to stay syntactical.

Note that these optimisations serve no purpose at all to one interested only in
the proof, it only becomes interesting when exploiting the constructiveness of
the formalised proof to get translated terms and the corresponding derivations
from a derivation in \acrshort{ETT}.