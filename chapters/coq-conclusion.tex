% \setchapterpreamble[u]{\margintoc}
\chapter{Conclusion}
\labch{coq-conclusion}

We have formalised an almost feature-complete type checker for \Coq in \Coq
and proven it correct. Thanks to the extraction mechanism of \Coq, this can
actually be turned into an independent type checker program.
For instance it can be run within Coq in the manner of a plugin:
\begin{minted}{coq}
MetaCoq SafeCheck nat.
\end{minted}

Because we do not deal with template polymorphism, the module system and
\(\eta\)-expansion, we are not yet able to typecheck the standard library
or the formalisation itself (as it relies on the standard library).
We were however able to test it on reasonable proof terms coming from
the \acrshort{HoTT} library~\sidecite{bauer2017hott}.
\marginnote[1.1cm]{
  Examples of the checker in action, including the \acrshort{HoTT} theorems
  can be found in the file
  \href{https://github.com/MetaCoq/metacoq/blob/popl-artifact-eval/test-suite/safechecker_test.v}{test-suite/safechecker\_test.v}.
}%
For instance, we have been able to typecheck the proof that an isomorphism can
be turned into an equivalence.
It currently is about one order of magnitude slower than the \Coq
implementation (0.015s vs 0.002s averaged over 10 runs for each checking).
This can in particular be attributed to our very ineficient representation of
global environments as association lists indexed by character lists, where \Coq
uses efficient hash-maps on strings.

Another point I would like to address is the lack of completeness proof.
Our tests suggest that we are close to completeness if not there yet for the
fragment we are considering. It would be however really interesting to have a
proof that conversion and inference are complete with respect to their
specifications.
At the moment, dealing with \(\eta\)-conversion seems like a much more pressing
matter as it would make the checker usable in practice at least as a standalone
independent checker.

There is also the long term dream of having a \Coq kernel fully verified and
running in \Coq.