% \setchapterpreamble[u]{\margintoc}
\chapter{Syntax and formalisation of type theory}
\labch{formalisation}

An interesting fact of type theory (and perhaps one its main selling points) is
that it is a suitable framework in which to reason about type theory.
Representing type theory in itself isn't entirely straightforward, and some care
must be taken. There are actually several choices to be made when representing
type theory and they are not all equivalent or with the same pros and cons.
I will detail some of them, spending more time on those I ended up choosing
and will try to motivate my choice.