% \setchapterpreamble[u]{\margintoc}
\chapter{Syntax and formalisation of type theory}
\labch{formalisation}

An interesting fact of type theory (and perhaps one its main selling points) is
that it is a suitable framework in which to reason about type theory.
That being said, representing type theory in itself isn't entirely
straightforward, and some care must be taken. There are actually several choices
to be made when representing type theory and they are not all equivalent or with
the same pros and cons.
I will detail some of them, spending more time on those I ended up choosing
and will try to motivate my choice.

In this chapter I will refer to work done in conjunction with Andrej Bauer and
Philipp Haselwarter called \ftt~\sidecite{formaltypetheory}.

\section{Representation of syntax}

I will first focus on the syntactical side of type theory.
The first important choice being how to represent variables.

\subsection{How to deal with variables}

When writing programs or expressions with binders on paper or on the computer
we will usually use \emph{names}, identifiers for variables like in
\(\lambda x. \lambda y. x\ y\), \(x\) refers to the variable bound by le
outermost \(\lambda\), while \(y\) referes to the variable bound by the
innermost.
The names aren't fundamental in what the term represents:
\(\lambda z. \lambda w. z\ w\) represents \emph{exactly} the same term.
We call this operation \(\alpha\)-renaming, here I \(\alpha\)-renamed \(x\)
to \(z\) and \(y\) to \(w\). This defines the notion of \(\alpha\)-equality
or \(\alpha\)-equivalence.
\[
  \lambda x. \lambda y. x\ y =_\alpha \lambda z. \lambda w. z\ w
\]
However variable names should only be thought of as an abstraction to represent
such terms and not a part of the syntax in itself.

Thinking in terms of variables name can lead to unpleasant examples where
\(\alpha\)-renaming might become a necessity.
For instance, \(\lambda x. \lambda x. x\) is perfectly valid but is easier to
read when renamed to \(\lambda y. \lambda x. x\). This process is called
\emph{shadowing}, when several variables bear the same name in scope, it is the
innermost that takes precedence. This principle is crucial for compositionality.

Even more problems arise when considering substitutions (after all, that is what
variables are for: to be substituted).
If you consider the term \(t \coloneqq x\ (\lambda x. x)\) we have two
occurrences of the name \(x\) but they do \emph{not} represent the same
variable, the first \(x\) is \emph{free} in the term, while the second is
\emph{bound} by the only \(\lambda\).
Now when substituting \(x\) for term \(u\) in \(t\), one has to be careful not
to replace the bound variable \(x\). The expected result is
\[
  t[x \sto u] = u\ (\lambda x. x)
\]
This used to be called \emph{capture-avoiding} substitutions, but I will call
them substitutions, because the operation yielding \(u\ (\lambda x. u)\)
is utterly rubbish and not deserving of a name.

Several solutions have been proposed to this ``problem'' like nominal
sets~\sidecite[-0.4cm]{pitts2001nominal},
\acrfull{HOAS}~\sidecite[0.3cm]{pfenning1988higher} and de Bruijn indices or
levels~\sidecite[0.6cm]{de1978lambda}.
In think de Bruijn indices are exactly what we want when dealing with
\(\lambda\)-terms as they carry the right amount of information.
The idea is to use natural numbers instead of names to indicate how many binders
to traverse before reaching the one introducing the variable.
\[
  \begin{array}{rcl}
    \lambda x.\ \lambda y.\ \lambda z.\ z
    &\to& \lambda\ \lambda\ \lambda\ \db{0} \\
    \lambda x.\ \lambda y.\ \lambda z.\ y
    &\to& \lambda\ \lambda\ \lambda\ \db{1} \\
    \lambda x.\ \lambda y.\ \lambda z.\ x
    &\to& \lambda\ \lambda\ \lambda\ \db{2}
  \end{array}
\]
In this setting the same variable can be represented in different ways:
\[
  \lambda x.\ x\ (\lambda y.\ x\ y)
\]
becomes
\[
  \lambda\ \db{0}\ (\lambda\ \db{1}\ \db{0})
\]
so that \(x\) is now written \(\db{0}\) and \(\db{1}\) depending on whether it
is referenced under the second \(\lambda\) or not.
The following diagram should make things more explicit.

\begin{center}
  \begin{tikzpicture}[remember picture]
    \node (term) {
      \(\subnode{la}{\(\lambda\)}\ \subnode{va}{\(\db{0}\)}\
      (\subnode{lb}{\(\lambda\)}\ \subnode{vb}{\(\db{1}\)}\
      \subnode{vc}{\(\db{0}\)})\)
    } ;
    \draw[barrow, bend right] (va.north) to (la.north) ;
    \draw[barrow, bend left] (vb.south) to (la.south) ;
    \draw[barrow, bend right] (vc.north) to (lb.north) ;
  \end{tikzpicture}
\end{center}

Using this representation, both \(\lambda x.x\) and \(\lambda y.y\) are written
\[
  \lambda\ \db{0}
\]
so that \(\alpha\)-equality is purely syntactic equality.
\(\alpha\)-renaming is not only a problem for pretty-printing.
This also solves the problem of substitutions potentially capturing free
variables: the term \(t \coloneqq x\ (\lambda x.\ x)\) of before is now
\(\db{n}\ (\lambda\ \db{0})\) where \(n\) is some number which should point to
somewhere in the context (same as the \(x\) it replaces).

Notice however that using de Bruijn indices, weakening---\ie putting a term into
an extended context---will now affect the term itself.
If you consider the following weakening, adding one variable \(z\) in the middle
of the context, doesn't affect the term using names.
\marginnote[1cm]{
  I use variables in the scope to make clear where the new variable is inserted
  in the nameless case.
}
\[
  x : A, y : B \vdash x\ y\ (\lambda u.\ x\ y\ u)
  \leadsto
  x : A, z : C, y : B \vdash x\ y\ (\lambda u.\ x\ y\ u)
\]
In the context of de Bruijn indices this becomes
\[
  A, B \vdash \db{1}\ \db{0}\ (\lambda\ \db{2}\ \db{1}\ \db{0})
  \leadsto
  A, C, B \vdash
  \highlight{\db{2}}\ \db{0}\ (\lambda\ \highlight{\db{3}}\ \db{1}\ \db{0})
\]


\subsection{Substitutions}

Another choice that is close to the representation of variable is that of
substitutions. There are several ways to represent a substitution in itself,
but I think the main question is whether to make them \emph{explicit} or not,
\ie part of the syntax or not.

With explicit substitutions \(t[\sigma]\) is a term in itself and things like
evaluation of substitutions come as reduction rules:
\[
  x[\sigma] \red \sigma(x)
\]
The question of how you represent substitutions is more curcial in this case,
and there are several ways to do so, the first being introduced in
\sidecite{abadi1991explicit}.

In \ftt where we formalised syntax of type theory in \Coq using explicit
substitutions we settled on the following constructions (I will give them
using typing rules to make their behaviour explicit):
\begin{mathpar}
  \infer
    {\Ga \vdash u : A}
    {\sbzero{A}{u} : \Ga \to \Ga,A}
  %

  \infer
    {\Ga \vdash A}
    {\sbweak{A} : \Ga \to \Ga, A}
  %

  \infer
    {
      \sigma : \Ga \to \D \\
      \D \vdash A
    }
    {\sbshift{A}{\sigma} : \Ga, A[\sigma] \to \D, A}
  %

  \infer
    {\vdash \Ga}
    {\sbid : \Ga \to \Ga}
  %

  \infer
    {
      \sigma : \Ga \to \D \\
      \theta : \D \to \Xi
    }
    {\theta \circ \sigma : \Ga \to \Xi}
  %

  \infer
    {\vdash \Ga}
    {\sbterminal : \Ga \to \ctxempty}
  %
\end{mathpar}

With computation rules such as
\[
  \begin{array}{rcl}
    \db{0}[\sbzero{A}{u}] &\red& u \\
    (\db{n+1})[\sbzero{A}{u}] &\red& \db{n} \\
    \db{n}[\sbweak{A}] &\red& \db{n+1} \\
    \db{0}[\sbshift{A}{\sigma}] &\red& \db{0} \\
    (\db{n+1})[\sbshift{A}{\sigma}] &\red& \db{n}[\sigma][\sbweak{A[\sigma]}] \\
    &\dots&
  \end{array}
\]

However, I find the other option of having substitutions as a meta-operation,
outside of the syntax, more natural. It also helps in keeping the syntax and
rules to a minimum while turning the substitutions notions above into
definitions and properties.
For instance weakening becomes a lemma and not something postulated with the
constructor \(\sbweak{}\).
I will not make a strong case for either choice however as they both have their
own interest.
Note that in \MetaCoq we go with meta-level substitutions.

\subsection{Annotations}
\todo{Curry vs Church for annotation of domain}
\todo{Annotation paranoia formal-type-theory (including typed or untyped equality?)}
\todo{Cardinal model}

\subsection{Universes and types}
\todo{Tarski/Russel universes}

\section{Representation of typing}

\todo{And conversion}
\todo{Well-typed syntax that I did not study in depth but needs to be
mentioned → maybe expose my idea that it isn't exactly the same,
notion of computation in the meta (→ translations)., HOAS?}