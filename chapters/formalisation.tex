% \setchapterpreamble[u]{\margintoc}
\chapter{Syntax and formalisation of type theory}
\labch{formalisation}

An interesting fact of type theory (and perhaps one its main selling points) is
that it is a suitable framework in which to reason about type theory.
That being said, representing type theory in itself isn't entirely
straightforward, and some care must be taken. There are actually several choices
to be made when representing type theory and they are not all equivalent or with
the same pros and cons.
I will detail some of them, spending more time on those I ended up choosing
and will try to motivate my choice.

\section{How to deal with variables}

When writing programs or expressions with binders on paper or on the computer
we will usually use \emph{names}, identifiers for variables like in
\(\lambda x. \lambda y. x\ y\), \(x\) refers to the variable bound by le
outermost \(\lambda\), while \(y\) referes to the variable bound by the
innermost.
The names aren't fundamental in what the term represents:
\(\lambda z. \lambda w. z\ w\) represents \emph{exactly} the same term.
We call this operation \(\alpha\)-renaming, here I \(\alpha\)-renamed \(x\)
to \(z\) and \(y\) to \(w\). This defines the notion of \(\alpha\)-equality
or \(\alpha\)-equivalence.
\[
  \lambda x. \lambda y. x\ y =_\alpha \lambda z. \lambda w. z\ w
\]
However variable names should only be thought of as an abstraction to represent
such terms and not a part of the syntax in itself.

Thinking in terms of variables name can lead to unpleasant examples where
\(\alpha\)-renaming might become a necessity.
For instance, \(\lambda x. \lambda x. x\) is perfectly valid but is easier to
read when renamed to \(\lambda y. \lambda x. x\). This process is called
\emph{shadowing}, when several variables bear the same name in scope, it is the
innermost that takes precedence. This principle is crucial for compositionality.

Even more problems arise when considering substitutions (after all, that is what
variables are for: to be substituted).
If you consider the term \(t \coloneqq x\ (\lambda x. x)\) we have two
occurrences of the name \(x\) but they do \emph{not} represent the same
variable, the first \(x\) is \emph{free} in the term, while the second is
\emph{bound} by the only \(\lambda\).
Now when substituting \(x\) for term \(u\) in \(t\), one has to be careful not
to replace the bound variable \(x\). The expected result is
\[
  t[x \sto u] = u\ (\lambda x. x)
\]
This used to be called \emph{capture-avoiding} substitutions, but I will call
them substitutions, because the operation yielding \(u\ (\lambda x. u)\)
is utterly rubbish and not deserving of a name.

Several solutions have been proposed to this ``problem'' like nominal
sets~\sidecite[-2.1cm]{pitts2001nominal},
\acrfull{HOAS}~\sidecite[-1.4cm]{pfenning1988higher} and de Bruijn indices or
levels~\sidecite[-1.1cm]{de1978lambda}.
In think de Bruijn indices are exactly what we want when dealing with
\(\lambda\)-terms as they carry the right amount of information.
The idea is to use natural numbers instead of names to indicate how many binders
to traverse before reaching the one introducing the variable.
\[
  \begin{array}{rcl}
    \lambda x.\ \lambda y.\ \lambda z.\ z
    &\to& \lambda\ \lambda\ \lambda\ \db{0} \\
    \lambda x.\ \lambda y.\ \lambda z.\ y
    &\to& \lambda\ \lambda\ \lambda\ \db{1} \\
    \lambda x.\ \lambda y.\ \lambda z.\ x
    &\to& \lambda\ \lambda\ \lambda\ \db{2}
  \end{array}
\]
In this setting the same variable can be represented in different ways:
\[
  \lambda x.\ x\ (\lambda y.\ x\ y)
\]
becomes
\[
  \lambda\ \db{0}\ (\lambda\ \db{1}\ \db{0})
\]
so that \(x\) is now written \(\db{0}\) and \(\db{1}\) depending on whether it
is referenced under the second \(\lambda\) or not.

\makeatletter
\def\pgfsys@hboxsynced#1{%
  {%
    \pgfsys@beginscope%
    \setbox\pgf@hbox=\hbox{%
      \hskip\pgf@pt@x%
      \raise\pgf@pt@y\hbox{%
        \pgf@pt@x=0pt%
        \pgf@pt@y=0pt%
        \special{pdf: content q}%
        \pgflowlevelsynccm%
        \pgfsys@invoke{q -1 0 0 -1 0 0 cm}%
        \special{pdf: content -1 0 0 -1 0 0 cm q}% translate to original coordinate system
        \pgfsys@invoke{0 J [] 0 d}% reset line cap and dash
        \wd#1=0pt%
        \ht#1=0pt%
        \dp#1=0pt%
        \box#1%
        \pgfsys@invoke{n Q Q Q}%
      }%
      \hss%
    }%
    \wd\pgf@hbox=0pt%
    \ht\pgf@hbox=0pt%
    \dp\pgf@hbox=0pt%
    \pgfsys@hbox\pgf@hbox%
    \pgfsys@endscope%
  }%
}
\makeatother

% \tikzstyle{arrow} = [thick,->,>=stealth]
\tikzstyle{barrow} = [->, bend angle=40]

\begin{center}
  \begin{tikzpicture}[remember picture]
    \node (term) {
      \(\subnode{la}{\(\lambda\)}\ \subnode{va}{\(\db{0}\)}\
      (\subnode{lb}{\(\lambda\)}\ \subnode{vb}{\(\db{1}\)}\
      \subnode{vc}{\(\db{0}\)})\)
    } ;
    \draw[barrow, bend right] (va.north) to (la.north) ;
    \draw[barrow, bend left] (vb.south) to (la.south) ;
    \draw[barrow, bend right] (vc.north) to (lb.north) ;
  \end{tikzpicture}
\end{center}

\todo{Explicit substitutions}
\todo{Maybe this section is about syntax}
\todo{Curry vs Church for annotation of domain}
\todo{Annotation paranoia formal-type-theory (including typed or untyped equality?)}
\todo{Cardinal model}
\todo{Tarski/Russel universes}

\section{Representation of typing}

\todo{And conversion}
\todo{Well-typed syntax that I did not study in depth but needs to be
mentioned → maybe expose my idea that it isn't exactly the same,
notion of computation in the meta (→ translations)., HOAS?}