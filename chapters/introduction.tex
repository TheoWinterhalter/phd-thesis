% \setchapterpreamble[u]{\margintoc}
\chapter{Introduction}
\labch{intro}

Type theory is set at the interface between programming and formal logic,
feeding off and nourishing both worlds. Proof assistants can be built on it,
making them full-fledged programming languages as well, with the added benifit
of producing certified programs.

My main interest lies in the study of type theory while relying on the tools it
provides: I study type theory \emph{within} type theory.
As such I focused on the formalisation of type theory in the \Coq proof
assistant~\sidecite[-0.7cm]{coq} and particularly on two points:
\marginnote[0.7cm]{
  Reflection is defined in \arefsubsec{ett-def}, and weak equality in
  \arefsubsec{wtt}, both in \refch{flavours}.
}
\begin{itemize}
  \item How can we effectively turn a proof using a very strong notion of
  equality called \emph{reflection}, into a proof relying on a very weak notion
  of equality?
  \item How can we improve the trust in our system, in my case \Coq, using this
  system itself as a framework to study it?
\end{itemize}

\paradot{Contributions}
My contributions will mainly be found in \arefpart{elim-reflection} and
\arefpart{coq-in-coq} corresponding to the following published
articles~\sidecite{winterhalter:hal-01849166,sozeau2019coq,sozeau:hal-02167423}.
While the chapters before these two parts are mainly introductory and
corresponding to a rough state-of-the-art, they actually contain other
contributions, namely work I did with Andrej Bauer on the cardinal model
in \nrefch{models} that we did not publish, and work I did with Andrej Bauer
and Philipp Haselwarter on formalising type theory called
\ftt~\sidecite{formaltypetheory} and that I present briefly in
\nrefch{formalisation}.
\todo{Make some statements precise for the reader in a hurry? Or should I have
an abstract instead?}

\section{Proof assistants}

One of my goals is improving and understanding better proof assistants, but what
is a proof assistant?
I think we can see them as chatbots---that is, those programs that you can
converse with---that are there to help you assert and prove theorems.
They are not particularly smart and will not get the job done for you, but they
are very annoying because they do not always understand what you say and you
have to be very precise or they will point out your mistakes.

\marginnote[1.2cm]{
  The user will have blue speech bubbles, slightly on the left, while the proof
  assistant will answer in green (for good) or red (for bad) on the right-hand
  side.
}%
\begin{center}
  \includegraphics[width=0.6\textwidth]{coq-chatbot.pdf}
\end{center}

Here, I represent the user on the left, conversing with the proof assistant, on
the right, that I picture as a laptop with a rooster inside.

Let us now see in detail how such a conversation can go.

\begin{center}
  \includegraphics[width=0.8\textwidth]{modern-art.pdf}
\end{center}

This piece of modern art will be the setting for the exchange between the user
and the proof assistant, and we will assume they both agree on it.
We have cats in boxes (although one of the cats is rather a lion), of different
colours, and subject to different gravitation forces.

At the beginning, the user makes their purpose know to the proof assistant,
saying for which theorem they require their assistance.

\begin{center}
  \includegraphics[width=0.8\textwidth]{iwanterror.pdf}
\end{center}

The sentence above doesn't sound like a statement we can prove, and the proof
assistant will rightfully complain.

\begin{center}
  \includegraphics[width=0.8\textwidth]{error-incomplete.pdf}
\end{center}

Indeed, we were to fast and forgot half of what we wanted to say. Next we give
it a complete sentence.

\begin{center}
  \includegraphics[width=0.8\textwidth]{iwant-is.pdf}
\end{center}

This time, the statement is slightly incorrect, or not precise enough for our
good assistant.

\begin{center}
  \includegraphics[width=0.8\textwidth]{error-box-feline.pdf}
\end{center}

We have to revise our wanted theorem, and realise that we do not want to prove
that all boxes are feline, although a human might understand what we mean by
that, but that all of the boxes contain, each, a feline.

\begin{center}
  \includegraphics[width=0.8\textwidth]{box-contains-feline.pdf}
  \includegraphics[width=0.8\textwidth]{how-to-prove.pdf}
\end{center}

The proof assistant has now understood what we meant. The interactive part of
the prove now begins. We want to prove a property on the boxes, as there are
four to consider, we can tell the proof assistant that we intend to look at each
case, one by one.

\begin{center}
  \includegraphics[width=0.8\textwidth]{convo7.pdf}
  \includegraphics[width=0.8\textwidth]{convo8.pdf}
\end{center}

We are presented with four cases, and as many statements to prove.
In this case, we are rather lazy, especially since all four cases will be
similar. Fortunately for us, the proof assistant is capable of understanding
that you want to deal with several cases in similar ways, \emph{and} is also
capable of doing very basic proofs without the user.

\begin{center}
  \includegraphics[width=0.8\textwidth]{convo9.pdf}
  \includegraphics[width=0.8\textwidth]{convo10.pdf}
\end{center}

Once the proof is complete, the proof assistant tells you so and you can move on
to other theorems to prove. I already showed that you have to be explicit and
non ambiguous when talking to the proof assistant. You also have to be correct.
The proof assistant will not trust blindly into what you say.
%
Let us now consider a theorem that is not provable.

\begin{center}
  \includegraphics[width=0.8\textwidth]{convo11.pdf}
  \includegraphics[width=0.8\textwidth]{how-to-prove.pdf}
  \includegraphics[width=0.8\textwidth]{convo7.pdf}
  \includegraphics[width=0.8\textwidth]{convo12.pdf}
  \includegraphics[width=0.8\textwidth]{convo9.pdf}
  \includegraphics[width=0.8\textwidth]{convo13.pdf}
\end{center}

Here the proof assistant tells us that not all cases are trivial so we decide
to go over them one by one to understand what is wrong.

\begin{center}
  \includegraphics[width=0.8\textwidth]{convo-focus.pdf}
  \includegraphics[width=0.8\textwidth]{convo-case1.pdf}
\end{center}

When we focus on a case, the proof assistant reminds us of what we have to prove
specifically. Here we revert to using our earlier strategy and simply say that
it is plain to see.

\begin{center}
  \includegraphics[width=0.8\textwidth]{convo-trivial.pdf}
  \includegraphics[width=0.8\textwidth]{convo-indeed.pdf}
  \includegraphics[width=0.8\textwidth]{convo-case2.pdf}
  \includegraphics[width=0.8\textwidth]{convo-trivial.pdf}
  \includegraphics[width=0.8\textwidth]{convo-indeed.pdf}
  \includegraphics[width=0.8\textwidth]{convo-case3.pdf}
  \includegraphics[width=0.8\textwidth]{convo-abort.pdf}
\end{center}

This time we see we do not have a cat so we do not even try to prove it.
We abort the proof, that is we give up on proving a statement we know to be
wrong now.

A proof assistant is thus both helping us do proofs by building parts of it on
its own, and verifying that they are correct, for instance by not forgetting one
case were we had a lion rather than a cat.

Proof assistants are also interesting and useful for other people than the user
doing the proof. Say that I have proven a complicated theorem using my favourite
proof assistant, people who trust my proof assistant will simply have to ask it
if it agrees with my claim.
In fact, proof assistants usually generate so-called \emph{certificates} that
can be checked by others without having to understand the details of the proof.
\begin{center}
  \includegraphics[width=0.4\textwidth]{cerificate.pdf}
\end{center}

Even then, why should we trust the proof assistant? What exactly can we prove
with them? And how can we make them better?
Part of my work is focused on these questions, and on the study of proof
assistants based on something called type theory.

\section{Type theory}

\todo{Some sort of summary of the intro part?}

\section{Summary of part 1}

\todo{Outlining the problem mainly}

\section{Summary of part 2}

\todo{First talk about how verifying is important, increase trust}