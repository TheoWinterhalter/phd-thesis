\setchapterpreamble[u]{\margintoc}
\chapter{Introduction}
\labch{intro}


\section{Meta}

Nothing organised for now...

\paragraph{Things to put in (high level)}

\begin{itemize}
  \item ETT to ITT
  \item ETT to Weak TT
  \item Coq Coq Correct
\end{itemize}
%
Probably merging the two firsts. Since there is a CPP article for ETT to ITT
it might be worth it to instead write a chapter on ETT to WTT and say we can
derive the other from WTT to ITT.
However, this would not allow me to really talk about the ``plugin''.

\paragraph{Introduction}

\begin{itemize}
  \item Proof theory proof as object(?)
  \item Type theory
  \item Curry-Howard for SLT first?
  \item Gödel?
  \item Why formalise type theory
  \item Dependent types (also for programming?)
\end{itemize}

\paragraph{Outline}

For now only three chapters excluding intro but it feels like there should be
more. Maybe these aren't chapters but parts.
The Card model for instance could be its own chapter.
HoTT should be mentioned to state ETT to ITT/WTT extends to 2TT/HTS.

Maybe a first part on type theory itself?
It feels like maybe it should be part of the introduction though.
There I should also mention models (otherwise it'd be a bit rough to talk
about models to justify syntax choices).

\begin{enumerate}
  \item Formalisation / Representation of Type Theories
    \begin{itemize}
      \item Translations: syntactical or not
      \item Annotations for applications etc (Card model?)
      \item MetaCoq / TemplateCoq?
    \end{itemize}
  \item Computation in Type Theory
    \begin{itemize}
      \item ETT, ITT, WTT
      \item Funext, UIP? (HoTT?)
    \end{itemize}
  \item Kernel of \Coq written in \Coq
    \begin{enumerate}
      \item Mainly the checker
      \item Briefly mention extraction (paper + Yannik's thesis?)
      \item Meta-theory (although it's not my main contrib)
    \end{enumerate}
\end{enumerate}

Regarding the meta-theory, even if I'm not the author of that of MetaCoq/PCUIC
I still have done similar (but simpler) work with ett-to-itt and my
formal-type-theory with Andrej and Philipp.

\paragraph{What can already be written?}

Unclear as of now. There will surely be ETT/ITT/WTT and Coq presentations,
along with a MetaCoq presentation. If they are mixed or separate and what
notions will already be introduced are big unknowns though.

I assumed I would be doing the introductory chapters later on but it's hard not
to rely on the definitions they would introduce.

\paragraph{New attempt at outline}

\begin{itemize}
  \item Proof theory\sidenote{As first? Do I have much to say?}
    \begin{itemize}
      \item How to prove something
      \item Logics\sidenote{Consistency, frameworks}
      \item Self-evident argument\sidenote{Ref de Bruijn?}
      \item Mechanised proofs\sidenote{Too early?}
      \item Constructiveness, classical logic, linear logic
    \end{itemize}
  \item Simple type theory
    \begin{itemize}
      \item λ-calculus
      \item Types for programs
      \item Curry-Howard
    \end{itemize}
  \item Dependent types
    \begin{itemize}
      \item For programming
      \item For verification
      \item For maths
      \item Constructiveness by computation (forward ref canonicity)
    \end{itemize}
  \item Usual definitions in type theory
    \begin{itemize}
      \item Inductive types (nat, lists, vectors)
      \item Equality\sidenote{UIP and funext may be mentioned here, transports
      also}
    \end{itemize}
  \item Flavours or type theory\sidenote{Or computation in type theory?
  This is not clear in which order to introduce them.}
    \begin{itemize}
      \item ITT / MLTT / PCUIC
      \item ETT
      \item WTT
      \item HoTT, CubicalTT, 2TT, HTS
    \end{itemize}
  \item Usual/desirable properties of type theory
    \begin{itemize}
      \item Weakening, substitution
      \item Inversion of typing
      \item Subject reduction
      \item Unique / principal typing
      \item Confluence
      \item Termination
      \item Canonicity
      \item Decidability of type checking
      \item Consistency
    \end{itemize}
  \item Models of type theory
    \begin{itemize}
      \item Categorical models
      \item Set models
      \item Type theoretic ones (→ translations)
      \item Gödel
    \end{itemize}
  \item Syntax and formalisations of type theory\sidenote{Is it the moment
  to talk about MetaCoq though? It seems best to save it for later\dots
  A forward mention wouldn't hurt.}
    \begin{itemize}
      \item De Bruijn indices
      \item Annotations and paranoia. Contribs\sidenote{These are contributions
      but this part seems more to be state-of-the-art/introduction, however if
      it makes sense, best to put here no?}:
        \begin{itemize}
          \item formal-type-theory
          \item Card model (ref to model section)
        \end{itemize}
      \item Explicit vs implicit substitutions
      \item Well-typed syntax that I did not study in depth but needs to be
      mentioned → maybe expose my idea that it isn't exactly the same,
      notion of computation in the meta (→ translations).
    \end{itemize}
  \item (Syntactical) translations
    \begin{itemize}
      \item Simon's PhD + Next 700
      \item Follow up on other translations using derivations (or in-between
      objects)
      \item Digression on notion of proof here? (ETT, de Bruijn's plea for
      weaker frameworks)
    \end{itemize}
  \item Coq Kernel Specification: MetaCoq
  \item ETT → ITT/WTT\sidenote{It's a bit sad that it doesn't follow directly
  the section on translations but it seems best to talk about MetaCoq
  beforehand.}
    \begin{itemize}
      \item Translation
      \item Plugin
      \item Type-checker (→ transition to Coq Coq Correct\sidenote{As in real
      life})
    \end{itemize}
  \item A verified type-checker for Coq in Coq
\end{itemize}