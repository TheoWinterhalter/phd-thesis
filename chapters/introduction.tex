% \setchapterpreamble[u]{\margintoc}
\chapter{Introduction}
\labch{intro}

\todo{Nicolas: Speak about ``derivation checking''. Where? How?}

Type theory places itself at the interface between programming and formal logic,
feeding off and nourishing both worlds. Proof assistants can be built on it,
making them full-fledged programming languages as well.

My main interest lies in the study of type theory while relying on the tools it
provides: I study type theory \emph{in} type theory.

\paradot{Contributions}
My contributions will mainly be found in \arefpart{elim-reflection} and
\arefpart{coq-in-coq} corresponding to the following published
articles~\sidecite{winterhalter:hal-01849166,sozeau2019coq,sozeau:hal-02167423}.
While the chapters before these two parts are mainly introductory and
corresponding to a rough state-of-the-art, they actually contain other
contributions, namely work I did with Andrej Bauer on the cardinal model
in \nrefch{models} that we didn't publish, and work I did with Andrej Bauer
and Philipp Haselwarter on formalising type theory called
\ftt~\sidecite{formaltypetheory} and that I present briefly in
\nrefch{formalisation}.

\paradot{Outline}
\todo{TODO}