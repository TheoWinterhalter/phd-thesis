\setchapterpreamble[u]{\margintoc}
\chapter{Introduction}
\labch{intro}


\section{Meta}

Nothing organised for now...

\paragraph{Things to put in (high level)}

\begin{itemize}
  \item ETT to ITT
  \item ETT to Weak TT
  \item Coq Coq Correct
\end{itemize}
%
Probably merging the two firsts. Since there is a CPP article for ETT to ITT
it might be worth it to instead write a chapter on ETT to WTT and say we can
derive the other from WTT to ITT.
However, this would not allow me to really talk about the ``plugin''.

\paragraph{Introduction}

\begin{itemize}
  \item Proof theory proof as object(?)
  \item Type theory
  \item Curry-Howard for SLT first?
  \item Gödel?
  \item Why formalise type theory
  \item Dependent types (also for programming?)
\end{itemize}

\paragraph{Outline}

For now only three chapters excluding intro but it feels like there should be
more. Maybe these aren't chapters but parts.
The Card model for instance could be its own chapter.
HoTT should be mentioned to state ETT to ITT/WTT extends to 2TT/HTS.

Maybe a first part on type theory itself?
It feels like maybe it should be part of the introduction though.
There I should also mention models (otherwise it'd be a bit rough to talk
about models to justify syntax choices).

\begin{enumerate}
  \item Formalisation / Representation of Type Theories
    \begin{itemize}
      \item Translations: syntactical or not
      \item Annotations for applications etc (Card model?)
      \item MetaCoq / TemplateCoq?
    \end{itemize}
  \item Computation in Type Theory
    \begin{itemize}
      \item ETT, ITT, WTT
      \item Funext, UIP? (HoTT?)
    \end{itemize}
  \item Kernel of \Coq written in \Coq
    \begin{enumerate}
      \item Mainly the checker
      \item Briefly mention extraction (paper + Yannik's thesis?)
      \item Meta-theory (although it's not my main contrib)
    \end{enumerate}
\end{enumerate}

Regarding the meta-theory, even if I'm not the author of that of MetaCoq/PCUIC
I still have done similar (but simpler) work with ett-to-itt and my
formal-type-theory with Andrej and Philipp.

\paragraph{What can already be written?}

Unclear as of now. There will surely be ETT/ITT/WTT and Coq presentations,
along with a MetaCoq presentation. If they are mixed or separate and what
notions will already be introduced are big unknowns though.

I assumed I would be doing the introductory chapters later on but it's hard not
to rely on the definitions they would introduce.