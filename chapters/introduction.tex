% \setchapterpreamble[u]{\margintoc}
\chapter{Introduction}
\labch{intro}

Type theory is set at the interface between programming and formal logic,
feeding off and nourishing both worlds. Proof assistants can be built on it,
making them full-fledged programming languages as well, with the added benifit
of producing certified programs.

My main interest lies in the study of type theory while relying on the tools it
provides: I study type theory \emph{within} type theory.
As such I focused on the formalisation of type theory in the \Coq proof
assistant~\sidecite[-0.7cm]{coq} and particularly on two points:
\marginnote[0.7cm]{
  Reflection is defined in \refsubsec{ett-def}, and weak equality in
  \refsubsec{wtt}, both in \nrefch{flavours}.
}
\begin{itemize}
  \item How can we effectively turn a proof using a very strong notion of
  equality called \emph{reflection}, into a proof relying on a very weak notion
  of equality?
  \item How can we improve the trust in our system, in my case \Coq, using this
  system itself as a framework to study it?
\end{itemize}

\paradot{Contributions}
My contributions will mainly be found in \arefpart{elim-reflection} and
\arefpart{coq-in-coq} corresponding to the following published
articles~\sidecite{winterhalter:hal-01849166,sozeau2019coq,sozeau:hal-02167423}.
While the chapters before these two parts are mainly introductory and
corresponding to a rough state-of-the-art, they actually contain other
contributions, namely work I did with Andrej Bauer on the cardinal model
in \nrefch{models} that we didn't publish, and work I did with Andrej Bauer
and Philipp Haselwarter on formalising type theory called
\ftt~\sidecite{formaltypetheory} and that I present briefly in
\nrefch{formalisation}.