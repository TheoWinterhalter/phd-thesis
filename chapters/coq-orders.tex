% \setchapterpreamble[u]{\margintoc}
\chapter{Well-founded induction and well-orders}
\labch{coq-orders}

Before delving into the definiton of the type checker which will rely on strong
normalisation, we need to study well-founded induction and orders in more
detail (remember that strong normalisation is defined in terms of
well-foundedness of coreduction).

\section{General setting}

The \emph{classical} definition of a well-founded order basically says that one
cannot \emph{go down} indefinitely in that order.

\begin{definition}[Well-order (classically)]
  An order \(\prec\) is said to be well-founded when there is no infinitely
  decreasing sequence for that order, \ie there is no sequence
  \((u_i)_{i \in \mathbb{N}}\) such that for all \(i\), \(u_{i+1} \prec u_i\).
\end{definition}

When \(\prec\) is a well-order, every non-increasing sequence has to become
stationary at some point.
The prototypical example of this the canonical order on \(\mathbb{N}\): if you
take a natural number and start going down, at some point you have to stop lest
you go below \(0\) and leave the realm of natural numbers.
We say that \(\mathbb{N}\) is a well-founded set in that case.

This is particularly useful to show that a process terminates. If after
completing each task you are left with the completion of smaller tasks,
you will eventually run out of tasks to complete.
This can be used to justify the termination of an \ocaml function like the
ever-famous factorial function.
\marginnote[0.5cm]{
  Even though the \ocaml \mintinline{ocaml}{int} type is not comprised only of
  natural numbers, the \mintinline{ocaml}{n < 1} condition ensures that the
  recursive call only happens when \(n \ge 1\).
}
\begin{minted}{ocaml}
let rec fact n =
  if n < 1
  then 1
  else n * (fact (n - 1))
\end{minted}
The recursive call is always on a smaller natural number and at some point it
will reach some \(n < 1\) and return a value.

This kind of reasoning is the key to proofs of termination in general, but this
presentation is not really suited to a constructive setting like \Coq because of
the negative formulation.

\section{Constructive setting}

In an intuitionistic or even constructive setting, the notion of
well-foundedness is stated in a positive way: instead of saying that there is no
infinitely decreasing sequence, we say that every element---for instance every
natural number---is \emph{reachable} in a finite number of steps from the base
case(s).

\begin{definition}[Accessibility]
  A term \(t\) of type \(A\) is accessible for a relation \(\prec\) on \(A\)
  when all smaller elements for \(\prec\) are also accessible.
\end{definition}
\marginnote[-1.3cm]{
  In particular, a minimal element \(t\)---in the sense that there are no \(u\)
  such that \(u \prec t\)---is always accessible.
}

For instance, to show that \(2\) is accessible for \(<\) we would need to show
that \(0\) and \(1\) are accessible, and for \(1\) we would need to show that
\(0\) is accessible. This leaves us with \(0\). It is accessible because there
are no smaller natural numbers. \(0\) is the base case.

\marginnote[0.1cm]{
  Note that we define well-founded relations and not well-orders, this is more
  general and actually sufficient for us. A well-order is an order that forms a
  well-founded relation.
}
\begin{definition}[Well-founded relation]
  A relation \(\prec\) on \(A\) is well-founded when all the inhabitants of
  \(A\) are accessible for \(\prec\).
\end{definition}

Again, this is the case that all natural numbers are accessible for \(<\)
and this can be shown by induction.
From this we can do well-founded induction: if a property holds for \(t\) when
it holds for smaller elements, it holds for every accessible element.

In \Coq the accessibility predicate is defined as follows:
\begin{minted}{coq}
Inductive Acc (A : Type) (R : A -> A -> Prop) (x : A) : Prop :=
  Acc_intro : (forall y : A, R y x -> Acc R y) -> Acc R x
\end{minted}
This corresponds to a translation of the definition above to the syntax of \Coq.
The induction principle produced by \Coq for \mintinline{coq}{Acc} yields
well-founded induction.
\begin{minted}{coq}
Acc_ind :
  forall (A : Type) (R : A -> A -> Prop) (P : A -> Prop),
    (forall x : A,
      (forall y : A, R y x -> P y) ->
      P x
    ) ->
    forall x : A, Acc R x -> P x
\end{minted}
It says that if for any given \(x : A\) such that all \(y : A\) smaller than
\(x\) are accesible and verify the predicate \(P\), \(x\) also verifies \(P\),
then all accessible elements verify \(P\).

\reminder[-0.7cm]{Fibonacci sequence}{
  The Fibonacci sequence \((\mathcal{F}_n)_{n \in \mathbb{N}}\) is defined by
  \(\mathcal{F}_0 = 0\), \(\mathcal{F}_1 = 1\) and
  \(\mathcal{F}_{n+2} = \mathcal{F}_{n+1} + \mathcal{F}_n\).
}
Let us say this time we want to define the Fibonacci sequence, the following
definition is not accepted by \Coq's termination checker.
\begin{minted}{coq}
Fail Fixpoint fib n :=
  match n with
  | 0 => 1
  | 1 => 1
  | S (S n) => fib (S n) + fib n
  end.
\end{minted}
It does not like the \mintinline{coq}{S n} bit in the recursive call.
I am cheating a bit because it can still be defined by telling \Coq that
\mintinline{coq}{S n} here is indeed a subterm of the matched term.
\begin{minted}{coq}
Fixpoint fib n :=
  match n with
  | 0 => 1
  | 1 => 1
  | S ((S n) as m) => fib m + fib n
  end.
\end{minted}
\marginnote[0.2cm]{
  If you are not convinced by this argument, just read the rest of the thesis
  to see well-founded induction in action.
}
For the sake of the argument I will still write this function using well-founded
induction.
\begin{minted}{coq}
Require Import Lia Arith.

Lemma wf_nat :
  well_founded lt.
Proof.
  intro n. induction n as [| n ih].
  - constructor. intros m h. lia.
  - constructor. intros m h.
    destruct (eq_nat_dec m n).
    + subst. assumption.
    + destruct ih as [ih]. apply ih. lia.
Qed.

Definition fib n :=
  Acc_rec (fun _ => nat)
    (fun n =>
      match n with
      | 0 => fun _ _ => 1
      | 1 => fun _ _ => 1
      | S (S m) =>
        fun acc_lt_n fib_lt_n =>
          fib_lt_n (S m) (Nat.lt_succ_diag_r (S m)) +
          fib_lt_n m
            (Nat.lt_trans _ _ _
              (Nat.lt_succ_diag_r m)
              (Nat.lt_succ_diag_r (S m)))
      end
    )
    (wf_nat n).
\end{minted}
\marginnote[-10.6cm]{
  The first line loads the \mintinline{coq}{lia} tactic which solves arithmetic
  goals.
}
\marginnote[-9.6cm]{
  \mintinline{coq}{well_founded R} is defined as
  \mintinline{coq}{forall x : A, Acc R x} for
  \mintinline{coq}{R : A -> A -> Prop}.
  Also \mintinline{coq}{n < m} is just a notation for
  \mintinline{coq}{lt n m}.
}
The \mintinline{coq}{fib} function we tried to define earlier is barely
recognisable here. Fortunately for us there are tools to write well-founded
inductions in a nicer way. One such tool is \Equations.
\begin{minted}{coq}
From Equations Require Equations.

Equations fib (n : nat) : nat
  by wf n lt :=

  fib 0 := 1 ;
  fib 1 := 1 ;
  fib (S (S n)) := fib (S n) + fib n.
\end{minted}
The syntax differs a bit from vanilla \Coq but should be pretty
self-explanatory. We write the function just the way we want it to be and
specify that we are using well-founded recursion on \mintinline{coq}{n}
using the standard \mintinline{coq}{lt} order with \mintinline{coq}{by wf n lt}.
Then it generates some proof obligations saying that the recursive calls are
indeed made on smaller arguments for \mintinline{coq}{lt}, but as they are
pretty simple, it solves them for us.
Moreover it uses the type-class mechanism to automatically find a proof that
the order is indeed well-founded.
Thanks to \Equations, well-founded recursion is not too painful and pretty
useful.

\section{Lexicographic orders}

I will dedicate this section to a special case of order that will be useful for
the formalisation.

\subsection{Simple lexicographic orders}

Generally speaking, a lexicographic order is the combination of two orders to
yield an order on pairs. The idea is similar to the alphabetical order:
you first compare the first letter of each word, and if they match you
proceed to the second, etc.

\begin{definition}[Lexicographic order]
  Given two types \(A\) and \(B\) and two orders \(\prec_A\) and \(\prec_B\)
  on them, the lexicographic order \(\prec_{\mathsf{lex}}\) on \(A \times B\)
  is defined as
  \[
    (a,b) \prec_{\mathsf{lex}} (a',b')
    \text{ when }
    \left\{
    \begin{array}{ll}
      \text{either} & a \prec_A a' \\
      \text{or} & a = a' \text{ and } b \prec_B b'
    \end{array}
    \right.
  \]
\end{definition}

So, like the alphabetical order, when the two first elements are in the
relation, we will not even look at the rest.
\marginnote[0.2cm]{
  They also remain orders.
}
This is pretty convenient, especially since lexicographic orders preserve
well-foundedness.

\begin{lemma}[Well-founded lexicographic orders]
  If \(\prec_A\) and \(\prec_B\) are both well-founded, then the lexicographic
  order they induce is also well-founded.
\end{lemma}

With lexicographic orders we can define the Ackermann function for instance.
\marginnote[1cm]{
  \mintinline{coq}{lexprod} is the lexicographic order in \Coq as provided by
  \Equations itself.
}
\begin{minted}{coq}
Equations ack (x : nat * nat) : nat
  by wf x (lexprod _ _ lt lt) :=

  ack (0,   n)   := S n ;
  ack (S m, 0)   := ack (m, 1) ;
  ack (S m, S n) := ack (m, ack (S m, n)).
\end{minted}
Once again, \Equations saves the day by providing us with a readable version of
the function, while hiding the well-foundedness proof: the definition is
accepted as-is with no more input required of the user.

In the context of dependent types, these might not prove sufficient, hence the
need for dependent lexicographic orders.

\subsection{Dependent lexicographic orders}

In the same way that simple products \(A \times B\) can be extended to their
dependent counterpart, namely \(\Sigma\)-types: \(\Sigma (x:A). B\),
lexicographic orders have a dependent version.
As before, we take an order \(\prec_A\) on \(A\), but as \(B\) depends on
\(x : A\), we need the second order to also depend on \(x\). Fortunately for us,
the only times we are interested in \(\prec_B\) is when the first two components
are the same (\(a = a'\)), meaning the same type \(B[x \sto a]\).

\begin{definition}[Dependent lexicographic order]
  Given a type \(A\) and a type \(B\) dependent on \(A\), as well as an order
  \(\prec_A\) on \(A\) and an order \(\prec_B\) on \(B\), dependent on \(A\),
  we define their dependent lexicographic order on \(\Sigma (x:A). B\),
  \(\prec_{\mathsf{lex}}\) as
  \[
    \begin{array}{lcl}
      \dpair{a,b} \prec_{\mathsf{lex}} \dpair{a',b'} &\text{when}&
      a \prec_A a' \\
      \dpair{a,b} \prec_{\mathsf{lex}} \dpair{a,b'} &\text{when}&
      b \prec_{B[x \sto a]} b'
    \end{array}
  \]
\end{definition}
\marginnote[-1cm]{
  Notice in the second case that there is no \(a'\), it is twice \(a\).
}

I find the \Coq definition can shed more light than the paper one in this case.
In it the dependencies appear in a more explicit manner so it can lift some
confusion I hope.

\marginnote[1.25cm]{
  \mintinline{coq}{sigT B} has implicit parameter \mintinline{coq}{A} and
  corresponds to the \(\Sigma\)-type \mintinline{coq}{∑ (x : A), B x}.
}
\begin{minted}{coq}
Inductive dlexprod {A} {B : A -> Type}
  (leA : A -> A -> Prop) (leB : forall x, B x -> B x -> Prop)
  : sigT B -> sigT B -> Prop :=

| left_lex :
    forall x x' y y',
      leA x x' ->
      dlexprod leA leB (x;y) (x';y')

| right_lex :
    forall x y y',
      leB x y y' ->
      dlexprod leA leB (x;y) (x;y').
\end{minted}
\marginnote[-1cm]{
  In \Coq I write \mintinline{coq}{(x;y)} for \(\dpair{x,y}\).
}

I use notations for the simple and dependent lexicographic orders:
\begin{minted}{coq}
Notation "R1 ⊗ R2" :=
  (lexprod R1 R2)
  (at level 20, right associativity).

Notation "x ⊩ R1 ⨶ R2" :=
  (dlexprod R1 (fun x => R2))
  (at level 20, right associativity).
\end{minted}

Once again, the dependent lexicographic order of two well-founded orders is also
well-founded. In \Coq I show something even more precise, saying that a pair
\(\dpair{x,y}\) is accessible if \(x\) is accessible and the second order is
well-founded.
\begin{minted}{coq}
Lemma dlexprod_Acc :
  forall A B leA leB,
    (forall x, well_founded (leB x)) ->
    forall x y,
      Acc leA x ->
      Acc (@dlexprod A B leA leB) (x;y).
\end{minted}

\subsection{Lexicographic order modulo a bisimulation}

With usual lexicographic orders, the second order is involved when the first
components are \emph{equal}. In some cases, this requirement is just too strong
and we want to relax it a bit, by accepting them to be equal modulo another
relation. Of course we cannot accept any relation, otherwise we would have no
hope of showing the resulting order well-founded.

Again I will stick to the \Coq definition because I find it better to convey
information.

\begin{minted}{coq}
Inductive dlexmod {A} {B : A -> Type}
    (leA : A -> A -> Prop)
    (eA : A -> A -> Prop)
    (coe : forall x x', eA x x' -> B x -> B x')
    (leB : forall x, B x -> B x -> Prop)
  : sigT B -> sigT B -> Prop :=

| left_dlexmod :
    forall x x' y y',
      leA x x' ->
      dlexmod leA eA coe leB (x;y) (x';y')

| right_dlexmod :
    forall x x' y y' (e : eA x x'),
      leB x' (coe _ _ e y) y' ->
      dlexmod leA eA coe leB (x;y) (x';y').

Notation "x ⊨ e \ R1 'by' coe ⨷ R2" :=
  (dlexmod R1 e coe (fun x => R2))
  (at level 20, right associativity).
\end{minted}

It is very similar to the dependent lexicographic order of the previous section,
only this time we compare the second components when they are related by
\mintinline{coq}{eA}. Because of the dependency however, we have
\mintinline{coq}{y : B x} and \mintinline{coq}{y' : B x'} while
\mintinline{coq}{leB} will want only one of either \mintinline{coq}{x} or
\mintinline{coq}{x'}.
For this reason we need to somehow relate \mintinline{coq}{B x} and
\mintinline{coq}{B x'}; that is the purpose of the \mintinline{coq}{coe}
function we take as argument. It \emph{transports} terms of
\mintinline{coq}{B x} to \mintinline{coq}{B x'}; that way we compare
\mintinline{coq}{coe x x' e y} and \mintinline{coq}{y'} for
\mintinline{coq}{leB}.
The existence of such a coercion function is not automatic, hence the need to
have it explicitly in the notation \mintinline{coq}{x ⊨ e \ R1 by coe ⨷ R2}.

As you probably have guessed, dependent lexicographic order modulo a relation
can also be shown to be well-founded, provided it involves well-orders and that
the coercion function is well-behaved.
Here is the lemma---whose statement is longer than its proof so I will break it
down right after---in \Coq.
\begin{minted}{coq}
Lemma dlexmod_Acc :
  forall A B (leA : A -> A -> Prop) (eA : A -> A -> Prop)
    (coe : forall x x', eA x x' -> B x -> B x')
    (leB : forall x : A, B x -> B x -> Prop)
    (sym : forall x y, eA x y -> eA y x)
    (trans : forall x y z, eA x y -> eA y z -> eA x z),
    (forall x, well_founded (leB x)) ->
    (forall x x' y, eA x x' -> leA y x' -> leA y x) ->
    (forall x,
      exists e : eA x x, forall y, coe _ _ e y = y
    ) ->
    (forall x x' y e,
      coe x x' (sym _ _ e) (coe _ _ e y) = y
    ) ->
    (forall x0 x1 x2 e1 e2 y,
      coe _ _ (trans x0 x1 x2 e1 e2) y =
      coe _ _ e2 (coe _ _ e1 y)
    ) ->
    (forall x x' e y y',
      leB _ y (coe x x' e y') ->
      leB _ (coe _ _ (sym _ _ e) y) y'
    ) ->
    forall x y,
      Acc leA x ->
      Acc (@dlexmod A B leA eA coe leB) (x ; y).
\end{minted}

Admittedly this formulation is not really pretty, and comes as the minimal
subset of requirement I found with which I could still prove well-foundedness.
The first thing to notice, is the addition of two operators on the relation
\mintinline{coq}{eA}: I ask that it is symmetric and transitive as witnessed
by \mintinline{coq}{sym} and by \mintinline{coq}{trans}.
Then I list several properties that I require, I shall list them one by one:
\begin{enumerate}[label=(\roman*)]
  \item \mintinline{coq}{leB} has to be well-founded;
  \item \label{item:dlexmod-sim} objects in relation for \mintinline{coq}{eA}
  can substituted in \mintinline{coq}{leA} on the right;
  \item \mintinline{coq}{eA} should be reflexive and coercing using the proof
  of reflexivity is the identity;
  \item \mintinline{coq}{sym _ _ e} and \mintinline{coq}{e} cancel each other
  out for \mintinline{coq}{coe};
  \item coercing a transitivity is the same as coercing twice;
  \item if you have a coercion of the right of \mintinline{coq}{leB} you can
  instead put its symmetric on the left.
\end{enumerate}
\marginnote[-2cm]{
  There is probably a hidden notion here like an order isomorphism but this
  would require a more thorough investigation.
}

All these should feel natural if you have equality in mind.
I did mention bisimulations however and did not even talk about it.
Let me now address this point.

\begin{definition}[Simulation]
  Given a relation \(\red\) on \(A\), a relation \(\sim\) on \(A\) is called a
  simulation if whenever \(p \red q\) and \(p \sim p'\) there exists \(q'\)
  such that \(q \sim q'\) and \(p' \red q'\).
\end{definition}

This is summarised by the following diagram.
\begin{center}
  \begin{tikzpicture}[baseline=(p.base), node distance=2cm]
    \node (p) { \(p\) } ;
    \node (p') [right of = p] { \(p'\) } ;
    \node (q) [below of = p] { \(q\) } ;
    \node (q') [below of = p'] { \(\exists q'\) } ;
    \draw[tred] (p) to (q) ;
    \draw[tred, dashed] (p') to (q') ;
    \draw[sim] (p) -- (p') ;
    \draw[sim, dashed] (q) -- (q') ;
  \end{tikzpicture}
\end{center}

\begin{definition}[Bisimulation]
  A relation \(\sim\) on \(A\) is a bisimulation for \(\red\) when \(\sim\)
  and \(\sim^{-1}\) are both simulations for \(\red\).
\end{definition}

A bisimulation does not quite fit the bill yet though. The simulation property
does look like \ref{item:dlexmod-sim} but it would conclude
\mintinline{coq}{leA y' x} for some \mintinline{coq}{eA y y'} instead of plain
\mintinline{coq}{leA y x}. Also, there is no reason for a bisimulation to
be an equivalence relation and even less for the existence of a well-behaved
coercion function for the second order.

If we go back to the simple---\ie not dependent---case for a bit, a bisimulation
that is also an equivalence relation is sufficient.

\begin{minted}{coq}
Inductive lexmod {A B}
    (leA : A -> A -> Prop)
    (eA : A -> A -> Prop)
    (leB : B -> B -> Prop)
  : A * B -> A * B -> Prop :=

| left_lexmod :
    forall x x' y y',
      leA x x' ->
      lexmod leA eA leB (x,y) (x',y')

| right_lexmod :
    forall x x' y y',
      eA x x' ->
      leB y y' ->
      lexmod leA eA leB (x,y) (x',y').

Notation "e \ R1 ⊕ R2" :=
  (lexmod R1 e R2)
  (at level 20, right associativity).

Lemma lexmod_Acc :
  forall A B (leA : A -> A -> Prop) (eA : A -> A -> Prop)
    (leB : B -> B -> Prop),
    well_founded leB ->
    (forall x x' y,
      eA x x' ->
      leA y x' ->
      exists y',
        leA y' x *
        eA y' y
    ) ->
    (forall a, eA a a) ->
    (forall a b c, eA a c -> eA b c -> eA a b) ->
    forall x y,
      Acc leA x ->
      Acc (eA \ leA ⊕ leB) (x, y).
\end{minted}

Regarding the requirement of the simulation to also be an equivalence relation,
we can actually use the notion of bisimilarity, a special kind of bisimulation.

\begin{definition}[Bisimilarity]
  Given a relation \(\red\) on type \(A\), two terms \(p\) and \(q\) in \(A\)
  are called \emph{bisimilar} when there exist a bisimulation \(\sim\) for
  \(\red\) such that \(p \sim q\).
\end{definition}

The bisimilarity relation is itself an equivalence relation meaning we can use
it to quotient simple lexicographic orders.

\marginnote[0.2cm]{
  In the simple case, the coercion function is always the indentity, so all
  its properties follow trivially.
}
Again, in the dependent case this is not enough but I still kept the name
because it is roughly a generalisation of the simple case.