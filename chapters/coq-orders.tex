% \setchapterpreamble[u]{\margintoc}
\chapter{Well-founded induction and well-orders}
\labch{coq-orders}

This chapter is perhaps a bit off-track compared to the others in this chapter
as it is a bit general but I have it here because I am going to rely on it in
the next chapters.

\section{General setting}

The \emph{classical} definition of a well-founded order basically says that one
cannot \emph{go down} indefinitely in that order.

\begin{definition}[Well-order (classically)]
  An order \(\prec\) is said to be well-founded when there is not infinitely
  decreasing sequence for that order, \ie there is no sequence
  \((u_i)_{i \in \mathbb{N}}\) such that for all \(i\), \(u_{i+1} \prec u_i\).
\end{definition}

When \(\prec\) is a well-order, every non-increasing sequence has to become
stationary at some point.
The prototypical example of this the canonical order on \(\mathbb{N}\): if you
take a natural number and start going down, at some point you have to stop lest
you go below \(0\) and leave the realm of natural numbers.
We say that \(\mathbb{N}\) is a well-founded set in that case.

This is particularly useful to show that a process terminates. If after
completing each task you are left with the completion of smaller tasks,
you will eventually run out of tasks to complete.
This can be used to justify the termination of an \ocaml function like the
ever-famous factorial function.
\marginnote[0.5cm]{
  Even though the \ocaml \mintinline{ocaml}{int} type is not comprised only of
  natural numbers, the \mintinline{ocaml}{n < 1} condition ensures that the
  recursive call only happens when \(n \ge 1\).
}
\begin{minted}{ocaml}
let rec fact n =
  if n < 1
  then 1
  else n * (fact (n - 1))
\end{minted}
The recursive call is always on a smaller natural number and at some point it
will reach some \(n < 1\) and return a value.

This kind of reasoning is the key to proofs of termination in general, but this
presentation is not really suited to a constructive setting like \Coq because of
the negative formulation.

\section{Constructive setting}