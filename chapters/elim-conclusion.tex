% \setchapterpreamble[u]{\margintoc}
\chapter{Limitations and related works}
\labch{elim-conclusion}

\section{Limitations and Axioms}
\label{sec:axioms}

\marginnote[0.2cm]{
  I would like to use a minimal representation of derivations that does not contain
  everything that can be inferred automatically for one.
}
Currently, the representation of terms and derivations and the
computational content of the proof only allow us to deal with the
translation of relatively small terms but I hope to improve that in
the future. As we have seen, the actual translation involves the
computational content of lemmata of inversion, substitution, weakening
and equational reasoning and thus cannot be presented as a simple
recursive definition on derivations.


As I already mentioned, the axioms K and \acrshort{funext} are both
necessary in \acrshort{ITT} if we want the translation to be conservative as
they are provable in \acrshort{ETT}~\sidecite{hofmann1995conservativity}.
However, one might still be concerned about having axioms
as they can for instance hinder canonicity of the system.
In that respect, K is not really a restriction since it preserves canonicity.
The best proof of that is probably \Agda itself which natively features K---in
fact, one needs to explicitly deactivate it with a flag if they wish to work
without.

The case of \acrshort{funext} is trickier. It should be possible to realise
the axiom by composing our translation with a setoid
interpretation~\sidecite{altenkirch99} which validates it, or by going into a
system featuring it, for instance by implementing
\acrlong{OTT}~\sidecite{altenkirch2007observational} like
\Epigram~\sidecite[1cm]{mcbride2004epigram}.

However, these two axioms are not used to define the translation itself,
but only to witness \acrshort{UIP} and \acrlong{funext} in the translation to
\Coq.
The translation itself only relies on one axiom, called
\mintinline{coq}|conv_trans_AXIOM|, stating that conversion
of \acrshort{ITT} is transitive.
The translation to \acrshort{WTT} is totally axiom-free however!

The proof of transivity in \acrshort{ITT} basically sums up to the confluence of
the reduction rules which is out of scope for this thesis and has
recently been formalised in Agda~\sidecite{Abel:2017:DCT:3177123.3158111} (in a
simpler setting with only one universe).
\MetaCoq also features a proof of confluence for
\Coq~\sidecite[0.7cm]{sozeau2019coq}.
Regardless, this axiom inhabits a proposition (the type of conversion is in
\mintinline{coq}|Prop|) and is thus irrelevant for computation. Actually no
information about the derivation leaks to the production of the \acrshort{ITT}
term.

On a different note, the \mintinline{coq}|candidate| axiom allows us to derive
\mintinline{coq}|False| but is merely used to write ill-typed terms in \Coq.
The translated term will never make us of it and one can always check if a
term is relying on unsafe assumptions thanks to the
\mintinline{coq}|Print Assumptions| command.

\section{Related Works}
\label{sec:related-works}

The seminal works on the precise connection between \acrshort{ETT} and
\acrshort{ITT} go back to \sidecite{streicher1993investigations} and
\sidecite[0.7cm]{hofmann1995conservativity,HofmannPhD}.
%
In particular, the work of Hofmann provides a categorical answer to
the question of consistency and conservativity of \acrshort{ETT} over
\acrshort{ITT} with \acrshort{UIP} and \acrshort{funext}.
%
Ten years later, Oury~\sidecite[1.6cm]{oury2005extensionality,Oury2006}
provided a translation from \acrshort{ETT} to \acrshort{ITT} with
\acrshort{UIP} and \acrshort{funext} and other axioms (mainly due to
technical difficulties).
%
Although a first step towards a move from categorical semantics to a
syntactic translation, his work does not stress any constructive
aspect of the proof and shows that there merely exist translations in
\acrshort{ITT} of a typed term in \acrshort{ETT}.

Door \emph{et al.}~\sidecite[0.5cm]{van2013explicit} have later proposed and
formalised a similar translation between a \acrshort{PTS} with and without
explicit conversion. This does not entail anything about \acrshort{ETT} to
\acrshort{ITT} but we can find similarities in that there is a witness of
conversion between any term and itself under an explicit conversion, which
internalises irrelevance of explicit conversions. This morally corresponds to a
\emph{Uniqueness of Conversions} principle.

The \name{Program} \sidecite{sozeau:icfp07} extension of \Coq performs a
related coercion insertion algorithm, between objects in subsets on the
same carrier or in different instances of the same inductive family,
assuming a proof-irrelevance axiom. Inserting coercions locally is not
as general as the present translation from \acrshort{ETT} to \acrshort{ITT}
which can insert transports in any context.


In both formalisations, I implemented type-checkers for the theories involved.
In \acrshort{WTT}, since the meta-theory is so simple, I was able to prove the
type-checker to be sound and complete. I do not detail them in this thesis as
I will dedicate \arefpart{coq-in-coq} to the implemtation and verification of
a full-fledged type-checker of \Coq in \Coq.