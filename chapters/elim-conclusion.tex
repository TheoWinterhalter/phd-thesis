% \setchapterpreamble[u]{\margintoc}
\chapter{Conclusions regarding elimination of reflection}
\labch{elim-conclusion}

\todo{Copied}

\section{Limitations and Axioms}
\label{sec:axioms}

Currently, the representation of terms and derivations and the
computational content of the proof only allow us to deal with the
translation of relatively small terms but we hope to improve that in
the future. As we have seen, the actual translation involves the
computational content of lemmata of inversion, substitution, weakening
and equational reasoning and thus cannot be presented as a simple
recursive definition on derivations.


As we already mentioned, the axioms K and FunExt are both
necessary in ITT if we want the translation to be conservative as they are
provable in ETT~\sidecite{hofmann1995conservativity}.
However, one might still be concerned about having axioms
as they can for instance hinder canonicity of the system.
In that respect, K isn't really a restriction since it preserves canonicity.
The best proof of that is probably \Agda itself which natively features K---in
fact, one needs to explicitly deactivate it with a flag if they wish to work
without.

The case of FunExt is trickier. It should be possible to realise
the axiom by composing our translation with a setoid
interpretation~\sidecite{altenkirch99} which validates it, or by going into a
system featuring it, for instance by implementing Observational Type
Theory~\sidecite{altenkirch2007observational} like
\Epigram~\sidecite{mcbride2004epigram}.

However, these two axioms are not used to define the translation itself,
but only to witness UIP and function extensionality in the translation to
\Coq.
The translation only relies on one axiom, called
\mintinline{coq}|conv_trans_AXIOM| in the formalisation, stating that conversion
of ITT is transitive.
%
The proof of this property basically sums up to the confluence of the
reduction rules of ITT which is out of scope for this paper and has
recently been formalised in Agda~\sidecite{Abel:2017:DCT:3177123.3158111} (in a
simpler setting with only one universe).
Regardless, this axiom inhabits a proposition (the type of conversion is in
\mintinline{coq}|Prop|) and is thus irrelevant for computation. Actually no
information about the derivation leaks to the production of the ITT term.

On a different note, the \mintinline{coq}|candidate| axiom allows us to derive
\mintinline{coq}|False| but is merely used to write ill-typed terms in \Coq.
The translated term will never make us of it and one can always check if a
term is relying on unsafe assumptions thanks to the
\mintinline{coq}|Print Assumptions| command.

\section{Related Works and Conclusion}
\label{sec:related-works}

The seminal works on the precise connection between ETT and ITT go
back to \sidecite{streicher1993investigations} and
\sidecite{hofmann1995conservativity,HofmannPhD}.
%
In particular, the work of Hofmann provides a categorical answer to
the question of consistency and conservativity of ETT over ITT with
UIP and FunExt.
%
Ten years later, \sidecite{oury2005extensionality,Oury2006} provided
a translation from ETT to ITT with
UIP and FunExt and other axioms (mainly due to
technical difficulties).
%
Although a first step towards a move from categorical semantics to a
syntactic translation, his work does not stress any constructive
aspect of the proof and shows that there merely exist translations in
ITT to a typed term in ETT.

\sidecite{van2013explicit} have later proposed and
formalised a similar translation between a PTS with and without explicit
conversion. This does not entail anything about ETT to ITT but we can
find similarities in that there is a witness of conversion between any
term and itself under an explicit conversion, which internalises
irrelevance of explicit conversions. This morally corresponds to a
Uniqueness of Conversions principle.

The Program \sidecite{sozeau:icfp07} extension of \Coq performs a
related coercion insertion algorithm, between objects in subsets on the
same carrier or in different instances of the same inductive family,
assuming a proof-irrelevance axiom. Inserting coercions locally is not
as general as the present translation from ETT to ITT which can insert
transports in any context.

In this paper we provide the first effective translation from ETT to ITT
with UIP and FunExt. The translation has been
formalised in \Coq using \TemplateCoq, a meta-programming plugin of
\Coq. This translation is also effective in the sense that we can
produce in the end a \Coq term using the \TemplateCoq denotation
machinery.
%
With ongoing work to extend the translation to the inductive fragment
of \Coq, we are paving the way to an extensional version of the \Coq
proof assistant which could be translated back to its intensional
version, allowing the user to navigate between the two modes, and in
the end produce a proof term checkable in the intensional fragment.

% \begin{acks}
  We would like to thank Andrej Bauer and Philipp Haselwarter with whom we had
  fruitful discussions on the subject, prior to this work.
  We also would like to thank the attendees of the Aarhus EUTypes 2018 meeting
  for their insightful feedback on the plugin stemming from the translation.
% \end{acks}