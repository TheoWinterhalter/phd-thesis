% \setchapterpreamble[u]{\margintoc}
\chapter{Desirable properties of type theories}
\labch{desirable-props}

To compare different type theories there are several measures we can use in
form of usual or desirable properties that they migh satisfy or not.
After a brief presentation of the main ones I will summarise which of the
theories of \vrefch{flavours} has which properties in a table.

\section{Properties}

\subsection{Weakening and substitutivity}

Variables and binders are essential to type theory and as such we have to treat
them with care, in particular we want our theories to be \emph{compositional}
meaning that different blocks that make sense can be assembled into something
that still makes sense.
This is embodied in the two following properties.

\begin{definition}[Weakening]
  A type theory enjoys weakening when for any \(\Ga, \Xi \vdash t : A\) and
  \(\vdash \Ga, \Delta\) we have \(\Ga, \Delta, \Xi \vdash t : A\).
\end{definition}
\marginnote[-1.2cm]{
  Note that in more generality you might have to rename variables when
  weakening, so for this we usually introduce a \emph{lifting} operator
  \(\lift{}{}\) such that we have
  \(\Ga, \Delta, \Xi \vdash \lift{n}{k}\ t : \lift{n}{k}\ A\)
  where \(n = |\Delta|\) and \(k = |\Xi|\).
}

Weakening means that you can plug a term into a larger context.

Susbstitutions are the way to instantiate the variables that are bound.
This happens for instance after a \(\beta\)-reduction.
\reminder[-0.9cm]{\(\beta\)-reduction}{
  \[
    (\lambda (x:A). t)\ u \red_\beta t[x \sto u]
  \]
}
Substitutions are typed using two contexts: \(\sigma : \Ga \to \D\) basically
states that \(\sigma\) maps variables of \(\D\) to terms typed in \(\Ga\).
\begin{mathpar}
  \infer
    {\forall (x : A) \in \D,\ \Ga \vdash \sigma(x) : A\sigma}
    {\sigma : \Ga \to \D}
\end{mathpar}
This is sometimes written \(\Ga \vdash \sigma : \D\) instead, and typing
definitons vary a little depending on the definition of substitution but this
is the basic idea.

\begin{definition}[Substitutivity]
  A type theory is substitutive when for any \(\D \vdash t : A\)
  and any substitution \(\sigma : \Ga \to \D\), we have
  \(\Ga \vdash t\sigma : A\sigma\).
\end{definition}

More often than not, weakening and substitutivity also hold for reduction and
conversion.

\subsection{Inversion of typing}

Inversion of typing is not really a measure as it is always present in some way.
It is nonetheless a very useful property to state when reasoning on a type
theory.
It is saying than when we have \(\Ga \vdash t : A\), by analysing the shape of
\(t\) we can get information on \(A\) (and sometimes even \(\Ga\)).

\reminder[-0.6cm]{Application rule}{
  \[
    \infer
      {
        \Ga \vdash \Pi (x:A).\ B : s \\
        \Ga \vdash t : \Pi (x:A).\ B \\
        \Ga \vdash u : A
      }
      {\Ga \vdash t\ u : B[x \sto u]}
  \]
}
For instance, if we have \(\Ga \vdash t\ u : T\), then by inversion of typing
we know that there must exist \(A\) and \(B\) such that
\begin{mathpar}
  \Ga \vdash \Pi (x:A).\ B : s

  \Ga \vdash t : \Pi (x:A).\ B

  \Ga \vdash u : A

  \Ga \vdash T \equiv B[x \sto u]
\end{mathpar}

This can be proved by seeing that only two typing rules can be concluded with
\(\Ga \vdash t\ u : T\): the application rule and the conversion rule. The
result follows from a simple induction.

Now, it won't always be stated this way depending on the premises of the
application rule, but also depending on the presence or not of a conversion
rule. In the case of \acrshort{WTT} for instance, there is no conversion so
instead of \(\Ga \vdash T \equiv B[x \sto u]\) we will have syntactic equality
\(T =_{\alpha} B[x \sto u]\).

You usually prove inversion of typing for every term constructor, but I won't
do it here.

\subsection{Validity}

The term \emph{validity} might be a bit overloaded, and maybe not the norm
when it comes to type theory, but I will use it to be consistent with the
notion for \Coq.
This property states that the type on the right-hand side of the colon is indeed
a type.

\begin{definition}[Validity]
  A type theory enjoys validity when from \(\Ga \vdash t : A\) one can deduce
  \(\Ga \vdash A\) (\ie \(\Ga \vdash A : s\) for some sort \(s\)).
\end{definition}

Depending on the theory, we can often prove a similar property regarding
contexts: namely that \(\Ga \vdash t : A\) implies that \(\Ga\) is well-formed
(\(\vdash \Ga\)). Having this property mainly depends on whether the typing
rules of things like sorts and variables ask for the context to be well-formed.
%
\begin{mathpar}
  \infer
    {(x : A) \in \Ga}
    {\Ga \vdash x : A}
  %

  \text{vs}

  \infer
    {
      \vdash \Ga \\
      (x : A) \in \Ga
    }
    {\Ga \vdash x : A}
  %
\end{mathpar}
%
In a theory which does not have this requirement / property, many lemmata will
only apply assuming the contexts involved are well-formed.
The difference of presentation like this will be studied in
\nrefch{formalisation}.

\subsection{Unique / principal typing}

\subsection{Properties of reduction}
\todo{Confluence, termination}

\subsection{Canonicity}
\todo{With reduction or weaker with conversion (for ETT)}

\subsection{Decidability of type checking}

\subsection{Consistency}

\section{Summarising table}

\todo{Fill}
\begin{table*}
  \rowcolors{1}{SkyBlue!10!White}{}
  \caption[Properties of type theories]{Properties of type theories.}
  \begin{tabular}{l|c|c|c|c|c|c|c|c|c|}
    \cline{2-10}
    & \acrshort{WTT} & \acrshort{ITT} & \acrshort{ETT} & \acrshort{MLTT}
    & \acrshort{PCUIC} & ?HoTT & ?CubicalTT & ?2TT & ?HTS \\
    \hline
    \multicolumn{1}{ |c|  }{Weakening / substitutivity} &
    \multicolumn{9}{c|}{Yes} \\
    \hline
    \multicolumn{1}{ |c|  }{Inversion of typing} &
    \multicolumn{9}{c|}{Yes} \\
    \hline
    \multicolumn{1}{ |c|  }{Unique (U) / Principal type (P)} &
    U & U/P & No? & U & P & ? & ? & ? & ? \\
    \hline
  \end{tabular}
\end{table*}