% \setchapterpreamble[u]{\margintoc}
\chapter{How to read this thesis}
\labch{how-to-read}

I believe that a document of this size will more often viewed on computer than
on paper. This how I personally view most papers I read anyway.
As such I believe that the document should be adapted for such a medium, unlike
most \LaTeX{} documents.
While trying to make my own kind of document I stumbled upon the great
\href{https://github.com/fmarotta/kaobook/}{kaobook} class which corresponds
almost exactly to what I was looking for: citations, notes, reminders can be put
in a large margin.

Citations like~\sidecite{boulier17:next-syntac-model-type-theor} will go in the
margin, as well as in the \refbib.
The margin also contain sidenotes\sidenote{
  No more going back and forth between the end and top of the page.
}, replacing the usual footnotes.

I will also use margin notes that are not anchored in the main document but are
usually relevant to the paragraph or figure on their left.
\marginnote[-0.4cm]{
  They are not placed entirely automatically, I tend to adjust the height so it
  is at the right place.
}

\begin{theorem}
  Theorems and definitions will be in these yellow boxes.
\end{theorem}

\reminder[-0.7cm]{of something}{
  Body of the reminder.
}
I will also use the margin to set up reminders in green boxes. Their purpose
is to avoid the reader having to go back to a previous chapter they might not
have even read and still grasp the meaning of the statements at hand.

\sidedef[-0.7cm]{of something else}{
  This is supposedly a definition
}
On the other hand, I will sometimes refer to concepts for the first time without
taking the time to introduce them because they are secondary to the main point.
In some cases those will be accompagnied by a short definition on the side, in
a yellowish box.