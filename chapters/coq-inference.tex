% \setchapterpreamble[u]{\margintoc}
\chapter{Type inference and checking}
\labch{coq-inference}

Now that we have our conversion algorithm, we can move on to type-checking.
This work is not mine but that of Simon Boulier so I will go over it briefly.
It needs to be mentioned as it is the goal of both the weak head reduction
and the conversion algorithms.

\paradot{Inference versus checking}
In \Coq, without existential variables, type inference is decidable. That is,
given a term of \acrshort{PCUIC} in a well-formed environment \(\Ga\), there is
an algorithm that decides whether there exists a type \(A\) such that
\[
  \Ga \vdash t : A
\]
\marginnote[-0.3cm]{
  This implies that the inference algorithm should not return any type, but the
  principal type of \(t\).
}%
Checking that \(t\) has type \(B\) is then implemented by inferring the type of
\(A\) and verifying that it is a subtype of \(B\).

The good news is that this time, termination is structural so it does not require
extra work like we had for reduction and conversion.
Inference looks like this
\begin{minted}{coq}
Program Fixpoint infer
  (Γ : context) (HΓ : ∥ wf_local Σ Γ ∥) (t : term) {struct t}
  : typing_result ({ A : term & ∥ Σ ;;; Γ |- t : A ∥ }) :=
  match t with
  | tRel n =>
    match nth_error Γ n with
    | Some c => ret ((lift0 (S n)) (decl_type c) ; _)
    | None   => raise (UnboundRel n)
    end

  | tSort u =>
    match Universe.get_is_level u with
    | Some l =>
      check_eq_true
        (LevelSet.mem l (global_ext_levels Σ))
        (Msg ("undeclared level " ++ string_of_level l)) ;;
      ret (tSort (Universe.super l) ; _)
    | None  =>
      raise (Msg (string_of_sort u ++ " is not a level"))
    end

  | tProd na A B =>
    s1 <- infer_type infer Γ HΓ A ;;
    s2 <- infer_type infer (Γ ,, vass na A) _ B ;;
    ret (tSort (Universe.sort_of_product s1.π1 s2.π1) ; _)

  | tLambda na A t =>
    s1 <- infer_type infer Γ HΓ A ;;
    B <- infer (Γ ,, vass na A) _ t ;;
    ret (tProd na A B.π1 ; _)

  | tLetIn n b b_ty b' =>
    infer_type infer Γ HΓ b_ty ;;
    infer_cumul infer Γ HΓ b b_ty _ ;;
    b'_ty <- infer (Γ ,, vdef n b b_ty) _ b' ;;
    ret (tLetIn n b b_ty b'_ty.π1 ; _)

  | tApp t u =>
    ty <- infer Γ HΓ t ;;
    pi <- reduce_to_prod Γ ty.π1 _ ;;
    infer_cumul infer Γ HΓ u pi.π2.π1 _ ;;
    ret (subst10 u pi.π2.π2.π1 ; _)

  | tConst cst u =>
    match lookup_env (fst Σ) cst with
    | Some (ConstantDecl d) =>
      check_consistent_instance d.(cst_universes) u ;;
      let ty := subst_instance_constr u d.(cst_type) in
      ret (ty ; _)
    |  _ => raise (UndeclaredConstant cst)
    end

  (* ... *)

  end.
\end{minted}

We use \Program this time to allow us to solve some parts using tactics as,
once again, the function is defined to be correct by construction.
The function is written in monadic style, using the following monad
\begin{minted}{coq}
Inductive typing_result (A : Type) :=
| Checked (a : A)
| TypeError (t : type_error).
\end{minted}
which corresponds to either returning a type or an error message.

We can then decude checking
\begin{minted}{coq}
Program Definition infer_cumul Γ HΓ t A (hA : wellformed Σ Γ A)
  : typing_result (∥ Σ ;;; Γ |- t : A ∥) :=
  A' <- infer Γ HΓ t ;;
  X <- convert_leq Γ A'.π1 A _ _ ;;
  ret _.
\end{minted}

I do not expect you to read the definition in detail. For a variable it checks if
it is bound, and if so returns the type it is assigned in the environment,
otherwise it raises an error.
For a sort, it checks if is algebraic, if not and it is a declared level,
it returns the successor level.
For a \(\Pi\)-type if infers the types of the domain and codomain, checking that
they are indeed sorts, and returns the sort of the product (\ie the maximum
of the two sorts if the second one is not \(\Prop\), and \(\Prop\) otherwise).
In the case of application we use \mintinline{coq}{infer_cumul} to \emph{check}
rather than infer the type of the argument. The checking function is defined
afterwards so this is another version of it (as you can see it takes
\mintinline{coq}{infer} as argument).

All this is as one should expect, and concludes our quest of a type-checker
defined in \Coq and proven correct while relying only on a few axioms stating
that the theory is well behaved.