% \setchapterpreamble[u]{\margintoc}
\chapter{Reduction}
\labch{coq-reduction}

A key element to defining a type checker is to have a conversion checker.
Conversion in turn relies on reduction. The reduction I describe in
\nrefch{coq-spec} is just a relation and is non-deterministic. In this chapter
I will present an algorithmic implementation of reduction.
This implementation relies on the axiom of strong normalisation, but even then
it requires some more work to show the process is terminating.

\section{Weak head normalisation}

Since reduction is non deterministic, we have to pick a strategy to implement.
When checking terms for conversion, one thing we want to do is verify that they
have the same head constructor: if two terms are \(\lambda\)-abstractions
\(\lambda (x:A).t\) and \(\lambda (x:A').t'\) we need only compare \(A\) and
\(A'\), as well as \(t\) and \(t'\) for conversion.
Weak head reduction is a strategy that allows us to access the head of a term
rather efficiently.

The idea is to only deal with redexes (\ie left-hand side of reduction rules)
that appear at the top-level and might be hiding the head constructor. As such
when considering \(\lambda (x:A).t\), neither \(A\) nor \(t\) will be reduced
because we already know the head, or the shape, of the term: it is a
\(\lambda\). However, if we have an application \(t\ u\), we will first weak
head reduce \(t\) to see if it reaches a \(\lambda\): if it does we substitute
\(u\) and repeat the process, if it does not, then we have reached a weak head
normal form.

Weak head normal forms are by definition the terms that cannot be reduced
further by weak head reduction.

\begin{definition}[Normal form]
  A term \(t\) is a normal form for a reduction \(\red\) if it doesn't reduce
  for \(\red\).
  \[
    t \not \red
  \]
\end{definition}

It is however possible to characterise weak
head normal forms syntactically. For this we need to talk about \emph{neutral}
forms.

\begin{definition}[Neutral forms]
  A term \(t\) is neutral for a reduction \(\red\) if substituting it inside
  another term doesn't introduce redexes. Equivalently substituting a neutral
  term \(t\) in a normal form \(u\) will yield another normal form.
  \[
    u[x \sto t] \not \red
  \]
\end{definition}

\todo{Syntactic characterisation, on simple case and/or code version?}