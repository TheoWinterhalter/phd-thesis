% \setchapterpreamble[u]{\margintoc}
\chapter{What I mean by elimination of reflection}
\labch{elim-reflection-intro}

We presented earlier \acrshort{ETT} and its defining reflection rule.
%
\reminder[-1.0cm]{Reflection rule}{
  \begin{equation*}
    \label{eq:reflection}
    \infer[]
      {\xisterm{\Ga}{e}{\Eq{A}{u}{v}}}
      {\xeqterm{\Ga}{u}{v}{A}}
    %
  \end{equation*}
}
%
The next few chapters are going to be dedicated to its elimination from type
theory: that is how to make a translation from \acrshort{ETT} to type theories
that do not feature reflection.

This work gave rise to a publication~\sidecite{winterhalter:hal-01849166} that
focused on translating \acrshort{ETT} to \acrshort{ITT}.
However, with Simon Boulier we worked on another version translating directly to
\acrshort{WTT} which I'm going to present here.

\section{No syntactical translation}

\todo{Have reminders clickable to refer to actual definition.}
\todo{Use the notation table to include the notation of typing etc. especially
the ETT mark.}

First of all we have to wonder about what kind of translation is possible.
We presented earlier~\misref{} the notion of syntactical translation.
Unfortunately it is not possible to devise a syntactical translation to
eliminate reflection from type theory.

Assume we have such a translation given by \(\transl{.}\) and \([.]\) such that
whenever \(\xisterm{\Gamma}{t}{A}\) we have
\(\isterm{\transl{\Gamma}}{[t]}{\transl{A}}\) in the target type theory.