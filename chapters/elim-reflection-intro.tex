% \setchapterpreamble[u]{\margintoc}
\chapter{What I mean by elimination of reflection}
\labch{elim-reflection-intro}

We presented earlier \acrshort{ETT} and its defining reflection rule.
%
\reminder[-1.0cm]{Reflection rule}{
  \begin{equation*}
    \infer[]
      {\xisterm{\Ga}{e}{\Eq{A}{u}{v}}}
      {\xeqterm{\Ga}{u}{v}{A}}
    %
  \end{equation*}
  See \refdef{reflection}
}
%
The next few chapters are going to be dedicated to its elimination from type
theory: that is how to make a translation from \acrshort{ETT} to type theories
that do not feature reflection.

This work gave rise to a publication~\sidecite[0.2cm]{winterhalter:hal-01849166} that
focused on translating \acrshort{ETT} to \acrshort{ITT}.
However, with Simon Boulier we worked on another version translating directly to
\acrshort{WTT} which I'm going to present here.

\section{Nature of the translation}

\subsection{Syntactical translations are not possible}

First of all we have to wonder about what kind of translation is possible.
We presented earlier~\misref{} the notion of syntactical translation.
Unfortunately it is not possible to devise a syntactical translation to
eliminate reflection from type theory.

Assume we have such a translation given by \(\transl{.}\) and \([.]\) such that
whenever \(\xisterm{\Gamma}{t}{A}\) we have
\(\isterm{\transl{\Gamma}}{[t]}{\transl{A}}\) in the target type theory.
Now let's say we have an inconsistent context in \acrshort{ETT}: \(\Gamma_\bot\)
(one can for instance assume \(0 = 1\) or \(\forall A, A\)), in such a context
anything can have any type because conversion has become trivial.
%
\marginnote[1.4cm]{
  Since \(\Gamma_\bot\) is inconsistent, anything can be proved from it,
  including the equality between the two types \(\mathbb{N}\) and \(\bot\).
  We then use reflection and conversion.
}
\begin{mathpar}
  \infer
    {
      \infer
        {\vdots}
        {\xisterm{\Gamma_\bot}{0}{\mathbb{N}}}
      \\
      \infer
        {
          \infer
            {\vdots}
            {\xisterm{\Gamma_\bot}{\_}{\mathbb{N} = \bot}}
          %
        }
        {\xeqterm{\Gamma_\bot}{\mathbb{N}}{\bot}{\Type}}
    }
    {\xisterm{\Gamma_\bot}{0}{\bot}}
  %
\end{mathpar}
%
It thus follows that in the target you have
\( \isterm{\transl{\Gamma_\bot}}{[0]}{\transl{\bot}} \).
Similarly you would have
\( \isterm{\transl{\Gamma_\bot}}{[0]}{\transl{\mathbb{N}}} \). This means
that both \(\bot\) and \(\mathbb{N}\) should be translated to similar things
(to convertible types in case the target theory has uniqueness of type), without
being able to exploit the knowledge that \(\Gamma_\bot\) is inconsistent
because of the syntactical nature of the translation.

\todo{Make things clearer}
Even worse, translations should preserve falsehood meaning in particular that
the translation of \(0\) should imply a proof of \(\bot\) in the target.
This is not a concrete proof that it is impossible but rather an argument
to see that such a translation would not behave well. One of the reasons is
that it would translate terms, types and contexts independently when it cannot.
Another is that an \acrshort{ETT} term does not contain any hints with respect
to the uses of reflection.

\subsection{Our translation(s)}

If we go back to the notion of proof~\misref, it becomes apparent that
syntactical translations do not work because we would not be translating proofs
but only partial ones; in other words terms in \acrshort{ETT} are not proofs
because they are insufficient to recover a typing derivation\sidenote{Because
type-checking is undecidable.}.
Thus comes the question of what is a \emph{suitable proof} in \acrshort{ETT}.
There is probably an intermediate structure between the term and the full
derivation that fits this role in the form of a term together with explicit
casts, however, in the setting of this thesis, we will still use complete
typing derivations in \acrshort{ETT} as \emph{proofs}.

Many problems stem from this approach unfortunately. Since we're translating
derivations there is no guarantee that the same term \(t\) in
\(\xisterm{\Gamma}{t}{A}\) and \(\xisterm{\Delta}{t}{B}\) will be translated
twice to the same term, this actually goes even for two different derivations
of the same judgement \(\xisterm{\Gamma}{t}{A}\).
This seems like a big obstacle to compositionality and would be iredeemable
without extra care regarding how individual judgements are translated.

We solve these problems by relating translations of a term (respectively
type and context) to the term itself, \emph{syntactically}.

\section{Target(s) of the translation}

Strictly speaking, elimination of reflection should be a translation from a
certain type theory \(T\) extended with reflection to \(T\) itself.
To fit this framework, we would provide a translation from \acrshort{ETT}
to \acrshort{ITT}. With Simon Boulier we discovered however that it is possible
to do something even stronger and go directly to a much weaker theory where
the notion of conversion is removed, \acrshort{WTT}.

\begin{figure}[hb]
  \includegraphics[width=0.9\textwidth]{elim-reflection-summary}
\end{figure}

From a translation of \acrshort{ETT} to \acrshort{WTT} we get an indirect one
from \acrshort{ETT} to \acrshort{ITT} as well as one from \acrshort{ITT} to
\acrshort{WTT}.
\marginnote[-0.5cm]{In both cases we exploit the fact that \acrshort{ETT}
extends \acrshort{ITT} which in turn extends \acrshort{WTT}.}

Note that in both cases, we're not dealing with arbitrary notions of
\acrshort{ITT} and \acrshort{WTT} but ones extended with some principles on
equality that will be described in \vrefsec{ext-principles}.

\section{Goal of the translation}

Why would we want to eliminate reflection? The first interest is that it
justfies that adding the reflection rule preserves consistency.
The main take-away however comes from the fact that we can show that
\acrshort{ETT} is conservative over both \acrshort{ITT} and \acrshort{WTT},
meaning that in order to prove a statement in one of those, it is enough
to prove it using the refleciton rule.
%
\reminder[-2.3cm]{Conservativity (roughly)}{
  A theory \(\mathcal{S}\) is said to be \emph{conservative} over a
  theory \(\mathcal{T}\) when every statement in \(\mathcal{T}\) that
  is provable in \(\mathcal{S}\) is also provable in \(\mathcal{T}\).
}
%

This can have practical use when proving theorems in a proof assistant like \Coq
which doesn't have reflection. If the translation is constructive, it gives
rise to an algorithm to transform a proof using reflection into a proof without.

The deduced \acrshort{ITT} to \acrshort{WTT} even teaches us that computation
is more of a commodity than a necessity as proofs can be transformed not to
exploit any computational behaviour.

\section{Extensionality principles on equality}
\labsec{ext-principles}

As we said earier, the \acrshort{ITT} and \acrshort{WTT} we consider are
actually extended with extensional principles on equality.
The main principles we require are \acrshort{UIP} and fonctional extensionality.
These two principles are valid statements of \acrshort{ITT} and \acrshort{WTT}
which are provable in \acrshort{ETT}, to show \acrshort{ETT} is a conservative
extension of the target, it must be provable in it; as it isn't the case we need
to extend the target with those principles.
\reminder[-3.6cm]{\acrshort{UIP}}{
\[ \Pi\ x \ y\ (e \ e' : x = y).\ e = e' \]
}
\reminder[-2.1cm]{\Acrshort{funext}}{
\[ \Pi\ f \ g .\ (\Pi x.\ f\ x = g\ x) \to f = g \]
}

For \acrshort{UIP} it can be show equivalent to Streicher's axiom K
\[
\begin{array}{rcl}
  \mathtt{K} & : & \Prod{A: s} \Prod{x :A} \Prod{e : x = x} e = \refl{x} \\
\end{array}
\]
where \(s\) is a sort, using the elimination on the identity type.
K is provable in \acrshort{ETT} by considering the type
\[
  \Prod{A: s} \Prod{x \ y:A} \Prod{e : x = y} e = \refl{x} \\
\]
which is well typed (using the reflection rule to show that $e$ has
type $x= x$) and which can be inhabited by elimination of the identity
type.

In the same way, \acrshort{funext} is provable in \acrshort{ETT} as shown by the
following diagram.
\[
\begin{array}{cll}
  &\Prod{x : A} f\ x = g\ x \\
  \to & x : A \vdash f\ x \equiv g\ x &
  \mbox{by reflection} \\
  \to &  (\lambda (x:A). {f \ x}) \equiv (\lambda{(x:A)}.{g \ x}) &
  \mbox{by congruence of $\equiv$} \\
  \to &  f \equiv g & \mbox{by $\eta$-law} \\
  \to &  f = g
\end{array}
\]

Therefore, applying our translation to the proofs of those theorems in
\acrshort{ETT} gives corresponding proofs of the same theorems in the target.

As I said, \acrshort{UIP} is independent from \acrshort{ITT}, as first shown by
Hofmann and Streicher using the groupoid model~\sidecite{groupoid-interp}, which
has recently been extended in the setting of univalent type theory using the
simplicial or cubical models~\sidecite{kapulkin2012simplicial,coquand:cubical}.

Similarly, \acrshort{funext} is independent from \acrshort{ITT}, it is folklore
but has recently been formalised by Boulier \emph{et al.} using a simple
syntactical translation~\sidecite{boulier17:next-syntac-model-type-theor}.

As previously said in \vrefsubsec{wtt}, \acrshort{WTT} has to be extended with
more principles in the vein of functional extensionality, like extensionality
of \(\Pi\)-types for similar reasons.

\section{Basic idea of the translation}
\todo{Presnet naive idea and paste sec 1.2 from CPP, maybe not hetero equality
which should be intro before (reminder)}